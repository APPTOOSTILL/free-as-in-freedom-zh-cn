%% Copyright (c) 2002, 2010 Sam Williams
%% Copyright (c) 2010 Richard M. Stallman
%% Permission is granted to copy, distribute and/or modify this
%% document under the terms of the GNU Free Documentation License,
%% Version 1.3 or any later version published by the Free Software
%% Foundation; with no Invariant Sections, no Front-Cover Texts, and
%% no Back-Cover Texts. A copy of the license is included in the
%% file called ``gfdl.tex''.


\chapter{\ifdefined\eng
A Brief Journey through Hacker Hell
\fi
\ifdefined\chs
开往黑客地狱的短暂旅途
\fi
}
\ifdefined\eng
\chaptermark{A Brief Journey}
\fi
\thispagestyle{empty}
%% \ifdefined\eng
%% [RMS: In this chapter my only change is to add a few notes labeled like
%% this one.]
%% \fi

%% \ifdefined\chs
%% [RMS: 在这一章中,我只添加了少量的注释,以这样的形式出现。]
%% \fi


\ifdefined\eng
Richard Stallman stares, unblinking, through the windshield of a rental car, waiting for the light to change as we make our way through downtown Kihei.
\fi

\ifdefined\chs
理查德·斯托曼双目凝视着车外,眼睛一眨不眨,我们正坐在这辆租来的汽车里途经津汇城区,等待着信号灯变绿。
\fi

\ifdefined\eng
The two of us are headed to the nearby town of Pa'ia, where we are scheduled to meet up with some software programmers and their wives for dinner in about an hour or so.
\fi

\ifdefined\chs
我们正在前往附近一个名为芭雅的小镇,去会见一些软件开发者和他们的妻子,然后一同去参加一小时后将在另一地点开场的晚宴。
\fi

\ifdefined\eng
It's about two hours after Stallman's speech at the Maui High Performance Center, and Kihei, a town that seemed so inviting before the speech, now seems profoundly uncooperative. Like most beach cities, Kihei is a one-dimensional exercise in suburban sprawl. Driving down its main drag, with its endless succession of burger stands, realty agencies, and bikini shops, it's hard not to feel like a steel-coated morsel passing through the alimentary canal of a giant commercial tapeworm. The feeling is exacerbated by the lack of side roads. With nowhere to go but forward, traffic moves in spring-like lurches. 200 yards ahead, a light turns green. By the time we are moving, the light is yellow again.
\fi

\ifdefined\chs
斯托曼刚刚在茂宜高性能计算中心完成一次演讲,晚宴的开始时间距离演讲结束只有两小时时间。演讲开始前,津汇对我们的到来显得如此的热情,而现在,则让人感觉到它处处在为难我们。跟很多海边城市一样,津汇只有一条主干道贯穿全城。行驶在津汇的主干道上,看着路边的汉堡店、房产中介和比基尼商店,让人感觉到像是一条巨大的绦虫正在吞下一片穿着钢盔铁甲的食物。除了一直向前开,没有别的选择,车流就像一条蜿蜒的溪流。200码开外,信号灯变绿了。当我们开始往前挪动车子时,灯又变黄了。
\fi

\ifdefined\eng
For Stallman, a lifetime resident of the east coast, the prospect of spending the better part of a sunny Hawaiian afternoon trapped in slow traffic is enough to trigger an embolism. %% [RMS: Since I was driving, I was also losing time to answer my email, and that's a real pain since I can barely keep up anyway.] 
Even worse is the knowledge that, with just a few quick right turns a quarter mile back, this whole situation easily could have been avoided. Unfortunately, we are at the mercy of the driver ahead of us, a programmer from the lab who knows the way and who has decided to take us to Pa'ia via the scenic route instead of via the nearby Pilani Highway.
\fi

\ifdefined\chs
对于斯托曼这样长期在东部生活的人来说,把夏威夷午后明媚的阳光浪费在拥堵的马路上足以诱发他的脑血栓。%% [RMS:因为如果我在开车,我就无法回复电子邮件,这对于我来说实在是太痛苦了。] 
更糟糕的是,其实在1/4英里前,有一些可以右转的道路,如果车辆趁早转弯,就可以避免这可怕的交通拥堵。不幸的是,我们需要跟着前面车辆,那辆车的司机是一名实验室的程序员,他认识去往目的地的路,并且是他决定带我们从芭雅的观光道路通行,而不是绕行附近的彼拉尼高速公路。
\fi

\ifdefined\eng
``This is terrible,'' says Stallman between frustrated sighs. ``Why didn't we take the other route?''
\fi

\ifdefined\chs
斯托曼叹着气失望的说:``这实在是太可怕了,我们为什么没走另一条路呢?''
\fi

\ifdefined\eng
Again, the light a quarter mile ahead of us turns green. Again, we creep forward a few more car lengths. This process continues for another 10 minutes, until we finally reach a major crossroad promising access to the adjacent highway.
\fi

\ifdefined\chs
终于,我们前方1/4英里处的信号灯变绿了。但我们仍然只能前进很短的一段距离。10分钟内这样的过程重复了好几次,直到我们最终慢慢的挪到了相临一条公路的十字路口后。
\fi

\ifdefined\eng
The driver ahead of us ignores it and continues through the intersection.
\fi

\ifdefined\chs
我们前面的车没有理会这个十字路口,继续向前行驶。
\fi

\ifdefined\eng
``Why isn't he turning?'' moans Stallman, throwing up his hands in frustration. ``Can you believe this?''
\fi

\ifdefined\chs
斯托曼沮丧的挥动着双手,抱怨道:``他为什么不拐弯呢?你们不觉得奇怪吗?''
\fi

\ifdefined\eng
I decide not to answer either. I find the fact that I am sitting in a car with Stallman in the driver seat, in Maui no less, unbelievable enough. Until two hours ago, I didn't even know Stallman knew how to drive. Now, listening to Yo-Yo Ma's cello playing the mournful bass notes of ``Appalachian Journey'' on the car stereo and watching the sunset pass by on our left, I do my best to fade into the upholstery.
\fi

\ifdefined\chs
我决定不回答这两个问题。我知道我正与斯托曼在茂宜岛同乘一辆车,他在开车,这让人很难以至信。事实上,两小时前我还不知道斯托曼会不会开车。而现在,我们在车上欣赏着马友友的``阿巴拉契旅行''专辑中大提琴所发出的让人感到悲伤的低音音符,看着太阳渐渐从我们的左边落下,我尽可能的让自己沉浸在这样的氛围中。
\fi

\ifdefined\eng
When the next opportunity to turn finally comes up, Stallman hits his right turn signal in an attempt to cue the driver ahead of us. No such luck. Once again, we creep slowly through the intersection, coming to a stop a good 200 yards before the next light. By now, Stallman is livid.
\fi

\ifdefined\chs
当下一个转弯的机会出现的时候,斯托曼打开了右转向灯,试图提醒前车的司机。不过仍然没有好运气。我们又一次慢慢的驶过了十字路口,在距离下一个信号灯至少200码的地方停了下来。斯托曼终于忍不住大发雷霆。
\fi

\ifdefined\eng
``It's like he's deliberately ignoring us,'' he says, gesturing and pantomiming like an air craft carrier landing-signals officer in a futile attempt to catch our guide's eye. The guide appears unfazed, and for the next five minutes all we see is a small portion of his head in the rearview mirror.
\fi

\ifdefined\chs
``他简直就是故意无视我们的存在。''他愤愤地说。一面又像机场的信号官指挥飞机降落一样打着手势,尝试着吸引向导的注意力。向导像是完全没有注意到这一点,在接下来的5分钟里,我们只能在他的后视镜里看到他的一小部分脑袋。
\fi

\ifdefined\eng
I look out Stallman's window. Nearby Kahoolawe and Lanai Islands provide an ideal frame for the setting sun. It's a breathtaking view, the kind that makes moments like this a bit more bearable if you're a Hawaiian native, I suppose. I try to direct Stallman's attention to it, but Stallman, by now obsessed by the inattentiveness of the driver ahead of us, blows me off.
\fi

\ifdefined\chs
我从斯托曼的车窗向外看,附近的卡胡拉威岛和拉拿夷岛与落日一起构成了一副美丽的画面。这是一副美得让人窒息的画面,我想,如果你一个夏威夷本地人,一定会因为这样的美景而忘记了堵车的烦恼。我试图把斯托曼的注意力引向这里,但是他的注意力仍然一心集中在前面那个无视我们的司机身上,完全不搭理我。
\fi

\ifdefined\eng
When the driver passes through another green light, completely ignoring a ``Pilani Highway Next Right,'' I grit my teeth. I remember an early warning relayed to me by BSD programmer Keith Bostic. ``Stallman does not suffer fools gladly,'' Bostic warned me. ``If somebody says or does something stupid, he'll look them in the eye and say, `That's stupid.'\hspace{0.01in}''
\fi

\ifdefined\chs
司机开过另一个亮着绿灯的路口,完全无视边上``下一个路口向右转驶入彼拉尼高速''的标识。我咂了咂嘴。我记起以前一位BSD程序员基思·博斯蒂克(Keith Bostic)警告过我:``斯托曼无法容忍傻瓜,如果有人说了或者做了一些什么蠢事,他会看着他的眼睛说:`这样做太愚蠢了。'\hspace{0.01in}''
\fi

\ifdefined\eng
Looking at the oblivious driver ahead of us, I realize that it's the stupidity, not the inconvenience, that's killing Stallman right now.
\fi

\ifdefined\chs
看着前方心不在焉的司机,我觉得他做的就是所谓的蠢事,而不只是一些不太聪明的事,这些蠢事正在让斯托曼倍受煎熬。
\fi

\ifdefined\eng
``It's as if he picked this route with absolutely no thought on how to get there efficiently,'' Stallman says.
\fi

\ifdefined\chs
``他好像完全没有想过应该如何有效的到达目的地,所以才选择了这么一条路。''斯托曼说。
\fi

\ifdefined\eng
The word ``efficiently'' hangs in the air like a bad odor. Few things irritate the hacker mind more than inefficiency. It was the inefficiency of checking the Xerox laser printer two or three times a day that triggered Stallman's initial inquiry into the printer source code. It was the inefficiency of rewriting software tools hijacked by commercial software vendors that led Stallman to battle Symbolics and to launch the GNU Project. If, as Jean Paul Sartre once opined, hell is other people, hacker hell is duplicating other people's stupid mistakes, and it's no exaggeration to say that Stallman's entire life has been an attempt to save mankind from these fiery depths.
\fi

\ifdefined\chs
``有效''这个词像是一股坏气味停留在了空中。很少有事情能比``低效''更刺激到一个黑客的神经了。当年,正是因为施乐的打印机一天要检修两三次所来带的低效才激发了斯托曼想要获取打印机源代码的想法。也正是因为要重头编写被商业公司所绑架的商用软件的低效,才引发了斯托曼与Symbolics公司之间的斗争,从而诞生了GNU工程。让-保罗·萨特(Jean Paul Sartre)曾经说,如果别人是地狱,那么黑客的地狱就是重复其它人愚蠢的错误,毫不夸张的说,斯托曼的一生,就是在尝试把人类从地狱的火焰中拯救出来。
\fi

\ifdefined\eng
This hell metaphor becomes all the more apparent as we take in the slowly passing scenery. With its multitude of shops, parking lots, and poorly timed street lights, Kihei seems less like a city and more like a poorly designed software program writ large. Instead of rerouting traffic and distributing vehicles through side streets and expressways, city planners have elected to run everything through a single main drag. From a hacker perspective, sitting in a car amidst all this mess is like listening to a CD rendition of nails on a chalkboard at full volume.
\fi

\ifdefined\chs
这种有关地狱的隐喻在我们缓慢通过这被美景环绕的城市时变得更为明显。四处都是商场、停车场和缺乏设计的信号灯,这看上去不像是个城市,倒更像是一个设计得很糟糕的软件。城市的规划者把城市设计成所有的车辆都要通过主干道,而不是把车流分散到支路和高速公路上。从黑客的角度来说,坐在车里并被围困在这一团糟的交通中,就像是用最大的音量听一张录有在木板上钉钉子声音的CD。
\fi

\ifdefined\eng
``Imperfect systems infuriate hackers,'' observes Steven Levy, another warning I should have listened to before climbing into the car with Stallman. ``This is one reason why hackers generally hate driving cars -- the system of randomly programmed red lights and oddly laid out one-way streets causes delays which are so goddamn \textit{unnecessary} [Levy's emphasis] that the impulse is to rearrange signs, open up traffic-light control boxes . . . redesign the entire system.''\endnote{See Steven Levy, \textit{Hackers} (Penguin USA [paperback], 1984): 40.}
\fi

\ifdefined\chs
``不完美的系统会激怒黑客。''史蒂芬·李维(Steven Levy)说过这样的话,这是我决定与斯托曼同坐一辆车前应该听取的另一个忠告,``这是黑客们通常不喜欢开车的原因之一:这是一个充满不确定性的程序,交通信号灯总是随机的变化,还有横七竖八的单行道,导致交通经常堵塞。这实在是太\textit{不必要}了(李维强调说),应该让黑客们重新安排一下信号灯,打开交通灯控制盒,重新设计整个系统。'' \endnote{参考史蒂芬·李维的《\textit{黑客}》一书(企鹅出版社,美国,平装,1984年)第40页。}
\fi

\ifdefined\eng
More frustrating, however, is the duplicity of our trusted guide. Instead of searching out a clever shortcut -- as any true hacker would do on instinct -- the driver ahead of us has instead chosen to play along with the city planners' game. Like Virgil in Dante's \textit{Inferno}, our guide is determined to give us the full guided tour of this hacker hell whether we want it or not.
\fi

\ifdefined\chs
更让人感到沮丧的,是我们向导的愚笨,他没有像一个黑客那样本能的做出最聪明的选择,选择一条更为聪明的捷径,而是坚持陪着城市设计者玩他们愚蠢的游戏。像但丁《\textit{神曲}》中的维吉尔一样,不管我们是否希望,他都打定主意要让我们完整的体验这个黑客的地狱。
\fi

\ifdefined\eng
Before I can make this observation to Stallman, the driver finally hits his right turn signal. Stallman's hunched shoulders relax slightly, and for a moment the air of tension within the car dissipates. The tension comes back, however, as the driver in front of us slows down. ``Construction Ahead'' signs line both sides of the street, and even though the Pilani Highway lies less than a quarter mile off in the distance, the two-lane road between us and the highway is blocked by a dormant bulldozer and two large mounds of dirt.
\fi

\ifdefined\chs
在我还没有来得及告诉斯托曼我的发现以前,前方的司机终于在路口右转了。斯托曼耸起的肩膀总于放松了一些,车里紧张的空气稍稍消散了一些。然而,当前面的司机把车慢慢停下来时,紧张的气氛又回来了。道路两侧的放着``前方施工''的标识,虽然彼拉尼高速公路就在前方不到四分之一英里的距离,我们的车与高速公路间的一条两车道的公路被一辆停着的推土车和两大堆土方完全堵住了。
\fi

\ifdefined\eng
It takes Stallman a few seconds to register what's going on as our guide begins executing a clumsy five-point U-turn in front of us. When he catches a glimpse of the bulldozer and the ``No Through Access'' signs just beyond, Stallman finally boils over.
\fi

\ifdefined\chs
我们的向导在我们的面前忽然把车笨拙的调了个头,斯托曼一时都没有反应过来。当他看到了面前的推土机和``禁止通行''标识后,他终于再一次爆发了。
\fi

\ifdefined\eng
``Why, why, why?'' he whines, throwing his head back. ``You should have known the road was blocked. You should have known this way wouldn't work. You did this deliberately.''  %% [RMS: I meant that he chose the slow road deliberately.  As explained below, I think these quotes are not exact.] 
\fi

\ifdefined\chs
``为什么,为什么,为什么?''他抱怨道,把头一仰,``你早该知道前面的路被封住了,你早该知道不能走这条路。你是故意这么做的。'' %%[RMS:我原来的意思是说他故意选择了那条拥堵的路,跟下面所解释的一样,我觉得这里所引用的我说的话并不准确。]
\fi

\ifdefined\eng
The driver finishes the turn and passes us on the way back toward the main drag. As he does so, he shakes his head and gives us an apologetic shrug. Coupled with a toothy grin, the driver's gesture reveals a touch of mainlander frustration but is tempered with a protective dose of islander fatalism. Coming through the sealed windows of our rental car, it spells out a succinct message: ``Hey, it's Maui; what are you gonna do?''
\fi

\ifdefined\chs
向导的车已经调好了头,从我们身边开过,开回那条拥塞的主干道。在他经过我们身边时,他摇了摇了头,给了我们一个抱歉的耸肩。他露出牙齿,司机的姿势显示出了一个外地人的失望,但是这已经被岛国人的宿命所中和了。从我们租来的车紧闭的窗户望出去,我们似乎看到了这么一句话:``嘿,这就是茂宜岛,你想做什么?''
\fi

\ifdefined\eng
Stallman can take it no longer.
\fi

\ifdefined\chs
斯托曼再也无法忍受了。
\fi

\ifdefined\eng
``Don't you fucking smile!'' he shouts, fogging up the glass as he does so. ``It's your fucking fault. This all could have been so much easier if we had just done it my way.'' %%[RMS: These quotes appear to be inaccurate, because I don't use ``fucking'' as an adverb.  This was not an interview, so Williams would not have had a tape recorder running.  I'm sure things happened overall as described, but these quotations probably reflect his understanding rather than my words.]
\fi

\ifdefined\chs
``你可不可以不要再笑了!''他咆哮道,雾气蒙上了他的眼镜片,``这全他妈是你的错。如果听我的走另一条路就不会这样了。''%%[RMS:这些话记录的似乎并不准确,因为我从来不把``他妈的''当成副词来使用。由于我说这些话里并不是在接受采访,所以威廉姆斯并没有录下我说的话。我很确定这里的文字确实还愿了当时的情景,但是这些话也许是反应了他对我所说的话的个人理解,并不是我的原话。]
\fi

\ifdefined\eng
Stallman accents the words ``my way'' by gripping the steering wheel and pulling himself towards it twice. The image of Stallman's lurching frame is like that of a child throwing a temper tantrum in a car seat, an image further underlined by the tone of Stallman's voice. Halfway between anger and anguish, Stallman seems to be on the verge of tears.
\fi

\ifdefined\chs
斯托曼加重了``听我的''一词,紧紧的抓住方向盘并两次把它拽向自己。斯托曼的样子就像是一个在汽车座位里发脾气的小孩子,他的声音进一步加强了这样的形象。斯托曼又生气又郁闷,眼泪几乎就要掉下来。
\fi

\ifdefined\eng
Fortunately, the tears do not arrive. Like a summer cloudburst, the tantrum ends almost as soon as it begins. After a few whiny gasps, Stallman shifts the car into reverse and begins executing his own U-turn. By the time we are back on the main drag, his face is as impassive as it was when we left the hotel 30 minutes earlier.
\fi

\ifdefined\chs
幸运的是,眼泪最终没有到来。就像夏天的暴雨,斯托曼的愤怒转瞬即逝。他轻轻叹了口气,把车挂到倒档,并开始调头。当我们回到城市主干道上时,他的表情让人过目不忘,就跟我们提前半小时离开酒店时的表情一样。
\fi

\ifdefined\eng
It takes less than five minutes to reach the next cross-street. This one offers easy highway access, and within seconds, we are soon speeding off toward Pa'ia at a relaxing rate of speed. The sun that once loomed bright and yellow over Stallman's left shoulder is now burning a cool orange-red in our rearview mirror. It lends its color to the gauntlet wili wili trees flying past us on both sides of the highway.
\fi

\ifdefined\chs
我们花了不到五分钟时间,到达了下一个十字路口。这里可以容易的驶入高速公路,很快,我们就加大马力向着芭雅驶去。刚才斯托曼左肩上若隐若现的黄色太阳现在变成了橙红色出现在我们的后视镜中。金色的阳光撒向高速公路两边的树木。
\fi

\ifdefined\eng
For the next 20 minutes, the only sound in our vehicle, aside from the ambient hum of the car's engine and tires, is the sound of a cello and a violin trio playing the mournful strains of an Appalachian folk tune.
\fi

\ifdefined\chs
接下来的20分钟,只剩下了汽车的声音,包含着引擎和轮胎的轰鸣声。像是大提琴与小提琴的三重奏,演绎着阿巴拉契的民歌旋律。
\fi

\theendnotes
\setcounter{endnote}{0}
