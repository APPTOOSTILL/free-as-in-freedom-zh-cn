%% Copyright (c) 2002, 2010 Sam Williams
%% Copyright (c) 2010 Richard M. Stallman
%% Copyright (c) 2014 Posts & Telecom Press
%% Permission is granted to copy, distribute and/or modify this
%% document under the terms of the GNU Free Documentation License,
%% Version 1.3 or any later version published by the Free Software
%% Foundation; with no Invariant Sections, no Front-Cover Texts, and
%% no Back-Cover Texts. A copy of the license is included in the
%% file called ``gfdl.tex''.

%% DONE 

\chapter{\ifdefined\eng
Appendix C -- GNU Free Documentation License (GFDL)
\fi
\ifdefined\chs
附录C:GNU 自由文档许可证(GFDL)
\fi} \label{Appendix C}
\phantomsection
\thispagestyle{empty}
% fdl.tex 
% This file is a chapter.  It must be included in a larger document to work
% properly.
\ifdefined\eng
\begin{center}
Version 1.1, March 2000\\

 Copyright \copyright\ 2000  Free Software Foundation, Inc.\\
     51 Franklin St, Fifth Floor, Boston, MA  02110-1301  USA

\bigskip

 Everyone is permitted to copy and distribute verbatim copies
 of this license document, but changing it is not allowed.
\end{center}

\begin{center}
{\bf\large Preamble}
\end{center}

The purpose of this License is to make a manual, textbook, or other
written document ``free'' in the sense of freedom: to assure everyone
the effective freedom to copy and redistribute it, with or without
modifying it, either commercially or noncommercially.  Secondarily,
this License preserves for the author and publisher a way to get
credit for their work, while not being considered responsible for
modifications made by others.

This License is a kind of ``copyleft'', which means that derivative
works of the document must themselves be free in the same sense.  It
complements the GNU General Public License, which is a copyleft
license designed for free software.

We have designed this License in order to use it for manuals for free
software, because free software needs free documentation: a free
program should come with manuals providing the same freedoms that the
software does.  But this License is not limited to software manuals;
it can be used for any textual work, regardless of subject matter or
whether it is published as a printed book.  We recommend this License
principally for works whose purpose is instruction or reference.

\begin{center}
{\Large\bf 1. APPLICABILITY AND DEFINITIONS\par}
\phantomsection
\end{center}

This License applies to any manual or other work that contains a
notice placed by the copyright holder saying it can be distributed
under the terms of this License.  The ``Document'', below, refers to any
such manual or work.  Any member of the public is a licensee, and is
addressed as ``you''.

A ``Modified Version'' of the Document means any work containing the
Document or a portion of it, either copied verbatim, or with
modifications and/or translated into another language.

A ``Secondary Section'' is a named appendix or a front-matter section of
the Document that deals exclusively with the relationship of the
publishers or authors of the Document to the Document's overall subject
(or to related matters) and contains nothing that could fall directly
within that overall subject.  (For example, if the Document is in part a
textbook of mathematics, a Secondary Section may not explain any
mathematics.)  The relationship could be a matter of historical
connection with the subject or with related matters, or of legal,
commercial, philosophical, ethical or political position regarding
them.

The ``Invariant Sections'' are certain Secondary Sections whose titles
are designated, as being those of Invariant Sections, in the notice
that says that the Document is released under this License.

The ``Cover Texts'' are certain short passages of text that are listed,
as Front-Cover Texts or Back-Cover Texts, in the notice that says that
the Document is released under this License.

A ``Transparent'' copy of the Document means a machine-readable copy,
represented in a format whose specification is available to the
general public, whose contents can be viewed and edited directly and
straightforwardly with generic text editors or (for images composed of
pixels) generic paint programs or (for drawings) some widely available
drawing editor, and that is suitable for input to text formatters or
for automatic translation to a variety of formats suitable for input
to text formatters.  A copy made in an otherwise Transparent file
format whose markup has been designed to thwart or discourage
subsequent modification by readers is not Transparent.  A copy that is
not ``Transparent'' is called ``Opaque''.

Examples of suitable formats for Transparent copies include plain
ASCII without markup, Texinfo input format, \LaTeX~input format, SGML
or XML using a publicly available DTD, and standard-conforming simple
HTML designed for human modification.  Opaque formats include
PostScript, PDF, proprietary formats that can be read and edited only
by proprietary word processors, SGML or XML for which the DTD and/or
processing tools are not generally available, and the
machine-generated HTML produced by some word processors for output
purposes only.

The ``Title Page'' means, for a printed book, the title page itself,
plus such following pages as are needed to hold, legibly, the material
this License requires to appear in the title page.  For works in
formats which do not have any title page as such, ``Title Page'' means
the text near the most prominent appearance of the work's title,
preceding the beginning of the body of the text.


\begin{center}
{\Large\bf 2. VERBATIM COPYING\par}
\phantomsection
\end{center}


You may copy and distribute the Document in any medium, either
commercially or noncommercially, provided that this License, the
copyright notices, and the license notice saying this License applies
to the Document are reproduced in all copies, and that you add no other
conditions whatsoever to those of this License.  You may not use
technical measures to obstruct or control the reading or further
copying of the copies you make or distribute.  However, you may accept
compensation in exchange for copies.  If you distribute a large enough
number of copies you must also follow the conditions in section 3.

You may also lend copies, under the same conditions stated above, and
you may publicly display copies.


\begin{center}
{\Large\bf 3. COPYING IN QUANTITY\par}
\phantomsection
\end{center}


If you publish printed copies of the Document numbering more than 100,
and the Document's license notice requires Cover Texts, you must enclose
the copies in covers that carry, clearly and legibly, all these Cover
Texts: Front-Cover Texts on the front cover, and Back-Cover Texts on
the back cover.  Both covers must also clearly and legibly identify
you as the publisher of these copies.  The front cover must present
the full title with all words of the title equally prominent and
visible.  You may add other material on the covers in addition.
Copying with changes limited to the covers, as long as they preserve
the title of the Document and satisfy these conditions, can be treated
as verbatim copying in other respects.

If the required texts for either cover are too voluminous to fit
legibly, you should put the first ones listed (as many as fit
reasonably) on the actual cover, and continue the rest onto adjacent
pages.

If you publish or distribute Opaque copies of the Document numbering
more than 100, you must either include a machine-readable Transparent
copy along with each Opaque copy, or state in or with each Opaque copy
a publicly-accessible computer-network location containing a complete
Transparent copy of the Document, free of added material, which the
general network-using public has access to download anonymously at no
charge using public-standard network protocols.  If you use the latter
option, you must take reasonably prudent steps, when you begin
distribution of Opaque copies in quantity, to ensure that this
Transparent copy will remain thus accessible at the stated location
until at least one year after the last time you distribute an Opaque
copy (directly or through your agents or retailers) of that edition to
the public.

It is requested, but not required, that you contact the authors of the
Document well before redistributing any large number of copies, to give
them a chance to provide you with an updated version of the Document.


\begin{center}
{\Large\bf 4. MODIFICATIONS\par}
\phantomsection
\end{center}

You may copy and distribute a Modified Version of the Document under
the conditions of sections 2 and 3 above, provided that you release
the Modified Version under precisely this License, with the Modified
Version filling the role of the Document, thus licensing distribution
and modification of the Modified Version to whoever possesses a copy
of it.  In addition, you must do these things in the Modified Version:

\begin{itemize}

\item Use in the Title Page (and on the covers, if any) a title distinct
   from that of the Document, and from those of previous versions
   (which should, if there were any, be listed in the History section
   of the Document).  You may use the same title as a previous version
   if the original publisher of that version gives permission.
\item List on the Title Page, as authors, one or more persons or entities
   responsible for authorship of the modifications in the Modified
   Version, together with at least five of the principal authors of the
   Document (all of its principal authors, if it has less than five).
\item State on the Title page the name of the publisher of the
   Modified Version, as the publisher.
\item Preserve all the copyright notices of the Document.
\item Add an appropriate copyright notice for your modifications
   adjacent to the other copyright notices.
\item Include, immediately after the copyright notices, a license notice
   giving the public permission to use the Modified Version under the
   terms of this License, in the form shown in the Addendum below.
\item Preserve in that license notice the full lists of Invariant Sections
   and required Cover Texts given in the Document's license notice.
\item Include an unaltered copy of this License.
\item Preserve the section entitled ``History'', and its title, and add to
   it an item stating at least the title, year, new authors, and
   publisher of the Modified Version as given on the Title Page.  If
   there is no section entitled ``History'' in the Document, create one
   stating the title, year, authors, and publisher of the Document as
   given on its Title Page, then add an item describing the Modified
   Version as stated in the previous sentence.
\item Preserve the network location, if any, given in the Document for
   public access to a Transparent copy of the Document, and likewise
   the network locations given in the Document for previous versions
   it was based on.  These may be placed in the ``History'' section.
   You may omit a network location for a work that was published at
   least four years before the Document itself, or if the original
   publisher of the version it refers to gives permission.
\item In any section entitled ``Acknowledgements'' or ``Dedications'',
   preserve the section's title, and preserve in the section all the
   substance and tone of each of the contributor acknowledgements
   and/or dedications given therein.
\item Preserve all the Invariant Sections of the Document,
   unaltered in their text and in their titles.  Section numbers
   or the equivalent are not considered part of the section titles.
\item Delete any section entitled ``Endorsements''.  Such a section
   may not be included in the Modified Version.
\item Do not retitle any existing section as ``Endorsements''
   or to conflict in title with any Invariant Section.

\end{itemize}

If the Modified Version includes new front-matter sections or
appendices that qualify as Secondary Sections and contain no material
copied from the Document, you may at your option designate some or all
of these sections as invariant.  To do this, add their titles to the
list of Invariant Sections in the Modified Version's license notice.
These titles must be distinct from any other section titles.

You may add a section entitled ``Endorsements'', provided it contains
nothing but endorsements of your Modified Version by various
parties -- for example, statements of peer review or that the text has
been approved by an organization as the authoritative definition of a
standard.

You may add a passage of up to five words as a Front-Cover Text, and a
passage of up to 25 words as a Back-Cover Text, to the end of the list
of Cover Texts in the Modified Version.  Only one passage of
Front-Cover Text and one of Back-Cover Text may be added by (or
through arrangements made by) any one entity.  If the Document already
includes a cover text for the same cover, previously added by you or
by arrangement made by the same entity you are acting on behalf of,
you may not add another; but you may replace the old one, on explicit
permission from the previous publisher that added the old one.

The author(s) and publisher(s) of the Document do not by this License
give permission to use their names for publicity for or to assert or
imply endorsement of any Modified Version.


\begin{center}
{\Large\bf 5. COMBINING DOCUMENTS\par}
\phantomsection
\end{center}

You may combine the Document with other documents released under this
License, under the terms defined in section 4 above for modified
versions, provided that you include in the combination all of the
Invariant Sections of all of the original documents, unmodified, and
list them all as Invariant Sections of your combined work in its
license notice.

The combined work need only contain one copy of this License, and
multiple identical Invariant Sections may be replaced with a single
copy.  If there are multiple Invariant Sections with the same name but
different contents, make the title of each such section unique by
adding at the end of it, in parentheses, the name of the original
author or publisher of that section if known, or else a unique number.
Make the same adjustment to the section titles in the list of
Invariant Sections in the license notice of the combined work.

In the combination, you must combine any sections entitled ``History''
in the various original documents, forming one section entitled
``History''; likewise combine any sections entitled ``Acknowledgements'',
and any sections entitled ``Dedications''.  You must delete all sections
entitled ``Endorsements.''


\begin{center}
{\Large\bf 6. COLLECTIONS OF DOCUMENTS\par}
\phantomsection
\end{center}

You may make a collection consisting of the Document and other documents
released under this License, and replace the individual copies of this
License in the various documents with a single copy that is included in
the collection, provided that you follow the rules of this License for
verbatim copying of each of the documents in all other respects.

You may extract a single document from such a collection, and distribute
it individually under this License, provided you insert a copy of this
License into the extracted document, and follow this License in all
other respects regarding verbatim copying of that document.


\begin{center}
{\Large\bf 7. AGGREGATION WITH INDEPENDENT WORKS\par}
\phantomsection
\end{center}

A compilation of the Document or its derivatives with other separate
and independent documents or works, in or on a volume of a storage or
distribution medium, does not as a whole count as a Modified Version
of the Document, provided no compilation copyright is claimed for the
compilation.  Such a compilation is called an ``aggregate'', and this
License does not apply to the other self-contained works thus compiled
with the Document, on account of their being thus compiled, if they
are not themselves derivative works of the Document.

If the Cover Text requirement of section 3 is applicable to these
copies of the Document, then if the Document is less than one quarter
of the entire aggregate, the Document's Cover Texts may be placed on
covers that surround only the Document within the aggregate.
Otherwise they must appear on covers around the whole aggregate.


\begin{center}
{\Large\bf 8. TRANSLATION\par}
\phantomsection
\end{center}

Translation is considered a kind of modification, so you may
distribute translations of the Document under the terms of section 4.
Replacing Invariant Sections with translations requires special
permission from their copyright holders, but you may include
translations of some or all Invariant Sections in addition to the
original versions of these Invariant Sections.  You may include a
translation of this License provided that you also include the
original English version of this License.  In case of a disagreement
between the translation and the original English version of this
License, the original English version will prevail.


\begin{center}
{\Large\bf 9. TERMINATION\par}
\phantomsection
\end{center}

You may not copy, modify, sublicense, or distribute the Document except
as expressly provided for under this License.  Any other attempt to
copy, modify, sublicense or distribute the Document is void, and will
automatically terminate your rights under this License.  However,
parties who have received copies, or rights, from you under this
License will not have their licenses terminated so long as such
parties remain in full compliance.


\begin{center}
{\Large\bf 10. FUTURE REVISIONS OF THIS LICENSE\par}
\phantomsection
\end{center}

The Free Software Foundation may publish new, revised versions
of the GNU Free Documentation License from time to time.  Such new
versions will be similar in spirit to the present version, but may
differ in detail to address new problems or concerns. See
http://www.gnu.org/copyleft/.

Each version of the License is given a distinguishing version number.
If the Document specifies that a particular numbered version of this
License ``or any later version'' applies to it, you have the option of
following the terms and conditions either of that specified version or
of any later version that has been published (not as a draft) by the
Free Software Foundation.  If the Document does not specify a version
number of this License, you may choose any version ever published (not
as a draft) by the Free Software Foundation.

\begin{center}
{\Large\bf ADDENDUM: How to use this License for your documents\par}
\phantomsection
\end{center}

To use this License in a document you have written, include a copy of
the License in the document and put the following copyright and
license notices just after the title page:

\begin{quote}

      Copyright \copyright\ YEAR  YOUR NAME.
      Permission is granted to copy, distribute and/or modify this document
      under the terms of the GNU Free Documentation License, Version 1.1
      or any later version published by the Free Software Foundation;
      with the Invariant Sections being LIST THEIR TITLES, with the
      Front-Cover Texts being LIST, and with the Back-Cover Texts being LIST.
      A copy of the license is included in the section entitled ``GNU
      Free Documentation License''.

\end{quote}

If you have no Invariant Sections, write ``with no Invariant Sections''
instead of saying which ones are invariant.  If you have no
Front-Cover Texts, write ``no Front-Cover Texts'' instead of
``Front-Cover Texts being LIST''; likewise for Back-Cover Texts.

If your document contains nontrivial examples of program code, we
recommend releasing these examples in parallel under your choice of
free software license, such as the GNU General Public License,
to permit their use in free software.
\fi

\ifdefined\chs
以下翻译参考自``\url{http://www.linuxfocus.org/Chinese/team/fdl.html}''。原译者: 王旭 <wangxu(at)linuxfocus.org>

\begin{center}
{\large GNU 自由文档许可证 \\ 1.1版本 \\ 2000年3月}

版权所有 (C) 2000 自由软件基金会 59 Temple Place, Suite 330, Boston, MA 02111-1307 USA 
\end{center}

\bigskip

任何人都可以复制并分发本许可证文档的原始拷贝,但不允许进行修改。


\begin{center}
{\bf\large 前言}
\end{center}

本许可证用于使得手册、教材或其他功能的有用的文档是自由的(``free'') :保证任何人确实可以自由地拷贝与分发经过或未经改动的该文档,无论是否用于商业目的。此外,本许可证保护作者和出版者对他们的作品的信誉,不需要对其他人的改动负责。


这个许可证是一种``copyleft'',这意味着基于遵循 GFPL 文档的衍生作品必须也是同样自由的。这个许可证是为自由软件设计的 copyleft 许可证 GPL 的补充。


我们为了用于自由软件的手册设计了这个许可证,因为自由软件需要自由的文档:一个自由的程序应该跟随着一个给出同样自由的文档一同发布。但这个许可证不仅仅限于软件的手册,它可以被用于任何文本作品,无论它的主题或它是否作为一本印刷的书籍被发行。我们建议将这个许可证主要用于说明或参考性的文献作品。


\begin{center}
{\Large\bf 1. 适用性与定义\par}
\phantomsection
\end{center}

本许可证适用于包含一个宣布该作品在本许可证下发布的声明的使用任何介质的任何手册或其他作品。这个声明授权在这里声明的条件下,在任何时间、地点、无版税地使用作品的许可。下述的``文档''指任何这样的手册或作品。任何公众中的一员都是被授予许可的人,这里被称为``你''。如果你在需要版权法律许可的情况下复制、修改或分发这个作品,你就接受了这个许可证。

文档的``修订版本''意味着包含文档或文档的一部分的作品,或者是原始拷贝,或者进行了修改和/或翻译成为了其他的语言。

``次要章节''(Secondary Section) 是文档中特定的附录或导言中的节,专用于处理文档的出版者或作者与文档的全部主题(或相关问题)的关系,不包含任何在全部主题之中的内容。(也就是说,如果文档是一本数学教材,``次要章节''可能不包含任何数学内容。)其中的关系可能是和主题的历史关联,或者是相关问题,或者是关于主题的有关法律、商业、哲学、伦理或政治关系。

``不可变章节''(Invariant Sections) 是指定的标题的次要章节,这个标题在文档以本许可证发布的声明中被声明为不可变章节。如果一个章节不符合上述的次要章节的定义,它就不可以被声明为不可变章节。文档可以没有不可变章节。如果文档没有指定不可变章节,则视为没有不可变章节。

``封皮文本''(Cover Texts) 是一些在文档以本许可证发布的声明中被列为封面文本(Front-Cover Text) 或封底文本(Back-Cover Text) 的特定的短段落。封面文本最多 5 个单词,封底文本最多 25 个单词。

文档的``透明''(Transparent) 拷贝是一个机器可读的拷贝,使用公众可以得到其规范的格式表达,这样的拷贝适合于使用通用文本编辑器、 (对于像素构成的图像(image)) 通用绘图(paint)程序、(对于绘制的图形(draw))广泛使用的绘画(draw)程序直接修改文档,也适用于输入到文本格式处理程序或自动翻译成各种适于适用于输入到文本格式处理程序的格式。一个用其他透明文件格式表示的拷贝,如果该格式的标记(或缺少标记)已经构成了对读者的后续的修改的障碍,那么就不是透明的。如果用一个图像格式表示确实有效的文本,不论数量多少,都不是透明各式的。不``透明''的拷贝称为``不透明''(Opaque)。

适于作为透明拷贝的格式的例子有:没有标记的纯 ASCII 文本、Texinfo 的输入格式、LaTeX 的输入格式、使用公众可用的 DTD 的 SGML 或 XML,符合标准的简单 HTML、可以手工修改的 PostScript 或 PDF。透明的图像格式的例子有 PNG, XCF 和 JPG。不透明的格式包括:尽可以被私有版权的字处理软件使用的私有版权格式、所用的 DTD 和/或处理工具不是广泛可用的 SGML 或 XML,机器生成的 HTML,一些字处理器生成的只用于输出目的的 PostScript 或 PDF。

对于被印刷的书籍,``扉页''(Title Page)就是扉页本身以及随后的一些用于补充的页,本许可需要出现在扉页上。对于那些没有扉页的作品形式,扉页代表接近作品最突出的标题的、在文本正文之前的文本。


\begin{center}
{\Large\bf 2. 逐字地复制\par}
\phantomsection
\end{center}

你可以以任何媒质拷贝并分发文档,无论是否处于商业目的,只要保证本许可证、版权声明和宣称本许可证应用于文档的声明都在所有复制品中被完整地、无任何附加条件地给出即可。你不能使用任何技术手段阻碍或控制你制作或发布的拷贝的阅读或再次复制。不过你可以在复制品的交易中得到报酬。如果你发布足够多的拷贝,你必须遵循下面第三节中的条件。

你也可以在和上面相同的条件下出租拷贝和向公众放映拷贝。

\begin{center}
{\Large\bf 3. 适用性与定义\par}
\phantomsection
\end{center}


如果你发行文档的印刷版的拷贝 (或是有印制封皮的其他媒质的拷贝) 多于 100 份,而文档的许可证声明中要求封皮文本,你必须将它清晰明显地置于封皮之上,封面文本在封面上,封底文本在封底上。封面和封底上还必须标明你是这些拷贝的发行者。封面必须以同等显著、可见地完整展现标题的所有文字。你可以在标题上加入其他的材料。改动仅限于封皮的复制,只要保持文档的标题不变并满足这些条件,可以在其他方面被视为是的逐字的复制。

如果需要加上的文本对于封面或封底过多,无法明显地表示,你应该在封皮上列出前面的(在合理的前提下尽量多),把其它的放在邻近的页面上。

如果你出版或分发了超过 100 份文档的不透明拷贝,你必须在每个不透明拷贝中包含一份机器可读的透明拷贝,或是在每个不透明拷贝中给出一个计算机网络地址,通过这个地址,使用计算机网络的公众可以使用标准的网络协议没有任何附加条件的下载一个文档的完整的透明拷贝。如果你选择后者,你必须在开始大量分发非透明拷贝的时候采用相当谨慎的步骤,保证透明拷贝在其所给出的位置在(直接或通过代理和零售商)分发最后一次该版本的非透明拷贝的时间之后一年之内始终是有效的。

在重新大量发布拷贝之前,请你(但不是必须)与文档的作者联系,以便可以得到文档的更新版本。


\begin{center}
{\Large\bf 4. 修改\par}
\phantomsection
\end{center}



在上述第 2、3 节的条件下,你可以复制与分发文档的修改后的版本,只要严格的按照本许可证发布修改后的文档,在文档的位置填入修改后的版本,也就是许可任何得到这个修改版的拷贝的人分发或修改这个修改后的版本。另外,在修改版中,你需要做到如下几点:

使用和被修改文档与以前的各个版本 (如果有的话,应该被列在文档的版本历史章节中) 有显著不同的扉页(和封面,如果有的话)。如果那个版本的原始发行者允许的话,你可以使用和以前版本相同的标题。 

与作者一样,在扉页上列出承担修改版本中的修改的作者责任的一个或多个人或实体和至少五个文档的原作者 (如果原作者不足五个就全部列出),除非他们免除了你的这个责任。在修改版本的扉页上注明该版本的发布者。

保持文档的全部版权声明不变。 

在与其他版权声明邻近的位置加入恰当的针对你的修改的版权声明。 

在紧接着版权声明的位置加入许可声明,按照下面附录中给出的形式,以本许可证给公众授予使用修订版本的权利。 

保持原文档的许可声明中的全部不可变章节、封面文字和封底文字的声明不变。 

包含一份未作任何修改的本许可证的拷贝。

保持命名为特殊标题``History''(版本历史) 的章节不变,保持它的标题不变,并在其中加入一项,该项至少声明扉页上的修改版本的标题、年、新作者和新发行者。如果文档中没有一个特殊标题版本历史的章节,就新建这一章节,并加入声明原文档扉页上所列的标题、年、作者与发行者的项,之后在后面加入如上句所说的描述修改版本的项。 

如果问当中有用于公众访问的文档透明拷贝的网址的话,保持网址不变,并同样把它所基于的以前版本的网址给出。这些网址可以放在特殊标题版本历史章节。你可以不给出那些在原文档发行之前已经发行至少四年的版本给出的网址,或者该版本的发行者授权不列出网址。 

对于任何以特殊标题``Acknowledgements''(致谢) 特殊标题``Dedications'' (献给)命名的章节,保持标题的章节不变,并保持其全部内容和对每个贡献者的感谢与列出的奉献的语气不变。 

保持文档的所有不可变章节不变,不改变它们的标题和内容。章节的编号或等价的东西不被认为是章节标题的一部分。 

删除以特殊标题``Endorsements''(签名)命名的章节。这样的章节不可以被包含在修改后的版本中。 

不要把一个已经存在的章节重命名为特殊标题``Endorsements''(签名)或和任何不可变章节的名字相冲突的名字。 

如果修改版本加入了新的符合次要章节定义的引言或附录章节,并且不含有从原文档中复制的内容,你可以按照你的意愿将它标记为不可变。如果需要这样做,就把它们的标题加入修改版本的许可声明的不可变章节列表之中。这些标题必须和其他章节的标题相区分。

你可以加入一个命名为特殊标题``Endorsements''的章节,只要它只包含对你的修改版本由不同的各方给出的签名--例如书评或是声明文本已经被一个组织认定为一个标准的权威定义。

你可以加入一个最多五个字的段落作为封面文本和一个最多25个字的段落作为封底文本,把它们加入到修改版本的封皮文本列表的末端。一个实体之可以加入 (或通过排列(arrangement)制作) 一段封面或封底文本。如果原文档已经为该封皮(封面或封底)包含了封皮文本,由你或你所代表的实体先前加入或排列的文本,你不能再新加入一个,但你可以在原来的发行者的显示的许可下替换掉原来的那个

作者和发行者不能通过本许可证授权公众使用他们的名字推荐或暗示认可任何一个修改版本。

\begin{center}
{\Large\bf 5. 合并文档\par}
\phantomsection
\end{center}


遵照第4节所说的修改版本的规定,你可以将文档和其他文档合并并以本许可证发布,只要你在合并结果中包含原文档的所有不可变章节,对它们不加以任何改动,并在合并结果的许可声明中将它们全部列为不可变章节,而且维持原作者的免责声明不变。

合并的作品仅需要包含一份本许可证,多个相同的不可变章节可以由一个来取代。如果有多个名称相同、内容不同的不可变章节,通过在章节的名字后面用包含在括号中的文本加以原作者、发行者的名字 (如果有的话) 来加以区别,或通过唯一的编号加以区别。并对合并作品的许可声明中的不可变章节列表中的章节标题做相同的修改。

在合并过程中,你必须合并不同原始文档中任何以特殊标题``版本历史''命名的章节,从而形成新的特殊标题``版本历史''的章节;类似地,还要合并特殊标题``致谢''和``献给''命名的章节。你必须删除所有以特殊标题``签名''(Endorsements)命名的章节。

\begin{center}
{\Large\bf 6. 文档的合集\par}
\phantomsection
\end{center}


你可以制作一个文档和其他文档的合集,在本许可证下发布,并在合集中将不同文档中的多个本许可证的拷贝以一个单独的拷贝来代替,只要你在文档的其他方面遵循本许可证的逐字地拷贝的条款即可。

你可以从一个这样的合集中提取一个单独的文档,并将它在本许可证下单独发布,只要你想这个提取出的文档中加入一份本许可证的拷贝,并在文档的其他方面遵循本许可证的逐字地拷贝的原则。

\begin{center}
{\Large\bf 7. 独立作品的聚合体\par}
\phantomsection
\end{center}


文档或文档的派生品和其它的与之相分离的独立文档或作品编辑在一起,在一个大包中或大的发布媒质上,如果其结果著作权对编辑作品的使用者的权利的限制没有超出原来的独立作品的许可范围,称为文档的``聚合体''(aggregate)。当以本许可证发布的文档被包含在一个聚合体中的时候,本许可证不施加于聚合体中的本来不是该文档的派生作品的其他作品。

如果第3节中的封皮文本的需求适用于文档的拷贝,那么如果文档在聚合体中所占的比重小于全文的一半,文档的封皮文本可以被放置在聚合体内包含文档的部分的封皮上,或是电子文档中的等效部分。否则,它必须位于整个聚合体的印刷的封皮上。


\begin{center}
{\Large\bf 8. 翻译\par}
\phantomsection
\end{center}


翻译被认为是一种修改,所以你可以按照第4节的规定发布文档的翻译版本。如果要将文档的不可变章节用翻译版取代,需要得到著作权人的授权,但你可以将部分或全部不可变章节的翻译版附加在原始版本的后面。你可以包含一个本许可证和所有许可证声明、免责声明的翻译版本,,只要你同时包含他们的原始英文版本即可。当翻译版本和英文版发生冲突的时候,原始版本有效。

\begin{center}
{\Large\bf 9. 许可的终止\par}
\phantomsection
\end{center}


除非确实遵从本许可证,你不可以对遵从本许可证发布的文档进行复制、修改、附加许可证或发布。任何其它的试图复制、修改、附加许可、发布本文档的行为都是无效的,并自动终止本许可证所授予你的权利。然而其他从你这里依照本许可证得到的拷贝或权力的人(或组织)得到的许可证都不会终止,只要他们仍然完全遵照本许可证。

\begin{center}
{\Large\bf 10. 本许可证的未来修订版本\par}
\phantomsection
\end{center}


未来的某天,自由软件基金会 (FSF) 可能会发布 GNU 自由文档许可证的修订版本。这些版本将会和现在的版本体现类似的精神,但可能在解决某些问题和利害关系的细节上有所不同。参见 http://www.gnu.org/copyleft/ 。

本许可证的每个版本都有一个唯一的版本号。如果文档指定服从一个特定的本许可证版本``或任何之后的版本''(or any later version),你可以选择遵循指定版本或自由软件基金会的任何更新的已经发布的版本 (不是草案) 的条款和条件来遵循。如果文档没有指定本许可证的版本,那么你可以选择遵循任何自由软件基金会曾经发布的版本 (不是草案)。

\begin{center}
{\Large\bf 附录:如何使用本许可证\par}
\phantomsection
\end{center}


要使用本许可证发布你写的文档,请在文档中包含本许可证的一个副本,并在紧接着扉页之后加入如下版权声明与许可声明:

\begin{quote}

Copyright (C) YEAR YOUR NAME.
Permission is granted to copy, distribute and/or modify this document under the terms of the GNU Free Documentation License, Version 1.1 or any later version published by the Free Software Foundation; with the Invariant Sections being LIST THEIR TITLES, with the Front-Cover Texts being LIST, and with the Back-Cover Texts being LIST. A copy of the license is included in the section entitled ``GNU Free Documentation License''.

\end{quote}

如果你有不可变章节、封面文本和封底文本,请将``with...Texts.''一行替换为:

如果你有不可变章节而没有封皮文本或上述三项的其他组合,将两个版本结合使用以满足实际情况。

如果你的文档包含有意义的程序示例代码,我们建议您同时将代码按照您的选择以自由软件许可证发布,比如 GNU 通用公共许可证。以授权它们作为自由软件被使用。
\fi
