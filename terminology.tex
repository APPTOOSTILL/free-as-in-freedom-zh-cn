%% Copyright (c) 2002 Sam Williams
%% Permission is granted to copy, distribute and/or modify this
%% document under the terms of the GNU Free Documentation License,
%% Version 1.3 or any later version published by the Free Software
%% Foundation; with no Invariant Sections, no Front-Cover Texts, and
%% no Back-Cover Texts. A copy of the license is included in the
%% file called ``gfdl.tex''.

%% DONE

\chapter{\ifdefined\eng
Appendix A -- Terminology
\fi
\ifdefined\chs
附录A:术语
\fi
}

\ifdefined\eng
For the most part, I have chosen to use the term GNU/Linux in reference to the free software operating system and Linux when referring specifically to the kernel that drives the operating system. The most notable exception to this rule comes in \autoref{chapter:gpl}. In the final part of that chapter, I describe the early evolution of Linux as an offshoot of Minix. It is safe to say that during the first two years of the project's development, the operating system Torvalds and his colleagues were working on bore little similarity to the GNU system envisioned by Stallman, even though it gradually began to share key components, such as the GNU C Compiler and the GNU Debugger.
\fi

\ifdefined\chs
在大多数情况下,我使用GNU/Linux这个词来表示基于Linux内核的自由软件操作系统,使用Linux这个词来特指驱动这个操作系统的内核。但在\autoref{chapter:gpl}中,这个原则不太适用。在这章的最后一部分,我介绍了Linux作为Minix的衍生作品的早期演化历史。在这个项目开始开发的前两年中,托瓦兹和他的同事们所开发的操作系统与斯托曼所设想的GNU系统并没有很相似之处,但它们慢慢的开始共享一些关键的组件,比如GNU C编译器和GNU调试器。这样的决定使得斯托曼在1993年以前并没有很重视要求在Linux前加上GNU的标识。
\fi

\ifdefined\eng
This decision further benefits from the fact that, prior to 1993, Stallman saw little need to insist on credit.
\fi

\ifdefined\chs
这样的决定使得斯托曼在1993年以前并没有很重视要求在Linux前加上GNU的标识。
\fi

\ifdefined\eng
Some might view the decision to use GNU/Linux for later versions of the same operating system as arbitrary. I would like to point out that it was in no way a prerequisite for gaining Stallman's cooperation in the making of this book. I came to it of my own accord, partly because of the operating system's modular nature and the community surrounding it, and partly because of the apolitical nature of the Linux name. Given that this is a biography of Richard Stallman, it seemed inappropriate to define the operating system in apolitical terms.
\fi

\ifdefined\chs
有些人可能会认为在后期把这个操作系统称为GNU/Linux有点专制。我想指出的是,这是让斯托曼与我合作完成本书的一个基本前提。对于这个问题,我有自己的观点,一方面在于这个操作系统的模块化设计和围绕它的社区,另一方面在于Linux这个名字本身并不具备政治意义。由于这是一本理查德·斯托曼的传记,所以用政治术语来定义一个操作系统并不合适。
\fi

\ifdefined\eng
In the final phases of the book, when it became clear that O'Reilly \& Associates would be the book's publisher, Stallman did make it a condition that I use ``GNU/Linux'' instead of Linux if O'Reilly expected him to provide promotional support for the book after publication. When informed of this, I relayed my earlier decision and left it up to Stallman to judge whether the resulting book met this condition or not. At the time of this writing, I have no idea what Stallman's judgment will be.
\fi

\ifdefined\chs
在本书即将完稿之际,确定了O'Reilly \& Associates将成为本书的出版商,斯托曼要求我在书中使用“GNU/Linux”而不是“Linux”,只有这样,他才会答应给在本书出版后,帮助O'Reilly推销本书。当我得知这个消息以后,我坚定了我先前的决定,由斯托曼自己去判断这本书最终是否满足了他的期望。截至现在,我还不知道斯托曼的判断结果会是什么样的。
\fi

\ifdefined\eng
A similar situation surrounds the terms ``free software'' and ``open source.'' Again, I have opted for the more politically laden ``free software'' term when describing software programs that come with freely copyable and freely modifiable source code. Although more popular, I have chosen to use the term ``open source'' only when referring to groups and businesses that have championed its usage. But for a few instances, the terms are completely interchangeable, and in making this decision I have followed the advice of Christine Peterson, the person generally credited with coining the term. ``The `free software' term should still be used in circumstances where it works better,'' Peterson writes. ``[`Open source'] caught on mainly because a new term was greatly needed, not because it's ideal.''
\fi

\ifdefined\chs
在“自由软件”和“开源”这两个词上也有类似的问题。我仍然选用政治意味稍重的“自由软件”一词用于描述那些包含可以随意复制和修改的源代码的软件。虽然”开源”一词更为大众化,但我只在描述那些声援这个词的团体和公司时才使用。在少数的场合下,这两个词也是完全可以互换的,我是依据克里斯汀·彼德森的建议做出的决定,他是如此定义这两个名词的:“应该尽可能使用’自由软件‘一词,”彼德森说,“[‘开源’]这个词只是为了需要而创造出来,它并不是一个理想的描述。”
\fi

%%\theendnotes
\setcounter{endnote}{0}
