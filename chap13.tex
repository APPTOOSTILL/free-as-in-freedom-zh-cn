%% Copyright (c) 2002, 2010 Sam Williams
%% Copyright (c) 2010 Richard M. Stallman
%% Permission is granted to copy, distribute and/or modify this
%% document under the terms of the GNU Free Documentation License,
%% Version 1.3 or any later version published by the Free Software
%% Foundation; with no Invariant Sections, no Front-Cover Texts, and
%% no Back-Cover Texts. A copy of the license is included in the
%% file called ``gfdl.tex''.

%% DONE 

\chapter{\ifdefined\eng
Continuing the Fight
\fi
\ifdefined\chs
斗争仍在继续
\fi}

\ifdefined\eng
For Richard Stallman, time may not heal all wounds, but it does provide a convenient ally.
\fi

\ifdefined\chs
对于理查德·斯托曼来说,时间并不能治愈所有的伤口,但却也算是一个不错的医生。
\fi

\ifdefined\eng
Four years after ``The Cathedral and the Bazaar,'' Stallman still chafes over the Raymond critique. He also grumbles over Linus Torvalds' elevation to the role of world's most famous hacker. He recalls a popular T-shirt that began showing at Linux tradeshows around 1999. Designed to mimic the original promotional poster for Star Wars, the shirt depicted Torvalds brandishing a lightsaber like Luke Skywalker, while Stallman's face rides atop R2D2. The shirt still grates on Stallmans nerves not only because it depicts him as a Torvalds' sidekick, but also because it elevates Torvalds to the leadership role in the free software/open source community, a role even Torvalds himself is loath to accept. ``It's ironic,'' says Stallman mournfully. ``Picking up that sword is exactly what Linus refuses to do. He gets everybody focusing on him as the symbol of the movement, and then he won't fight. What good is it?''
\fi

\ifdefined\chs
``教堂与集市''一书出版后四年,雷蒙德给斯托曼留下伤口虽已收疤,但仍然在隐隐作痛。斯托曼一直在抱怨林纳斯·托瓦兹被人们提升到了世界上最知名黑客的行列的这件事。他回忆起一件1999在Linux商展台前一度很流行的T恤衫,这件T恤衫上的图案模仿了星球大战当年的宣报海报,上面的托瓦兹像卢克·天行者一样挥舞着激光剑,而斯托曼的脸出现在R2D2的上方。这件T恤衫之所以惹怒了斯托曼是因为它把斯托曼描绘为托瓦兹的死党,同时还把托瓦兹提升到了自由软件和开源社区的领袖位置,即便这个位置托瓦兹本人也不愿意担当。``这实在是太讽刺了!''斯托曼悲哀的说,``拿起武器正是林纳斯所不愿意去做的。他希望自己成为一次运动的焦点,所以他并不想发起战争。战争有什么好的呢!''
\fi

\ifdefined\eng
Then again, it is that same unwillingness to ``pick up the sword,'' on Torvalds part, that has left the door open for Stallman to bolster his reputation as the hacker community's ethical arbiter. Despite his grievances, Stallman has to admit that the last few years have been quite good, both to himself and to his organization. Relegated to the periphery by the unforeseen success of GNU/Linux, Stallman has nonetheless successfully recaptured the initiative. His speaking schedule between January 2000 and December 2001 included stops on six continents and visits to countries where the notion of software freedom carries heavy overtones -- China and India, for example.
\fi

\ifdefined\chs
与此同时,对于托瓦兹来说,``拿起武器''也确实不是他想做的,他敞开大门让斯托曼去提升他作为黑客社区的道德仲裁人的影响力。不考虑他所受的委屈的话,斯托曼必须承认,在过去的几年中,无论是对他还是对他的组织,这样的结果都是很好的。虽然GNU/Linux出人意料的成功让自由软件软件运动变得有些边缘化,但是斯托曼还是成功的找到了他的位置。在2000年1月到2001年12月之间,他的演讲覆盖了六大洲并且访问了一些对于软件自由这个话题可能存在更敏感的理解的国家,比如中国和印度。
\fi

\ifdefined\eng
Outside the bully pulpit, Stallman has also learned how to leverage his power as costeward of the GNU General Public License (GPL). During the summer of 2000, while the air was rapidly leaking out of the 1999 Linux IPO bubble, Stallman and the Free Software Foundation scored two major victories. In July, 2000, Troll Tech, a Norwegian software company and developer of Qt, a valuable suite of graphics tools for the GNU/Linux operating system, announced it was licensing its software under the GPL. A few weeks later, Sun Microsystems, a company that, until then, had been warily trying to ride the open source bandwagon without giving up total control of its software properties, finally relented and announced that it, too, was dual licensing its new OpenOffice %%\endnote{Sun was compelled by a trademark complaint to use the clumsy name ``OpenOffice.org.''}
application suite under the Lesser GNU Public License (LGPL) and the Sun Industry Standards Source License (SISSL).
\fi

\ifdefined\chs
除了像一个出色的布道者那样,斯托曼也学会了如何去去利用他作为GNU GPL的海岸警卫队员的权力。在2000年夏天,当1999年Linux IPO的泡沫散去的时候,斯托曼和自由软件基金会获得两个重要的胜利。2000年7月,挪威软件公司Troll Tech宣布对他们所开发的Qt库使用GPL许可证,Qt库是GNU/Linux操作系统中一个重要的图形开发库。几周后,一个对开源潮流跃跃欲试却又不想放弃对他们的私有软件控制权的Sun公司宣布对他们的OpenOffice%%\endnote{Sun公司为了遵守商标相关的规定,使用了一个很拙劣的名字,叫``OpenOffice.org''}
应用套件采用LGPL和Sun Industry Standards Source Licese(SISSL)进行双重许可。
\fi

%% \ifdefined\eng
%% In the case of Trolltech, this victory was the result of a protracted effort by the GNU Project. The nonfreeness of Qt was a serious problem for the free software community because KDE, a free graphical desktop environment that was becoming popular, depended on it. Qt was non-free software but Trolltech had invited free software projects (such as KDE) to use it gratis. Although KDE itself was free software, users that insisted on freedom couldn't run it, since they had to reject Qt. Stallman recognized that many users would want a graphical desktop on GNU/Linux, and most would not value freedom enough to reject the temptation of KDE, with Qt hiding within. The danger was that GNU/Linux would become a motor for the installation of KDE, and therefore also of non-free Qt. This would undermine the freedom which was the purpose of GNU.
%% \fi

%% \ifdefined\chs
%% 对于这些胜利,其实斯托曼并没有做什么有针对性的努力。对于Trolltech的案例来说,斯托曼只是扮演了自由软件教主的角色。1999年,Trolltech想寻求一种与自由软件基金会要求相符的软件许可证,但在仔细审阅许可证后,斯托曼发现了一些条款在法律上存在冲突,使得Qt无法与GPL许可证的自由软件绑定在一起。Trolltech的管理团队疲于与斯托曼在许可证条款上斤斤计较,决定把Qt拆分成两个版本,一个使用GPL协议,另一个使用QPL协议,让开发者可以有办法绕开斯托曼所发现的协定冲突。...............
%% \fi

%% \ifdefined\eng
%% To deal with this danger, Stallman recruited people to launch two parallel counterprojects. One was GNOME, the GNU free graphical desktop environment. The other was Harmony, a compatible free replacement for Qt. If GNOME succeeded, KDE would not be necessary; if Harmony succeeded, KDE would not need Qt. Either way, users would be able to have a graphical desktop on GNU/Linux without nonfree Qt.
%% \fi

%% \ifdefined\chs
%% ..................
%% \fi

%% In 1999, these two efforts were making good progress, and the management of Trolltech were starting to feel the pressure. So Trolltech released Qt under its own free software license, the QPL. The QPL qualified as a free license, but Stallman pointed out the drawback of incompatibility with the GPL: in general, combining GPL-covered code with Qt in one program was impossible without violating one license or the other. Eventually the Trolltech management recognized that the GPL would serve their purposes equally well, and released Qt with dual licensing: the same Qt code, in parallel, was available under the GNU GPL and under the QPL. After three years, this was victory.

%% Once Qt was free, the motive for developing Harmony (which wasn't complete enough for actual use) had disappeared, and the developers abandoned it. GNOME had acquired substantial momentum, so its development continued, and it remains the main GNU graphical desktop.

\ifdefined\eng
Underlining each victory was the fact that Stallman had done little to fight for them. In the case of Troll Tech, Stallman had simply played the role of free software pontiff. In 1999, the company had come up with a license that met the conditions laid out by the Free Software Foundation, but in examining the license further, Stallman detected legal incompatibles that would make it impossible to bundle Qt with GPL-protected software programs. Tired of battling Stallman, Troll Tech management finally decided to split the Qt into two versions, one GPL-protected and one QPL-protected, giving developers a way around the compatibility issues cited by Stallman.
\fi

\ifdefined\chs
对于这些胜利,其实斯托曼并没有做什么特别的努力。对于Troll Tech的案例来说,斯托曼只是扮演了自由软件教主的角色。1999年,Troll Tech想寻求一种与自由软件基金会要求相符的软件许可证,但在仔细审阅许可证后,斯托曼发现了一些条款在法律上存在冲突,使得Qt无法与GPL许可证的自由软件绑定在一起。Troll Tech的管理团队疲于与斯托曼在许可证条款上斤斤计较,决定把Qt拆分成两个版本,一个使用GPL协议,另一个使用QPL协议,让开发者可以有办法绕开斯托曼所发现的协议冲突。
\fi

\ifdefined\eng
In the case of Sun, they desired to play according to the Free Software Foundation's conditions. At the 1999 O'Reilly Open Source Conference, Sun Microsystems cofounder and chief scientist Bill Joy defended his company's ``community source'' license, essentially a watered-down compromise letting users copy and modify Sun-owned software but not charge a fee for said software without negotiating a royalty agreement with Sun. A year after Joy's speech, Sun Microsystems vice president Marco Boerries was appearing on the same stage spelling out the company's new licensing compromise in the case of OpenOffice, an office-application suite designed specifically for the GNU/Linux operating system.
\fi

\ifdefined\chs
在Sun公司的例子里,他们希望能按照自由软件基金会的游戏规则来进行。在1999年的O'Reilly开源大会上,Sun公司的共同创始人和首席科学家比尔·乔伊为该公司的``community source''许可证辩护,这个许可证本质是放松了限制,用户无须与Sun公司签署协议就可以免费复制或修改Sun的软件。乔伊发表演讲一年后,Sun公司的的副总经理马科·博瑞斯再次登上这个演讲台,发布该公司的对于OpenOffice使用的新许可证。OpenOffice是为GNU/Linux系统设计的办公套件。
\fi

%% \ifdefined\eng
%% Sun desired to play according to the Free Software Foundation's conditions. At the 1999 O'Reilly Open Source Conference, Sun Microsystems cofounder and chief scientist Bill Joy defended his company's ``community source'' license, essentially a watered-down compromise letting users copy and modify Sun-owned software but not sell copies of said software without negotiating a royalty agreement with Sun. (With this restriction, the license did not qualify as free, nor for that matter as open source.) A year after Joy's speech, Sun Microsystems vice president Marco Boerries was appearing on the same stage spelling out the company's new licensing compromise in the case of OpenOffice, an office-application suite designed specifically for the GNU/Linux operating system.
%% \fi

%% \ifdefined\chs
%% 在Sun公司的例子里,他们希望能按照自由软件基金会的游戏规则来进行。在1999年的O'Reilly开源大会上,Sun Microsystem的共同创始人和首席科学家比尔·乔伊对该公司的``community source''许可证辩护,这个许可证本质是放松了限制,允许用户不与Sun签署协议免费复制或修改Sun的软件。乔伊发表演讲一年后,Sun公司的的副总经理马科·博瑞斯再次登上这个演讲台,发布该公司的对于OpenOffice使用的新许可证。OpenOffice是为GNU/Linux系统设计的办公套件。
%% \fi

\ifdefined\eng
``I can spell it out in three letters,'' said Boerries. ``GPL.''
\fi

\ifdefined\chs
博瑞斯说:``我可以用三个字每来拼写它,GPL。''
\fi

\ifdefined\eng
At the time, Boerries said his company's decision had little to do with Stallman and more to do with the momentum of GPL-protected programs. ``What basically happened was the recognition that different products attracted different communities, and the license you use depends on what type of community you want to attract,'' said Boerries. ``With [OpenOffice], it was clear we had the highest correlation with the GPL community.''\endnote{Marco Boerries, interview with author (July, 2000).}  %%Alas, this victory was incomplete, since OpenOffice recommends the use of nonfree plug-ins.
\fi

\ifdefined\chs
在那时候,博瑞斯说Sun的决定与斯托曼没有什么直接的关系,而是更多的因为GPL软件的蓬勃发展的势头。``现在的实际情况是,很多人认识到不同的软件产品与不同的社区,选用什么样的许可证就取决于你希望赢得哪个社区的注意力'',博瑞斯说,``对于OpenOffice,我们希望能与GPL的社区密切合作。''\endnote{参考作者对马科·博瑞斯的采访(2000年7月)。}%%然而这个斗争的成果也并不完全,因为OpenOffice仍然会推荐用户去使用一些非自由的插件。
\fi

\ifdefined\eng
Such comments point out the under-recognized strength of the GPL and, indirectly, the political genius of man who played the largest role in creating it. ``There isn't a lawyer on earth who would have drafted the GPL the way it is,'' says Eben Moglen, Columbia University law professor and Free Software Foundation general counsel. ``But it works. And it works because of Richard's philosophy of design.''
\fi

\ifdefined\chs
这样的评论一方面承认了GPL在业界的地位,另一方面也承认了在幕后创建这种许可证人的政治天才。伊本·莫格林说:``他一定不是一个律师,不然怎么可能把GPL写成这个样子。''莫格林是哥伦比亚大学的法学教授,也是自由软件基金会的法律顾问,他接着说:``但是它很有效,它之所以有效是因为理查德·斯托曼所坚持的设计哲学。''
\fi

\ifdefined\eng
A former professional programmer, Moglen traces his pro bono work with Stallman back to 1990 when Stallman requested Moglen's legal assistance on a private affair. Moglen, then working with encryption expert Phillip Zimmerman during Zimmerman's legal battles with the National Security Administration, says he was honored by the request.%%\endnote{For more information on Zimmerman's legal travails, read Steven Levy's \textit{Crypto}, p. 287-288. In the original book version of \textit{Free as in Freedom}, I reported that Moglen helped defend Zimmerman against the National Security Agency. According to Levy's account, Zimmerman was investigated by the U.S. Attorney's office and U.S. Customs, not the NSA.}
``I told him I used Emacs every day of my life, and it would take an awful lot of lawyering on my part to pay off the debt.''
\fi

\ifdefined\chs
作为一个曾经的专业程序员,莫格林追溯他自己1990年时为斯托曼提供的无偿专业服务,那时斯托曼在一件私人事件上请求莫格林的法律援助。莫格林后来与加密学专家菲利普·齐默曼合作与联邦政府进行相关的法律斗争,他说他对于斯托曼这次请求觉得很荣幸:%%\endnote{有关齐默曼的官司,可以参考史蒂芬·李维的《\textit{加密}》一书,第287至288页。在本书印刷版的第一版中,把这次官司写成是莫格林帮助齐默曼与国家安全局(NSA)打官司。跟据李维提供的信息,齐默曼是被美国检察长办公室和美国海关调查,而不是NSA。}
``我告诉他说我每天都在使用Emacs,打这个官司可以赚到不少的律师费来付帐单。''
\fi

\ifdefined\eng
Since then, Moglen, perhaps more than any other individual, has had the best chance to observe the crossover of Stallman's hacker philosophies into the legal realm. Moglen says the difference between Stallman's approach to legal code and software code are largely the same. ``I have to say, as a lawyer, the idea that what you should do with a legal document is to take out all the bugs doesn't make much sense,'' Moglen says. ``There is uncertainty in every legal process, and what most lawyers want to do is to capture the benefits of uncertainty for their client. Richard's goal is the complete opposite. His goal is to remove uncertainty, which is inherently impossible. It is inherently impossible to draft one license to control all circumstances in all legal systems all over the world. But if you were to go at it, you would have to go at it his way. And the resulting elegance, the resulting simplicity in design almost achieves what it has to achieve. And from there a little lawyering will carry you quite far.''
\fi

\ifdefined\chs
从那以后,莫格林也许比任何一个人都能更有机会去近距离研究斯托曼的黑客哲学与现实领域的的交叉。莫格林说斯托曼对待法律文书和软件源代码的方式基本上是一致的。``作为一个律师,我必须得说,对待一份法律文书的正确方式是找出他里面存在的问题。''莫格林说,``在每一个法律的立法过程中都有很多的不确定性,每一个律师都希望能为他的代理人抓往这些能带来好处的不确定性。理查德的目标是从另一个方面达到相同的目的。他的目标是去除那些不确定性,但这自古以来就被证明是不可能的。要起草一个在任何国家任何情况下在任何法律系统中都适用的许可证是不可能的。但如果你朝着这个目标去努力,你就会走到跟他相同的一条路上。最终结果的简洁设计几乎到达了期望的效果。一个真正的律师很难把你引导到这么好的结果上。''
\fi

\ifdefined\eng
As the person charged with pushing the Stallman agenda, Moglen understands the frustration of would-be allies. ``Richard is a man who does not want to compromise over matters that he thinks of as fundamental,'' Moglen says, ``and he does not take easily the twisting of words or even just the seeking of artful ambiguity, which human society often requires from a lot of people.''
\fi

\ifdefined\chs
作为一个负责推动斯托曼的自由软件运动进程的人,莫格林非常理解那种不确定的同盟所带来的挫败感。莫格林说:``理查德是一个从不妥协的人,在他认为是一些核心的根本问题上,他从不尝试通过语言上技巧或换用一种更委婉的说法来取悦别人,但这种技能又是在人类社会中所必不可少的。''
\fi

%% \ifdefined\eng
%% In addition to helping the Free Software Foundation, Moglen has provided legal aid to other copyright defendants, such as Dmitry Sklyarov, and distributors of the DVD decryption program deCSS.
%% \fi

%% \ifdefined\chs
%% ...................
%% \fi

%% \ifdefined\eng
%% Sklyarov had written and released a program to break digital copy-protection on Adobe e-Books, in Russia where there was no law against it, as an employee of a Russian company.  He was then arrested while visiting the US to give a scientific paper about his work. Stallman eagerly participated in protests condemning Adobe for having Sklyarov arrested, and the Free Software Foundation denounced the Digital Millennium Copyright Act as ``censorship of software,'' but it could not intervene in favor of Sklyarov's program because that was nonfree.  Thus, Moglen worked for Sklyarov's defense through the Electronic Frontier Foundation.  The FSF avoided involvement in the distribution of deCSS, since that was illegal, but Stallman condemned the U.S. government for prohibiting deCSS, and Moglen worked as direct counsel for the defendants.
%% \fi

%% \ifdefined\chs
%% .............
%% \fi

%% \ifdefined\eng
%% Following the FSF's decision not to involve itself in those cases, Moglen has learned to appreciate the value of Stallman's stubbornness. ``There have been times over the years where I've gone to Richard and said, `We have to do this. We have to do that. Here's the strategic situation. Here's the next move. Here's what he have to do.' And Richard's response has always been, `We don't have to do anything.' Just wait. What needs doing will get done.''
%% \fi

%% \ifdefined\chs
%% ...................
%% \fi

\ifdefined\eng
Because of the Free Software Foundation's unwillingness to weigh in on issues outside the purview of GNU development and GPL enforcement, Moglen has taken to devoting his excess energies to assisting the Electronic Frontier Foundation, the organization providing legal aid to recent copyright defendants such as Dmitri Skylarov. In 2000, Moglen also served as direct counsel to a collection of hackers that were joined together from circulating the DVD decryption program deCSS. Despite the silence of his main client in both cases, Moglen has learned to appreciate the value of Stallman's stubbornness. ``There have been times over the years where I've gone to Richard and said, `We have to do this. We have to do that. Here's the strategic situation. Here's the next move. Here's what he have to do.' And Richard's response has always been, `We don't have to do anything.' Just wait. What needs doing will get done.''
\fi

\ifdefined\chs
由于自由软件基金会不愿意在一些GNU开发和GPL限制以外的问题上费太多心思,莫格林开始把他过盛的精力放在帮助电子前哨基金会的事务上面,这个组织向近期一些版权相关的被告提供法律援助,比如德米特里·斯柯里亚夫诺。2000年,莫格林同时也是一群黑客的直接律师,帮助他们处理有DVD的解密程序deCSS相关的事务。在这两个案子中,他的主要代理人的都没有什么明确的想法,那时莫格林学会了去赞扬斯托曼的倔强的价值。``这些年中,我不止一次的跟理查德说`我们需要这么做,我们需要那么做。这是一个战略格局。下一步我们这么做。这是我们需要去做的',但是理查德的回答永远都是`我们不需要做任何事'。只需要静静的等着,那些需要完成的事都会慢慢完成的。''

\fi

\ifdefined\eng
``And you know what?'' Moglen adds. ``Generally, he's been right.''
\fi

\ifdefined\chs
``你知道么?''莫格林补充说,``总体上来说,他是对的。''
\fi

\ifdefined\eng
Such comments disavow Stallman's own self-assessment: ``I'm not good at playing games,'' Stallman says, addressing the many unseen critics who see him as a shrewd strategist. ``I'm not good at looking ahead and anticipating what somebody else might do. My approach has always been to focus on the foundation, to say `Let's make the foundation as strong as we can make it.'\hspace{0.01in}''
\fi

\ifdefined\chs
莫格林这样的评价否定了斯托曼对自己的评价:``我不善于玩游戏'',斯托曼针对一些从未谋面的人认为他是一个精明的战略家的评价,曾这样说:``我不善于看到未来的形势并且参与一般人所会做的那些事情。我的方法是永远关注事情的基础,也就是说`让我们一起尽可能的打好基础吧'。''
\fi

\ifdefined\eng
The GPL's expanding popularity and continuing gravitational strength are the best tributes to the foundation laid by Stallman and his GNU colleagues. While no longer capable of billing himself as the ``last true hacker,'' Stallman nevertheless can take sole credit for building the free software movement's ethical framework. Whether or not other modern programmers feel comfortable working inside that framework is immaterial. The fact that they even have a choice at all is Stallman's greatest legacy.
\fi

\ifdefined\chs
GPL变得越来越有名,并且继续保持它对外界的吸引力,这是对斯托曼和他的GNU同事所构建的基础架构的最好礼物。斯托曼已经不能再标榜自己为``最后一个真正的黑客'',但他可以通过建立自由软件运动的道德框架来树立他的新的形象。其它现代的程序员在这个框架中工作是否舒服并不是最重要的事情,最重要的是他们已经选择了使用斯托曼最伟大的遗产这一事实。
\fi

\ifdefined\eng
Discussing Stallman's legacy at this point seems a bit premature. Stallman, 48 at the time of this writing, still has a few years left to add to or subtract from that legacy. Still, the autopilot nature of the free software movement makes it tempting to examine Stallman's life outside the day-to-day battles of the software industry and within a more august, historical setting.
\fi

\ifdefined\chs
在这个时候来讨论斯托曼的遗产似乎有点太早。在写作本书时,斯托曼只有48岁,还有不少的时间可以从这些``遗产''中增加或减少一些东西。并且,自由软件运动自身运行的模式,也让超脱于天天都在进行的软件工业的斗争去分析斯托曼的一生变得更有吸引力,我们应该在一种更为严肃的历史的背景中来分析他。
\fi

\ifdefined\eng
To his credit, Stallman refuses all opportunities to speculate. ``I've never been able to work out detailed plans of what the future was going to be like,'' says Stallman, offering his own premature epitaph. ``I just said `I'm going to fight. Who knows where I'll get?'\hspace{0.01in}''
\fi

\ifdefined\chs
为了维持他的形象,斯托曼拒绝一切投机的机会。``对于那些无法预知未来会怎样的东西,我没有办法做出相应的计划,''斯托曼说,用他的那早来的墓志铭来总结便是:``我说过,`我要去斗争,没有人知道我会到达哪里'。''
\fi

\ifdefined\eng
There's no question that in picking his fights, Stallman has alienated the very people who might otherwise have been his greatest champions. It is also a testament to his forthright, ethical nature that many of Stallman's erstwhile political opponents still manage to put in a few good words for him when pressed. The tension between Stallman the ideologue and Stallman the hacker genius, however, leads a biographer to wonder: how will people view Stallman when Stallman's own personality is no longer there to get in the way?
\fi

\ifdefined\chs
在斗争中,斯托曼协调睦和了那些原本可能会成为他最大竞争对手的人们。这也是他直率、讲究道德性格的一个体现。在必要的时候,他以前的一些政治上的对手仍然愿意帮他说一些好话。斯托曼的理论与斯托曼的黑客天才很容易让一名传记作者感到疑惑:如果有一天斯托曼由于某些原因丢失了他的强烈的个性,人们会如何去看待他?
\fi

\ifdefined\eng
In early drafts of this book, I dubbed this question the ``100 year'' question. Hoping to stimulate an objective view of Stallman and his work, I asked various software-industry luminaries to take themselves out of the current timeframe and put themselves in a position of a historian looking back on the free software movement 100 years in the future. From the current vantage point, it is easy to see similarities between Stallman and past Americans who, while somewhat marginal during their lifetime, have attained heightened historical importance in relation to their age. Easy comparisons include Henry David Thoreau, transcendentalist philosopher and author of On \textit{Civil Disobedience}, and John Muir, founder of the Sierra Club and progenitor of the modern environmental movement. It is also easy to see similarities in men like William Jennings Bryan, a.k.a. ``The Great Commoner,'' leader of the populist movement, enemy of monopolies, and a man who, though powerful, seems to have faded into historical insignificance.
\fi

\ifdefined\chs
在本书的早期草稿中,我把这个问题称为``100年''问题。期望用这个词来刺激斯托曼的目标和他的工作,我找了几位软件工业的杰出人物,让他们把自己置身当前的时间外,把他们自己放到一个历史学家的角度,从100年以后往回看自由软件运动。从当前有利的观点来看,不难发现斯托曼与以前的美国人之间的相似之处,他们在有生之年所做的一小点东西,在历史上就能影响整整的一代人。比如,与亨利·戴维·梭罗相比,他是一个先验论的哲学家,曾经写出《\textit{公民不服从}》一文;还有约翰·缪尔,塞拉俱乐部的创始人,也是现代环保运动的先驱;类似的人还有威廉·詹宁斯·布莱恩,他有着``伟大的平民''的称号,他是平民主义运动的领导人,垄断部门的敌人,他虽然曾经很强势,但是他的光辉已经在历史的长河中渐渐的褪去。
\fi

\ifdefined\eng
Although not the first person to view software as public property, Stallman is guaranteed a footnote in future history books thanks to the GPL. Given that fact, it seems worthwhile to step back and examine Richard Stallman's legacy outside the current time frame. Will the GPL still be something software programmers use in the year 2102, or will it have long since fallen by the wayside? Will the term ``free software'' seem as politically quaint as ``free silver'' does today, or will it seem eerily prescient in light of later political events?
\fi

\ifdefined\chs
斯托曼并不是第一个软件看成是公有财产的人,但是因为他创造了GPL,让他必定会在历史书籍中留下一笔。基于这样事实,看上去很值得脱离现在的时间去审视理查德·斯托曼的遗产。假设到了2102年,软件开发者们还会继续在使用GPL吗?还是它早已经被抛弃到了路边?``自由软件''这个词会像``自由白银''一词在今天一样的古朴还是会体现出对未来政治事件的神奇预测?
\fi

\ifdefined\eng
Predicting the future is risky sport, but most people, when presented with the question, seemed eager to bite. ``One hundred years from now, Richard and a couple of other people are going to deserve more than a footnote,'' says Moglen. ``They're going to be viewed as the main line of the story.''
\fi

\ifdefined\chs
试图预测未来是一件很冒险的事情,但是大部分人在遇到问题时,总是会迫切的想去解决它。莫格林说:``一百年以后,理查德和其他一些人也许不仅仅是能在历史中留下一笔,他们会成为一个时代的主角。''
\fi

\ifdefined\eng
The ``couple other people'' Moglen nominates for future textbook chapters include John Gilmore, Stallman's GPL advisor and future founder of the Electronic Frontier Foundation, and Theodor Holm Nelson, a.k.a. Ted Nelson, author of the 1982 book, \textit{Literary Machines}. Moglen says Stallman, Nelson, and Gilmore each stand out in historically significant, nonoverlapping ways. He credits Nelson, commonly considered to have coined the term ``hypertext,'' for identifying the predicament of information ownership in the digital age. Gilmore and Stallman, meanwhile, earn notable credit for identifying the negative political effects of information control and building organizations-the Electronic Frontier Foundation in the case of Gilmore and the Free Software Foundation in the case of Stallman-to counteract those effects. Of the two, however, Moglen sees Stallman's activities as more personal and less political in nature.

%% The ``couple of other people'' Moglen nominates for future textbook chapters include John Gilmore, who beyond contributing in various ways to free software has founded the Electronic Frontier Foundation, and Theodor Holm Nelson, a.k.a. Ted Nelson, author of the 1982 book, \textit{Literary Machines}. Moglen says Stallman, Nelson, and Gilmore each stand out in historically significant, nonoverlapping ways. He credits Nelson, commonly considered to have coined the term ``hypertext,'' for identifying the predicament of information ownership in the digital age. Gilmore and Stallman, meanwhile, earn notable credit for identifying the negative political effects of information control and building organizations -- the Electronic Frontier Foundation in the case of Gilmore and the Free Software Foundation in the case of Stallman -- to counteract those effects. Of the two, however, Moglen sees Stallman's activities as more personal and less political in nature.
\fi

\ifdefined\chs
莫格林所说的应该出现在未来教科书中的``其他一些人''包括约翰·吉尔摩,他是斯托曼GPL的顾问,也是后来电子前哨基金会的创始人。还有西奥多·霍尔姆·尼尔森,也叫泰德·尼尔森,他是1982年出版的《\textit{文学机器}》一书的作者。莫格林说,斯托曼、尼尔森和吉尔摩都在历史上以他们各自特有方式做出了巨大的贡献。尼尔森创造了``超文本''的概念,解决了在数字时代标定信息所有权的方式。吉尔摩和斯托曼则是指出了信息控制所带来了负面的政治效果,并且他们还创建了一些组织,包括吉尔摩所创建的电子前哨基金会和斯托曼的自由软件基金会,这些组织都致力于解决这些不好的效果。在他们两个之间,莫格林觉得斯托曼的活动更为个人化,从本质上来说没有那么政治化。
\fi

\ifdefined\eng
``Richard was unique in that the ethical implications of unfree software were particularly clear to him at an early moment,'' says Moglen. ``This has a lot to do with Richard's personality, which lots of people will, when writing about him, try to depict as epiphenomenal or even a drawback in Richard Stallman's own life work.''
\fi

\ifdefined\chs
``理查德是一个独一无二的人,在很早的时候,他就认清了非自由软件一些不太道德的地方。''莫格林说,``这与斯托曼本人的个性非常相关,这种个性引起很多人研究的兴趣,尝试把它描写成是一种附带现象或者甚至是理查德·斯托曼个人生活中的一个败笔。''
\fi

\ifdefined\eng
Gilmore, who describes his inclusion between the erratic Nelson and the irascible Stallman as something of a ``mixed honor,'' nevertheless seconds the Moglen argument. Writes Gilmore:
\fi

\ifdefined\chs
吉尔摩对自己能够同时与古怪的尼尔森和暴躁的斯托曼打交道这件事认为是一种``混杂的荣幸'',他很赞成莫格林的观点,写道:
\fi

\begin{quote}
\ifdefined\eng
My guess is that Stallman's writings will stand up as well as Thomas Jefferson's have; he's a pretty clear writer and also clear on his principles\ldots Whether Richard will be as influential as Jefferson will depend on whether the abstractions we call ``civil rights'' end up more important a hundred years from now than the abstractions that we call ``software'' or ``technically imposed restrictions.''
\fi

\ifdefined\chs
我的猜测是,斯托曼的作品会像托马斯·杰斐逊的一样伟大:他是一个思路清淅的作家,并且对他的原则非常清楚……至于理查德能否像杰斐逊一样具有影响力,主要取决于我们今天所说的``公民权利''这个抽象的概念在一百年以后是不是比今天的``软件''或``技术限制''更为重要。
\fi
\end{quote}

\ifdefined\eng
Another element of the Stallman legacy not to be overlooked, Gilmore writes, is the collaborative software-development model pioneered by the GNU Project. Although flawed at times, the model has nevertheless evolved into a standard within the software-development industry. All told, Gilmore says, this collaborative software-development model may end up being even more influential than the GNU Project, the GPL License, or any particular software program developed by Stallman:
\fi

\ifdefined\chs
另一项被人们所忽视的斯托曼的遗产,吉尔摩写道,是由GNU工程所领导的合作式的软件开发模式。这种模式虽然有时也存在一些缺陷,但还是成为了软件开发工业中的一项标准。吉尔摩说,这种合作式的软件开发模式事实上比GNU工程、GPL许可证或任何其它一个由斯托曼开发的软件本身更具影响力:
\fi

\begin{quote}
\ifdefined\eng
Before the Internet, it was quite hard to collaborate over distance on software, even among teams that know and trust each other. Richard pioneered collaborative development of software, particularly by disorganized volunteers who seldom meet each other. Richard didn't build any of the basic tools for doing this (the TCP protocol, email lists, diff and patch, tar files, RCS or CVS or remote-CVS), but he used the ones that were available to form social groups of programmers who could effectively collaborate.
\fi
\ifdefined\chs
在Internet出现以前,很难实现异地的软件合作开发,即使是一个互相了解和信任的团队也是如此。理查德引领了软件合作开发的潮流,而且还是未曾谋面的无组织的开发志愿者们。为了实现这个目标,理查德没有创建什么基础工具(比如TCP协议、电子邮件列表、diff和patch工具,tar文件格式,RCS或CVS或remote-CVS),但他选择了那些现有的可以帮助程序员群体开展有效合作的工具。
\fi
\end{quote}

%% \ifdefined\eng
%% Stallman thinks that evaluation, though positive, misses the point. ``It emphasizes development methods over freedom, which reflects the values of open source rather than free software.  If that is how future users look back on the GNU Project, I fear it will lead to a world in which developers maintain users in bondage, and let them aid development occasionally as a reward, but never take the chains off them.''
%% \fi

%% \ifdefined\chs
%% ........................
%% \fi

\ifdefined\eng
Lawrence Lessig, Stanford law professor and author of the 2001 book, \textit{The Future of Ideas}, is similarly bullish. Like many legal scholars, Lessig sees the GPL as a major bulwark of the current so-called ``digital commons,'' the vast agglomeration of community-owned software programs, network and telecommunication standards that have triggered the Internet's exponential growth over the last three decades. Rather than connect Stallman with other Internet pioneers, men such as Vannevar Bush, Vinton Cerf, and J. C. R. Licklider who convinced others to see computer technology on a wider scale, Lessig sees Stallman's impact as more personal, introspective, and, ultimately, unique:
\fi

\ifdefined\chs
2001出版的《\textit{思想的未来}》一书的作者,斯坦福大学法学教授劳伦斯·莱斯格表达了类似的乐观观点。与其它法律界的学者一样,莱斯格把GPL看成是现在所谓``数字大众''的主要保障。通过把很多社区开发的软件、网络和电信标准聚集起来,在过去的30年中,带来了Internet的指数级的增长势头。与其它的Internet先驱相比,比如万尼瓦尔·布什、文顿·瑟夫和约瑟夫·利克莱德这些让大众看到计算机技术的发展人,莱斯格认为斯托曼的影响力更为个人化和内省式的,并且是独一无二的:
\fi

\begin{quote}
\ifdefined\eng
[Stallman] changed the debate from is to ought. He made people see how much was at stake, and he built a device to carry these ideals forward\ldots That said, I don't quite know how to place him in the context of Cerf or Licklider. The innovation is different. It is not just about a certain kind of code, or enabling the Internet. [It's] much more about getting people to see the value in a certain kind of Internet. I don't think there is anyone else in that class, before or after.
\fi

\ifdefined\chs
[斯托曼]把这些争论从``是不是''变为了``应不应该''。他让人们看到危机,并且他创造了一种方式让这些理想得以延续……也就是说,我并不确定应该怎么把他与瑟夫或利克莱德去比较。他们的创新之处截然不同。这并不是某一段代码或者是让Internet活跃起来,这更多的是让人们去认识Internet的某种特殊的价值。我觉得这样的人前无古人,后无来者。
\fi
\end{quote}

\ifdefined\eng
Not everybody sees the Stallman legacy as set in stone, of course. Eric Raymond, the open source proponent who feels that Stallman's leadership role has diminished significantly since 1996, sees mixed signals when looking into the 2102 crystal ball:
\fi

\ifdefined\chs
当然,并不是所有人对斯托曼的看法都是一成不变的。埃里克·雷蒙德作为开源的拥护者,认为斯托曼的领导角色从1996年开始就弱化下去了,如果透过水晶球去看2102年的情景,也许会看到更为多样的信息:
\fi

\begin{quote}
\ifdefined\eng
I think Stallman's artifacts (GPL, Emacs, GCC) will be seen as revolutionary works, as foundation-stones of the information world. I think history will be less kind to some of the theories from which RMS operated, and not kind at all to his personal tendency towards territorial, cult-leader behavior.
\fi

\ifdefined\chs
我认为斯托曼的作品(GPL、Emacs和GCC)会被认为是革命性的作品,是信息世界的奠基石。我觉得历史不会像斯托曼理论所设想的那样,也不会像他个人所倡导的领地化和邪教领袖行为那样的。
\fi
\end{quote}

\ifdefined\eng
As for Stallman himself, he, too, sees mixed signals:
\fi

\ifdefined\chs
作为斯托曼本人,他也看到很多的可能性:
\fi

\begin{quote}
\ifdefined\eng
What history says about the GNU Project, twenty years from now, will depend on who wins the battle of freedom to use public knowledge. If we lose, we will be just a footnote. If we win, it is uncertain whether people will know the role of the GNU operating system -- if they think the system is ``Linux,'' they will build a false picture of what happened and why.
\fi

\ifdefined\chs
20年后,历史对GNU工程的评价,会取决于谁赢得了自由的斗争,正确的去使用公共的知识。如果我们失败了,我们就会成为沧海一粟。如果我们成功了,人们也许也不能正确的认识到GNU操作系统的角色,如果大家误以为这个系统叫``Linux''的话,那他们其实是搞错了情况,并且没有理解形成这一切的原因。
\fi

\ifdefined\eng
But even if we win, what history people learn a hundred years from now is likely to depend on who dominates politically.
\fi

\ifdefined\chs
但是,即使我们赢了,一百年以后人们所能了解到的历史,也更多的依据于政治上占有主导地位的一方的观点。
\fi
\end{quote}

\ifdefined\eng
Searching for his own 19th-century historical analogy, Stallman summons the figure of John Brown, the militant abolitionist regarded as a hero on one side of the Mason Dixon line and a madman on the other.
\fi

\ifdefined\chs
斯托曼找到了与他相似的历史人物:19世纪的约翰·布朗。他是一名激进的废奴主义者,他在梅森-迪克森线的一边被看成是一个英雄;而在另一边,则被看成是一个疯子。
\fi

\ifdefined\eng
John Brown's slave revolt never got going, but during his subsequent trial he effectively roused national demand for abolition. During the Civil War, John Brown was a hero; 100 years after, and for much of the 1900s, history textbooks taught that he was crazy. During the era of legal segregation, while bigotry was shameless, the US partly accepted the story that the South wanted to tell about itself, and history textbooks said many untrue things about the Civil War and related events.
\fi

\ifdefined\chs
约翰·布朗的奴隶起义最终没有真正开始,但是,他的行为唤醒了人们对于废奴的要求。在南北战争中,约翰·布朗是一个英雄:100年后,尤其在20世纪,历史教科书都把他描绘成一个疯子。在法律隔离的时期,当顽固成为一种无耻的行为,美国开始部分接受南部希望摆脱奴役的现实,历史书上对于南北战争和相关的事件的绘述都显得颇为偏袒一方。
\fi

\ifdefined\eng
Such comparisons document both the self-perceived peripheral nature of Stallman's current work and the binary nature of his current reputation. Although it's hard to see Stallman's reputation falling to the level of infamy as Brown's did during the post-Reconstruction period-Stallman, despite his occasional war-like analogies, has done little to inspire violence-it's easy to envision a future in which Stallman's ideas wind up on the ash-heap. In fashioning the free software cause not as a mass movement but as a collection of private battles against the forces of proprietary temptation, Stallman seems to have created a unwinnable situation, especially for the many acolytes with the same stubborn will.
%%Such comparisons document both the self-perceived peripheral nature of Stallman's current work and the binary nature of his current reputation. It's hard to see Stallman's reputation falling to the same level of infamy as Brown's did during the post-Reconstruction period. Stallman, despite his occasional war-like analogies, has done little to inspire violence. Still, it is easy to envision a future in which Stallman's ideas wind up on the ash-heap.%%\endnote{RMS: Sam Williams' further words here, ``In fashioning the free software cause not  not as a mass movement but as a collection of private battles against the forces of proprietary temptation,'' do not fit the facts. Ever since the first announcement of the GNU Project, I have asked the public to support the cause. The free software movement aims to be a mass movement, and the only question is whether it has enough supporters to qualify as ``mass.''  As of 2009, the Free Software Foundation has some 3000 members that pay the hefty dues, and over 20,000 subscribers to its monthly e-mail newsletter.}
\fi

\ifdefined\chs
这样的比较,记录了斯托曼工作的自省的理性性格特点,也记录了他当前声誉的两面性。仅管我们没有看到斯托曼的声誉降到美国内战后重建时声名狼藉的布朗那么低,但是除了他难得使用一些言辞激烈的类比去攻击别人,他所做的一切很难激发暴力,不难想象,即使在未来斯托曼的观点变得烟消云散时也会是一样。让自由软件这个词汇变得流行,不像是一个群体运动,倒像是一系列与私有化诱惑进行的个人战斗。斯托曼看上去处在一个不利的地位,尤其是在面对那些顽固的对手们时。%%\endnote{...............................}
\fi

\ifdefined\eng
Then again, it is that very will that may someday prove to be Stallman's greatest lasting legacy. Moglen, a close observer over the last decade, warns those who mistake the Stallman personality as counter-productive or epiphenomenal to the ``artifacts'' of Stalllman's life. Without that personality, Moglen says, there would be precious few artifiacts to discuss. Says Moglen, a former Supreme Court clerk:
\fi

\ifdefined\chs
再一次的,总有一种美好的愿望,那就是有一天历史会证明斯托曼留下的是一笔伟大的遗产。莫格林在过去的十年中密切关注着事态的变化,他提醒那些误解了斯托曼个性的人,以为他是生活的假象的副现象的人。莫格林说,如果没有这样的个性,就没有这么多可以讨论的作品。作为一名曾经的最高法院职员,莫格林说:
\fi

\begin{quote}
\ifdefined\eng
Look, the greatest man I ever worked for was Thurgood Marshall. I knew what made him a great man. I knew why he had been able to change the world in his possible way. I would be going out on a limb a little bit if I were to make a comparison, because they could not be more different. Thurgood Marshall was a man in society, representing an outcast society to the society that enclosed it, but still a man in society. His skill was social skills. But he was all of a piece, too. Different as they were in every other respect, that the person I most now compare him to in that sense, all of a piece, compact, made of the substance that makes stars, all the way through, is Stallman.
\fi

\ifdefined\chs
我曾经服务过的最伟大的人是瑟古德·马歇尔。我知道是什么让他成为了一个伟大的人。我明白为什么他可以用他的方式去改变世界。我不是在把他们两个作比较,因为他们两个并没有什么共同点。瑟古德·马歇尔是一个社会中的人,他代表着一个被驱逐的社会和那个包容他的社会,他存在于这个社会中。他的强项是社交能力。但他也是只是一个人。................................Different as they were in every other respect, the person I most compare him to in that sense, all of a piece, compact, made of the substance that makes stars, all the way through, is 斯托曼.
\fi
\end{quote}

\ifdefined\eng
In an effort to drive that image home, Moglen reflects on a shared moment in the spring of 2000. The success of the VA Linux IPO was still resonating in the business media, and a half dozen free software-related issues were swimming through the news. Surrounded by a swirling hurricane of issues and stories each begging for comment, Moglen recalls sitting down for lunch with Stallman and feeling like a castaway dropped into the eye of the storm. For the next hour, he says, the conversation calmly revolved around a single topic: strengthening the GPL.
\fi

\ifdefined\chs
为了把斯托曼的这种形象描述清楚,莫格林回忆起2000年春天的一幕。VA Linux IPO成功的故事仍在商业媒体中产生共呜,很多的自由软件相关的内容在这些新闻中被提及。莫格林回忆起那时他与斯托曼在一起共进午餐时的情景,他们正被暴风雨般的文章和评论所包围,午餐时间就像是坐在台风眼中的遇难者。在接下来的几个小时中,他说,他们的对话渐渐的转向一个主题:加强GPL的位置。
\fi

\ifdefined\eng
``We were sitting there talking about what we were going to do about some problems in Eastern Europe and what we were going to do when the problem of the ownership of content began to threaten free software,'' Moglen recalls. ``As we were talking, I briefly thought about how we must have looked to people passing by. Here we are, these two little bearded anarchists, plotting and planning the next steps. And, of course, Richard is plucking the knots from his hair and dropping them in the soup and behaving in his usual way. Anybody listening in on our conversation would have thought we were crazy, but I knew: I knew the revolution's right here at this table. This is what's making it happen. And this man is the person making it happen.''
\fi

\ifdefined\chs
``我们坐在哪里讨论对于东欧出现的一些问题我们应如何去解决,以及当内容的所有权问题成为威胁自由软件的一个问题时我们应怎么做。''莫格林回忆说,``在我们讨论的时候,我简单的想了想对于旁边路过行人来说,我们是什么样的一个形象。这就是我们,两个长着小胡子的无政府主义者,设计着下一步的行动。并且,理查德正在清理他头发里的小碎屑,并把它们若无其事的掉进了汤里。任何听到我们谈话内容的人都会觉得我们是疯子,但是我知道:我知道革命就将从我们这张桌子开始。这就是让革命开始的地方。这个人就是发起这场革命的领导者。''
\fi

\ifdefined\eng
Moglen says that moment, more than any other, drove home the elemental simplicity of the Stallman style.
\fi

\ifdefined\chs
莫格林说在那一刻,比其它任何时候,都准确的描述了斯托曼风格的简洁性。
\fi

\ifdefined\eng
``It was funny,'' recalls Moglen. ``I said to him, `Richard, you know, you and I are the two guys who didn't make any money out of this revolution.' And then I paid for the lunch, because I knew he didn't have the money to pay for it.'\hspace{0.01in}''%%\endnote{RMS: I never refuse to let people treat me to a meal, since my pride is not based on picking up the check.  But I surely did have the money to pay for lunch.  My income, which comes from around half the speeches I give, is much less than a law professor's salary, but I'm not poor.}
\fi

\ifdefined\chs
``这是件很有趣的事情,''莫格林回忆道,``我对他说,`理查德,你得明白,你和我两个人在这次革命中一分钱也没赚到。'然后我还得给他付午餐钱,因为他没有钱来付。'' %%\endnote{RMS:........................}
\fi

\theendnotes
\setcounter{endnote}{0}
