% Copyright (c) 2010 Free Software Foundation
%% Permission is granted to copy, distribute and/or modify this
%% document under the terms of the GNU Free Documentation License,
%% Version 1.3 or any later version published by the Free Software
%% Foundation; with no Invariant Sections, no Front-Cover Texts, and
%% no Back-Cover Texts. A copy of the license is included in the
%% file called ``gfdl.tex''.


\chapter{\ifdefined\eng
Epilogue from Sam Williams: Crushing Loneliness
\fi
\ifdefined\chs
萨姆·威廉姆斯跋语:粉碎孤独
\fi}
\chaptermark{Epilogue: Crushing Loneliness}

\ifdefined\eng
[RMS: Because this chapter is so personally from Sam Williams, I have indicated all changes to the text with square brackets or ellipses, and I have made such changes only to clear up technical or legal points, and to remove passages that I found to be hostile and devoid of information.  I have also added notes labeled `RMS:' to respond to certain points.  Williams has also changed the text of this chapter; changes made by Williams are not explicitly indicated.]
\fi

\ifdefined\chs
[RMS:由于这一章节的内容主要是萨姆·威廉姆斯的一些个人观点,我对正文的修改都将以方括号或省略号进行标记。我所做的改动主要是为了澄清一些技术或法律上的问题,并且我删除了一部分具有攻击性或是缺少信息量的短落。我也加了一些以“RMS:”打头标记的注释,用于回应文中的一些观点。在这一版中,威廉斯姆本人也进行了一些修改,他的改动没有明显的标识出来。]
\fi

\ifdefined\eng
Writing the biography of a living person is a bit like producing a play. The drama in front of the curtain often pales in comparison to the drama backstage.
\fi

\ifdefined\chs
为一个还没有去世的人撰写传记有点像是在写一个话剧。台前的话剧与幕后的故事相比,总是显得苍白无力。
\fi

\ifdefined\eng
In \textit{The Autobiography of Malcolm X}, Alex Haley gives readers a rare glimpse of that backstage drama. Stepping out of the ghostwriter role, Haley delivers the book's epilogue in his own voice. The epilogue explains how a freelance reporter originally dismissed as a ``tool'' and ``spy'' by the Nation of Islam spokesperson managed to work through personal and political barriers to get Malcolm X's life story on paper.
\fi

\ifdefined\chs
在《\textit{马尔科姆·艾克斯传记}》一书中,亚历克斯·哈里为广大读者展示了一部幕后的话剧。在这本书的后记中,哈里不再以代笔人的身份出现,而是表达了他自己的心声。在后记中,哈里讲述了一名被伊斯兰民族组织发言人当作“工具”和“间谍”的自由记者是如何突破个人和政治壁垒最终出版马尔科姆·艾克斯传记的故事。
\fi

\ifdefined\eng
While I hesitate to compare this book with \textit{The Autobiography of Malcolm X}, I do owe a debt of gratitude to Haley for his candid epilogue. Over the last 12 months, it has served as a sort of instruction manual on how to deal with a biographical subject who has built an entire career on being disagreeable.  [RMS: I have built my career on saying no to things others accept without much question, but if I sometimes seem or am disagreeable, it is not through specific intention.]  From the outset, I envisioned closing this biography with a similar epilogue, both as an homage to Haley and as a way to let readers know how this book came to be.
\fi

\ifdefined\chs
我并不想把这本书与《\textit{马尔科姆·艾克斯自传}》做什么比较,不过我还欠哈里一个人情债,没有他的后记作榜样,我不会写出这么一部坦诚的作品。在过去的一年中,这个后记被我当成解决一个从来不愿意与他人合作的传记人物的问题指导手册。[RMS:我在自己的职业生涯中确实习惯于对一些别人不假思索就接受的事物说“不”,所以我有时看上去也确实是一个不太愿意与他人合作,但这并不是我做事的动机。]我从一开始就想为这本传记写一篇类似的后记,一方面是向哈里致敬,另一方面也是想让读者了解这本书是如何写成的。
\fi

\ifdefined\eng
The story behind this story starts in an Oakland apartment, winding its way through the various locales mentioned in the book -- Silicon Valley, Maui, Boston, and Cambridge. Ultimately, however, it is a tale of two cities: New York, New York, the book-publishing capital of the world, and Sebastopol, California, the book-publishing capital of Sonoma County.
\fi

\ifdefined\chs
这个故事背后的故事开始于奥克兰的一个公寓,就像书中所记载的一样,故事在硅谷、茂宜岛、波士顿和剑桥展开。最后,成了一部现实版的双城记:纽约————全球出版界的中心和加州的塞瓦斯托波尔————索诺马县的出版中心。
\fi

\ifdefined\eng
The story starts in April, 2000. At the time, I was writing stories for the ill-fated web site BeOpen.com. One of my first assignments was a phone interview with Richard M. Stallman. The interview went well, so well that Slashdot (\url{http://www.slashdot.org}), the popular ``news for nerds'' site owned by VA Software, Inc. (formerly VA Linux Systems and before that, VA Research), gave it a link in its daily list of feature stories. Within hours, the web servers at BeOpen were heating up as readers clicked over to the site.
\fi

\ifdefined\chs
故事开始于2000年4月,那时候我正在为命运多舛的BoOpen网站(\url{http://www.beopen.com})写一些文章。其中的第一项任务就是与理查德·马修·斯托曼进行一次电话采访。这次采访进行得很顺利,Slashdot(\url{http://www.splashdot.org})网站在他的每日精彩文章中给了这篇文章一个链接。Slashdot是VA软件有限公司(在此之前名为VA Linux系统公司,更早的时候叫VA研究所)的一个热门的Herd新闻聚合网站。短短几小时内,BeOpen网站的访问量激增,读者们纷拥而来。
\fi

\ifdefined\eng
For all intents and purposes, the story should have ended there. Three months after the interview, while attending the O'Reilly Open Source Conference in Monterey, California, I received the following email message from Tracy Pattison, foreign-rights manager at a large New York publishing house:
\fi

\ifdefined\chs
如果只是按预期的设想,故事到这里就该结束了。采访结束后3个月,当我在加州的蒙特雷参加O'Reilly的开源软件大会时,我收到一封来自特蕾西·帕蒂森的电子邮件,她是纽约一家大出版社的涉外版权经理。
\fi

\begin{quote}
\ifdefined\eng
To: \url{sam@BeOpen.com}\\Subject: RMS Interview\\Date: Mon, 10 Jul 2000 15:56:37 -0400
\fi
\ifdefined\chs
发送至:sam@BeOpen.com\\主题:RMS采访\\日期:2000年7月10日,星期一,15:56:37 -0400
\fi

\ifdefined\eng
Dear Mr. Williams,
\fi

\ifdefined\chs
亲爱的威廉姆斯先生,
\fi

\ifdefined\eng
I read your interview with Richard Stallman on BeOpen with great interest. I've been intrigued by RMS and his work for some time now and was delighted to find your piece which I really think you did a great job of capturing some of the spirit of what Stallman is trying to do with GNU-Linux and the Free Software Foundation.
\fi

\ifdefined\chs
我在BeOpen上读到了您对理查德·斯托曼先生所进行的访谈内容,它非常有趣。很久以来,我对RMS和他的作品都很有兴趣,看到您的作品感觉到很高兴,因为我觉得你的作品非常出色,它准确的表述了斯托曼从事GNU-Linux和自由软件基金会相关工作时所表达出来的一些精神。
\fi

\ifdefined\eng
What I'd love to do, however, is read more - and I don't think I'm alone. Do you think there is more information and/or sources out there to expand and update your interview and adapt it into more of a profile of Stallman? Perhaps including some more anecdotal information about his personality and background that might really interest and enlighten readers outside the more hardcore programming scene?
\fi

\ifdefined\chs
我来信的目的希望能读到更多这方面的文章,我相信我不是唯一一个提出这样要求的人。你那里还有更多的信息或消息源可以把你的访谈作进一步的展开,并把它写成是斯托曼的一个传记吗?也许可以在里面添加更多有关他个性的花絮和背景资料,这样的作品也许能让非程序员群体产生更大的兴趣。
\fi
\end{quote}

\ifdefined\eng
Tracy ended the email with a request that I give her a call to discuss the idea further. I did just that. Tracy told me her company was launching a new electronic book line, and it wanted stories that appealed to an early-adopter audience. The e-book format was 30,000 words, about 100 pages, and she had pitched her bosses on the idea of profiling a major figure in the hacker community. Her bosses liked the idea, and in the process of searching for interesting people to profile, she had come across my BeOpen interview with Stallman. Hence her email to me.
\fi

\ifdefined\chs
特蕾西让我打电话给她进一步讨论这个主意,我照做了。特蕾西告诉我,她的公司在准备出版一系列在线的电子书,这个系列中希望能有一些有趣的内容,以便吸引第一批读者。这个系列的电子书需要30000词左右的内容,大约100页,并且她已经说服她的老板出一本黑客群体中的主要人物的传记。她的老板很赞同这个建议并且已经开始着手物色有故事的人物做为选题,她正好看到了我在BeOpen上发表的斯托曼访谈,所以就发邮件给我了。
\fi

\ifdefined\eng
That's when Tracy asked me: would I be willing to expand the interview into a full-length feature profile?
\fi

\ifdefined\chs
特蕾西是这样问我的:你愿意把这篇访谈扩展成为一本完整的人物传记吗?
\fi

\ifdefined\eng
My answer was instant: yes. Before accepting it, Tracy suggested I put together a story proposal she could show her superiors. Two days later, I sent her a polished proposal. A week later, Tracy sent me a follow up email. Her bosses had given it the green light.
\fi

\ifdefined\chs
我立即就做出了答复:没问题。在签署合同之前,特蕾西建议我先准备一个故事的大纲,让她先给她的上级看一下。两天后,我把润色后的大纲发给了她。一周后,特蕾西回邮件通知我,她的老板开了绿灯。
\fi

\ifdefined\eng
I have to admit, getting Stallman to participate in an e-book project was an afterthought on my part. As a reporter who covered the open source beat, I knew Stallman was a stickler. I'd already received a half dozen emails at that point upbraiding me for the use of ``Linux'' instead of ``GNU/Linux.''
\fi

\ifdefined\chs
我必须得承认,刚开始的时候我并没想到要让斯托曼直接参与到这样一个电子书的项目中来。作为一个记录开源斗争的记者,我知道斯托曼是一个固执的人。我至少已经收到了半打他给我的邮件,指责我不应该使用“Linux”一词,而应该用“GNU/Linux”。
\fi

\ifdefined\eng
Then again, I also knew Stallman was looking for ways to get his message out to the general public. Perhaps if I presented the project to him that way, he would be more receptive. If not, I could always rely upon the copious amounts of documents, interviews, and recorded online conversations Stallman had left lying around the Internet and do an unauthorized biography.
\fi

\ifdefined\chs
除此之外,我也很清楚斯托曼一直是期望能通过一切的渠道向公众表达他的观点。如果我从这个角度来向他介绍我们的出版计划,也许他会更容易接受一些。如果不这样做的话,我就只能依赖网上的各种文档、访谈和对话录音,最终写出一本并不权威的传记。
\fi

\ifdefined\eng
During my research, I came across an essay titled ``Freedom -- Or Copyright?'' Written by Stallman and published in the June, 2000, edition of the MIT \textit{Technology Review}, the essay blasted e-books for an assortment of software sins. Not only did readers have to use proprietary software programs to read them, Stallman lamented, but the methods used to prevent unauthorized copying were overly harsh. Instead of downloading a transferable HTML or PDF file, readers downloaded an encrypted file. In essence, purchasing an e-book meant purchasing a nontransferable key to unscramble the encrypted content. Any attempt to open a book's content without an authorized key constituted a criminal violation of the Digital Millennium Copyright Act, the 1998 law designed to bolster copyright enforcement on the Internet. Similar penalties held for readers who converted a book's content into an open file format, even if their only intention was to read the book on a different computer in their home. Unlike a normal book, the reader no longer held the right to lend, copy, or resell an e-book. They only had the right to read it on an authorized machine, warned Stallman:
\fi

\ifdefined\chs
在我的研究过程中,我曾经看到一篇斯托曼在2000年6月时所写的名为“自由还是版权?“的短文,发表在麻省理工学院的《\textit{技术观察}》杂志上,这篇短文把电子书描述成是各种软件罪恶的混合物。斯托曼认为,电子书的罪孽不但在于读者们需要使用一些私有的软件才能去阅诚,而且电子书中用来保护作品内容不被非法复制的技术也非常的粗暴。读者不能下载一个可以自由传播的HTML或PDF文件,而是得下载一个被加密过的文件。本质上来说,购买电子书意味着购买了一把无法转让的钥匙,这把钥匙用于开启加密内容。任何试图不用进过授权的密钥解密电子书内容的行为都构成违反《数字千年版权法》的行为,这部1998年制定的法律用于加强Internet上的版权控制。读者把加密的图书转换成开放的文件格式也构成一样犯罪,即使他们这么做只是为了在自己家里的另一台电脑上阅读这本电子书。与拥有一本传统的纸质书不同,读者不可以出借、复制或转卖电子书,读者所拥有的所有权利,就只是在一台授权过的设备上阅读这本电子书,斯托曼警告说:
\fi

\begin{quote}
\ifdefined\eng
We still have the same old freedoms in using paper books. But if e-books replace printed books, that exception will do little good. With ``electronic ink,'' which makes it possible to download new text onto an apparently printed piece of paper, even newspapers could become ephemeral. Imagine: no more used book stores; no more lending a book to your friend; no more borrowing one from the public library -- no more ``leaks'' that might give someone a chance to read without paying. (And judging from the ads for Microsoft Reader, no more anonymous purchasing of books either.) This is the world publishers have in mind for us.\endnote{See ``Freedom -- Or Copyright?'' (May, 2000), \url{http://www.technologyreview.com/articles/stallman0500.asp}.}
\fi

\ifdefined\chs
如果我们阅读纸质书籍,则我们还能享有一些最基本的自由。但是,如果电子书取代了印刷的纸书,则没有什么好处。通过使用”电子墨水“,人们可以在已经印有内容的“纸”上重新“印”上其它内容,即使是报纸,也会变得没有实体,转瞬即逝。想象一下:不会再有旧书店;不能把书借给你的朋友;不能从公共图书馆借阅图书。不再让人有机会不付钱就阅读的”漏洞“。(如果从微软公司阅读器的广告的内容来看,还不再允许匿名购买书籍。)这就是那些出版社想让我们面临的问题。\endnote{参考《自由还是版权?》(2000年5月),\url{http://www.technologyreview.com/articles/stallman0500.asp}。}
\fi
\end{quote}

\ifdefined\eng
Needless to say, the essay caused some concern. Neither Tracy nor I had discussed the software her company would use nor had we discussed the type of copyright [license] that would govern the e-book's usage. I mentioned the \textit{Technology Review} article and asked if she could give me information on her company's e-book policies. Tracy promised to get back to me.
\fi

\ifdefined\chs
无容质疑,这篇文章引起了一些担忧。特蕾西和我都没有谈论过有关她的公司所使用的软件或是电子书的版权[许可证]协议会是什么样。我向特蕾西谈起过了这篇《\textit{技术观察}》上的文章并请她提供一些有关她所在公司的电子书政策的信息给我。特蕾西答应一有消息就通知我。
\fi

\ifdefined\eng
Eager to get started, I decided to call Stallman anyway and mention the book idea to him. When I did, he expressed immediate interest and immediate concern. ``Did you read my essay on e-books?'' he asked.
\fi

\ifdefined\chs
我很希望工作可以得到尽快的开展,所以我决定先打电话给斯托曼并跟他先大致聊一下有关这本书的计划。在电话中,斯托曼很明确的表达了他对这本书的兴趣和担心。”你读过我那篇有关电子书的文章吗?“他问。
\fi

\ifdefined\eng
When I told him, yes, I had read the essay and was waiting to hear back from the publisher, Stallman laid out two conditions: he didn't want to lend support to an e-book licensing mechanism he fundamentally opposed, and he didn't want to come off as lending support. ``I don't want to participate in anything that makes me look like a hypocrite,'' he said.
\fi

\ifdefined\chs
我告诉他我的确读过那篇文章,并且我正在等候出版社对于此事的回应。斯托曼给出两个条件:他不愿意对他反对的电子书许可证条款做出任何妥协,也不愿意被别人看成是他做出了妥协。”我不想参与任何让我看起来像个伪君子的事情。“他说。
\fi

\ifdefined\eng
For Stallman, the software issue was secondary to the copyright issue. He said he was willing to ignore whatever software the publisher or its third-party vendors employed just so long as the company specified within the copyright that readers were free to make and distribute verbatim copies of the e-book's content. Stallman pointed to Stephen King's \textit{The Plant} as a possible model. In June, 2000, King announced on his official web site that he was self-publishing \textit{The Plant} in serial form. According to the announcement, the book's total cost would be \$13, spread out over a series of \$1 installments. As long as at least 75\% of the readers paid for each chapter, King promised to continue releasing new installments. By August, the plan seemed to be working, as King had published the first two chapters with a third on the way.
\fi

\ifdefined\chs
对于斯托曼来说,想比软件的问题来说,版权是个更大的问题。他说他可以不考虑出版社或相关的第三方公司使用什么软件来发布电子书,只要公司可以在版权声明中说明,允许读者自由的发布电子书的内容。斯托曼说,可以参考斯蒂芬·金的《\textit{植物}》所使用的出版模式。2000年6月,金在他自己的网站上发布声明说他自己以连载的形式出版《\textit{植物}》一书。根据声明所描述的,这个本书定价为13美元,每一个章节是1美元。如果有75\%的读者为前一章付费,金就会发布一个新的章节。截止8月份,这个计划看上去实施得很不错,金已经发布了前两个章节,第三章也即将发布。
\fi

\ifdefined\eng
``I'd be willing to accept something like that,'' Stallman said. ``As long as it also permitted verbatim copying.'' [RMS: As I recall, I also raised the issue of encryption; the text two paragraphs further down confirms this.  I would not have agreed to publish the book in a way that \textit{required} a nonfree program to read it.]
\fi

\ifdefined\chs
“如果这本书允许被自由的原样复制的话,我就可以接受。” 斯托曼说。[RMS:我记得我当时还提出了有关加密的问题,后面的段落中会谈到这个问题。如果这本书\textit{必须}要用一个非自由软件才能阅读的话,我是不会答应的。]
\fi

\ifdefined\eng
I forwarded the information to Tracy. Feeling confident that she and I might be able to work out an equitable arrangement, I called up Stallman and set up the first interview for the book. Stallman agreed to the interview without making a second inquiry into the status issue. Shortly after the first interview, I raced to set up a second interview (this one in Kihei), squeezing it in before Stallman headed off on a 14-day vacation to Tahiti. [RMS: That was not a pure vacation; I gave a speech there too.]
\fi

\ifdefined\chs
我把这个信息传达给了特蕾西,并对于她和我应该可以有办法商讨出一个公平的协议抱有很大的信心。我打电话给斯托曼预约这本书的第一次采访。斯托曼同意接受采访,而且没有再次询问协商的进展。在第一次采访后不久,我又乘胜追击的安排了第二次采访(这次是在津汇),时间赶在他出发去塔希提岛进行为期14天的度假之前。[RMS:这不是一次纯粹度假,我在那里也发表了一次演讲。]
\fi

\ifdefined\eng
It was during Stallman's vacation that the bad news came from Tracy. Her company's legal department didn't want to adjust its [license] notice on the e-books. Readers who wanted to make their books transferable would [first have to crack the encryption code, to be able to convert the book to a free, public format such as HTML. This would be illegal and they might face criminal penalties.]
\fi

\ifdefined\chs
正当斯托曼在度假时,从特蕾西那里传来了坏消息。她们公司的法务部门不希望改变这本电子书的[许可证]声明。如果读者想要把这本电子书转移给别人,就必须[先对电子书进行破解,然后才能把它转换成其它的开放格式,比如HTML。这样的做法是违法的,并且会面临诉讼。]
\fi

\ifdefined\eng
With two fresh interviews under my belt, I didn't see any way to write the book without resorting to the new material. I quickly set up a trip to New York to meet with my agent and with Tracy to see if there was a compromise solution.
\fi

\ifdefined\chs
手里拿着两次新鲜出炉的采访内容,我不希望最终出版的书籍中无法包含这些内容。我很快决定动身去纽约,与我的代理人和特蕾西见面,商量一个折中的方案。
\fi

\ifdefined\eng
When I flew to New York, I met my agent, Henning Guttman. It was our first face-to-face meeting, and Henning seemed pessimistic about our chances of forcing a compromise, at least on the publisher's end. The large, established publishing houses already viewed the e-book format with enough suspicion and weren't in the mood to experiment with copyright language that made it easier for readers to avoid payment. As an agent who specialized in technology books, however, Henning was intrigued by the novel nature of my predicament. I told him about the two interviews I'd already gathered and the promise not to publish the book in a way that made Stallman ``look like a hypocrite.'' Agreeing that I was in an ethical bind, Henning suggested we make that our negotiating point.
\fi

\ifdefined\chs
当我飞抵纽约后,我见到了我的代理人,亨宁·格特曼。这是我们的第一次面对面的会面,亨宁看起来对于找到一个妥协的方案并不乐观,至少从出版社的角度来说是这样。这些历史悠久的大出版社本来就对电子书的出现充满疑惑,所以他们并不情愿对于可能让读者更容易逃避付费的版权条款进行改变。虽然亨宁是一个关注于技术图书的代理人,他仍对我所处的戏剧般的窘境感到好奇。我告诉他有关我已经进行的两次采访的内容,并且告诉他我已经承诺斯托曼我们不会以让他看起来是个”伪君子“的方式去出版本书。亨宁理解我被伦理所绑架的困境,建议我把这一点作为谈判的关键点。
\fi

\ifdefined\eng
Barring that, Henning said, we could always take the carrot-and-stick approach. The carrot would be the publicity that came with publishing an e-book that honored the hacker community's internal ethics. The stick would be the risks associated with publishing an e-book that didn't. Nine months before Dmitry Sklyarov became an Internet \textit{cause célèbre}, we knew it was only a matter of time before an enterprising programmer revealed how to hack e-books. We also knew that a major publishing house releasing an [encrypted] e-book on Richard M. Stallman was the software equivalent of putting ``Steal This E-Book'' on the cover.
\fi

\ifdefined\chs
亨宁说,险此之外,我们还可以采用萝卜加大棒的策略。所谓的萝卜就是出版一本向黑客社区内在的伦理致敬的电子书的会来带不小的公众影响力,而大棒就是出版一本并不能真正弘扬黑客文化的电子书。在德米特里·斯柯里亚诺夫的官司在Internet的引起轰动(\textit{cause célèbre})前九个月,我们就已经预料到程序员找到破解电子书的方法只是个时间问题。我们能想象,如果出版社出版一本内容是有关理查德·M·斯托曼的加密的电子书,那无异在封面上用大字标明“请破解这本电子书”。
\fi

\ifdefined\eng
After my meeting with Henning, I called Stallman. Hoping to make the carrot more enticing, I discussed a number of potential compromises. What if the publisher released the book's content under a [dual] license, something similar to what Sun Microsystems had done with OpenOffice, the free software desktop applications suite? The publisher could then release DRM-restricted\endnote{RMS: Williams wrote ``commercial'' here, but that is a misnomer, since it means ``connected with business.''  All these versions would be commercial if a company published them.} versions of the e-book under [its usual] format, taking advantage of all the bells and whistles that went with the e-book software, while releasing the copyable version under a less aesthetically pleasing HTML format.
\fi

\ifdefined\chs
我见过亨宁后,我打了个电话给斯托曼。为了能让我们的”萝卜“显得更具诱惑力,我向他介绍了一些我们可能会使用的妥协方案。比如,出版版以[双重]许可证的方式来发布书籍的内容,就像太阳计算机系统公司(Sun Microsystems)对OpenOffice这个自由桌面办公套件所做的那样。这样的话,出版社就可以以正常的电子书出版方式出版一个DRM加密的\endnote{RMS:威廉姆斯在这里用的词是“商业”,属于用词不当,因为这个词的意思是“与交易有关”。其实不管以哪种方式出版电子书,只要是由一个公司出版的,就都是“商业”的。}电子书,这样就可以应用到那些电子书软件的好处,同时再以HTML格式发布一个可以自由复制的版本。
\fi

\ifdefined\eng
Stallman told me he didn't mind the [dual-license] idea, but he did dislike the idea of making the freely copyable version inferior to the restricted version. Besides, he said [on second thought, this case was different precisely because he had] a way to control the outcome. He could refuse to cooperate.
\fi

\ifdefined\chs
斯托曼告诉我,他不反对双重[许可证]的方式,但他不同意自由版本比受限制的版本地位要更低一等。另外,他说,[他仔细想了想,OpenOffice的例子不适用现在的情况,因为但对于这本书,]他完全有能力控制最终的产出,他有权绝拒合作。
\fi

\ifdefined\eng
[RMS: The question was whether it would be wrong for me to agree to the restricted version.  I can endorse the free version of Sun's OpenOffice, because it is free software and much better than nothing, while at the same time I reject the nonfree version.  There is no self-contradiction here, because Sun didn't need or ask my approval for the non-free version; I was not responsible for its existence. In this case, if I had said yes to the non-freely-copyable version, the onus would fall on me.]
\fi

\ifdefined\chs
[RMS:问题在于,如果我同意出版本一个受限的版本,这会不会一个错误的决定。我可以支持自由版本的Sun OpenOffice,因为它是一个自由软件,并且有比没有要好,同时我也可以拒绝使用非自由的版本。在这个问题上,我并没有自相矛盾,因为Sun公司发布非自由的版本并不需要征求我的意见,我无须为这个版本的存在负责。但在这个问题上,如果我同意出版一个不允许自由复制的版本,那这个责任就都在我身上。]
\fi

\ifdefined\eng
I made a few more suggestions with little effect. About the only thing I could get out of Stallman was a concession [RMS: i.e., a further compromise] that the e-book's  [license] restrict all forms of file sharing to ``noncommercial redistribution.''
\fi

\ifdefined\chs
我的其它一些建议也没有什么实质性的效果,唯一可以得到斯托曼认可的是他可以做出一些让步[RMS:也就是说,又妥协了一些],允许电子书的[许可证]限制改为“允许除了商业使用外的各种形式的分享”。
\fi

\ifdefined\eng
Before I signed off, Stallman suggested I tell the publisher that I'd promised Stallman that the work would be [freely sharable]. I told Stallman I couldn't agree to that statement [RMS: though it was true, since he had accepted my conditions at the outset] but that I did view the book as unfinishable without his cooperation. Seemingly satisfied, Stallman hung up with his usual sign-off line: ``Happy hacking.''
\fi

\ifdefined\chs
在我挂断电话前,斯托曼建议我告诉出版社说我已经答应了斯托曼说这本作品必须是[可以自由分享的]。我告斯托曼我不能答应这个说法[RMS:确实是这样,因为他从一开始就已经接受了我的条件],但我很清楚如果没有他的合作我是无法完成这本书的。斯托曼看上去很满意,以他惯有的结尾方式说了声“开心黑客”,然后挂断了电话。
\fi

\ifdefined\eng
Henning and I met with Tracy the next day. Tracy said her company was willing to publish copyable excerpts in a unencrypted format but would limit the excerpts to 500 words. Henning informed her that this wouldn't be enough for me to get around my ethical obligation to Stallman. Tracy mentioned her own company's contractual obligation to online vendors such as Amazon.com. Even if the company decided to open up its e-book content this one time, it faced the risk of its partners calling it a breach of contract. Barring a change of heart in the executive suite or on the part of Stallman, the decision was up to me. I could use the interviews and go against my earlier agreement with Stallman, or I could plead journalistic ethics and back out of the verbal agreement to do the book.
\fi

\ifdefined\chs
次日,我和亨宁见到了特蕾西。特蕾西说她的公司同意以不加密的方式发布本书的一些摘要,但摘要的长度需要限制在500个词以内。亨宁告诉她,这样的让步不足以把我从对斯托曼的伦理义务中解救出来。特蕾西说,她的公司的对于一些像Amazon.com这样的在线商家也有一些合同上的限制。即使出版社同意把这本书的电子版开放出去,这也还面临着可能被其它合作伙伴的诉讼。是想要出版社管理层的妥协还是让斯托曼妥协,这个决定权在我。我可以违反先前与斯托曼的约定,继续使用这些采访素材,也可以恪守新闻工作者的职业操守,对出版这本书的口头承诺食言。
\fi

\ifdefined\eng
Following the meeting, my agent and I relocated to a pub on Third Ave. I used his cell phone to call Stallman, leaving a message when nobody answered. Henning left for a moment, giving me time to collect my thoughts. When he returned, he was holding up the cell phone.
\fi

\ifdefined\chs
会议后,我和代理人转移到了第三大街上的一个酒吧。我用他的手机给斯托曼打了个电话,不过没人接电话,所以只能给他留了个言。亨宁离开了一会儿,给我一些时间来思考。当他回来时,他拿着电话:
\fi

\ifdefined\eng
``It's Stallman,'' Henning said.
\fi

\ifdefined\chs
“是斯托曼打来的。” 亨宁说。
\fi

\ifdefined\eng
The conversation got off badly from the start. I relayed Tracy's comment about the publisher's contractual obligations.
\fi

\ifdefined\chs
这次对话从一开始就很不愉快。我把特蕾西所说的有关出版社的合同限制问题传告给斯托曼。
\fi

\ifdefined\eng
``So,'' Stallman said bluntly. ``Why should I give a damn about their contractual obligations?''
\fi

\ifdefined\chs
斯托曼很粗鲁的回复道:”那么,为什么我需要因为他们的合同限制而作出妥协?“
\fi

\ifdefined\eng
Because asking a major publishing house to risk a legal battle with its vendors over a 30,000-word e-book is a tall order, I suggested. [RMS: His unstated premise was that I couldn't possibly refuse this deal for mere principle.]
\fi

\ifdefined\chs
我说,因为让一个大出版公司为了一本30000多字的电子书去冒一个圈入法律斗争的风险是一个非常离谱的要求。[RMS:他的假设是我不可能因为一点小小的原则去拒绝这个机会。]
\fi

\ifdefined\eng
``Don't you see?'' Stallman said. ``That's exactly why I'm doing this. I want a signal victory. I want them to make a choice between freedom and business as usual.''
\fi

\ifdefined\chs
“难道你还没有明白吗?”斯托曼说。“这正是我要坚持我的立场的原因,因为我需要发出一个胜利的信号,我要他们像以前一样,在自由和商业之间作出选择。”
\fi

\ifdefined\eng
As the words ``signal victory'' echoed in my head, I felt my attention wander momentarily to the passing foot traffic on the sidewalk. Coming into the bar, I had been pleased to notice that the location was less than half a block away from the street corner memorialized in the 1976 Ramones song, ``53rd and 3rd,'' a song I always enjoyed playing in my days as a musician. Like the perpetually frustrated street hustler depicted in that song, I could feel things falling apart as quickly as they had come together. The irony was palpable. After weeks of gleefully recording other people's laments, I found myself in the position of trying to pull off the rarest of feats: a Richard Stallman compromise. When I continued hemming and hawing, pleading the publisher's position and revealing my growing sympathy for it, Stallman, like an animal smelling blood, attacked. 
\fi

\ifdefined\chs
“胜利的信号”这个词回响在我的脑海中,我感觉到自己的思维一瞬间游走到了路边的人行道上,人行道上满是匆匆而过的行人。来到约定的洒吧,我很高兴的发现,这个地方距离那首1976年拉莫内斯的歌曲“第53和第3”所纪念的街角只有半个街区的距离,当我还是一个音乐人时,我非常喜欢演奏这个曲子。就像歌曲中所描绘的那个一直受挫折的家伙,当这些事情放到一起,我感觉到一点崩溃。这种反讽是非常明显的。经历了几周高兴的记录别人的不幸的日子,我发现自己正处于一个罕见的胜利边缘:让理查德.斯托曼做出妥协。当我还在犹豫不决是不是要恳求出版社改变立场并透露出我的对他们的同情时,斯托曼就像一头闻到血腥味的野兽,对我发起了攻击。
\fi

\ifdefined\eng
``So that's it? You're just going to screw me? You're just going to bend to their will?''
\fi

\ifdefined\chs
“这是怎么回事?你是在耍我吗?你是想按他们的意志去办吗?”
\fi

\ifdefined\eng
[RMS: The quotations show that Williams' interpretation of this conversation was totally wrong.  He compares me to a predator, but I was only saying no to the deal he was badgering me to accept.  I had already made several compromises, some described above; I just refused to compromise my principles entirely away.  I often do this; people who aren't satisfied say I ``refused to compromise at all,'' but that is an exaggeration; see \url{http://www.gnu.org/philosophy/compromise.html}. Then I feared he was going to disregard the conditions he had previously agreed to, and publish the book with DRM despite my refusal.  What I smelled was not his ``blood'' but possible betrayal.]
\fi

\ifdefined\chs
[RMS:从这句话中可以看出威廉姆斯没有正确的理解这次对话的内容。他把我比作是一只食肉动物,但事实上我只是拒绝了他试图让我接受的条件。正如前文所述,我已经作出了不少的让步,我只不过是拒绝对于原则性的问题作出让步。我经常这样做;一些对此不满的人就说我”完全不接受让步“,但这完全就是夸张了;有关这个问题,可以能考我的文章:\url{http://www.gnu.org/philosophy/compromise.html}。那时候,我所担心的是他完全不顾我们先前达成的二人协议,不管我的反对出版一本通过DRM保护的图书。我所嗅到的不是“血腥味”,而是存在背叛的可能性。]
\fi

\ifdefined\eng
I brought up the issue of a dual-copyright again.
\fi

\ifdefined\chs
我再次提出有关双重版权的方案。
\fi

\ifdefined\eng
``You mean license,'' Stallman said curtly.
\fi

\ifdefined\chs
“你所说的版权应该是指许可证,”斯托曼冷冷的说。
\fi

\ifdefined\eng
``Yeah, license. Copyright. Whatever,'' I said, feeling suddenly like a wounded tuna trailing a rich plume of plasma in the water.
\fi

\ifdefined\chs
“哦,是的,是许可证,或者版权,差不多吧”,我说,像是水里一条受了伤的金枪鱼。
\fi

\ifdefined\eng
``Aw, why didn't you just fucking do what I told you to do!'' he shouted.  [RMS: I think this quotation was garbled, both because using ``fucking'' as an adverb was never part of my speech pattern, and because the words do not fit the circumstances.  The words he quotes are a rebuke to a disobedient subordinate.  I felt he had an ethical obligation, but he was not my subordinate, and I would not have spoken to him as one.  Using notes rather than a recorder, he could not reliably retain the exact words.]
\fi

\ifdefined\chs
“你他妈的就不能按我所交待你的去做吗?”他咆哮道。[RMS:我觉得这句话可能会引起误解,首先把“他妈的”当成副词来使用绝不是我的说话风格,其次,这句话也不太符合当时的语境。这句话听起来像是在责备一名不听话的下属。我觉得他有一种伦理上的义务,但是他不是我的下属,所以我也不会用这样的语气说话。通过笔记而不是录音来记录对话,他没法保证记录下准确的内容。]
\fi

\ifdefined\eng
I must have been arguing on behalf of the publisher to the very end, because in my notes I managed to save a final Stallman chestnut: ``I don't care. What they're doing is evil. I can't support evil. Good-bye.''  [RMS: It sounds like I had concluded that he would never take no for an answer, and the only way to end the conversation without accepting his proposition was to hang up on him.]
\fi

\ifdefined\chs
我应该一直站在出版社的立场上跟他争辨到底了,因为我的记事本上记着这次对话中斯托曼的最后那段老生长谈的立场:“我不关心这些。他们在做的就是一些邪恶的事情。我不能支持作恶。再见。”[RMS:这样说听起来像是我总结说他不能接受一个否定的答复,但是想不接受他的建议而结束这次对话的唯一办法就是挂断电话。]
\fi

\ifdefined\eng
As soon as I put the phone down, my agent slid a freshly poured Guinness to me. ``I figured you might need this,'' he said with a laugh. ``I could see you shaking there towards the end.''
\fi

\ifdefined\chs
当我放下电话,我的代理人递给我一杯刚倒的健力士啤酒。“我想你现在需要这个,”他笑着说,“我看到你打电话时混身在发抖。”
\fi

\ifdefined\eng
I was indeed shaking. The shaking wouldn't stop until the Guinness was more than halfway gone. It felt weird, hearing myself characterized as an emissary of ``evil.'' [RMS: My words as quoted criticize the publisher, not Williams personally.  If he took it personally, perhaps that indicates he was starting to take ethical responsibility for the deal he had pressed me to accept.]  It felt weirder still, knowing that three months before, I was sitting in an Oakland apartment trying to come up with my next story idea. Now, I was sitting in a part of the world I'd only known through rock songs, taking meetings with publishing executives and drinking beer with an agent I'd never even laid eyes on until the day before. It was all too surreal, like watching my life reflected back as a movie montage.
\fi

\ifdefined\chs
我确实在发抖,直到半杯健力士啤酒下了肚,才慢慢平静下来。我有一种奇怪的感觉,似乎自己的变成“邪恶”的代言人。三个月前,当然坐在奥克兰的公寓中构思我的第一部作品时,也有过类似但更强烈的感觉。现在,我正坐在一个我只从摇滚歌曲中接触到过的世界,与出版社的管理层开会、与我一天前才见到的代理人喝酒。这确实太超现实了,就像是自己的生活出现在了电影里一样。
\fi

\ifdefined\eng
About that time, my internal absurdity meter kicked in. The initial shaking gave way to convulsions of laughter. To my agent, I must have looked like a another fragile author undergoing an untimely emotional breakdown. To me, I was just starting to appreciate the cynical beauty of my situation. Deal or no deal, I already had the makings of a pretty good story. It was only a matter of finding a place to tell it. When my laughing convulsions finally subsided, I held up my drink in a toast.
\fi

\ifdefined\chs
正刚这个时候,我内心的荒谬爆发了。最初的震动让我流露出一丝笑意。我的代理人一定认为我像是一个脆弱的作家,正在经历一次前所未有的情感崩溃。而对于我自己来说,我只是开始认识到自己的境遇是如此的讥诮。无论最终能不能成交,我已经完成了一个漂亮的故事。只不过是我需要找一个地方来讲述这个故事。我渐渐收起了笑容,干了杯中的酒。
\fi

\ifdefined\eng
``Welcome to the front lines, my friend,'' I said, clinking pints with my agent. ``Might as well enjoy it.''
\fi

\ifdefined\chs
“欢迎来到斗争的前线,我的朋友,”我与代理人一边碰杯一边说,“并且享受斗争吧。”
\fi

\ifdefined\eng
If this story really were a play, here's where it would take a momentary, romantic interlude. Disheartened by the tense nature of our meeting, Tracy invited Henning and me to go out for drinks with her and some of her coworkers. We left the bar on Third Ave., headed down to the East Village, and caught up with Tracy and her friends.
\fi

\ifdefined\chs
如果这个故事真的是一个戏剧的话,这里就是应该加入一个短暂、浪漫的插曲的地方了。我们的会议很紧张,结果也很令人感到沮丧,于是特蕾西邀请我和亨宁与她和她的一些同事们一起出去喝酒。我们离开了第三大街上的酒巴,前往东村,见到了特蕾西和她的朋友们。
\fi

\ifdefined\eng
Once there, I spoke with Tracy, careful to avoid shop talk. Our conversation was pleasant, relaxed. Before parting, we agreed to meet the next night. Once again, the conversation was pleasant, so pleasant that the Stallman e-book became almost a distant memory.
\fi

\ifdefined\chs
一到那里,我就跟特蕾西展开了交谈,并且小心的避免谈论工作。我们的交谈很愉快,也很放松。在告别前,我们约定次日晚上再见。这次交谈确实非常的愉快,以至于斯托曼的电子书的问题都快消失在记忆中了。
\fi

\ifdefined\eng
When I got back to Oakland, I called around to various journalist friends and acquaintances. I recounted my predicament. Most upbraided me for giving up too much ground to Stallman in the preinterview negotiation. [RMS: Those who have read the whole book know that I would never have dropped the conditions.] A former j-school professor suggested I ignore Stallman's ``hypocrite'' comment and just write the story. Reporters who knew of Stallman's media-savviness expressed sympathy but uniformly offered the same response: it's your call.
\fi

\ifdefined\chs
当我回到奥克兰,我召集了一些新闻出牌界的朋友和熟人。我向他们讲述了我所遇到的困境。他们中的很多人都责备我在采访斯托曼前的谈判中放弃了太多的原则。一个新闻学院的老教建议我忘掉斯托曼所谓“两面派”的说法,把故事完整的写下来。很多了解斯托曼对媒体的精明能力的记者表达了对我的同情,但他们一致建议我:这是你自己的事。
\fi

\ifdefined\eng
I decided to put the book on the back burner. Even with the interviews, I wasn't making much progress. Besides, it gave me a chance to speak with Tracy without running things past Henning first. By Christmas we had traded visits: she flying out to the west coast once, me flying out to New York a second time. The day before New Year's Eve, I proposed. Deciding which coast to live on, I picked New York. By February, I packed up my laptop computer and all my research notes related to the Stallman biography, and we winged our way to JFK Airport. Tracy and I were married on May 11. So much for failed book deals.
\fi

\ifdefined\chs
我决定把这本书的事先放到一边。就算是说采访,我其实也并没有太多的进展。除此之处,这让给了我不先通亨宁直接与特蕾西沟通的机会。到了圣诞节,我们扯平了相互访问的次数,她飞到西海岸来了一次,我也第二次飞去了纽约。新年前夜,我向她求婚了。我决定选择纽约作为我们定居的地方。二月分,我带着我的笔记本电脑和所有有关斯托曼传记的研究笔记,我们一起飞去了肯尼迪机场。我和特蕾西在五月十一日结婚了。这就是有关这本书的不成功的交易的结果。
\fi

\ifdefined\eng
During the summer, I began to contemplate turning my interview notes into a magazine article. Ethically, I felt in the clear doing so, since the original interview terms said nothing about traditional print media. To be honest, I also felt a bit more comfortable writing about Stallman after eight months of radio silence. Since our telephone conversation in September, I'd only received two emails from Stallman. Both chastised me for using ``Linux'' instead of ``GNU/Linux'' in a pair of articles for the web magazine \textit{Upside Today}. Aside from that, I had enjoyed the silence. In June, about a week after the New York University speech, I took a crack at writing a 5,000-word magazine-length story about Stallman. This time, the words flowed. The distance had helped restore my lost sense of emotional perspective, I suppose.
\fi

\ifdefined\chs
整个夏天,我开始考虑把我的采访笔记改编成杂志的文章。从伦理上来说,我觉得这样做是合理的,因为最初的采访条款上没有规定采访内容的适用范围。实话说,隔了八个月的时间,我觉得为斯托曼写传记变得舒服了一些。9月份我们通过电话以后,我只收到过斯托曼的两封电子邮件。两封邮件的内容都是谴责我在\textit{Upside Today}这个网络杂志的文章中使用“Linux”一词而不是“GNU/Linux”。除此之外,我很享受这份安静。六月,在纽约大学的演讲后一周,我抽空写了一篇有关斯托曼的5000字的杂志稿件。这一次,文字变得非常的流畅。我想,是我跟斯托曼之间的距离感帮助我恢复了曾经失去的情感观点。
\fi

\ifdefined\eng
In July, a full year after the original email from Tracy, I got a call from Henning. He told me that O'Reilly \& Associates, a publishing house out of Sebastopol, California, was interested in the running the Stallman story as a biography. [RMS: I have a vague memory that I suggested contacting O'Reilly, but I can't be sure after all these years.] The news pleased me. Of all the publishing houses in the world, O'Reilly, the same company that had published Eric Raymond's \textit{The Cathedral and the Bazaar}, seemed the most sensitive to the issues that had killed the earlier e-book. As a reporter, I had relied heavily on the O'Reilly book \textit{Open Sources} as a historical reference. I also knew that various chapters of the book, including a chapter written by Stallman, had been published with [license] notices that permitted redistribution. Such knowledge would come in handy if the issue of electronic publication ever came up again.
\fi

\ifdefined\chs
七月,那是在收到特蕾西的第一封电子邮件后的整整一年,我收到了来自亨宁的一个电话。他告诉我,位于加州塞瓦斯托波尔的O'Reilly \& Associates出版集团对于出版斯托曼的传记很有兴趣。这个消息让我很开心。在全世界的所有出版社中,O'Reilly看上去是对于造成先前胎死腹中的图书的原因最为知晓的公司,他同时也是为埃里克·雷蒙德出版《\textit{教堂与集市}》一书的公司。作为一个记者,我以前也重度依赖O'Reilly出版的\textit{开源}一书来作为历史的参考。我也知道有不少书的章节,包话斯托曼曾经写过的一章,都是以允许随意分布的[许可证]条款来出版的。如果电子出版的问题再次出现,那这些信息就会很有用处。
\fi

\ifdefined\eng
Sure enough, the issue did come up. I learned through Henning that O'Reilly intended to publish the biography both as a book and as part of its new Safari Tech Books Online subscription service. The Safari user license would involve special restrictions,\endnote{See ``Safari Tech Books Online; Subscriber Agreement: Terms of Service'' \url{http://my.safaribooksonline.com/termsofservice}.  As of December, 2009, these e-books require nonfree reader software, so people should refuse to use them.} Henning warned, but O'Reilly was willing to allow for a copyright that permitted users to copy and share the book's text regardless of medium. Basically, as author, I had the choice between two licenses: the Open Publication License or the GNU Free Documentation License.
\fi

\ifdefined\chs
果不其然,电子出版还是遇到了一些困难。亨宁告诉我说,O'Reilly打开把这本传记同时以纸质书和它的新的Safari Tech Books Online订阅服务的形式出版。亨宁提醒我,Safari的用户许可证中有一项特殊的限制\endnote{参考“Safari在线科技书籍;订阅者协议:服务条款”,\url{http://my.safaribooksonline.com/termsofservice}。截止2009年12月,这些电子书需要非自由的阅读器软件来阅读,所以建议大家抵制这种电子书},但是O'Reilly同意允许让用户在任意媒界上复制并分享这本书的文字内容。所以,作为作者,我需要在两些许可证之间做出选则:Open Publication License或GNU自由文档许可证。
\fi

\ifdefined\eng
I checked out the contents and background of each license. The Open Publication License (OPL)\endnote{See ``The Open Publication License: Draft v1.0'' (June 8, 1999), \url{http://opencontent.org/openpub/}.} gives readers the right to reproduce and distribute a work, in whole or in part, in any medium ``physical or electronic,'' provided the copied work retains the Open Publication License. It also permits modification of a work, provided certain conditions are met. Finally, the Open Publication License includes a number of options, which, if selected by the author, can limit the creation of ``substantively modified'' versions or book-form derivatives without prior author approval.
\fi

\ifdefined\chs
我研究了这两种许可证的内容和背景。Open Publication License (OPL) \endnote{参考"The Open Publication License: Draft v1.0" (1999年7月8日),\url{http://opencontent.org/openpub/}。}允许读者以整体或部分的方式复制和重新分发相关作品,无论是在“物理的媒介还是电子的”,只需要保证继续保持OPL许可证即可。在满足这两个条件的前提下,它也允许对作品进行修改。同时,OPL也包含一些选项,可以由作者来选用,可以限制读者在没有得到作者允许前不得对作品进行“实质性的修改”或以书籍的形式出版。
\fi

\ifdefined\eng
The GNU Free Documentation License (GFDL), meanwhile, permits the copying and distribution of a document in any medium, provided the resulting work carries the same license.\endnote{See ``The GNU Free Documentation License: Version 1.3'' (November, 2008), \url{http://www.gnu.org/copyleft/fdl.html}.} It also permits the modification of a document provided certain conditions. Unlike the OPL, however, it does not give authors the option to restrict certain modifications. It also does not give authors the right to reject modifications that might result in a competitive book product. It does require certain forms of front- and back-cover information if a party other than the copyright holder wishes to publish more than 100 copies of a protected work, however.
\fi

\ifdefined\chs
GNU自由文档许可证(GFDL) 则允许在任何媒介上复制和分发文档,只要求保留相同的许可证。\endnote{参考 "The GNU Free Documentation License: Version 1.1" (2000年3月),\url{http://www.gnu.org/copyleft/fdl.html}。}它也允许在一些条件下对文档进行修改。与OPL不同的是,它没有给作者禁止他人修改的权力。它也不给作者拒绝修改内容以避免他人出版一本类似的竞争作品的权力。然而,除了版权所有人以外的他人如果要出版100份以上的内容,则需要在封面和封底上提供一些必要的信息。
\fi

\ifdefined\eng
In the course of researching the licenses, I also made sure to visit the GNU Project web page titled ``Various Licenses and Comments About Them.''\endnote{See \url{http://www.gnu.org/philosophy/license-list.html}.} On that page, I found a Stallman critique of the Open Publication License. Stallman's critique related to the creation of modified works and the ability of an author to select either one of the OPL's options to restrict modification. If an author didn't want to select either option, it was better to use the GFDL instead, Stallman noted, since it minimized the risk of the nonselected options popping up in modified versions of a document.
\fi

\ifdefined\chs
在研究这些许可证的时候,我也在GNU工程的官方网站上的”各种许可证与对此的说明“\endnote{参考:\url{http://www.gnu.org/philosophy/license-list.html}。}文章页面上找到了一些斯托曼对于Open PUblication License的批评。斯托曼对它的批评主要关注在有关创建修改后的版本以及作者有权选择拒绝修改这两点上。斯托曼认为,如果作者并不打算行使这两项限制权力,那他完全可以改用GFDL,这样可以避免在修改后版本中突然被加上相关限制条件的风险。
\fi

\ifdefined\eng
The importance of modification in both licenses was a reflection of their original purpose -- namely, to give software-manual owners a chance to improve their manuals and publicize those improvements to the rest of the community. Since my book wasn't a manual, I had little concern about the modification clause in either license. My only concern was giving users the freedom to exchange copies of the book or make copies of the content, the same freedom they would have enjoyed if they purchased a hardcover book. Deeming either license suitable for this purpose, I signed the O'Reilly contract when it came to me.
\fi

\ifdefined\chs
修改行为在两种许可证中的重要性都反映了这它们的设计初衷,那就是允许软件手册的读者有机会去改进这些手册并把它们贡献到社区中去。由于我的书并不是一本手册,所以我对于两种许可证中对于修改的条款都有一些顾虑。我只是希望能够让我读者有机会去交换图书的副本或是复制图书的内容,如果他们买了纸质的图书,也可以享有相同的自然。我感觉这两种许可证都可以满足我的需求,所以我就跟O'Reilly签了相关的合同。
\fi

\ifdefined\eng
Still, the notion of unrestricted modification intrigued me. In my early negotiations with Tracy, I had pitched the merits of a GPL-style license for the e-book's content. At worst, I said, the license would guarantee a lot of positive publicity for the e-book. At best, it would encourage readers to participate in the book-writing process. As an author, I was willing to let other people amend my work just so long as my name always got top billing. Besides, it might even be interesting to watch the book evolve. I pictured later editions looking much like online versions of the \textit{Talmud}, my original text in a central column surrounded by illuminating, third-party commentary in the margins.
\fi

\ifdefined\chs
但是,有关可以随意修改的条款还是引发了我的好奇心。在我与特蕾西的早期谈判中,我曾经谈过对电子书使用GPL风格的许可证好处。我说,再不济的情况下,这样的许可证也可以对这本电子书的宣传起到正面的作用。在最好的情况下,这可以鼓励读者参与到图书的写作过程中来。作为作者,我非常希望其它能跟帮我一起改进我的作品,而然我也可以坐享其成。另外,看着一本书不断的演化也是一件很有趣的事情。我把未来的版本描绘成类似于《\textit{塔木德}》的在线版本,我原始的文字在中间,两边的空白由第三方的评论来装饰。
\fi

\ifdefined\eng
My idea drew inspiration from Project Xanadu (\url{http://www.xanadu.com}), the legendary software concept originally conceived by Ted Nelson in 1960. During the O'Reilly Open Source Conference in 1999, I had seen the first demonstration of the project's [free] offshoot Udanax and had been wowed by the result. In one demonstration sequence, Udanax displayed a parent document and a derivative work in a similar two-column, plain-text format. With a click of the button, the program introduced lines linking each sentence in the parent to its conceptual offshoot in the derivative. An e-book biography of Richard M. Stallman didn't have to be Udanax-enabled, but given such technological possibilities, why not give users a chance to play around?\endnote{Anybody willing to ``port'' this book over to Udanax, the free software version of Xanadu, will receive enthusiastic support from me. To find out more about this intriguing technology, visit \url{http://www.udanax.com}.}
\fi

\ifdefined\chs
我的灵感来自于Xanadu项目(\url{http://www.xanadu.com/}),这是泰德·尼尔森在1960年构想的传奇般的软件概念。在1999年的O'Reilly开源大会上,我看到了这个项目的开源版本Udanax,它的演示效果很让人震惊。在一个演示中,Uadanx把一篇父文档和它的衍生作品分成两栏纯文本进行展示。只要点击一下按钮,程序就可以把原文档和衍生作品中的每一个相关的句子联系起来。理查德.马修.斯托曼传记的电子版不需要真正的去用Udanax实现,但是既然我们在技术上有这样的可能性,那为什么不让读者能有一个参与的机会呢?\endnote{我会很乐意向任何愿意把本书“移植”到Uadanax平台的人提供帮助,Uadanax是Xanadu的自由软件版本。请访问\url{http://www.udanax.com/}了解这项引人入胜的技术。}
\fi

\ifdefined\eng
When Laurie Petrycki, my editor at O'Reilly, gave me a choice between the OPL or the GFDL, I indulged the fantasy once again. By September of 2001, the month I signed the contract, e-books had become almost a dead topic. Many publishing houses, Tracy's included, were shutting down their e-book imprints for lack of interest. I had to wonder. If these companies had treated e-books not as a form of publication but as a form of community building, would those imprints have survived?
\fi

\ifdefined\chs
当O'Reilly的编辑劳里·帕瑞其给我在OPL和GFDL中选择一个的机会时,我又一次萌发了这样的奇思妙想。到了2001年9月,也就是我签合同的时候,电子书几乎已经成了一个无人关心的话题。包括特蕾西所在的出版社在内的大部分出版社,都已经因为读者缺乏兴趣而停止了标名电子书出版印记。我开始怀疑,如果这些出版社都不把电子书看成是一种出版形式而只是看成是一种社区建设,那为什么这些出版印记还需要存在?
\fi

\ifdefined\eng
After I signed the contract, I notified Stallman that the book project was back on. I mentioned the choice O'Reilly was giving me between the Open Publication License and the GNU Free Documentation License. I told him I was leaning toward the OPL, if only for the fact I saw no reason to give O'Reilly's competitors a chance to print the same book under a different cover. Stallman wrote back, arguing in favor of the GFDL, noting that O'Reilly had already used it several times in the past. Despite the events of the past year, I suggested a deal. I would choose the GFDL if it gave me the possibility to do more interviews and if Stallman agreed to help O'Reilly publicize the book. Stallman agreed to participate in more interviews but said that his participation in publicity-related events would depend on the content of the book. Viewing this as only fair, I set up an interview for December 17, 2001 in Cambridge.
\fi

\ifdefined\chs
签完合同后,我通知斯托曼我们的图书项目可以继续下去了。我提到了O'Reilly在许可证方面给我们的选择,可以选择OPL或者GFDL。我告诉他我倾向于使用OPL,其原因只是因为我觉得没有理由要给O'Reilly的竞争者机会去只是换个封面就出版一本内容一样的新书。斯托曼回信说建议使用GFDL,因为O'Reilly以前也多次使用这个许可证。不管去年所发生过的那些事情,我建议他跟我做个交易。如果斯托曼能给我更多采访的机会,并且促成本书在O'Reilly出版,我就选用GFDL许可证。斯托曼同意接受更多的采访,但是对于出版相关的事宜的支持程度会取决于这本书的内容。这看上去很公平,于是我约定在2001年12月7日在剑桥对斯托曼再进行一次采访。
\fi

\ifdefined\eng
I set up the interview to coincide with a business trip my wife Tracy was taking to Boston. Two days before leaving, Tracy suggested I invite Stallman out to dinner.
\fi

\ifdefined\chs
这次采访的时间与我的妻子特蕾西去波士顿出差的行程正好一致。在出发前两天,特蕾西建议我邀请斯托曼一起吃个饭。
\fi

\ifdefined\eng
``After all,'' she said, ``he is the one who brought us together.''
\fi

\ifdefined\chs
她说:“不管怎么说,他是促成我们联姻的那个人。”
\fi

\ifdefined\eng
I sent an email to Stallman, who promptly sent a return email accepting the offer. When I drove up to Boston the next day, I met Tracy at her hotel and hopped the T to head over to MIT. When we got to Tech Square, I found Stallman in the middle of a conversation just as we knocked on the door.
\fi

\ifdefined\chs
我给斯托曼写了一封电子邮件,他很快回复接受这次邀请。次日,我开车去波士顿,在洒店接上特蕾西,就转去了麻省理工学院。当我们到达科技广场,敲门进去时,斯托曼正在跟别人交谈。
\fi

\ifdefined\eng
``I hope you don't mind,'' he said, pulling the door open far enough so that Tracy and I could just barely hear Stallman's conversational counterpart. It was a youngish woman, mid-20s I'd say, named Sarah.
\fi

\ifdefined\chs
“我希望你们能不介意”,他一边说,一边把门打开,门开得很大,我和特蕾西勉强可以听清斯托曼的说话的对象的声音。她是莎拉,一个20多岁的年轻女子。
\fi

\ifdefined\eng
``I took the liberty of inviting somebody else to have dinner with us,'' Stallman said, matter-of-factly, giving me the same cat-like smile he gave me back in that Palo Alto restaurant.
\fi

\ifdefined\chs
“我私自邀请了一个其它人一起参加我们的饭局”,斯托曼诚恳的说,他给了一个跟在帕罗奥多的饭店里见面时一样的狡猾的微笑。
\fi

\ifdefined\eng
To be honest, I wasn't too surprised. The news that Stallman had a new female friend had reached me a few weeks before, courtesy of Stallman's mother. ``In fact, they both went to Japan last month when Richard went over to accept the Takeda Award,'' Lippman told me at the time.\endnote{Alas, I didn't find out about the Takeda Foundation's decision to award Stallman, along with Linus Torvalds and Ken Sakamura, with its first-ever award for ``Techno-Entrepreneurial Achievement for Social/Economic Well-Being'' until after Stallman had made the trip to Japan to accept the award. For more information about the award and its accompanying \$1 million prize, visit the Takeda site, \url{http://www.takeda-foundation.jp}.}
\fi

\ifdefined\chs
实话说,我并没感到太多的意外。我几周前就听斯托曼的母亲说斯托曼有了个新的异性朋友。“事实上,他们上个月一起去了日本,理查德去接受Takeda奖”,李普曼当时是这么跟我说的。\endnote{另外,我没有找到有关Takeda基金会要嘉奖斯托曼、林纳斯·托瓦兹和坂村健的相关信息,似乎斯托曼前往日本去领取的是第一次也是最后一次所谓的“Techno-Entrepreneurial Achievement for Social/Economic Well-Being”。有关这个奖项和它100万美元的奖金的更多信息,可以参考Takeda的网站。\url{http://www.takea-foundation.jp}。}
\fi

\ifdefined\eng
On the way over to the restaurant, I learned the circumstances of Sarah and Richard's first meeting. Interestingly, the circumstances were very familiar. Working on her own fictional book, Sarah said she heard about Stallman and what an interesting character he was. She promptly decided to create a character in her book on Stallman and, in the interests of researching the character, set up an interview with Stallman. Things quickly went from there. The two had been dating since the beginning of 2001, she said.
\fi

\ifdefined\chs
在去饭店的路上,我了解到了Sarah和理查德头一次见面的情景。 有意思的事,这个情景让我觉得很熟悉。莎拉在构思一部小说时,听说了斯托曼,并对于他是什么样的性格觉得很好奇。她想在自己的作品中以斯托曼为原型设定一个人物形象,为了能了解这个形象的个性,她对斯托曼进行了一次采访。事情就是这样开始的,他们两个人从2001年年初开始约会了,她说。
\fi

\ifdefined\eng
``I really admired the way Richard built up an entire political movement to address an issue of profound personal concern,'' Sarah said, explaining her attraction to Stallman.
\fi

\ifdefined\chs
“我非常景仰理查德建立起这个完整的政治运动的方式,而这个运动只是为了满足他个人的一想法。”莎拉向我解释着他关注斯托曼的原因。
\fi

\ifdefined\eng
My wife immediately threw back the question: ``What was the issue?''
\fi

\ifdefined\chs
我的妻子立刻抛回一个问题:“所以,真正问题是什么?”
\fi

\ifdefined\eng
``Crushing loneliness.''
\fi

\ifdefined\chs
“粉碎孤独。”
\fi

\ifdefined\eng
During dinner, I let the women do the talking and spent most of the time trying to detect clues as to whether the last 12 months had softened Stallman in any significant way. I didn't see anything to suggest they had. Although more flirtatious than I remembered, Stallman retained the same general level of prickliness. At one point, my wife uttered an emphatic ``God forbid'' only to receive a typical Stallman rebuke.
\fi

\ifdefined\chs
晚餐的时候,我安排女士们与斯托曼多聊聊天,尝试发现他在过去一年中为什么会变成不那么强硬。不过我没有找到任何线索。斯托曼的眼光数次停留在我妻子的胸部,表现得比我印象中的样子更为卖俏,但他还是跟平时一样保持着敏感的状态。一度我的妻子加重语气说“但愿不会”,也只是得到了一个典型的斯托曼式的指责。
\fi

\ifdefined\eng
``I hate to break it to you, but there is no God,'' Stallman said. [RMS: I must have been too deadpan. He could justly accuse me of being a wise guy, but not of rebuking.]
\fi

\ifdefined\chs
“我不想打断你,但是上帝并不存在”,斯托曼说。[RMS:我当时一定是个冷面笑匠。他可以把我当成是一个聪明的家伙,但不能责备我。]
\fi

\ifdefined\eng
Afterwards, when the dinner was complete and Sarah had departed, Stallman seemed to let his guard down a little. As we walked to a nearby bookstore, he admitted that the last 12 months had dramatically changed his outlook on life. ``I thought I was going to be alone forever,'' he said. ``I'm glad I was wrong.''
\fi

\ifdefined\chs
过了一会儿,晚餐结束后,莎拉离开了,斯托曼开始看上去放松了一些警惕。当我们路过附近一家书店时,他承认过去的一年让他改变了对他生活的愿景。“我曾经以为我会一辈子孤独,”他说,“不过很幸运的时我错了。”
\fi

\ifdefined\eng
Before parting, Stallman handed me his ``pleasure card,'' a business card listing Stallman's address, phone number, and favorite pastimes (``sharing good books, good food and exotic music and dance'') so that I might set up a final interview.
\fi

\ifdefined\chs
告别前,斯托曼给了我一张他的“好运卡”,一张写有斯托曼的地址、电话号码和爱好(“分享好书、好吃的和国外的音乐和舞蹈”)的名片,以便于我安排最后一次采访。
\fi

\ifdefined\eng
The next day, over another meal of dim sum, Stallman seemed even more lovestruck than the night before. Recalling his debates with Currier House dorm maters over the benefits and drawbacks of an immortality serum, Stallman expressed hope that scientists might some day come up with the key to immortality. ``Now that I'm finally starting to have happiness in my life, I want to have [a longer life],'' he said.
\fi

\ifdefined\chs
次日,我们几个一起去吃点心,斯托曼看上去比前一天晚上更沉浸于热恋中。回忆起他于克利尔舍堂的宿舍管理员有关“永生血清”好坏的争执,斯托曼表示希望有一天科学家们能研究出不朽的关键。“现在,我终于开始在生活中找到快乐,我希望我能一直快乐下去。”他说。
\fi

\ifdefined\eng
When I mentioned Sarah's ``crushing loneliness'' comment, Stallman failed to see a connection between loneliness on a physical or spiritual level and loneliness on a hacker level. ``The impulse to share code is about friendship but friendship at a much lower level,'' he said. Later, however, when the subject came up again, Stallman did admit that loneliness, or the fear of perpetual loneliness [RMS: at the hacker-to-hacker, community level, that is], had played a major role in fueling his determination during the earliest days of the GNU Project.
\fi

\ifdefined\chs
当我提到莎拉所说的“粉碎孤独”的评价,斯托曼没有看到在身体和精神层面的孤独和黑客层面的孤独之前的联系。“分享代码的推动力来自朋友情谊,而友情在更低的一个层面。”他说。不过,后来又提到这个话题时,斯托曼承认了孤独,或者说是对持续的孤独的恐惧[RMS:在黑客与黑客之间,社区的层面上]在GNU工程最初的岁月里,是他做出的各种决定起到了一个非常主要作用。
\fi

\ifdefined\eng
``My fascination with computers was not a consequence of anything else,'' he said. ``I wouldn't have been less fascinated with computers if I had been popular and all the women flocked to me. However, it's certainly true the experience of feeling I didn't have a home, finding one and losing it, finding another and having it destroyed, affected me deeply. The one I lost was the dorm. The one that was destroyed was the AI Lab. The precariousness of not having any kind of home or community was very powerful. It made me want to fight to get it back.''
\fi

\ifdefined\chs
他说:“我对计算机的迷恋不是任何别的东西的结果。如果我人缘更好,有更多追随我的女性,我也对计算机的迷恋也不会有什么改变。当然,我感觉到自己没家,找到一个人又失去她, 找到另一个又摧毁她,这对我一生都有很大的影响。我失去的是我们的宿舍,我摧毁的是人工智能实验室。没有任何家和社区的归属感所带来的不安全感非常巨大,它们迫使我去战斗,把它们抢回来。”
\fi

\ifdefined\eng
After the interview, I couldn't help but feel a certain sense of emotional symmetry. Hearing Sarah describe what attracted her to Stallman and hearing Stallman himself describe the emotions that prompted him to take up the free software cause, I was reminded of my own reasons for writing this book. Since July, 2000, I have learned to appreciate both the seductive and the repellent sides of the Richard Stallman persona. Like Eben Moglen before me, I feel that dismissing that persona as epiphenomenal or distracting in relation to the overall free software movement would be a grievous mistake. In many ways the two are so mutually defining as to be indistinguishable.
\fi

\ifdefined\chs
这次采访过后,我禁不住感受到一种情感上的相似。听到莎拉说斯托曼所吸引她的地方,听到斯找曼本人描述激发他投身自由软件运动的情感因素,我体会到了我自己撰写这本书的本质原因。从2000年7月开始,我就学会去赞扬理查德.斯托曼让人赞扬又让人排斥的双重性格。就像我的前辈伊本·莫格林,如果把这样的人格当成是一种附带现象或是把它与整个自由软件运动隔离开来去分析是一种巨大的错误。在某种程度上,这两者是完全互补,不可分割的。
\fi

\ifdefined\eng
[RMS: Williams objectifies his reactions, both positive and negative, as parts of me, but they are functions also of his own attitudes about appearance, conformity, and business success.]
\fi

\ifdefined\chs
[RMS:威廉姆斯把他的感受进行了客观的描述,包括正面的和负面的,作为对我的分析,但是这些同时也是他自己对于外观、服从和商业成功的态度。]
\fi

\ifdefined\eng
While I'm sure not every reader feels the same level of affinity for Stallman\ldots I'm sure most will agree [that] few individuals offer as singular a human portrait as Richard M. Stallman. It is my sincere hope that, with this initial portrait complete and with the help of the GFDL, others will feel a similar urge to add their own perspective to that portrait.
\fi

\ifdefined\chs
我肯定不会每一位读者对于斯托曼都有相似程度的认同,在读完这本书后,有些人仍然会完全不认同斯托曼,但我肯定有些人会认同他。很少人能像理查德·马修·斯托曼这样立独行。我真诚的希望有这样的一个完整的榜样,并且有GFDL的帮忙,其它人能够有类似的动力向这个肖像上补充更多他们的观点。
\fi

\theendnotes
\setcounter{endnote}{0}
