%% Copyright (c) 2002, 2010 Sam Williams
%% Copyright (c) 2010 Richard M. Stallman
%% Permission is granted to copy, distribute and/or modify this
%% document under the terms of the GNU Free Documentation License,
%% Version 1.3 or any later version published by the Free Software
%% Foundation; with no Invariant Sections, no Front-Cover Texts, and
%% no Back-Cover Texts. A copy of the license is included in the
%% file called ``gfdl.tex''.

%% DONE 

\chapter{\ifdefined\eng
Appendix B -- Hack, Hackers, and Hacking
\fi
\ifdefined\chs
附录B:黑客的三层含义
\fi} \label{Appendix B}

\ifdefined\eng
To understand the full meaning of the word ``hacker,'' it helps to examine the word's etymology over the years.
\fi

\ifdefined\chs
要理解``黑客''一词内在的含义,需要追朔这个词的词源和这么多年来的演化过程。
\fi

\ifdefined\eng
\textit{The New Hacker Dictionary}, an online compendium of software-programmer jargon, officially lists nine different connotations of the word ``hack'' and a similar number for ``hacker.'' Then again, the same publication also includes an accompanying essay that quotes Phil Agre, an MIT hacker who warns readers not to be fooled by the word's perceived flexibility. ``Hack has only one meaning,'' argues Agre. ``An extremely subtle and profound one which defies articulation.''  %% v2 Richard Stallman tries to articulate it with the phrase, ``Playful cleverness.''
\fi

\ifdefined\chs
在\textit{新黑客词典}这本在线的软件程序员专业术语词典中,对``Hack''一词列出了9种不同的含义,``Hacker''一词也有相对应的多种含义。同时,上面也引用了一段麻省理工学院的黑客费尔·阿格雷所撰写的短文,这篇文篇提醒读者不要被这个词给人的第一印象所迷惑。``Hack这个词只有一种意思,''阿格雷争辩道,``这个意思非常微妙和有深度,难以用语言来描述。'' %% v2 理查德·斯托曼把这个词表述为一个短语:``顽皮的小聪明''。
\fi

\ifdefined\eng
Regardless of the width or narrowness of the definition, most modern hackers trace the word back to MIT, where the term bubbled up as popular item of student jargon in the early 1950s. In 1990 the MIT Museum put together a journal documenting the hacking phenomenon. According to the journal, students who attended the institute during the fifties used the word ``hack'' the way a modern student might use the word ``goof.'' Hanging a jalopy out a dormitory window was a ``hack,'' but anything harsh or malicious -- e.g., egging a rival dorm's windows or defacing a campus statue -- fell outside the bounds. Implicit within the definition of ``hack'' was a spirit of harmless, creative fun.
\fi

\ifdefined\chs
不管是讨论这个词的狭义定义还是广义定义,大部分现代的黑客都会把这个词追朔到麻省理工学院,从20世纪50年代起,这个词就成为学生们互相交流时的使用的一句黑话。1990年,麻省理工学院博物馆还发表了一篇有关黑客现象的论文。根据这篇论文的描述,50年代时,这个学校的学生们使用``Hack''这个词的方式与现在的学生使用``Goof(混混)''一词的含义类似。在宿舍的窗户上挂上一个破飞机的行为就可以算是Hack。但是,任何残忍的或恶意的行为就不能算是Hack,比如怂恿另人去打破宿舍窗户或损坏学院的雕像。``Hack''一词内在含义是在不损害他人的前提下寻找创新点和乐趣的精神。
\fi

\ifdefined\eng
This spirit would inspire the word's gerund form: ``hacking.'' A 1950s student who spent the better part of the afternoon talking on the phone or dismantling a radio might describe the activity as ``hacking.'' Again, a modern speaker would substitute the verb form of ``goof'' -- ``goofing'' or ``goofing off'' -- to describe the same activity.
\fi

\ifdefined\chs
这种精神赋于了Hack这个词的动名词形式``Hacking''更多活力。在20世纪50年代,如果一个学生花上大半天的时间打电话或拆装收音机,这样行为就可以被称作是``Hacking''。而在现代,人们通常会用``Goof''一词的动词形式``Goofing''或``Goofing Off''来描述这类行为。
\fi

\ifdefined\eng
As the 1950s progressed, the word ``hack'' acquired a sharper, more rebellious edge. The MIT of the 1950s was overly competitive, and hacking emerged as both a reaction to and extension of that competitive culture. Goofs and pranks suddenly became a way to blow off steam, thumb one's nose at campus administration, and indulge creative thinking and behavior stifled by the Institute's rigorous undergraduate curriculum. With its myriad hallways and underground steam tunnels, the Institute offered plenty of exploration opportunities for the student undaunted by locked doors and ``No Trespassing'' signs. Students began to refer to their off-limits explorations as ``tunnel hacking.'' Above ground, the campus phone system offered similar opportunities. Through casual experimentation and due diligence, students learned how to perform humorous tricks. Drawing inspiration from the more traditional pursuit of tunnel hacking, students quickly dubbed this new activity ``phone hacking.''
\fi

\ifdefined\chs
经过整个50年代的演化,``Hack''这个词的含义变得更加尖锐和反叛。50年代的麻省理工学院校园里充满了竞争,而且黑客的出现,正是这种竞争文化的产物和延续。Goof和Pranks(喜欢恶作剧的人)突然成为学生们发泄多余精力的方式,他们甚至会跑去学校的教务处去做鬼脸,沉溺于各种稀奇古怪的想法和被繁重的研究生课程所遏制的行为。学院众多走廊和地下蒸气管道提供了无畏于紧闭的大们和``禁止进入''标识的学生们足够多探索的机会。学生们开始把这种不被规章制度允许的探索称为``地道黑客''。在地面上,校园的电话系统也提供了类似的探索机会。通过业余的实验和一些专业的研究,学生找到一些恶作剧的办法。学生们从相对古老的``地道黑客''活动中获得了灵感,很快就参与到了这种新型的``电话黑客''活动中。
\fi

\ifdefined\eng
The combined emphasis on creative play and restriction-free exploration would serve as the basis for the future mutations of the hacking term. The first self-described computer hackers of the 1960s MIT campus originated from a late 1950s student group called the Tech Model Railroad Club. A tight clique within the club was the Signals and Power (S\&P) Committee -- the group behind the railroad club's electrical circuitry system. The system was a sophisticated assortment of relays and switches similar to the kind that controlled the local campus phone system. To control it, a member of the group simply dialed in commands via a connected phone and watched the trains do his bidding.
\fi

\ifdefined\chs
强调创新游戏和无限制的探索活动,成为后来黑客活动的文化基础。20世纪60年代出现了一些最早自称为是计算机黑客的人,他们来自麻省理工学院里一个起源于50年代末期、名为铁路科技模型俱乐部(Tech Model Railroad Clus)的学生组织。这个俱乐部中有一个紧密的小团队叫做``信号与电源(S\&P)委员会'',是这个俱乐部的电子电路系统的主要负责团队。这个系统是一个由继电器和开关组成的复杂系统,有点像当时学校电话的控制系统。俱乐部的成员只须通过电话拨打一些特定的号码就可以控制火车的运行。
\fi

\ifdefined\eng
The nascent electrical engineers responsible for building and maintaining this system saw their activity as similar in spirit to phone hacking. Adopting the hacking term, they began refining it even further. From the S\&P hacker point of view, using one less relay to operate a particular stretch of track meant having one more relay for future play. Hacking subtly shifted from a synonym for idle play to a synonym for idle play that improved the overall performance or efficiency of the club's railroad system at the same time. Soon S\&P committee members proudly referred to the entire activity of improving and reshaping the track's underlying circuitry as ``hacking'' and to the people who did it as ``hackers.''
\fi

\ifdefined\chs
负责建造和维护这套系统的电子工程爱好者们把这样的活动认为是与电话黑客活动一脉相成。他们不但接受了黑客这个概念,而且还把它变得更为完善。从S\&P黑客的观点来看,如果某一段火车铁轨的控制电路中可以少用一个继电器,就意味着可以节省一个继电器供以后使用。黑客这个词的含义渐渐从``闲玩''的同义词演化成了在闲玩的同时去提高俱乐部的铁路系统的整体性能和效率。不久以后,S\&P委员会的成员就自豪的宣布,把改进和重构铁路的底层电路的活动称为``黑客活动'',而参与这样活动的人就是``黑客''。
\fi

\ifdefined\eng
Given their affinity for sophisticated electronics -- not to mention the traditional MIT-student disregard for closed doors and ``No Trespassing'' signs -- it didn't take long before the hackers caught wind of a new machine on campus. Dubbed the TX-0, the machine was one of the first commercially marketed computers. By the end of the 1950s, the entire S\&P clique had migrated en masse over to the TX-0 control room, bringing the spirit of creative play with them. The wide-open realm of computer programming would encourage yet another mutation in etymology. ``To hack'' no longer meant soldering unusual looking circuits, but cobbling together software programs with little regard to ``official'' methods or software-writing procedures. It also meant improving the efficiency and speed of already-existing programs that tended to hog up machine resources. True to the word's roots, it also meant writing programs that served no other purpose than to amuse or entertain.
\fi

\ifdefined\chs
复杂的电子电路对黑客们有着天然的亲和力,抛开那些无视紧锁的大门和``禁止进入''标识的传统的麻省理工学院的学生,新一代的黑客们很快就在校园中找到了一种新的玩具。这种被称为TX-0的机器,是最早进入商用市场的计算机之一。到了20世纪50年代末,整个S\&P的团队都已经把兴趣转移到TX-0的控制室里,用他们的创新精神来折腾这个机器。从那时起,计算机程序设计领域开始进入``黑客''一词的词源。``黑客行为''不再是指焊接一些看起来很奇怪的电路板,而是更多的是与软件程序开发联系起来,还包括少许``正式''的软件开发方法或过程。它同样意味着提高效率和改进现有程序让它们运行的更快,因为只有这样才能节省更多的系统资源。从词源上来讲,这个词也指仅仅为了好玩或娱乐而编写一些没有实质性用途的程序。
\fi

\ifdefined\eng
%% v1v2 s
A classic example of this expanded hacking definition is the game Spacewar, the first interactive video game. Developed by MIT hackers in the early 1960s, Spacewar had all the traditional hacking definitions: it was goofy and random, serving little useful purpose other than providing a nightly distraction for the dozen or so hackers who delighted in playing it. From a software perspective, however, it was a monumental testament to innovation of programming skill. It was also completely free. Because hackers had built it for fun, they saw no reason to guard their creation, sharing it extensively with other programmers. By the end of the 1960s, Spacewar had become a favorite diversion for mainframe programmers around the world. %% v2 By the end of the 1960s, Spacewar had become a diversion for programmers around the world, if they had the (then rather rare) graphical displays.
\fi

\ifdefined\chs
``空间大战(Spacewar)''游戏是体现这种黑客定义的经典例子,这是世界上第一个在电脑上运行的交互式的视频游戏。早在20世纪60年代,麻省理工学院的黑客们就开发出了这个游戏,``空间大战''游戏诠释了所有传统黑客的定义:它既无聊又随机,除了可以让一群黑客们通宵达旦的娱乐外,没有什么实质性的用途。从软件的观点来说,它倒也算是程序开发技能上的一次伟大的创新。同时,它也是一个完全自由的软件。由于黑客们只是为了好玩而设计了它,所以他们觉得没有什么理由要去刻意保护这样的创作成果。因此,这个游戏可以在程序员之间广为分享。到了20世纪60年代末,``空间大战''已经成为全世界大型机程序员们最爱的一种消遣方式。
\fi

\ifdefined\eng
This notion of collective innovation and communal software ownership distanced the act of computer hacking in the 1960s from the tunnel hacking and phone hacking of the 1950s. The latter pursuits tended to be solo or small-group activities. Tunnel and phone hackers relied heavily on campus lore, but the off-limits nature of their activity discouraged the open circulation of new discoveries. Computer hackers, on the other hand, did their work amid a scientific field biased toward collaboration and the rewarding of innovation. Hackers and ``official'' computer scientists weren't always the best of allies, but in the rapid evolution of the field, the two species of computer programmer evolved a cooperative -- some might say symbiotic -- relationship.
\fi

\ifdefined\chs
60年代的计算机黑客与50年代的地道黑客和电话黑客的最大区别,就在于这种合作创新与公有的软件所有权。传统的黑客更崇尚个人或小团体的活动。地道黑客和电话黑客活动与校园里的情况紧密相关,他们这种不被规章制度所允许的活动天然的不鼓励通过公开的交流来分享新的发现。计算机黑客则不同,他们在科学领域中工作,这是一个完全基于协作和鼓励创新的氛围。黑客们与``官方''的计算机科学家并不一定是最好的搭档,但是由于计算机技术的高速发展,这两种计算机程序员之间很快就形成了一种或多或少的合作共生关系。
\fi

%% v2
%% \ifdefined\eng
%% Hackers had little respect for bureaucrats' rules.  They regarded computer security systems that obstructed access to the machine as just another bug, to be worked around or fixed if possible.  Thus, breaking security (but not for malicious purposes) was a recognized aspect of hacking in 1970, useful for practical jokes (the victim might say, ``I think someone's hacking me'') as well as for gaining access to the computer.  But it was not central to the idea of hacking.  Where there was a security obstacle, hackers were proud to display their wits in surmounting it; however, given the choice, as at the MIT AI Lab, they chose to have no obstacle and do other kinds of hacking.  Where there is no security, nobody needs to break it.
%% \fi

%% \ifdefined\chs
%% 黑客们大多都很讨厌官僚的制度。他们把阻碍自由访问计算机的安全系统看作是系统中的一个Bug,只要有机会,就要想办法绕过它或把它修复。因此,在1970年时,绕过安全系统(但不是出于恶意的目的)被看成是一种典型的黑客行为,这样做不但获取了计算机的访问权限,还增加了饭后的谈资(受害者可能会说:``我觉得我被黑了。'')。但这并不是黑客活动的核心理念。黑客们在遇到一些安全系统阻碍时,会发挥自己的才智绕过它们。但是,如果可以选择的话,麻省理工学院人工智能实验室的黑客们更倾向于从一开始就不要构建这样阻碍,从而可以把自己的才智用到更有意义的地方。如果没有所谓的安全系统,也就没有人想着要去破坏它。
%% \fi

\ifdefined\eng
It is a testament to the original computer hackers' prodigious skill that later programmers, including Richard M. Stallman, aspired to wear the same hacker mantle. By the mid to late 1970s, the term ``hacker'' had acquired elite connotations. In a general sense, a computer hacker was any person who wrote software code for the sake of writing software code. In the particular sense, however, it was a testament to programming skill. Like the term ``artist,'' the meaning carried tribal overtones. To describe a fellow programmer as hacker was a sign of respect. To describe oneself as a hacker was a sign of immense personal confidence. Either way, the original looseness of the computer-hacker appellation diminished as computers became more common.
\fi

\ifdefined\chs
这是早期计算机黑客们惊人的技能的一种证明,很多后来的程序员,包括理查德·马修·斯托曼,都渴望能够做到这一点。到了70年代中后期,``黑客''这个词又有了更精确的内涵。在一般人看来,计算机黑客是指那些可以为了写代码而写代码的人。他们觉得这是对于编程技能的一种证明。``黑客''一词跟``艺术家''这个词很像,它的涵义超越了它字面上的意思。把一名程序员称为是黑客,是对他的一种尊称,而把自己称为是黑客则表明了极强的自信。不管是哪一种情况,随着计算机的普及,``计算机黑客''这种早年很少见的称谓也变得越来越常见。
\fi

\ifdefined\eng
%v1v2 m
As the definition tightened, ``computer'' hacking acquired additional semantic overtones. To be a hacker, a person had to do more than write interesting software; a person had to belong to the hacker ``culture'' and honor its traditions the same way a medieval wine maker might pledge membership to a vintners' guild. The social structure wasn't as rigidly outlined as that of a guild, but hackers at elite institutions such as MIT, Stanford, and Carnegie Mellon began to speak openly of a ``hacker ethic'': the yet-unwritten rules that governed a hacker's day-to-day behavior. In the 1984 book \textit{Hackers}, author Steven Levy, after much research and consultation, codified the hacker ethic as five core hacker tenets.
% v2
%% As the definition tightened, ``computer'' hacking acquired additional semantic overtones.  The hackers at the MIT AI Lab shared many other characteristics, including love of Chinese food, disgust for tobacco smoke, and avoidance of alcohol, tobacco and other addictive drugs.  These characteristics became part of some people's understanding of what it meant to be a hacker, and the community exerted an influence on newcomers even though it did not demand conformity.  However, these cultural associations disappeared with the AI Lab hacker community.  Today, most hackers resemble the surrounding society on these points.
\fi

\ifdefined\chs
对``黑客''的定义越来越精确,``计算机''黑客则还有一层超越了字面的涵义。要成为一个黑客,不仅仅需要去编写一些有意义的软件,同时还需要能够融入黑客``文化''并崇尚它的传统,就像一个中世纪的酿酒师需要加入一个酿酒协会一样。黑客社区的结构不像协会那么严格,但是在麻省理工学院、斯坦福和卡耐基梅隆大学这样的精英大学中,黑客们开始公开宣传``黑客道德'':这些不成文的规定影响着黑客们的日常行为。在1984年出版的《\textit{黑客}》一书中,作者史蒂芬·李维在进过了很多研究和咨询以后,把黑客道德定义为五种核心的黑客信条。
%% v2
%% 对``黑客''的定义越来越精确,``计算机''黑客则还有一层超越了字面的涵义。麻省理工学院人工智能实验室的黑客们有一些共同的特质,包括,喜欢吃中国菜、讨厌抽烟、不沾烟、酒和其它一些可能会上瘾的药物。这些特质也渐渐成为一些人心目中对黑客群体的印象,这种印象对于新加入这个群体的人也产生了潜移默化的影响。然而,这些黑客文化在人工智能实验室黑客社区中渐渐的消失了。如今,大部分黑客在这些点上表现通常都与他周围的环境保持一致。
%% 
\fi

\ifdefined\eng
In many ways, the core tenets listed by Levy continue to define the culture of computer hacking. Still, the guild-like image of the hacker community was undermined by the overwhelmingly populist bias of the software industry. By the early 1980s, computers were popping up everywhere, and programmers who once would have had to travel to top-rank institutions or businesses just to gain access to a machine suddenly had the ability to rub elbows with major-league hackers via the ARPAnet. The more these programmers rubbed elbows, the more they began to appropriate the anarchic philosophies of the hacker culture in places like MIT. Lost within the cultural transfer, however, was the native MIT cultural taboo against malicious behavior. As younger programmers began employing their computer skills to harmful ends-creating and disseminating computer viruses, breaking into military computer systems, deliberately causing machines such as MIT Oz, a popular ARPAnet gateway, to crash-the term ``hacker'' acquired a punk, nihilistic edge. When police and businesses began tracing computer-related crimes back to a few renegade programmers who cited convenient portions of the hacking ethic in defense of their activities, the word ``hacker'' began appearing in newspapers and magazine stories in a negative light. Although books like Hackers did much to document the original spirit of exploration that gave rise to the hacking culture, for most news reporters, ``computer hacker'' became a synonym for ``electronic burglar.''
\fi

\ifdefined\chs
从很多层面上来说,李维所列出的这些核心信条都进一步定义了计算机黑客的文化的核心思想。但是黑客社区给人们留下的协会般的印象,并没有被普通人占据优势地位的软件工业所压制。80年代初,计算机开始在各行各业广为应用,那些原本只在在顶级大学或公司才能接触到电脑的程序员,现在可以通过ARPAnet与一流的黑客进行交流。通过交流,这些程序员开始渐渐接触到像麻省理工学院那些黑客文化中的无政府主义的哲学。然而,那些麻省理工学院原生的黑客文化中的禁忌,比如禁止恶意行为,在文化的转移中却渐渐的消失了。一些年轻的程序员开始在一些错误的方向应用他们的计算机才能,发明和传播计算机病毒、攻击军事计算机网络、故意让类似于ARPAnet网关的MIT Oz这样的计算机宕机。``黑客''这个词的含义开始渐渐滑向一个无政府主义的边缘。当警察和工商界开始发现计算机犯罪的来源是少数不守规矩的程序员,而这些程序员又片面的引用一些黑客的信条来为他们的行为辩解,黑客这个词也就更多的在报纸和杂志上以负面形象出现。尽管像《\textit{黑客}》这样书用很多的篇幅去介绍黑客精神中的探索本质,并且弘扬黑客文化,但是对于大部分报纸记者和读者仍然把``计算机黑客''当成是``电子盗贼''的同义词。
\fi

\ifdefined\eng
Although hackers have railed against this perceived misusage for nearly two decades, the term's rebellious connotations dating back to the 1950s make it hard to discern the 15-year-old writing software programs that circumvent modern encryption programs from the 1960s college student, picking locks and battering down doors to gain access to the lone, office computer terminal. One person's creative subversion of authority is another person's security headache, after all. Even so, the central taboo against malicious or deliberately harmful behavior remains strong enough that most hackers prefer to use the term `` cracker''-i.e., a person who deliberately cracks a computer security system to steal or vandalize data-to describe the subset of hackers who apply their computing skills maliciously.
\fi

\ifdefined\chs
仅管黑客这个词被这样错误的使用了近二十年,这个词从50年代以来所象征的反判的含义让人很难分辩一个破解现代加密程序的15岁软件开发经验和一个破坏门锁或直接损坏房门才得以访问办公用电脑终端60年代的大学生有什么区别。毕竟,一个人想要颠覆权威的创新精神就是让另一个人头疼的安全问题。即使是这样,作为信条的核心禁忌,反对恶意行为这一条仍然有很深远的影响力,所以大部分黑客都喜欢用``骇客''这个词来描述这类故意破解计算机安全系统去盗取或破坏数据的人,这些所谓的``黑客''使用他们的计算机技能去做一些恶意的事情。
\fi

%% v2
%% \ifdefined\eng
%% As the hackers at elite institutions such as MIT, Stanford, and Carnegie Mellon conversed about hacks they admired, they also considered the ethics of their activity, and began to speak openly of a ``hacker ethic'': the yet-unwritten rules that governed a hacker's day-to-day behavior. In the 1984 book \textit{Hackers}, author Steven Levy, after much research and consultation, codified the hacker ethic as five core hacker tenets.
%% \fi

%% \ifdefined\chs
%% 麻省理工学院、斯坦福和卡耐基梅隆大学这样的精英大学中的黑客谈论他们所敬仰的黑客特质时,会谈到有关黑客活动的伦理,并开始公开宣传``黑客伦理'':这些不成文的规定影响着黑客们的日常行为。在1984年出版的《\textit{黑客}》一书中,作者史蒂文·利维在进行了很多研究和咨询以后,把黑客伦理定义为五种核心的黑客信条。
%% \fi

%% \ifdefined\eng
%% In the 1980s, computer use expanded greatly, and so did security breaking.  Mostly it was done by insiders with criminal intent, who were generally not hackers at all.  However, occasionally the police and administrators, who defined disobedience as evil, traced a computer ``intrusion'' back to a hacker whose idea of ethics was ``Don't hurt people.''   Journalists published articles in which ``hacking'' meant breaking security, and usually endorsed the administrators' view of the matter.  Although books like \textit{Hackers} did much to document the original spirit of exploration that gave rise to the hacking culture, for most newspaper reporters and readers the term ``computer hacker'' became a synonym for ``electronic burglar.''
%% \fi

%% \ifdefined\chs
%% 20世纪80年代,计算机开始普及,破坏安全系统的行为也变多了。大部分破坏安全系统的行为都具有某些犯罪动机,实施这些破坏的内鬼其实根本就不能被称作是黑客。然而,警察和计算机管理员都会把破坏规则的行为看成是一种恶意的行为,他们把这种``入侵''计算机的行为追溯到了那些持有``不能伤害别人''理念的黑客身上。一些杂志的报道中,常常会从计算机管理员的视角出发,把``黑客行为''当成是破坏安全系统的同义词。尽管像《\textit{黑客}》这样书用很多的篇幅去介绍黑客精神中的探索本质,并且弘扬黑客文化,但是对于大部分报纸记者和读者仍然把``计算机黑客''当成是``电子盗贼''的同义词。
%% \fi

%% \ifdefined\eng
%% By the late 1980s, many U.S. teenagers had access to computers.  Some were alienated from society; inspired by journalists' distorted picture of ``hacking,'' they expressed their resentment by breaking computer security much as other alienated teens might have done it by breaking windows. They began to call themselves ``hackers,'' but they never learned the MIT hackers' principle against malicious behavior. As younger programmers began employing their computer skills to harmful ends -- creating and disseminating computer viruses, breaking into computer systems for mischief, deliberately causing computers to crash -- the term ``hacker'' acquired a punk, nihilistic edge which attracted more people with similar attitudes.
%% \fi

%% \ifdefined\chs
%% 到了80年代末,很多美国的青少年都开始有机会接触计算机。他们中的一部分有些叛逆的性格,受到杂志上歪曲了的``黑客''形象的误导,开始通过破坏计算机安全系统来发泄他们的对社会的不满,就像一些人通过砸玻璃来发泄一样。他们自称为``黑客'',但是他们从来没了解过麻省理工学院的黑客们反对恶意行为的信条。一些年轻的程序员开始在一些错误的方向利用他们的计算机才能:制作和传播计算机病毒、恶作剧入侵计算机系统、故意让计算机宕机。``黑客''这个词开始渐渐成为一个反叛、无政府主义的象征,这样的形象吸引了更多不了解黑客本质的人开始走向错误的方向。
%% \fi

%% \ifdefined\eng
%% Hackers have railed against this perceived misusage of their self-designator for nearly two decades.  Stallman, not one to take things lying down, coined the term ``cracking'' for ``security breaking'' so that people could more easily avoid calling it ``hacking.''  But the distinction between hacking and cracking is often misunderstood. These two descriptive terms are not meant to be exclusive.  It's not that ``Hacking is here, and cracking is there, and never the twain shall meet.''  Hacking and cracking are different attributes of activities, just as ``young'' and ``tall'' are different attributes of persons.
%% \fi

%% \ifdefined\chs
%% 黑客们为了改变这种错误的认识已经自发的进行了近二十年的斗争。斯托曼为了解决这个问题,创造了一个新词来表达``破坏计算机安全系统的人''的意思:``骇客'',避免人们错误的使用``黑客''一词。而且,黑客和骇客这两个词的还是常常引起人们的误解。这两个词表达的含义还是有一些重叠的地方,并不是``黑客在这里,骇客在那里,我们水火不容''。黑客行为和骇客行为只是同一种行为的不同层面上的属性,就像``年轻''和``高''是一个人的两个不同层面的属性。
%% \fi

%% \ifdefined\eng
%% Most hacking does not involve security, so it is not cracking.  Most cracking is done for profit or malice and not in a playful spirit, so it is not hacking.  Once in a while a single act may qualify as cracking and as hacking, but that is not the usual case.  The hacker spirit includes irreverence for rules, but most hacks do not break rules.  Cracking is by definition disobedience, but it is not necessarily malicious or harmful.  The computer security field distinguishes between ``black hat'' and ``white hat'' crackers -- i.e., crackers who turn toward destructive, malicious ends versus those who probe security in order to fix it.
%% \fi

%% \ifdefined\chs
%% 其实,大部的黑客活动都与安全无关,所以不能称之为骇客活动。大部分骇客活动都是为了牟取利益或出于恶意,不会是出于``玩''的心态,所以不能称之为黑客活动。在少数情况下,可能会有个别的行为可以同时看成是黑客行为与骇客行为,但大部分情况下都不会是这样。黑客精神中包含对规则的不敬,但大部分黑客并不会去破坏规则。骇客活动从定义上来说就是不守规则,但是也并不一定就是恶意或有害的。在计算机安全领域中还区分``黑帽''骇客和``白帽''骇客,如果一个骇客的目的是搞破坏、恶意的,他就是一个``黑帽'';那些在安全系统中寻找漏洞是为了修复它的骇客,被称作为''白帽''。
%% \fi

\ifdefined\eng
%% v1v2 s
This central taboo against maliciousness remains the primary cultural link between the notion of hacking in the early 21st century and hacking in the 1950s. %% v2 The hacker's central principle not to be malicious remains the primary cultural link between the notion of hacking in the early 21st century and hacking in the 1950s.
It is important to note that, as the idea of computer hacking has evolved over the last four decades, the original notion of hacking -- i.e., performing pranks or exploring underground tunnels -- remains intact. In the fall of 2000, the MIT Museum paid tradition to the Institute's age-old hacking tradition with a dedicated exhibit, the Hall of Hacks. The exhibit includes a number of photographs dating back to the 1920s, including one involving a mock police cruiser. In 1993, students paid homage to the original MIT notion of hacking by placing the same police cruiser, lights flashing, atop the Institute's main dome. The cruiser's vanity license plate read IHTFP, a popular MIT acronym with many meanings. The most noteworthy version, itself dating back to the pressure-filled world of MIT student life in the 1950s, is ``I hate this fucking place.'' In 1990, however, the Museum used the acronym as a basis for a journal on the history of hacks. Titled, \textit{The Institute for Hacks Tomfoolery and Pranks}, the journal offers an adept summary of the hacking.
\fi

\ifdefined\chs
这种忌讳恶意行为的核心原则成为联系21世纪初的黑客与20世纪50年代的黑客的文化纽带。值得注意的是,在过去的40年中,计算机黑客概念的演化了很多。最初这个词的意思是指有点恶作剧的行为,比如发现一些隐蔽的地道,这层含义现在依然存在。2000年秋天,麻省理工学院博物馆举行了一个专门向古老的黑客传统致敬的展览,名为黑客殿堂。这个展览上展出了很多20世纪20年代的老照片,其中有一张是有人愚弄警察的巡逻车的照片。1993年,有学生向这种行为致敬,把一模一样的巡逻车、警灯放到了学院主楼的圆顶上。巡逻车的仿制车牌上写着IHTFP,这是一个在麻省理工学院很流行的缩写词,包含着很多的含义。其中最引人注意的意思是始创于20世纪50年代麻省理工学院压力最大的时候的流行语,``我憎恨这个鬼地方(I hate this fucking place)''。然而,在1990年,博物馆基于这个缩写词创作了一本期刊:《\textit{黑客、蠢举与恶作剧学报(The Journal of the Institute for Hacks, Tomfoolery, and Pranks)}》,这本期刊的主要内容有是关黑客的历史,它精妙描述了黑客活动的本质。
\fi

\ifdefined\eng
``In the culture of hacking, an elegant, simple creation is as highly valued as it is in pure science,'' writes \textit{Boston Globe} reporter Randolph Ryan in a 1993 article attached to the police car exhibit. ``A Hack differs from the ordinary college prank in that the event usually requires careful planning, engineering and finesse, and has an underlying wit and inventiveness,'' Ryan writes. ``The unwritten rule holds that a hack should be good-natured, non-destructive and safe. In fact, hackers sometimes assist in dismantling their own handiwork.''
\fi

\ifdefined\chs
``在黑客的文化中,优雅、简洁的创新是被高度评价的,就像在纯科学的领域,''《\textit{波士顿环球报}》的记者伦道夫·赖安在1993年的一篇警车展览相关的文章中写道,``黑客与一般校园中的那些喜欢恶作剧的人在一些需要细致计划、工程化和使用手段的事件中的表现不太一样,他们本质上更为聪明和具有创造力。''赖安写道,``黑客的本质应该是好的,不能去搞破坏,应该是安全的,这是一条不成文的规则。事实上,黑客们有时也参与破坏他们自己的作品''。
\fi

\ifdefined\eng
%% v1v2 m
The urge to confine the culture of computer hacking within the same ethical boundaries is well-meaning but impossible. Although most software hacks aspire to the same spirit of elegance and simplicity, the software medium offers less chance for reversibility. Dismantling a police cruiser is easy compared with dismantling an idea, especially an idea whose time has come. Hence the growing distinction between ``black hat'' and ``white hat''-i.e., hackers who turn new ideas toward destructive, malicious ends versus hackers who turn new ideas toward positive or, at the very least, informative ends.
\fi

\ifdefined\chs
主张把计算机黑客的文化限定在特定的伦理边界中是一个不错的想法,但却是不太可能的。尽管大部分软件黑客们渴望优雅和简单的精神气质,但是软件媒体很少提供回头的机会。摆脱警察的巡逻车要比破碎一个想法更为容易,尤其是一个出现的正是时候的想法。所以``黑帽''与``白帽''之间的差异也越来越明显:``黑帽''黑客把新的想法转变为破坏性的、恶意的行为,而``白帽''黑客把新想法转变为正面的或者至少是有益的方向。
\fi

\ifdefined\eng
Once a vague item of obscure student jargon, the word ``hacker'' has become a linguistic billiard ball, subject to political spin and ethical nuances. Perhaps this is why so many hackers and journalists enjoy using it. Where that ball bounces next, however, is anybody's guess. %% v2 We cannot predict how people will use the word in the future.  We can, however, decide how we will use it ourselves.  Using the term ``cracking'' rather than ``hacking,'' when you mean ``security breaking,'' shows respect for Stallman and all the hackers mentioned in this book, and helps preserve something which all computer users have benefited from: the hacker spirit.
\fi

\ifdefined\chs
``黑客''这个词曾经只是学生们一个含糊其辞的黑话,现在却成为了一个语言学上的一个台球,服从于政治的束缚和伦理上的差异。这也许就是为什么很多黑客和记者们喜欢使用这个词语的原因。至于这个台球下一步会弹去哪个方向,人们还是只能众说纷纭。%% 没有人可以预料这个词以后会如何被使用,但是,我们可以决定我们自己如何去使用它。当你在谈论``破坏安全系统''时,请使用``骇客''这个词,而不要使用``黑客'',这样不但可以表现出你对斯托曼和本书中提到的其他黑客们的尊敬,也可以让黑客精神可以不断传承下去,造福更多计算机用户。
\fi
