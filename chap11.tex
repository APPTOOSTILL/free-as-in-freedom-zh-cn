%% Copyright (c) 2002, 2010 Sam Williams
%% Copyright (c) 2010 Richard M. Stallman
%% Permission is granted to copy, distribute and/or modify this
%% document under the terms of the GNU Free Documentation License,
%% Version 1.3 or any later version published by the Free Software
%% Foundation; with no Invariant Sections, no Front-Cover Texts, and
%% no Back-Cover Texts. A copy of the license is included in the
%% file called ``gfdl.tex''.

%% WITHOUT ENDNOTES/ NAMES/ NOT REVIEWED
\chapter{\ifdefined\eng
Open Source
\fi
\ifdefined\chs
开源
\fi} \label{chapter:open source}
\thispagestyle{empty}
%% \ifdefined\eng
%% [RMS: In this chapter only, I have deleted some quotations.  The
%% material deleted was about open source and didn't relate to my life or
%% my work.]
%% \fi

%% \ifdefined\chs
%% [RMS:在这一章中,我删除了一些引文。因这些内容都只是与开源有关的,与我的生活和工作都没有什么关系。]
%% \fi

\ifdefined\eng
In November , 1995, Peter Salus, a member of the Free Software Foundation and author of the 1994 book, \textit{A Quarter Century of Unix}, issued a call for papers to members of the GNU Project's ``system-discuss'' mailing list. Salus, the conference's scheduled chairman, wanted to tip off fellow hackers about the upcoming Conference on Freely Redistributable Software in Cambridge, Massachusetts. Slated for February, 1996 and sponsored by the Free Software Foundation, the event promised to be the first engineering conference solely dedicated to free software and, in a show of unity with other free software programmers, welcomed papers on ``any aspect of GNU, Linux, NetBSD, 386BSD, FreeBSD, Perl, Tcl/tk, and other tools for which the code is accessible and redistributable.'' Salus wrote:
\fi

\ifdefined\chs
1995年11月,自由软件基金会成员彼得·萨卢斯(Peter Salus)(他于1994年出版了《\textit{Unix的四分之一世纪}》一书),在GNU工程的system-discuss邮件列表中发布了一个通知,为即将在马萨诸塞州的剑桥召开的``自由发布软件大会''征集来自黑客们的论文。这场由自由软件基金会赞助的大会将于1996年2月召开,这是第一个纯由自由软件开发工程师的参加的大会,为了表现出自由软件程序员的团结一致,大会欢迎``各种有关GNU、Linux、NetBSD、386BSD、FreeBSD、Perl、Tcl/tk以及其它任何可以自由获取和分发的工具软件''相关的文章来大会交流。萨卢斯在通知中写道:
\fi

\begin{quote}
\ifdefined\eng
Over the past 15 years, free and low-cost software has become ubiquitous. This conference will bring together implementers of several different types of freely redistributable software and publishers of such software (on various media). There will be tutorials and refereed papers, as well as keynotes by Linus Torvalds and Richard Stallman.\endnote{See Peter Salus, ``FYI-Conference on Freely Redistributable Software, 2/2, Cambridge'' (1995) (archived by Terry Winograd), \url{http://bat8.inria.fr/~lang/hotlist/free/licence/fsf96/call-for-papers.html}.}
\fi

\ifdefined\chs
在过去的15年中,自由软件和低成本的软件变得越来越重要。本次大会将召集各种可自由分发软件的开发者和通过各种渠道发布这类软件的发布者来参加。大会上会有各种教程和参考论文,也会有林纳斯·托瓦兹和理查德·斯托曼的主题演讲。\endnote{参考彼得·萨卢斯, ``FYI-Conference on Freely Redistributable Software, 2/2, Cambridge'' (1995) (由特里·威诺格拉德提供)。\url{http://bat8.inria.fr/~lang/hotlist/free/licence/fsf96/call-for-papers.html}}
\fi
\end{quote}

\ifdefined\eng
One of the first people to receive Salus' email was conference committee member Eric S. Raymond. Although not the leader of a project or company like the various other members of the list, Raymond had built a tidy reputation within the hacker community as a major contributor to GNU Emacs and as editor of \textit{The New Hacker Dictionary}, a book version of the hacking community's decade-old Jargon File.
\fi

\ifdefined\chs
大会组委会成员埃里克·斯蒂芬·雷蒙德是最早收到萨卢斯的电子邮件的人之一,虽然他不像邮件列表中其它人那样是某个项目的开发负责人或是代表某个公司,但他在GNU Emacs的中的贡献以及作为《\textit{新黑客词典}》(一本收录黑客社会各种黑话的书)的编辑的身份,使他在黑客社区中享有一定的声誉。
\fi

\ifdefined\eng
For Raymond, the 1996 conference was a welcome event. Active in the GNU Project during the 1980s, Raymond had distanced himself from the project in 1992, citing, like many others before him, Stallman's ``micro-management'' style. ``Richard kicked up a fuss about my making unauthorized modifications when I was cleaning up the Emacs LISP libraries,'' Raymond recalls. ``It frustrated me so much that I decided I didn't want to work with him anymore.''
\fi

\ifdefined\chs
对于雷蒙德来说,1996年的这次大会是一个重要的契机。二十世纪80年代,他在GNU工程中非常活跃,但到了1992年以后,他就已经不再直接参与这个工程。与许多跟他经历类似的人一样,他对斯托曼的``微管理''风格颇有微词。``理查德对于我在清理Emacs LISP库中的代码时进行的修改非常不满。''雷蒙德回忆到,``这使得我感觉到很沮丧,我再也不想与他合作了。''
\fi

\ifdefined\eng
Despite the falling out, Raymond remained active in the free software community. So much so that when Salus suggested a conference pairing Stallman and Torvalds as keynote speakers, Raymond eagerly seconded the idea. With Stallman representing the older, wiser contingent of ITS/Unix hackers and Torvalds representing the younger, more energetic crop of Linux hackers, the pairing indicated a symbolic show of unity that could only be beneficial, especially to ambitious younger (i.e., below 40) hackers such as Raymond. ``I sort of had a foot in both camps,'' Raymond says.
\fi

\ifdefined\chs
除了不再直接参与GNU工程,雷蒙德在自由软件社区中仍然非常活跃。当萨卢斯建议在大会上由斯托曼和托瓦兹做主题演讲时,雷蒙德表示了非常赞同。斯托曼是老一辈ITS/Unix黑客的传承人,而托瓦兹则是新一代的更有活力的Linux黑客的代表,这样的搭配是一种联盟的象征,尤其会受到像雷蒙德这样的年经有抱负的黑客的欢迎。``我像是个脚踏两只船的人。'' 雷蒙德说。
\fi

\ifdefined\eng
By the time of the conference, the tension between those two camps had become palpable. Both groups had one thing in common, though: the conference was their first chance to meet the Finnish \textit{wunderkind} in the flesh. Surprisingly, Torvalds proved himself to be a charming, affable speaker. Possessing only a slight Swedish accent, Torvalds surprised audience members with his quick, self-effacing wit.\endnote{Although Linus Torvalds is Finnish, his mother tongue is Swedish. ``The Rampantly Unofficial Linus FAQ'' at \url{http://catb.org/~esr/faqs/linus/} offers a brief explanation:

\begin{quote}

Finland has a significant (about 6\%) Swedish-speaking minority population. They call themselves \textit{finlandssvensk} or \textit{finlandssvenskar} and consider themselves Finns; many of their families have lived in Finland for centuries. Swedish is one of Finland's two official languages.

\end{quote}

} Even more surprising, says Raymond, was Torvalds' equal willingness to take potshots at other prominent hackers, including the most prominent hacker of all, Richard Stallman. By the end of the conference, Torvalds' half-hacker, half-slacker manner was winning over older and younger conference-goers alike.
\fi

\ifdefined\chs
会议召开的日子一天天临近,两个开发阵营之间的紧张关系却变得更加明显。两个团体的开发者在一个问题上的观点是一致的:这次会议是他们第一次有机会当面见到芬兰神童托瓦兹。在会议上,托瓦兹出人意料的证明了自己是一个出色而又平易进人的演讲者。除了一点点轻微的瑞典口音,托瓦兹敏捷的思路和谦虚的才智让到场的听众感到惊讶。\endnote{尽管林纳斯·托瓦兹是芬兰人,但他的母语是瑞典语。``The Rampantly Unofficial Linus FAQ''(\url{http://catb.org/~esr/faqs/linus/})对此做出如下解释:

\begin{quote}
芬兰有很多(大约6\%)说瑞典语的少数人口。他们把自己叫作``\textit{finlandssvensk}''或``\textit{finladsvenskar}'',并且把他们自己认定是芬兰人。他们很多的家庭都在芬兰居住了很多年。瑞典语是芬兰两种官方语言中的一种。
\end{quote}

}雷蒙德说,更让人吃惊的是,托瓦兹在大会现场对其它杰出的黑客们进行了无情的抨击,包括所有黑客中最杰出的那位理查德·斯托曼。在大会的最后,托瓦兹的半黑客半懒鬼的作风赢得了新老两代与会人员的青睐。
\fi

\ifdefined\eng
``It was a pivotal moment,'' recalls Raymond. ``Before 1996, Richard was the only credible claimant to being the ideological leader of the entire culture. People who dissented didn't do so in public. The person who broke that taboo was Torvalds.''
\fi

\ifdefined\chs
``这是一个关键的转折点,''雷蒙德回忆道,``在1996年之前,理查德是整个自由软件文化的唯一一个权威的精神领袖。持有不同意见的人都不会在公众面前直接表达。托瓦兹是第一个打破这个禁忌的人。''
\fi

\ifdefined\eng
The ultimate breach of taboo would come near the end of the show. During a discussion on the growing market dominance of Microsoft Windows or some similar topic, Torvalds admitted to being a fan of Microsoft's PowerPoint slideshow software program. From the perspective of old-line software purists, it was like a Mormon bragging in church about his fondness of whiskey. From the perspective of Torvalds and his growing band of followers, it was simply common sense. Why shun worthy proprietary software programs just to make a point? Being a hacker wasn't about suffering, it was about getting the job done.
\fi

\ifdefined\chs
打破禁忌的最后一击出现在大会的最后。在一个有关微软Windows在市场中占有垄断地位的讨论中,托瓦兹承认他自己是微软的PowerPoint这个幻灯片制作软件的发烧友。从老一辈的自由软件纯化论者的眼光中,这就像是一个摩门教徒在教堂中宣扬自己对威士忌酒的热爱。从托瓦兹和他越来越多的追随者的观点来看,这并不奇怪。为什么为了表明自己观点就得刻意去回避私有软件呢?作为黑客,不是为了去忍受,而是为了更好地完成工作。
\fi

\ifdefined\eng
``That was a pretty shocking thing to say,'' Raymond remembers. ``Then again, he was able to do that, because by 1995 and 1996, he was rapidly acquiring clout.''
\fi

\ifdefined\chs
``这是一个振聋发聩的观点,''雷蒙德回忆说,``然而,他可以做到这一点,因为1995年和1996年,他正不断的接受打击。''
\fi

\ifdefined\eng
Stallman, for his part, doesn't remember any tension at the 1996 conference, but he does remember later feeling the sting of Torvalds' celebrated cheekiness. ``There was a thing in the Linux documentation which says print out the GNU coding standards and then tear them up,'' says Stallman, recalling one example. ``OK, so he disagrees with some of our conventions. That's fine, but he picked a singularly nasty way of saying so. He could have just said `Here's the way I think you should indent your code.' Fine. There should be no hostility there.''
\fi

\ifdefined\chs
但斯托曼表示他完全不记得1996年的大会上有什么紧张的气氛,只不过他记得在那次大会上,他被托瓦兹的厚脸皮刺激到了。``在Linux的文档中,有一段描述,让人们把GNU编码标准打印出来,然后再把它们撕烂。''斯托曼举例道。``好吧,我知道他对我们的一些做法持有反对意见。有不同意见是可以的,但是他选择了一种格外令人生厌的方式去表达这样的观点。他完全可以换个说法:`我觉得你可以使用另一种方式来缩进你的代码。'这样就不会有攻击性了。''
\fi

\ifdefined\eng
For Raymond, the warm reception other hackers gave to Torvalds' comments merely confirmed his suspicions. The dividing line separating Linux developers from GNU/Linux developers was largely generational. Many Linux hackers, like Torvalds, had grown up in a world of proprietary software. Unless a program was clearly inferior, most saw little reason to rail against a program on licensing issues alone. Somewhere in the universe of free software systems lurked a program that hackers might someday turn into a free software alternative to PowerPoint. Until then, why begrudge Microsoft the initiative of developing the program and reserving the rights to it?
\fi

\ifdefined\chs
对于雷蒙德来说,黑客们给予托瓦兹高度的评价和真诚的欢迎从另一个角度上证实了他的怀疑。Linux开发者与GNU/Linux开发者之间几乎存在着一个代沟。很多像托瓦兹那样的Linux黑客是在私有软件的世界中成长起来的。除非软件的质量确实很低下,否则没什么人会对软件的许可证问题发出抱怨。在自由软件的世界中,也许隐藏着一个PowerPoint的自由软件替代品。但是,在黑客们真正开始转向使用这个软件前,为什么要嫉妒微软开发了这样一个好用的私有软件呢?
\fi

\ifdefined\eng
As a former GNU Project member, Raymond sensed an added dynamic to the tension between Stallman and Torvalds. In the decade since launching the GNU Project, Stallman had built up a fearsome reputation as a programmer. He had also built up a reputation for intransigence both in terms of software design and people management. Shortly before the 1996 conference, the Free Software Foundation would experience a full-scale staff defection, blamed in large part on Stallman. Brian Youmans, a current FSF staffer hired by Salus in the wake of the resignations, recalls the scene: ``At one point, Peter [Salus] was the only staff member working in the office.''
\fi

\ifdefined\chs
雷蒙德曾经是一名GNU工程的成员,他能感觉到斯托曼与托瓦兹之间的紧张关系。在GNU工程开始的前十年中,斯托曼已经在程序员中建立起了威信。他也在软件设计和人员管理中建立了一种不妥协声誉。在1996年大会开如前不久,自由软件基金会遭遇了一次大面积的离职事件,事件的起因很大程度是因为斯托曼。布莱恩·尤曼斯(Brian Youmans),萨卢斯招聘进入自由软件基金会的现任职员回忆到:``在那时候,彼得·萨卢斯是留在办公室中的唯一职员。''
\fi

%% \ifdefined\eng
%% This was an example of the growing dispute, within the free software community, between those who valued freedom as such, and those who mainly valued powerful, reliable software.  Stallman referred to the two camps as political parties within the community, calling the former the ``freedom party.''  The supporters of the other camp did not try to name it, so Stallman disparagingly called it the ``bandwagon party'' or the ``success party,'' because many of them presented ``more
%% users'' as the primary goal.
%% \fi

%% \ifdefined\chs
%% .......................
%% \fi

%% \ifdefined\eng
%% In the decade since launching the GNU Project, Stallman had built up a fearsome reputation as a programmer. He had also built up a reputation for intransigence both in terms of software design and people management. This was partly true, but the reputation provided a convenient excuse that anyone could cite if Stallman did not do as he wished.  The reputation has been augmented by mistaken guesses.
%% \fi

%% \ifdefined\chs
%% 在GNU工程开始的前十年中,斯托曼已经在程序员中建立起了威信。他也在软件设计和人员管理中建立了一种不妥协声望。................
%% \fi

%% \ifdefined\eng
%% For example, shortly before the 1996 conference, the Free Software Foundation experienced a full-scale staff defection.  Brian Youmans, a current FSF staffer hired by Salus in the wake of the resignations, recalls the scene: ``At one point, Peter [Salus] was the only staff member working in the office.''  The previous staff were unhappy with the executive director; as Bryt Bradley told her friends in December, 1995:
%% \fi

%% \ifdefined\chs
%% 在1996年大会开始前不久,自由软件基金会遭遇了一次大面积的离职事件,事件的起因很大程度是因为斯托曼。Brain Youmans,Salus招聘的现任的FSF职员回忆到:``在那时候,Peter [Salus]是留在办公室中的唯一职员''。.........................
%% \fi

%% \ifdefined\eng
%% \begin{quote}
%% [name omitted] (the Executive Director of the FSF) decided to come back from Medical/Political Leave last week. The office staff (Gena Bean, Mike Drain, and myself) decided we could not work with her as our supervisor because of the many mistakes she had made in her job tasks prior to her taking a leave.  Also, there had been numerous instances where individuals were threatened with inappropriate firing and there  were many instances of what we felt were verbal abuse from her to ALL members of the office staff.  We requested (many times) that she not come back as our supervisor, but stated that we were willing to work with her as a co-worker.  Our requests were ignored.  We quit.
%% \end{quote}
%% \fi

%% \ifdefined\chs
%% ................
%% \fi

%% \ifdefined\eng
%% The executive director in question then gave Stallman an ultimatum: give her total autonomy in the office or she would quit.  Stallman, as president of the FSF, declined to give her total control over its activities, so she resigned, and he recruited in Peter Salus to replace her.
%% \fi

%% \ifdefined\chs
%% ........................
%% \fi

%% \ifdefined\eng
%% When Raymond, an outsider, learned that these people had left the FSF, he presumed Stallman was at fault.  This provided confirmation for his theory that Stallman's personality was the cause of any and all problems in the GNU Project.
%% \fi

%% \ifdefined\chs
%% .......................
%% \fi

%% \ifdefined\eng
%% Raymond had another theory: recent delays such as the Hurd and recent troubles such as the Lucid-Emacs schism reflected problems normally associated with software project management, not software code development.
%% \fi

%% \ifdefined\chs
%% ................
%% \fi

%% \ifdefined\eng
%% Shortly after the Freely Redistributable Software Conference, Raymond began working on his own pet software project, a mail utility called ``fetchmail.'' Taking a cue from Torvalds, Raymond issued his program with a tacked-on promise to update the source code as early and as often as possible. When users began sending in bug reports and feature suggestions, Raymond, at first anticipating a tangled mess, found the resulting software surprisingly sturdy. Analyzing the success of the Torvalds approach, Raymond issued a quick analysis: using the Internet as his ``petri dish'' and the harsh scrutiny of the hacker community as a form of natural selection, Torvalds had created an evolutionary model free of central planning.
%% \fi

%% \ifdefined\chs
%% ....................
%% \fi

\ifdefined\eng
For Raymond, the defection merely confirmed a growing suspicion: recent delays such as the HURD and recent troubles such as the Lucid-Emacs schism reflected problems normally associated with software project management, not software code development. Shortly after the Freely Redistributable Software Conference, Raymond began working on his own pet software project, a popmail utility called `` fetchmail.'' Taking a cue from Torvalds, Raymond issued his program with a tacked-on promise to update the source code as early and as often as possible. When users began sending in bug reports and feature suggestions, Raymond, at first anticipating a tangled mess, found the resulting software surprisingly sturdy. Analyzing the success of the Torvalds approach, Raymond issued a quick analysis: using the Internet as his ``petri dish'' and the harsh scrutiny of the hacker community as a form of natural selection, Torvalds had created an evolutionary model free of central planning.
\fi

\ifdefined\chs
对于雷蒙德来说,这次离职事件几乎肯定了一种猜测:最近HURD项目的进度的延迟和类似于Lucid-Emacs分裂的麻烦反映出来的是软件项目管理中的问题,而不是代码开发的问题。自由发布软件大会召开后不久,雷蒙德开始做他自己的玩具项目,一个名为fetchmail的邮件管理工具。根据托瓦兹提供的线索,雷蒙德在发布这个程序时附加上了一个承诺,承诺尽可能的及时更新源代码。当用户们开始提交问题报告和新需求时,雷蒙德发现程序并不向他起初时想的那样陷入泥潭,而是惊人的健壮。通过分析托瓦兹成功的模式,雷蒙德很快得出了一个结论:把Internet当成一个``培养皿'',并且把黑客社区简单的监督看成一个自然选择的过程,托瓦兹建立了一种革命性的模式,这种模式不依赖于集中的计划。
\fi

\ifdefined\eng
What's more, Raymond decided, Torvalds had found a way around Brooks' Law. First articulated by Fred P. Brooks, manager of IBM's OS/360 project and author of the 1975 book, \textit{The Mythical Man-Month}, Brooks' Law held that adding developers to a project only resulted in further project delays. Believing as most hackers that software, like soup, benefits from a limited number of cooks, Raymond sensed something revolutionary at work. In inviting more and more cooks into the kitchen, Torvalds had actually found a way to make the resulting software \textit{better}.\endnote{Brooks' Law is the shorthand summary of the following quote taken from Brooks' book:
\begin{quote}
Since software construction is inherently a systems effort -- an exercise in complex interrelationships -- communication effort is great, and it quickly dominates the decrease in individual task time brought about by partitioning. Adding more men then lengthens, not shortens, the schedule.
\end{quote}
\noindent See Fred P. Brooks, \textit{The Mythical Man-Month} (Addison Wesley Publishing, 1995).
}
\fi

\ifdefined\chs
另一方面,雷蒙德觉得,托瓦兹找到了一种突破``布鲁克斯法则''的方式。在IBM OS/360项目经理佛瑞德·P·布鲁克斯(Fred P. Brooks)于1975年出版的《\textit{人月神话}》一书中,布鲁克斯提出了``布鲁克斯法则'':向一个软件项目中增加开发者的做法只会造成项目的进一步延迟。对于黑客们来说,写软件就像做汤,增加厨师的人数对于改善汤的口味并没有太多作用。雷蒙德从中感觉到了一些革命性的变化:托瓦兹确实在聘请更多的厨师进入厨房的同时,也做出了\textit{更好}的软件。\endnote{``布鲁克斯法则''是对他的书中如下的描述的简称:
\begin{quote}
由于设计软件本质上是一个系统性的工作,沟通的成本很高,把任务分割成多个子任务会使沟通成本的大大增加。所以,向软件项目中增加人手只会增加软件开发时间,而不是缩短。
\end{quote}
\noindent 参考佛瑞德·布鲁克斯, 《\textit{人月神话}》(Addison Wesley Publishing, 1995)
}
\fi

\ifdefined\eng
Raymond put his observations on paper. He crafted them into a speech, which he promptly delivered before a group of friends and neighbors in Chester County, Pennsylvania. Dubbed `` The Cathedral and the Bazaar,'' the speech contrasted the management styles of the GNU Project with the management style of Torvalds and the kernel hackers. Raymond says the response was enthusiastic, but not nearly as enthusiastic as the one he received during the 1997 Linux Kongress, a gathering of Linux users in Germany the next spring.
\fi

\ifdefined\chs
雷蒙德把他的发现写成了论文,并把它设计成若干次演讲,在宾夕法尼亚州切斯特县的一些朋友和邻居们面前宣传。在一次名为《大教堂与集市》的演讲中,他对GNU工程和托瓦兹等内核黑客们所使用的管理风格进行了比较。雷蒙德说这些演讲的反响都很强烈,不过反响最好的一次,还得是次年春天他在德国召开的1997 Linux Kongress上的那次演讲。
\fi

\ifdefined\eng
``At the Kongress, they gave me a standing ovation at the end of the speech,'' Raymond recalls. ``I took that as significant for two reasons. For one thing, it meant they were excited by what they were hearing. For another thing, it meant they were excited even after hearing the speech delivered through a language barrier.''
\fi

\ifdefined\chs
``在Kongress上,演讲结束后听众都起身鼓掌,''雷蒙德回忆道,``我把这次成功归纳成两个因素。首先,这说明听众们为所听到的内容感到激动。其次,这意味着即使存在一些语言交流上的障碍,听众们依然感到兴奋。''
\fi

\ifdefined\eng
Eventually, Raymond would convert the speech into a paper, also titled ``The Cathedral and the Bazaar.'' The paper drew its name from Raymond's central analogy. GNU programs were ``cathedrals,'' impressive, centrally planned monuments to the hacker ethic, built to stand the test of time. Linux, on the other hand, was more like ``a great babbling bazaar,'' a software program developed through the loose decentralizing dynamics of the Internet.
\fi

\ifdefined\chs
最后,雷蒙德把这些演讲的内容写成了论文,同样起名为《大教堂与集市》。这篇论文的题目是对雷蒙德的核心思想的精确概括。GNU程序就是``教堂'',它们都是有计划地修建而成的宏伟的黑客精神的纪念碑,经得起时间的考验。而另一方面,Linux则更像是一个``嘈杂的大集市'',它是在Internet去中心化的松散组织结构中开发出来的。
\fi

%% \ifdefined\eng
%% Raymond's paper associated the Cathedral style, which he and Stallman and many others had used, specifically with the GNU Project and Stallman, thus casting the contrast between development models as a comparison between Stallman and Torvalds. Where Stallman was his chosen example of the classic cathedral architect -- i.e., a programming ``wizard'' who could disappear for 18 months and return with something like the GNU C Compiler -- Torvalds was more like a genial dinner-party host. In letting others lead the Linux design discussion and stepping in only when the entire table needed a referee, Torvalds had created a development model very much reflective of his own laid-back personality. From Torvalds' perspective, the most important managerial task was not imposing control but keeping the ideas flowing.
%% \fi


\ifdefined\eng
Implicit within each analogy was a comparison of Stallman and Torvalds. Where Stallman served as the classic model of the cathedral architect-i.e., a programming ``wizard'' who could disappear for 18 months and return with something like the GNU C Compiler-Torvalds was more like a genial dinner-party host. In letting others lead the Linux design discussion and stepping in only when the entire table needed a referee, Torvalds had created a development model very much reflective of his own laid-back personality. From the Torvalds' perspective, the most important managerial task was not imposing control but keeping the ideas flowing.
\fi

\ifdefined\chs
相似的类比同样也适用于斯托曼和托瓦兹。斯托曼代表着经典的教堂架构,他像一个程序魔法师一样,在消失了18个月以后,带来了类似于GNU C编译器这样的神作。而托瓦兹则像是一个亲切的宴会主人。他让别人来领导Linux的设计讨论,并只在关键时候提出一些参考建议。托瓦兹所创建的这种开发模式正好反映了的懒散的个性。从托瓦兹的观点来看,最好的管理工作不是要加强对事情的控制,而是要保持思维的活跃度。
\fi

\ifdefined\eng
Summarized Raymond, ``I think Linus's cleverest and most consequential hack was not the construction of the Linux kernel itself, but rather his invention of the Linux development model.''\endnote{See Eric Raymond, ``The Cathedral and the Bazaar'' (1997).}
\fi

\ifdefined\chs
雷蒙德总结说:``我觉得林纳斯最聪明的地方和最重要的黑客作品就他是创造出来Linux开发模式,而不仅仅是他所创造的Linux内核本身。''\endnote{参考埃里克·雷蒙德的文章``大教堂与集市'',1997年。}
\fi

%% \ifdefined\eng
%% If the paper's description of these two styles of development was perceptive, its association of the Cathedral model specifically with Stallman (rather than all the others who had used it, including Raymond himself) was sheer calumny.  In fact, the developers of some GNU packages including the GNU Hurd had read about and adopted Torvalds' methods before Raymond tried them, though without analyzing them further and publicly championing them as Raymond's paper did. Thousands of hackers, reading Raymond's article, must have been led to a negative attitude towards GNU by this smear.
%% \fi

%% \ifdefined\chs

%% \fi

\ifdefined\eng
In summarizing the secrets of Torvalds' managerial success, Raymond himself had pulled off a coup. One of the audience members at the Linux Kongress was Tim O'Reilly, publisher of O'Reilly \& Associates, a company specializing in software manuals and software-related books (and the publisher of this book). After hearing Raymond's Kongress speech, O'Reilly promptly invited Raymond to deliver it again at the company's inaugural Perl Conference later that year in Monterey, California.
%% In summarizing the secrets of Torvalds' managerial success, Raymond attracted the attention of other members of the free software community for whom freedom was not a priority.  They sought to interest business in the use and development of free software, and to do so, decided to cast the issue in terms of the values that appeal to business: powerful, reliable, cheap, advanced.  Raymond became the best-known proponent of these ideas, and they reached the management of Netscape, whose proprietary browser was losing market share to Microsoft's equally proprietary Internet Explorer.  Intrigued by a speech by Raymond, Netscape executives took the message back to corporate headquarters. A few months later, in January, 1998, the company announced its plan to publish the source code of its flagship Navigator web browser in the hopes of enlisting hacker support in future development.
\fi

\ifdefined\chs
在总结托瓦兹的成功管理经验方面,雷蒙德自己也完成的非常出色。提姆·奥莱理(Tim O'Reilly)当时也在Linux Kongress的会场聆听演讲,他是O'Reilly \& Associates出版公司的创始人,这家公司专注于出版软件使用手册和软件相关的书籍(同时也是本书英文版的出版商)。在听完了雷蒙德在Kongress上的演讲后,奥莱理马上就邀请雷蒙德参加下半年在加州蒙特雷举办的首届Perl大会,请他在那次会议上再次分享这个主题。
%% 在总结托瓦兹的成功管理经验方面,雷蒙德做自己也完成得的非常出色。.................跟预期的情况一样,听众并不仅仅被黑客所震惊,也有很多人对自由软件运动的快速成长表达了强烈的兴趣。Netscap就是其中的一个,。这家成立于加州山景城的创业公司,刚刚经历一了一场场持续了三年的与微软争夺Web浏览器的市场的争夺。出于对雷蒙德演讲的好奇和希望赢回失去的市场份额的期望目的,Netscape的执行官们居们把这次演讲的信息带回了公司总部。几个月以后,1998年1月,Netscape宣布计划把它的旗舰级产品Navigator Web浏览器的代码公开,希望能得到黑客们的支持,完成以后的开发。...............
\fi

\ifdefined\eng
Although the conference was supposed to focus on Perl, a scripting language created by Unix hacker Larry Wall, O'Reilly assured Raymond that the conference would address other free software technologies. Given the growing commercial interest in Linux and Apache, a popular free software web server, O'Reilly hoped to use the event to publicize the role of free software in creating the entire infrastructure of the Internet. From web-friendly languages such as Perl and Python to back-room programs such as BIND (the Berkeley Internet Naming Daemon), a software tool that lets users replace arcane IP numbers with the easy-to-remember domain-name addresses (e.g., amazon.com), and sendmail, the most popular mail program on the Internet, free software had become an emergent phenomenon. Like a colony of ants creating a beautiful nest one grain of sand at a time, the only thing missing was the communal self-awareness. O'Reilly saw Raymond's speech as a good way to inspire that self-awareness, to drive home the point that free software development didn't start and end with the GNU Project. Programming languages, such as Perl and Python, and Internet software, such as BIND, sendmail, and Apache, demonstrated that free software was already ubiquitous and influential. He also assured Raymond an even warmer reception than the one at Linux Kongress.
\fi

\ifdefined\chs
虽然这次会议的主题是Perl,一种由著名的Unix黑客拉里·华尔(Larry Wall)设计的脚本语言。奥莱理承诺雷蒙德这次会议也会讨论其它自由软件的技术。那时商业界已经开始关注Linux和Apache,Apache是一种流行的自由Web服务器软件,奥莱理希望通过这次会议可以让自由软件更加流行,并用自由软件来建立一套完整的Internet的基础设施。从Perl或Python这些Web友好的开发语言到BIND(the Berkeley Internet Naming Daemon)这样后台的服务(用于把类似于amazon.com这样的容易记忆的域名转换为数字型式的IP地址),还有sendmail这样的流行的Internet电子邮件服务器程序,自由软件正创造着一个又一个的业界奇迹。自由软件社区存在的唯一问题就是缺少对整个社区的自我认识,总是像蚂蚁部落一样各自为营,在沙子里建造一个又一个的窝。奥莱理觉得雷蒙德的演讲可以引发人们对自由软件的认识,自由软件并不等同于GNU工程,它不会因为GNU工程的开始而诞生,同样也不会因为GNU工程的结束而消亡。Perl和Python这样的开发语言,BIND、sendmail和Apache这样的Internet软件已经证明了自由软件已经无处不在,并具有强大的影响力。奥莱理还承诺雷蒙德,在这次会议上的演讲会比在Linux Kongress上的演讲更加轰动。
\fi

\ifdefined\eng
O'Reilly was right. ``This time, I got the standing ovation before the speech,'' says Raymond, laughing.
\fi

\ifdefined\chs
奥莱理是对的。雷蒙德笑了笑,说:``这次,我的演讲获得了听众的起身鼓掌。''
\fi

\ifdefined\eng
As predicted, the audience was stocked not only with hackers, but with other people interested in the growing power of the free software movement. One contingent included a group from Netscape, the Mountain View, California startup then nearing the end game of its three-year battle with Microsoft for control of the web-browser market.
\fi

\ifdefined\chs
跟预期的情况一样,听众并不仅仅被黑客所震惊,也有很多人对自由软件运动的快速成长表达了强烈的兴趣。网景公司(Netscape)就是其中的一个。这家成立于加州山景城的创业公司刚刚经历了一场持续了三年的与微软争夺Web浏览器的市场的斗争。
\fi

\ifdefined\eng
Intrigued by Raymond's speech and anxious to win back lost market share, Netscape executives took the message back to corporate headquarters. A few months later, in January, 1998, the company announced its plan to publish the source code of its flagship Navigator web browser in the hopes of enlisting hacker support in future development.
\fi

\ifdefined\chs
出于对雷蒙德演讲的好奇和希望赢回失去的市场份额的目的,网景公司的高管们把这次演讲所传达的信息带回了公司总部。几个月以后,1998年1月,网景宣布计划开放它的旗舰级产品Navigator Web浏览器的源代码,以此希望能得到黑客们的支持,继续完成后续的开发。
\fi

\ifdefined\eng
When Netscape CEO Jim Barksdale cited Raymond's ``Cathedral and the Bazaar'' essay as a major influence upon the company's decision, the company instantly elevated Raymond to the level of hacker celebrity. Determined not to squander the opportunity, Raymond traveled west to deliver interviews, advise Netscape executives, and take part in the eventual party celebrating the publication of Netscape Navigator's source code. The code name for Navigator's source code was ``Mozilla'': a reference both to the program's gargantuan size-30 million lines of code-and to its heritage. Developed as a proprietary offshoot of Mosaic, the web browser created by Marc Andreessen at the University of Illinois, Mozilla was proof, yet again, that when it came to building new programs, most programmers preferred to borrow on older, modifiable programs.
\fi

\ifdefined\chs
网景公司的首席执行官吉姆·巴克斯代尔(Jim Barksdale)说:雷蒙德那篇``大教堂与集市''对于公司做出这样的决策有着深远的影响,公司很快把雷蒙德地位抬升到了黑客界的知名人士。为了紧紧抓住这次机会,雷蒙德出差去西海岸去进行采访,向网景的高管们提供建议,并且参与了开放Netscape Navigator源代码的庆祝聚会。Navigator的源代码的code name是``Mozilla'',这个名字一方面象征着这个程序的源代码有30万行的代码,同时也象征着它的尊贵血统。作为一个浏览器鼻祖Mosaic的商业化版本,马克·安德生(Marc Andreessen)在伊利诺伊大学开发Navigator浏览器时也参考了Mosaic的设计,这就再次证明了大部分程序员在打算开发一个新的程序时,总是会借鉴一些现有的、可以修改的程序。
\fi

%% \ifdefined\eng
%% When Netscape CEO Jim Barksdale cited Raymond's ``Cathedral and the Bazaar'' essay as a major influence upon the company's decision, the company instantly elevated Raymond to the level of hacker celebrity. He invited a few people including Larry Augustin, founder of VA Research which sold workstations with the GNU/Linux operating system preinstalled; Tim O'Reilly,  founder of the publisher O'Reilly \& Associates; and Christine Peterson, president of the Foresight Institute, a Silicon Valley think tank specializing in nanotechnology, to talk. ``The meeting's agenda boiled down to one item: how to take advantage of Netscape's decision so that other companies might follow suit?''
%% \fi

%% \ifdefined\chs
%% Netscape CEO Jim Barksdale说雷蒙德的``大教堂与集市''文章对于公司做出这样的决策有着深远的影响,公司很快把雷蒙德地位抬升到了黑客界的知名人士。为了紧紧抓住这次机会,雷蒙德出差去西海岸去进行采访,向Netscape的执行官们提供建议,并且参与了Netscape Navigator源代码开放庆祝的聚会。Navigator的源代码的code name是``Mozilla'',这个名字一方面象征着这个程序的源代码有30万行的代码,同时也象征着它的血统。作为一个Mosaic浏览器的非商业化版本,Marc Adressen在Illionis大学开发这个浏览器时也参考了Mosaic的设计,这就再次证明了大部分程序员在打算开发一个新的程序时,总是会借鉴一些现有的、可以修改的程序。..................Tim O'Reilly当时也在Linux Kongress的会场聆听演讲,他是O'Reilly \& Associcates出版公司的创始人,这家公司专注于出版软件使用手册和软件相关的书籍(同样也是本书的出版商)。听完了雷蒙德在Kongress上的演讲后,O'Reilly迅速地邀请雷蒙德参加下半年在加州Monterey举办的首届Perl大会,在会议上再次分享这个主题。............
%% \fi

\ifdefined\eng
While in California, Raymond also managed to squeeze in a visit to VA Research, a Santa Clara-based company selling workstations with the GNU/Linux operating system preinstalled. Convened by Raymond, the meeting was small. The invite list included VA founder Larry Augustin, a few VA employees, and Christine Peterson, president of the Foresight Institute, a Silicon Valley think tank specializing in nanotechnology.
\fi

\ifdefined\chs
雷蒙德在加州的那段时间,还找到了一次去VA Research参观的机会。VA Research公司坐落于圣克拉拉市,专职销售一些预装GNU/Linux操作系统的工作站。雷蒙德召集了一次小规模的会议,仅仅邀请了VA的创始人拉里·奥古斯丁(Larry Augustin)、少数几名VA的雇员以及克里斯汀·彼得森(Christine Peterson),他是硅谷Foresight Institude的主席,他们专注纳米科技。
\fi

\ifdefined\eng
``The meeting's agenda boiled down to one item: how to take advantage of Netscape's decision so that other companies might follow suit?'' Raymond doesn't recall the conversation that took place, but he does remember the first complaint addressed. Despite the best efforts of Stallman and other hackers to remind people that the word ``free'' in free software stood for freedom and not price, the message still wasn't getting through. Most business executives, upon hearing the term for the first time, interpreted the word as synonymous with ``zero cost,'' tuning out any follow up messages in short order. Until hackers found a way to get past this cognitive dissonance, the free software movement faced an uphill climb, even after Netscape.
\fi

\ifdefined\chs
``这次会议的议程最终归结为一点:如何来总结网景公司的决策,以便让别的公司可以去效仿?''雷蒙德不记得这次讨论的内容,但是他记得这次会议中提到的另一个问题:不管斯托曼和其他黑客们如何努力的去提醒人们自由软件中的``Free''一词是指``自由''而不是``免费'',这一点仍然没有赢得大众普遍的认知。很多商业管理者第一次听到这个名词时,都把它当然是``零成本''的同义词,而并不注意到其它相关的信息。除非黑客们能找到一种方法解决这个认知上的偏差,否则即使有网景公司作为榜样,自由软件运动想要获得最终成功,还是得面临一个困难的登顶过程。
\fi

%% \ifdefined\eng
%% Raymond doesn't recall the conversation that took place, but he does remember the first complaint addressed. Despite the best efforts of Stallman and other hackers to remind people that the word ``free'' in free software stood for freedom and not price, the message still wasn't getting through. Most business executives, upon hearing the term for the first time, interpreted the word as synonymous with ``zero cost,'' tuning out any follow-up messages in short order. Until hackers found a way to get past this misunderstanding, the free software movement faced an uphill climb, even after Netscape.
%% \fi

%% \ifdefined\chs
%% 雷蒙德不记得这次讨论的内容,但是他记得这次会议中的提到的另一个问题。不管斯托曼和其他黑客如何努力的去提醒人们自由软件中的``Free''一词是指``自由''而不是``免费'',但这一点仍然没有得到大众的真正认知。第一次听到这个名词时,许多商业管理者都把它当作是``零成本''的同义词,而并不注意它蕴含着的其它相关信息。除非黑客们能找到一种方法解决这个认识上的偏差,否则自由软件运动就会面临一个高耸难攀的山峰,即使有Netscape的案例供参考,也很难一举登顶。
%% \fi

\ifdefined\eng
Peterson, whose organization had taken an active interest in advancing the free software cause, offered an alternative: open source.
\fi

\ifdefined\chs
于是,彼德森成立了一个组织,专门致力于推广自由软件,他们使用了另一个词语:``开源''。
\fi

\ifdefined\eng
Looking back, Peterson says she came up with the open source term while discussing Netscape's decision with a friend in the public relations industry. She doesn't remember where she came upon the term or if she borrowed it from another field, but she does remember her friend disliking the term.\endnote{See Malcolm Maclachlan, ``Profit Motive Splits Open Source Movement,'' \textit{TechWeb News} (August 26, 1998), \url{http://www.techweb.com/article/showArticle?articleID=29102344}.}
\fi

\ifdefined\chs
彼德森回忆道,她是在与一个朋友讨论网景公司所作的决定的时候,想到了``开源''这个词语。她不记得是她自己首创了这个词语,还是从别的什么领域借用了这个词语,但是她确实记得她的朋友并不喜欢这个词语。\endnote{参考马尔科姆·麦克拉克伦(Malcolm Maclachlan)1998年8月26日发表在\textit{TechWeb News}网站的文章 ``利益驱使导致开源运动的分裂''一文。\url{http://www.techweb.com/article/showArticle?articleID=29102344}}
\fi

\ifdefined\eng
At the meeting, Peterson says, the response was dramatically different. ``I was hesitant about suggesting it,'' Peterson recalls. ``I had no standing with the group, so started using it casually, not highlighting it as a new term.'' To Peterson's surprise, the term caught on. By the end of the meeting, most of the attendees, including Raymond, seemed pleased by it.
\fi

\ifdefined\chs
彼德森回忆说,在那次会议上,这个词语得到的反馈却是截然不同的。她说:``我很犹豫要不要提出这个建议,因为我对这个组织并不了解,所以我故意在不经意中去使用这个词语,而不是把它作为一个新词语去强调。''然而,这个词语出乎意料的受到了人们的关注。在这次会议结束时,大部分的与会者,包括雷蒙德,对于这个词语都非常满意。
\fi

\ifdefined\eng
Raymond says he didn't publicly use the term ``open source'' as a substitute for free software until a day or two after the Mozilla launch party, when O'Reilly had scheduled a meeting to talk about free software. Calling his meeting ``the Freeware Summit,'' O'Reilly says he wanted to direct media and community attention to the other deserving projects that had also encouraged Netscape to release Mozilla. ``All these guys had so much in common, and I was surprised they didn't all know each other,'' says O'Reilly. ``I also wanted to let the world know just how great an impact the free software culture had already made. People were missing out on a large part of the free software tradition.''
\fi

\ifdefined\chs
Mozilla发布会前几天,奥莱理组织了一次讨论自由软件的会议。雷蒙德说,在这次会议之前,他从来不公开使用``开源''这个词语来替代``自由软件''。奥莱理把这次会议叫做``免费软件峰会'',期望把媒体和社区的注意力吸引到其它一些有价值的项目上去,因为这些项目对于网景公司决定发布Mozilla也起到了积极的作用。``参会的那些家伙有很多相似之处,但我很惊讶他们之前互相之间都不认识。''奥莱理说,``我同时也想让全世界知道自由软件文化已经产生了如此深远的影响。人们如果再不对自由软件引起重视,就是错过了一次很好的机会。''
\fi

\ifdefined\eng
In putting together the invite list, however, O'Reilly made a decision that would have long-term political consequences. He decided to limit the list to west-coast developers such as Wall, Eric Allman, creator of sendmail, and Paul Vixie, creator of BIND. There were exceptions, of course: Pennsylvania-resident Raymond, who was already in town thanks to the Mozilla launch, earned a quick invite. So did Virginia-resident Guido van Rossum, creator of Python. ``Frank Willison, my editor in chief and champion of Python within the company, invited him without first checking in with me,'' O'Reilly recalls. ``I was happy to have him there, but when I started, it really was just a local gathering.''
\fi

\ifdefined\chs
邀请这些人聚集在一起,是奥莱理一个深思熟虑的决定,并将产生深远的政治影响。他原本只打算邀请西海岸的软件开发者们,比如华尔、sendmail的作者埃里克·奥尔曼(Eric Allman),BIND的作者保罗·维克西(Paul Vixie)。不过也有一个人是例外,那就是长居在宾夕法尼亚州的雷蒙德,因为他正好来参加Mozilla的发布会,所以也马上邀请了他。同时还有来自弗吉尼亚州的吉多·范罗苏姆(Guido van Rossum),他是Python语言的作者。``弗兰克·威利森(Frank Willison)是出版社的主编,他是Python的斗士,他决定邀请吉多参加,甚至都没有跟我商量。'',奥莱理回忆说,``我非常高兴能邀请到他出席,不过刚开始的时候,我确实只是想组织一个本地聚会。''
\fi

\ifdefined\eng
For some observers, the unwillingness to include Stallman's name on the list qualified as a snub. ``I decided not to go to the event because of it,'' says Perens, remembering the summit. Raymond, who did go, says he argued for Stallman's inclusion to no avail. The snub rumor gained additional strength from the fact that O'Reilly, the event's host, had feuded publicly with Stallman over the issue of software-manual copyrights. Prior to the meeting, Stallman had argued that free software manuals should be as freely copyable and modifiable as free software programs. O'Reilly, meanwhile, argued that a value-added market for nonfree books increased the utility of free software by making it more accessible to a wider community. The two had also disputed the title of the event, with Stallman insisting on ``Free Software'' over the less politically laden ``Freeware.''
\fi

\ifdefined\chs
从一些观察者的视角来看,没有邀请斯托曼参加这次会议可以算是有点怠慢。``正是由于这个原因,我决定不去参加这次会议。''佩伦斯回忆这次峰会时回忆道。雷蒙德参加了这次会议,他却认为即使邀请斯托曼也是徒劳的。有关这次会议的组织者奥莱理怠慢斯托曼的传言,在奥莱理与斯托曼公开表明在软件手册的权限问题上的分歧后,变得更能让人信服。在会议前,斯托曼曾表示,自由软件手册应该可以像自由软件本身一样自由的复制和修改。而奥莱理那时认为非自由的图书是对自由软件的增值,它们能让自由软件被更多的社区所熟悉。奥莱理和斯托曼还为这次会议的名称发生了争论,斯托曼坚持要使用``自由软件''来代替不那么政治化的``免费软件''一词。
\fi

\ifdefined\eng
Looking back, O'Reilly doesn't see the decision to leave Stallman's name off the invite list as a snub. ``At that time, I had never met Richard in person, but in our email interactions, he'd been inflexible and unwilling to engage in dialogue. I wanted to make sure the GNU tradition was represented at the meeting, so I invited John Gilmore and Michael Tiemann, whom I knew personally, and whom I knew were passionate about the value of the GPL but seemed more willing to engage in a frank back-and-forth about the strengths and weaknesses of the various free software projects and traditions. Given all the later brouhaha, I do wish I'd invited Richard as well, but I certainly don't think that my failure to do so should be interpreted as a lack of respect for the GNU Project or for Richard personally.''
\fi

\ifdefined\chs
回头去看,奥莱理认为没有邀请斯托曼的决定并不是因为怠慢了他。奥莱理说:``在那个时候,我没有见过理查德·斯托曼本人,不过在我们的电子邮件沟通中,斯托曼看上去是个很强势的人,并且不太愿意加入对话。我希望能够保证在会议中展示出GNU的传统,所以我邀请了约翰·吉尔摩(John Gilmore)和迈克尔·蒂曼(Michael Tiemann),这两个人我都比较熟悉,我知道他们对于GPL的价值都非常有热情,不过看上去他们更愿意参与有关自由软件项目和传统的优势和弱点的讨论。从后面发生的事情来看,我有点后悔没有邀请理查德,但是我没有这么做并不能看成是对GNU工程或理查德·斯托曼本人的不敬。''
\fi

\ifdefined\eng
Snub or no snub, both O'Reilly and Raymond say the term ``open source'' won over just enough summit-goers to qualify as a success. The attendees shared ideas and experiences and brainstormed on how to improve free software's image. Of key concern was how to point out the successes of free software, particularly in the realm of Internet infrastructure, as opposed to playing up the GNU/Linux challenge to Microsoft Windows. But like the earlier meeting at VA, the discussion soon turned to the problems associated with the term ``free software.'' O'Reilly, the summit host, remembers a particularly insightful comment from Torvalds, a summit attendee.
\fi

\ifdefined\chs
无论是不是怠慢了斯托曼,奥莱理和雷蒙德回忆说,在那次峰会上,``开源''一词成功的赢得了很多与会者的支持。与会者分享了各种想法和经验,并且一起进行了一场头脑风暴,讨论如何提升自由软件的形象。讨论的焦点是如何才能展示自由软件的成功之处,尤其是在实现Internet的基础架构方面,而不是讨论GNU/Linux是如何与微软Windows进行竞争的问题。不过跟先前在VA召开的那次会议一样,这些讨论的内容在不经意间就转向了讨论``自由软件''这个词自身所存在的问题。作为会议的主办者,奥莱理记得峰会的与会者托瓦兹的一段很有见地评论:
\fi

\ifdefined\eng
``Linus had just moved to Silicon Valley at that point, and he explained how only recently that he had learned that the word `free' had two meanings -- free as in `libre' and free as in `gratis' -- in English.''
\fi

\ifdefined\chs
``那时候,林纳斯才刚刚搬到硅谷来,他解释说其实他本人也是最近才知道`Free'这个词在英文中有两个意思,一个意思是自由,另一个意思是免费''。\endnote{译注:原文在这里用了两个西班牙语词汇,libre和gratis。}
\fi

\ifdefined\eng
Michael Tiemann, founder of Cygnus, proposed an alternative to the troublesome ``free software'' term: sourceware. ``Nobody got too excited about it,'' O'Reilly recalls. ``That's when Eric threw out the term `open source.'\hspace{0.01in}''
\fi

\ifdefined\chs
Cygnus的创始人迈克尔·蒂曼建议用sourceware一词来代替容易引起误解的``自由软件''一词。``不过大部分人都对这个词没什么兴趣。''奥莱理回忆到,``直到埃里克抛出了`开源'一词。''
\fi

\ifdefined\eng
Although the term appealed to some, support for a change in official terminology was far from unanimous. At the end of the one-day conference, attendees put the three terms -- free software, open source, or sourceware -- to a vote. According to O'Reilly, 9 out of the 15 attendees voted for ``open source.'' Although some still quibbled with the term, all attendees agreed to use it in future discussions with the press. ``We wanted to go out with a solidarity message,'' O'Reilly says.
\fi

\ifdefined\chs
虽然``开源''这个词对一些人很有吸引力,但是要让它取代``自由软件''变为一个官方的用词还有很长的路要走。结束一天的会议前,与会者开始对三个词语进行投票,``自由软件''、``开源''和``sourceware''。奥莱理说,15名与会者中有9人把票投给了``开源''。虽然有一部分人对这个词还有些意见,不过所有的与会者都同意在以后与媒体的讨论中可以使用这个词汇。奥莱理说,``我们希望能向外界发出一些团结的信号''。
\fi

\ifdefined\eng
The term didn't take long to enter the national lexicon. Shortly after the summit, O'Reilly shepherded summit attendees to a press conference attended by reporters from the \textit{New York Times}, the \textit{Wall Street Journal}, and other prominent publications. Within a few months, Torvalds' face was appearing on the cover of \textit{Forbes} magazine, with the faces of Stallman, Perl creator Larry Wall, and Apache team leader Brian Behlendorf featured in the interior spread. Open source was open for business.
\fi

\ifdefined\chs
没过多久,``开源''这个词就进入了词典。这次峰会后不久,奥莱理带着与会者们参加了一个有\textit{纽约时报}、\textit{华尔街日报}和其它著名出版社记者参与的发布会。在短短几个月中,托瓦兹就登上了\textit{福布斯}杂志的封面,正文中还出现了斯托曼、Perl语言的作者拉里·华尔和Apache团队领导者布莱恩·贝伦多夫(Brian Behlendorf)的照片。开源开始向商业市场敞开怀抱。
\fi

\ifdefined\eng
For summit attendees such as Tiemann, the solidarity message was the most important thing. Although his company had achieved a fair amount of success selling free software tools and services, he sensed the difficulty other programmers and entrepreneurs faced.
\fi

\ifdefined\chs
对于像蒂曼这样的参会者来说,向外界发出团结的信息才是最重要的事情。尽管他的公司通过销售自由软件和相关服务已经捞到了第一桶金,他还是可以感觉到很多其它的程序员和企业在面对自由软件时还是觉得困难重重。
\fi

\ifdefined\eng
``There's no question that the use of the word free was confusing in a lot of situations,'' Tiemann says. ``Open source positioned itself as being business friendly and business sensible. Free software positioned itself as morally righteous. For better or worse we figured it was more advantageous to align with the open source crowd.‘’
\fi

\ifdefined\chs
蒂曼说:``毫无疑问,使用Free这个词在很多场合都会造成误解。开源这个词则对于商业市场来说更加友好,也更容易引起商业市场的注意。自由软件则更是关注于一种道德上的正义。不管好不好,我们都觉得与开源团队合作是有益无害的。''
\fi

\ifdefined\eng
For Stallman, the response to the new ``open source'' term was slow in coming. Raymond says Stallman briefly considered adopting the term, only to discard it. ``I know because I had direct personal conversations about it,'' Raymond says.
\fi

\ifdefined\chs
斯托曼没有马上对``开源''这个新词汇做出回应。雷蒙德说斯托曼只是简单对这个词汇进行了一下了解,其实目的只是为了找个理由抛弃它。``我知道这一点是因为我跟斯托曼就这个问题有过直接的对话。''
\fi

\ifdefined\eng
By the end of 1998, Stallman had formulated a position: open source, while helpful in communicating the technical advantages of free software, also encouraged speakers to soft-pedal the issue of software freedom. Given this drawback, Stallman would stick with the term free software.
\fi

\ifdefined\chs
1998年年底,斯托曼正式表明了立场:在讨论自由软件技术优点时,``开源''是一个很有利于沟通的词汇,但是它也诱使演讲者弱化软件自由的重要性。由于存在这样的不足,斯托曼将坚持继续使用``自由软件''这个词汇。
\fi

\ifdefined\eng
Summing up his position at the 1999 LinuxWorld Convention and Expo, an event billed by Torvalds himself as a ``coming out party'' for the Linux community, Stallman implored his fellow hackers to resist the lure of easy compromise.
\fi

\ifdefined\chs
1999年召开的LinuxWorld博览会,托瓦兹把它看成是宣传Linux社区产出的大好机会,在这次会议上,斯托曼总结了他的立场,他恳求他的追随者抵抗住诱惑。
\fi

\ifdefined\eng
``Because we've shown how much we can do, we don't have to be desperate to work with companies or compromise our goals,'' Stallman said during a panel discussion. ``Let them offer and we'll accept. We don't have to change what we're doing to get them to help us. You can take a single step towards a goal, then another and then more and more and you'll actually reach your goal. Or, you can take a half measure that means you don't ever take another step and you'll never get there.''
\fi

\ifdefined\chs
斯托曼在一次小组讨论上说:``我们已经展示了很多我们已经取得的成就,所以我们不需要为了去讨好公司而放弃我们的目标。如果某些公司能向我们提供帮助,我们可以接受。如果他们不愿意,我们也没有必要去改变自己来争取这些帮助。只要你能不断向着目标慢慢前进,最终一定能实现目标。如果你们什么事情都采用折中的方式,不真正往前前进,就永远也不能实现目标。''
\fi

\ifdefined\eng
Even before the LinuxWorld show, however, Stallman was showing an increased willingness to alienate his more conciliatory peers. A few months after the Freeware Summit, O'Reilly hosted its second annual Perl Conference. This time around, Stallman was in attendance. During a panel discussion lauding IBM's decision to employ the free software Apache web server in its commercial offerings, Stallman, taking advantage of an audience microphone, disrupted the proceedings with a tirade against panelist John Ousterhout, creator of the Tcl scripting language. Stallman branded Ousterhout a ``parasite'' on the free software community for marketing a proprietary version of Tcl via Ousterhout's startup company, Scriptics. ``I don't think Scriptics is necessary for the continued existence of Tcl,'' Stallman said to hisses from the fellow audience members.\endnote{See Malcolm Maclachlan, ``Profit Motive Splits Open Source Movement,'' \textit{TechWeb News} (August 26, 1998), \url{http://www.techweb.com/article/showArticle?articleID=29102344}.}
\fi

\ifdefined\chs
事实上,在LinuxWorld大会之前,斯托曼已经表现出他正在慢慢疏远他的同盟。免费软件峰会召开后几个月,奥莱理主办了他的第二次年度Perl大会。这一次,斯托曼参加了会议。在一次有关IBM决定在它们的商用方案中整合Apache Web服务器这个自由软件的小组讨论中,斯托曼抢过了听众的话筒,把矛头指向了与会者之一的约翰·欧斯特霍特(John Ousterhout),他是Tcl脚本语言的作者。斯托曼把欧斯特霍特称作是自由软件社区的寄生虫,因为他通过他自己开办的名为Scriptics的创业公司销售一个商业版本的Tcl。``我不觉得Scriptics是Tcl继续存活下去所必须的。''斯托曼在听众们的窃窃私语中表示。\endnote{参考马尔科姆·麦克拉克伦1998年8月26日发表在\textit{TechWeb News}网站的文章 ``利益驱使导致开源运动的分裂''一文。\url{http://www.techweb.com/article/showArticle?articleID=29102344}}
\fi

\ifdefined\eng
``It was a pretty ugly scene,'' recalls Prime Time Freeware's Rich Morin. ``John's done some pretty respectable things: Tcl, Tk, Sprite. He's a real contributor.''
\fi

\ifdefined\chs
``当时的场面非常尴尬,''Prime Time Freeware公司的李奇·莫林(Rich Morin)回忆说。``约翰做出了一些很让人值得尊敬的东西:Tcl,Tk和Sprite。他是自由软件社区中一个真正的贡献者。''
\fi

\ifdefined\eng
Despite his sympathies for Stallman and Stallman's position, Morin felt empathy for those troubled by Stallman's discordant behavior.
\fi

\ifdefined\chs
莫林很理解斯托曼以及他的立场,但他也同样也理解那些被斯托曼无理的行为扫兴的人们。
\fi

\ifdefined\eng
Stallman's Perl Conference outburst would momentarily chase off another potential sympathizer, Bruce Perens. In 1998, Eric Raymond proposed launching the Open Source Initiative, or OSI, an organization that would police the use of the term ``open source'' and provide a definition for companies interested in making their own programs. Raymond recruited Perens to draft the definition.\endnote{See Bruce Perens et al., ``The Open Source Definition,'' The Open Source Initiative (1998). \url{http://www.opensource.org/docs/definition.html}}
\fi

\ifdefined\chs
斯托曼在Perl大会上的爆发也影响了另一位潜在的追随者,布鲁斯·佩伦斯(Bruce Perens)。1998年,埃里克·雷蒙德打算发起开源倡议组织,OSI,一个监管``开源''这个词语使用并向感兴趣的公司提供标准的组织。雷蒙德请佩伦斯来起划这个标准。\endnote{参考布鲁斯·佩伦斯等人的``开源定义'',来自开源倡议组织,1998年。\url{http://www.opensource.org/docs/definition.html}}
\fi

%% \ifdefined\eng
%% Raymond called Stallman after the meeting to tell him about the new term ``open source'' and ask if he would use it.  Raymond says Stallman briefly considered adopting the term, only to discard it. ``I know because I had direct personal conversations about it,'' Raymond says.
%% \fi

%% \ifdefined\chs
%% 斯托曼没有马上对``开源''这个新词语作出回应。雷蒙德说斯托曼简单了解了一个这个词语,其实只是为了抛弃它。``我知道这一点是因为我跟斯托曼就这个问题有过直接的对话。''
%% \fi

%% \ifdefined\eng
%% Stallman's immediate response was, ``I'll have to think about it.''  The following day he had concluded that the values of Raymond and O'Reilly would surely dominate the future discourse of ``open source,'' and that the best way to keep the ideas of the free software movement in public view was to stick to its traditional term.
%% \fi

%% \ifdefined\chs
%% 。。。。。。。。。。。。。。。。。。。
%% \fi

%% \ifdefined\eng
%% Later in 1998, Stallman announced his position: ``open source,'' while helpful in communicating the technical advantages of free software also encouraged speakers to soft-pedal the issue of software freedom. It avoided the unintended meaning of ``gratis software'' and the intended meaning of ``freedom-respecting software'' equally.  As a means for conveying the latter meaning, it was therefore no use.  In effect, Raymond and O'Reilly had given a name to the nonidealistic political party in the community, the one Stallman did not agree with.
%% \fi

%% \ifdefined\chs
%% 到了1998年年底,斯托曼正式表明了立场:开源这个词在讨论自由软件技术优点时,是一个在沟通时便于使用的词语,但是它也诱使演讲者弱化软件自由的重要性。。。。。。。。。。。。。。。由于存在这样的不足,斯托曼坚持使用``自由软件''这个词汇。
%% \fi

%% \ifdefined\eng
%% In addition, Stallman thought that the ideas of ``open source'' led people to put too much emphasis on winning the support of business.  While such support in itself wasn't necessarily bad in itself, he expected that being too desperate for it would lead to harmful compromises.  ``Negotiation 101 would teach you that if you are desperate to get someone's agreement, you are asking for a bad deal,'' he says.  ``You need to be prepared to say no.''  Summing up his position at the 1999 LinuxWorld Convention and Expo, an event billed by Torvalds himself as a ``coming out party'' for the ``Linux'' community, Stallman implored his fellow hackers to resist the lure of easy compromise.
%% \fi

%% \ifdefined\chs
%% TODO。。。。。。。1999年LinuxWorld对话与博览会,托瓦兹把它看成是Linux社区产出的宣传会,在这次会议上,斯托曼总结了他的立场,他恳求他的追随者抵抗住诱惑。
%% \fi

%% \ifdefined\eng
%% ``Because we've shown how much we can do, we don't have to be desperate to work with companies or compromise our goals,'' Stallman said during a panel discussion. ``Let them offer and we'll accept. We don't have to change what we're doing to get them to help us. You can take a single step towards a goal, then another and then more and more and you'll actually reach your goal. Or, you can take a half measure that means you don't ever take another step, and you'll never get there.''
%% \fi

%% \ifdefined\chs
%% 斯托曼在一次小组讨论上说:``我们已经展示了很多我们已经取得的成就,所以我们不需要去讨好公司而放弃我们的目标。如果公司能向我们提供帮助,我们可以接受。如果公司不愿意,我们也没有必要去改变自己来争取这些帮助。只要你能不断向着目标慢慢前进,最终一定能实现目标。如果你们什么事情都采用折中的方式,不真正往前前进,就永远也不能实现目标。''
%% \fi

%% \ifdefined\eng
%% Even before the LinuxWorld show, however, Stallman was showing an increased willingness to alienate open source supporters. A few months after the Freeware Summit, O'Reilly hosted its second annual Perl Conference. This time around, Stallman was in attendance. During a panel discussion lauding IBM's decision to employ the free software Apache web server in its commercial offerings, Stallman, taking advantage of an audience microphone, made a sharp denunciation of panelist John Ousterhout, creator of the Tcl scripting language. Stallman branded Ousterhout a ``parasite'' on the free software community for marketing a proprietary version of Tcl via Ousterhout's startup company, Scriptics.  Ousterhout had stated that Scriptics would contribute only the barest minimum of its improvements to the free version of Tcl, meaning it would in effect use that small contribution to win community approval for much a larger amount of nonfree software development.  Stallman rejected this position and denounced Scriptics' plans. ``I don't think Scriptics is necessary for the continued existence of Tcl,'' Stallman said to hisses from the fellow audience members.\endnote{\textit{Ibid.}}
%% \fi

%% \ifdefined\chs
%% 事实上,在LinuWorld之前,斯托曼已经表现出他正在慢慢疏远他的同盟。Freeware峰会后几个月,O'Reilly的主办了他的第二次年度Perl大会。这一次,斯托曼参加了会议。在一次称赞IBM决定在他的商用方案中整合Apache Web服务器这个自由软件的小组讨论中,斯托曼抢过了听众的话筒,把矛头指向了讨论人John Ousterhout,他是Tcl脚本语言的作者。斯托曼把Ousterhout称为是自由软件社区的寄生虫,因为他通过他自己的名为Scriptics的创业公司稍售一个商业版本的Tcl。``我不觉得Scriptics是Tcl继续存活下去所必须的。''斯托曼在听众们的窃窃私语中坚定地表明道。
%% \fi

%% \ifdefined\eng
%% ``It was a pretty ugly scene,'' recalls Prime Time Freeware's Rich Morin. ``John's done some pretty respectable things: Tcl, Tk, Sprite. He's a real contributor.''  Despite his sympathies for Stallman and Stallman's position, Morin felt empathy for those troubled by Stallman's discordant words.
%% \fi

%% \ifdefined\chs
%% ``这是一种很尴尬的场面。'' Prime TIme Freeware的李奇·莫林回忆说,``John做出了一些很让人值得尊敬的东西——Tcl,Tk和Sprite。他是一个真正的贡献者''。莫林很理解斯托曼以及他的立场,但他也同样也理解那些被斯托曼``无理''的行为扫了兴的人们。
%% \fi

%% \ifdefined\eng
%% Stallman will not apologize.  ``Criticizing proprietary software isn't ugly -- proprietary software is ugly.  Ousterhout had indeed made real contributions in the past, but the point is that Scriptics was going to be nearly 100\% a proprietary software company.  In that conference, standing up for freedom meant disagreeing with nearly everyone. Speaking from the audience, I could only say a few sentences.  The only way to raise the issue so it would not be immediately forgotten was to put it in strong terms.''
%% \fi

%% \ifdefined\chs
%% TODO...............
%% \fi

%% \ifdefined\eng
%% ``If people rebuke me for `making a scene' when I state a serious criticism of someone's conduct, while calling Torvalds `cheeky' for saying nastier things about trivial matters, that seems like a double standard to me.''
%% \fi

%% \ifdefined\chs
%% TODO..............
%% \fi

%% \ifdefined\eng
%% Stallman's controversial criticism of Ousterhout momentarily alienated a potential sympathizer, Bruce Perens. In 1998, Eric Raymond proposed launching the Open Source Initiative, or OSI, an organization that would police the use of the term ``open source'' and provide a definition for companies interested in making their own programs. Raymond recruited Perens to draft the definition.\endnote{See Bruce Perens et al., ``The Open Source Definition,'' The Open Source Initiative (1998), \url{http://www.opensource.org/docs/definition.html}.}
%% \fi

%% \ifdefined\chs
%% 斯托曼在Perl大会上的爆发,影响了另一位潜在的追随者——布鲁斯·佩伦斯。1998年,埃里克·雷蒙德打算发行开源倡议组织,OSI,一个监管``开源''这个词语使用并向感兴趣的公司提供标准的组织。雷蒙德请佩伦斯来企起划这个标准。 \endnote{See Bruce Perens et al., ``The Open Source Definition,'' The Open Source Initiative (1998), \url{http://www.opensource.org/docs/definition.html}.}
%% \fi

\ifdefined\eng
Perens would later resign from the OSI, expressing regret that the organization had set itself up in opposition to Stallman and the FSF. Still, looking back on the need for a free software definition outside the Free Software Foundation's auspices, Perens understands why other hackers might still feel the need for distance. ``I really like and admire Richard,'' says Perens. ``I do think Richard would do his job better if Richard had more balance. That includes going away from free software for a couple of months.''
\fi

\ifdefined\chs
佩伦斯不久就从OSI的高管位置上辞职,他对这个组织与斯托曼和自由软件基金会对立的立场表示遗憾。回忆往事,从在自由软件基金会以外重新定义自由软件的需要来看,佩伦斯理解了为什么其它的黑客们仍然觉得需要与之保持距离。``我非常敬仰理查德,''佩伦斯说,``我觉得如果他能更为平衡的处理各方关系的话,他会把事情完成得更好。我觉得他应该花几个月时间远离自由软件,静下来好好思考一下。''
\fi

\ifdefined\eng
Stallman's monomaniacal energies would do little to counteract the public-relations momentum of open source proponents. In August of 1998, when chip-maker Intel purchased a stake in GNU/Linux vendor Red Hat, an accompanying \textit{New York Times} article described the company as the product of a movement ``known alternatively as free software and open source.''\endnote{See Amy Harmon, ``For Sale: Free Operating System,'' \textit{New York Times} (September 28, 1998), \url{http://www.nytimes.com/library/tech/98/09/biztech/articles/28linux.html}.} Six months later, a John Markoff article on Apple Computer was proclaiming the company's adoption of the ``open source'' Apache server in the article headline.\endnote{See John Markoff, ``Apple Adopts `Open Source' for its Server Computers,'' \textit{New York Times} (March 17, 1999), \url{http://www.nytimes.com/library/tech/99/03/biztech/articles/17apple.html}.}
\fi

\ifdefined\chs
斯托曼的个人偏执并不足以抵消与开源斗士对商业界的公关势头。1998年8月,芯片厂商Intel购买了GNU/Linux供应商红帽公司的股份,\textit{纽约时报}在随后的文章中评价红帽公司为``自由软件和开源''运动的产物。\endnote{参考艾米·哈蒙(Amy Harmon)``大减价:自由操作系统'',\textit{纽约时报},1998年9月28日。\url{http://www.nytimes.com/library/tech/98/09/biztech/articles/28linux.html}} 半年后,约翰·马尔科夫(John Markoff)在\textit{苹果电脑}杂志上发表文章,标题中称该公司开始使用``开源''的Apache服务器。\endnote{参考约翰·马尔科夫``苹果在其服务器上使用`开源'软件'',\textit{纽约时报},1999年3月17日。\url{http://www.nytimes.com/library/tech/99/03/biztech/articles/17apple.html}}
\fi

\ifdefined\eng
Such momentum would coincide with the growing momentum of companies that actively embraced the ``open source'' term. By August of 1999, Red Hat, a company that now eagerly billed itself as ``open source,'' was selling shares on Nasdaq. In December, VA Linux -- formerly VA Research -- was floating its own IPO to historical effect. Opening at \$30 per share, the company's stock price exploded past the \$300 mark in initial trading only to settle back down to the \$239 level. Shareholders lucky enough to get in at the bottom and stay until the end experienced a 698\% increase in paper wealth, a Nasdaq record.
\fi

\ifdefined\chs
这样的势头与很多公司拥抱``开源''一词的势头非常一致。到了1999年8月,红帽这家乐意把自己标榜为``开源''的公司在纳斯达克上市。同年12月,VA Linux公司(曾经的VA Research)正式IPO。这家公司的股票价格从开盘的每股30美元一路彪升到300美元,并最后回稳在239美元左右。股东们很幸运的抄了底,并且享受到了698\%的纸面收益,打破了纳斯达克的历史记录。
\fi

\ifdefined\eng
Among those lucky shareholders was Eric Raymond, who, as a company board member since the Mozilla launch, had received 150,000 shares of VA Linux stock. Stunned by the realization that his essay contrasting the Stallman-Torvalds managerial styles had netted him \$36 million in potential wealth, Raymond penned a follow-up essay. In it, Raymond mused on the relationship between the hacker ethic and monetary wealth:
\fi

\ifdefined\chs
埃里克·雷蒙德也是这些幸运的股东们中的一员,作为从Mozilla发布开始,他就是VA Linux公司董事会成员,他收获了150000股的VA Linux股票。雷蒙德那篇对比斯托曼和托瓦兹的管理风格的文章给他的净资产增值了36万美元,这让他自己也觉得很震惊,于是他又写了一篇文章,在这篇文章中,雷蒙德思考了黑客道德与货币财富之间的关系:
\fi

\begin{quote}
\ifdefined\eng
Reporters often ask me these days if I think the open-source community will be corrupted by the influx of big money. I tell them what I believe, which is this: commercial demand for programmers has been so intense for so long that anyone who can be seriously distracted by money is already gone. Our community has been self-selected for caring about other things-accomplishment, pride, artistic passion, and each other.\endnote{See Eric Raymond, ``Surprised by Wealth,'' \textit{Linux Today} (December 10, 1999). \url{http://linuxtoday.com/news_story.php3?ltsn=1999-12-10-001-05-NW-LF}}
\fi

\ifdefined\chs
最近记者们经常问我是否认为开源社区会因为大量资金的注入而被破坏。我对他们说,我所相信的是商业公司对程序员的需求一直非常旺盛,所以如果任何一个只是渴求金钱的人应该早就离开了社区。我们的社区是自我选择的结果,人们所关注的是其它的东西,比如:成就感、骄傲、艺术热情和彼此之间的关系。\endnote{参考埃里克·雷蒙德``被财富所震惊'',\textit{今日Linux},1999年12月10日。\url{http://linuxtoday.com/news_story.php3?ltsn=1999-12-10-001-05-NW-LF}}
\fi
\end{quote}

\ifdefined\eng
Whether or not such comments allayed suspicions that Raymond and other open source proponents had simply been in it for the money, they drove home the open source community's ultimate message: all you needed to sell the free software concept is a friendly face and a sensible message. Instead of fighting the marketplace head-on as Stallman had done, Raymond, Torvalds, and other new leaders of the hacker community had adopted a more relaxed approach-ignoring the marketplace in some areas, leveraging it in others. Instead of playing the role of high-school outcasts, they had played the game of celebrity, magnifying their power in the process.
\fi

\ifdefined\chs
不管这些评论是否减轻了雷蒙德和其它开源拥护者的行为是以金钱为目的嫌疑,但是他们向开源社区传达了一个最基本的信息:想要宣传自由软件的概念,只需要一张友善的面孔和合乎情理的信号。不用像斯托曼那样正面与商业市场斗争,雷蒙德、托瓦兹和其它黑客社区的新领导者们需要一种更宽松的方式:在某些领域忽略市场,通过别的方式去影响它。不能扮演被学校开除的那个角色,而应该成为社会的名流,在这个过程中不断扩大自己的影响力。
\fi

%% \ifdefined\eng
%% The open source proponents' message was simple: all you need, to sell the free software concept, is to make it business-friendly.  They saw Stallman and the free software movement as fighting the market; they sought instead to leverage it. Instead of playing the role of high-school outcasts, they had played the game of celebrity, magnifying their power in the process.
%% \fi

%% \ifdefined\chs
%% ..................
%% \fi

%% \ifdefined\eng
%% These methods won great success for open source, but not for the ideals of free software.  What they had done to ``spread the message'' was to omit the most important part of it: the idea of freedom as an ethical issue.  The effects of this omission are visible today: as of 2009, nearly all GNU/Linux distributions include proprietary programs, Torvalds' version of Linux contains proprietary firmware programs, and the company formerly called VA Linux bases its business on proprietary software.  Over half of all the world's web servers run some version of Apache, and the usual version of Apache is free software, but many of those sites run a proprietary modified version distributed by IBM.
%% \fi

%% \ifdefined\chs
%% 不管这些评论是否减轻了雷蒙德和其它开源拥护者是为金钱为目的嫌疑,但是他们向开源社区传达了一个最基本的信息:想要宣传自由软件的概念,只需要一张友善的面孔和合乎情理的信号。不用像斯托曼那样正面与商业市场斗争,雷蒙德、托瓦兹和其它黑客社区的新领导者们需要一种更宽松的方式:在某些领域忽略市场,通过别的方式去影响它。不能扮演被学校开除的那个角色,而应该成为社会的名流、在这个过程中不断扩大自己的影响力。
%% \fi

\ifdefined\eng
``On his worst days Richard believes that Linus Torvalds and I conspired to hijack his revolution,'' Raymond says. ``Richard's rejection of the term open source and his deliberate creation of an ideological fissure in my view comes from an odd mix of idealism and territoriality. There are people out there who think it's all Richard's personal ego. I don't believe that. It's more that he so personally associates himself with the free software idea that he sees any threat to that as a threat to himself.''
\fi

\ifdefined\chs
雷蒙德说:``在理查德最不如意的日子里,他认为我和林纳斯·托瓦兹串通搞阴谋绑架他的革命。理查德拒绝使用开源一词,并且故意认为我的观点是意识形态上的分歧,这是由于他自己的理想主义和领地主义的混合的结果。从外人的眼光看来,可能会以为这是理查德的个人野心膨胀的结果。但我不这样认为。我觉得这不仅仅是野心,这更是因为理查德把自己和自由软件的理念紧紧的绑在了一起,在他看来,任何对自由软件的威胁就是对他本人的威胁。''
\fi

%% \ifdefined\eng
%% Stallman responds, ``Raymond misrepresents my views: I don't think Torvalds `conspired' with anyone, since being sneaky is not his way. However, Raymond's nasty conduct is visible in those statements themselves.   Rather than respond to my views (even as he claims they are) on their merits, he proposes psychological interpretations for them. He attributes the harshest interpretation to unnamed others, then `defends' me by proposing a slightly less derogatory one.  He has often `defended' me this way.''
%% \fi

%% \ifdefined\chs
%% .............
%% \fi

\ifdefined\eng
Ironically, the success of open source and open source advocates such as Raymond would not diminish Stallman's role as a leader. If anything, it gave Stallman new followers to convert. Still, the Raymond territoriality charge is a damning one. There are numerous instances of Stallman sticking to his guns more out of habit than out of principle: his initial dismissal of the Linux kernel, for example, and his current unwillingness as a political figure to venture outside the realm of software issues.
%% Ironically, the success of open source and open source advocates such as Raymond would not diminish Stallman's role as a leader -- but it would lead many to misunderstand what he is a leader of.  Since the free software movement lacks the corporate and media recognition of open source, most users of GNU/Linux do not hear that it exists, let alone what its views are.  They have heard the ideas and values of open source, and they never imagine that Stallman might have different ones.  Thus he receives messages thanking him for his advocacy of ``open source,'' and explains in response that he has never been a supporter of that, using the occasion to inform the sender about free software.
\fi

\ifdefined\chs
让人觉得讽刺的是,开源和像雷蒙德这样的开源拥护者的胜利并不能撼动斯托曼作为领导者的角色。斯托曼的一举一动都有机会给他带来新的追随者。而雷蒙德则还没有到达这样的高度。斯托曼时常手持武器,充满攻击性。但这不是因为出于原则,而更多是出于习惯:比如,他一开始反对Linux内核,现在则是作为一个自由软件运动的政治形象考虑,不愿意在自由软件理想以外领域去冒险。
\fi

%% \ifdefined\eng
%% Some writers recognize the term ``free software'' by using the term ``FLOSS,'' which stands for ``Free/Libre and Open Source Software.''  However, they often say there is a single ``FLOSS'' movement, which is like saying that the U.S. has a ``Liberal/Conservative'' movement, and the views they usually associate with this supposed single movement are the open source views they have heard.
%% \fi

%% \ifdefined\chs
%% TODO...............
%% \fi

\ifdefined\eng
Then again, as the recent debate over open source also shows, in instances when Stallman has stuck to his guns, he's usually found a way to gain ground because of it. ``One of Stallman's primary character traits is the fact he doesn't budge,'' says Ian Murdock. ``He'll wait up to a decade for people to come around to his point of view if that's what it takes.''
%%Despite all these obstacles, the free software movement does make its ideas heard sometimes, and continues to grow in absolute terms.  By sticking to its guns, and presenting its ideas in contrast to those of open source, it gains ground. ``One of Stallman's primary character traits is the fact he doesn't budge,'' says Ian Murdock. ``He'll wait up to a decade for people to come around to his point of view if that's what it takes.''
\fi

\ifdefined\chs
除此之外,近期关于开源的争论也表明:当斯托曼手持武器时,通常是他找到了一种可以让他的事业有所进展的方式的时候。伊恩·默多克(Ian Murdock)说:``斯托曼最重要的一个性格特征就是他不会动摇自己的立场。如果需要,为了让别人接受他的观点,他可以等上十年。''
\fi

\ifdefined\eng
Murdock, for one, finds that unbudgeable nature both refreshing and valuable. Stallman may no longer be the solitary leader of the free software movement, but he is still the polestar of the free software community. ``You always know that he's going to be consistent in his views,'' Murdock says. ``Most people aren't like that. Whether you agree with him or not, you really have to respect that.''
\fi

\ifdefined\chs
默多克所发现的斯托曼这个不动摇的特质既新鲜又宝贵。就算斯托曼不再是自由软件运动的唯一领袖,他仍然是自由软件社区的北极星。默多克说:``你可以坚信他永远会坚持他自己的观点,虽然大部分人都不是这样的。不管你是否同意他的观点,你都必须尊重他的观点。'' 因为事实常常证明:他站得更高,看得更远。
\fi

\theendnotes
\setcounter{endnote}{0}
