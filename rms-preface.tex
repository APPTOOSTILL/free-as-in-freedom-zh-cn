%% Copyright (c) 2002, 2010 Sam Williams
%% Copyright (c) 2010 Richard M. Stallman
%% Permission is granted to copy, distribute and/or modify this
%% document under the terms of the GNU Free Documentation License,
%% Version 1.3 or any later version published by the Free Software
%% Foundation; with no Invariant Sections, no Front-Cover Texts, and
%% no Back-Cover Texts. A copy of the license is included in the
%% file called ``gfdl.tex''.
\chapter[Foreword by Richard M. Stallman]{\ifdefined\eng
Foreword\\by Richard M. Stallman
\fi
\ifdefined\chs
理查德·斯托曼的前言
\fi}

\ifdefined\eng
I have aimed to make this edition combine the advantages of my
knowledge and Williams' interviews and outside viewpoint.  The reader
can judge to what extent I have achieved this.
\fi

\ifdefined\chs

\fi

\ifdefined\eng
I read the published text of the English edition for the first time in
2009 when I was asked to assist in making a French translation of \textit{Free
as in Freedom}.  It called for more than small changes.
\fi

\ifdefined\chs

\fi

\ifdefined\eng
Many facts needed correction, but deeper changes were also needed.
Williams, a non-programmer, blurred fundamental technical and legal
distinctions, such as that between modifying an existing program's
code, on the one hand, and implementing some of its ideas in a new
program, on the other.  Thus, the first edition said that both Gosmacs
and GNU Emacs were developed by modifying the original PDP-10 Emacs,
which in fact neither one was.  Likewise, it mistakenly
described Linux as a ``version of Minix.''  SCO later made the same
false claim in its infamous lawsuit against IBM, and both Torvalds and
Tanenbaum rebutted it.
\fi

\ifdefined\chs

\fi

\ifdefined\eng
The first edition overdramatized many events by projecting spurious
emotions into them.  For instance, it said that I ``all but shunned''
Linux in 1992, and then made a ``a dramatic about-face'' by deciding in
1993 to sponsor Debian GNU/Linux.  Both my interest in 1993 and my
lack of interest in 1992 were pragmatic means to pursue the same end:
to complete the GNU system.  The launch of the GNU Hurd kernel in 1990
was also a pragmatic move directed at that same end.
\fi

\ifdefined\chs

\fi

\ifdefined\eng
For all these reasons, many statements in the original edition were
mistaken or incoherent.  It was necessary to correct them, but not
straightforward to do so with integrity short of a total rewrite,
which was undesirable for other reasons.  Using explicit notes for the
corrections was suggested, but in most chapters the amount of change
made explicit notes prohibitive.  Some errors were too pervasive or
too ingrained to be corrected by notes.  Inline or footnotes for the
rest would have overwhelmed the text in some places and made the text
hard to read; footnotes would have been skipped by readers tired of
looking down for them.  I have therefore made corrections directly in
the text.
\fi

\ifdefined\chs

\fi

\ifdefined\eng
However, I have not tried to check all the facts and quotations that
are outside my knowledge; most of those I have simply carried forward
on Williams' authority.
\fi

\ifdefined\chs

\fi

\ifdefined\eng
Williams' version contained many quotations that are critical of me.  I
have preserved all these, adding rebuttals when appropriate.  I have
not deleted any quotation, except in \autoref{chapter:open source} where I have deleted
some that were about open source and did not pertain to my life or
work.  Likewise I have preserved (and sometimes commented on) most of
Williams' own interpretations that criticized me, when they did not
represent misunderstanding of facts or technology, but I have freely
corrected inaccurate assertions about my work and my thoughts and
feelings.  I have preserved his personal impressions when presented as
such, and ``I'' in the text of this edition always refers to Williams
except in notes labeled ``RMS:''.
\fi

\ifdefined\chs

\fi

\ifdefined\eng
In this edition, the complete system that combines GNU and Linux is
always ``GNU/Linux,'' and ``Linux'' by itself always refers to Torvalds'
kernel, except in quotations where the other usage is marked with
``[\textit{sic}]''.  See \url{http://www.gnu.org/gnu/gnu-linux-faq.html} for more
explanation of why it is erroneous and unfair to call the whole system
``Linux.''
\fi

\ifdefined\chs

\fi

\ifdefined\eng
I would like to thank John Sullivan for his many useful criticisms and
suggestions.
\fi

\ifdefined\chs

\fi
