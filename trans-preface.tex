%% Copyright (c) 2014 Li Fanxi <lifanxi@freemindworld.com>
%% Permission is granted to copy, distribute and/or modify this
%% document under the terms of the GNU Free Documentation License,
%% Version 1.3 or any later version published by the Free Software
%% Foundation; with no Invariant Sections, no Front-Cover Texts, and
%% no Back-Cover Texts. A copy of the license is included in the
%% file called ``gfdl.tex''.
\chapter{译者序一\ 若为自由故}
\thispagestyle{empty}

\ifdefined\nov
``自由''这个词,有人崇尚,有人厌恶,还有人打心眼里对它咬牙切齿。而还有那么一小撮人,绞尽脑汁地要诉说着什么是自由,该如何保卫自由。而理查德·斯托曼(Richard Stallman)是这样一个另类:他定义了``软件用户的自由'',并且身体力行地为自己的自由理念宣传,在计算机技术领域掀起了一场``自由''运动。
\fi

\ifdefined\spv
``自由''这个词,有人崇尚,有人厌恶,还有人打心眼里对它咬牙切齿。而还有那么一小撮,绞尽脑汁地要诉说着什么是自由,该怎么保卫自由。理查德·斯托曼(Richard Stallman)就是这样一个另类:他在``言论自由'',``集会自由''这样的普世价值之外,定义了``软件用户的自由'',并且身体力行地为自己的自由理念宣传,在计算机技术领域掀起了一场政治运动。
\fi

瞎折腾——恐怕不少人会如此评价斯托曼。

\ifdefined\nov
如今,计算机与网络的普及改变了人们的生活方式。然而,技术的便利却没能带来意想中的地球村。用户面前不是一个信息流畅的互联网络,而是军阀割据的战乱天地。在利益与专制面前,自由再一次沦陷。费劲周折,人们重拾起自由这面大旗。此时才发现,斯托曼早已在这个战场上为大家奋战多年。
\fi

\ifdefined\spv
随着计算机软件和网络的普及,不少问题却逐渐显现:各大厂商开始毫无节制地收集用户数据,甚至把用户花钱购买的计算机当作战场,宣泄对竞争对手的不满;网络上的言论被莫名其妙地删除,而言论本身甚至也能招来血光之灾……在利益与专制面前,自由再一次沦陷。费劲周折,人们重拾起自由这面大旗。此时才发现,斯托曼早已在这个战场上为大家奋战多年。
\fi

在我看来,理查德·斯托曼注定会成为一位俄狄浦斯式的悲剧英雄。他固执,有原则,拒绝妥协。而这也就注定了他难以如愿。他追求着一份乌托邦式的理想,也恰恰是这个理想本身铸就了他的几分睿智、几分悲凉。作为自由软件运动的发起人,他大可从容地把同根通源的``开源运动''视作晚辈后生,在当今众多公司投身开源的热潮里,摘取本该属于他的果实。然而,他却认为开源运动强调的是软件开发模式,忽略了软件用户的自由。尽管绝大部分开源软件许可证也是自由软件许可证,尽管没几个人真的在乎``自由软件''和``开源软件''究竟有什么区别,他还是拒绝与开源运动有任何瓜葛。人们诉说着自由软件的成果,分享着GNU/Linux、Apache基金会的成功故事,斯托曼则在旁边不断泼冷水:自由软件运动远没有结束,我们离目标依旧很远。

正如科学永远要与误解相伴一样,自由也总有被私利与强权侵蚀的可能。追求自由如同追求真理,永无止境。

这本斯托曼的传记为人们描绘了一位计算机界的奇才。然而,与那些硅谷创业故事不同,这里没有一夜暴富的上市传奇,没有商场战场的尔虞我诈,更不会看到主人公的圆满谢幕。如今的理查德·斯托曼,依旧参与着自由软件的开发,依旧在世界各地演讲,宣传他的自由软件运动;也依旧会遭受人们的误解与不屑,被视作异类。斯托曼的执着与固执令他难见成功,也成就了他的传奇。

本书写成于21世纪初期,书中描写的一些环境当下早已不再:如今,自由/开源社区蓬勃发展,甚至硅谷的各大公司也纷纷抛来橄榄枝。自由软件与开源软件早已不是商业公司的敌人,反而是众多企业的得力助手,甚至是利润来源。哪怕是曾经自由软件运动的宿敌——微软公司,也投身到自由/开源软件的开发中,甚至公开了自家公司产品的代码。这一切的发展,都始于斯托曼当年的一份宣言。而其中原委,只有读过之后才能品味。

本书原题为Free as in Freedom。本意是强调Free一词中的``自由''意项。对应的中文翻译则让我们犯了难。为此,我们曾在GNU官方网站的中文翻译组征求意见,最终选定了半句匈牙利诗人裴多菲·山多尔的``若为自由顾''来作为本书的标题。裴多菲曾领导了1848年爆发的匈牙利革命,并为此献出了自己26岁的生命。尽管斯托曼不一定会赞同裴多菲的政治理念,但那份为了争取自由而拼进全力的执着,却是两位领袖的共通之处。

非常荣幸能与凡希合译本书。他翻译了第9章以后的内容,并完成了绝大部分的校对工作。是他的热情,让我这个习惯拖延的人追上进度。感谢人民邮电出版社陈冀康编辑的帮助,是他的耐心才让这本书最终得以出版。也感谢哲思自由软件社区,特别是徐继哲先生对本书的支持。

与很多翻译作品一样,本书也难免存在疏漏。但作为一本``自由''图书,读者可以参与改进,我们也会根据读者的建议更新本书的电子版。

\bigskip

\rightline{邓楠}

\rightline{2014年5月于山景城}

\chapter{译者序二\ 自由自在}
\thispagestyle{empty}

1999年,我第一次尝试安装GNU/Linux,这也应该是我第一次接触``自由软件''。这次尝试的结果就是让我损失了硬盘上积累的自从1994年第一次接触电脑以来所有的资料。虽然遭遇了这样的不幸,却仍然磨灭不了我的好奇心。1999年到2006年期间,我屡战屡败地安装过各种各样的GNU/Linux发行版,但最终还是选择了GNU/Linux作为我日常使用的操作系统。

2004年,我第一次接触Emacs编辑器。Emacs实在是个难以上手的家伙,好不容易通过Emacs内置的教程学到一点皮毛后,我就决定尝试把这份教程翻译成中文,造福后人。我向Emacs的邮件列表发了一封热情洋溢的邮件,咨询是否可以由我来完成翻译工作。很快收到了两封回信,说CVS中已经有人完成了教程的中文翻译,会随着下一个版本的Emacs发布,并欢迎我对现有的译文进行审阅。

在继续学习Emacs的过程中,我终于认识了GNU工程,也才真正明白了什么是``自由软件''。后来我加入了GNU官方网站的中文翻译小组,在翻译GNU网站的各种页面和文章的过程中,对自由软件、GPL以及自由软件运动发起人理查德·斯托曼有了进一步的认识。

2008年5月24日,我参加了在上海复旦大学举行的``哲思自由软件峰会'',第一次与斯托曼有了面对面的交流。除了他所使用的OLPC XO-1笔记本电脑让我垂涎欲滴外,他在现场所表现出来的强烈个性更是给我留下了深刻的印象。会后,我请他在某本图书上签名的请求被他毫不留情地拒绝了,``这不是一本自由图书,我不能在上面签名。''斯托曼的解释非常干脆。

2011年的某一天,我在无意中发现,七年前回复我有关Emacs教程翻译问题的人之一就是斯托曼本人。我与斯托曼的这两次直接交流,得到了截然不同的反馈:对于向自由软件社区贡献自己力量的行为,斯托曼会给予热情的支持和鼓励;对于非自由的软件和图书作品,斯托曼的抵触毫不留情。

在GNU官方网站的中文翻译小组中,我和邓楠一度是比较活跃的两位成员。作为翻译小组的协调员,邓楠需要负责所有翻译成果的发布工作。我们的合作非常愉快,在2010年左右完成了GNU网站大部分二级页面的翻译工作。

所以,当邓楠邀请我跟他一起完成本书的翻译工作时,我马上就答应了。虽然这只是一本薄薄的小册子,但是对于我这样习惯于技术言语的人来说,要把这么多词句组织得不是那么干涩,还是有着不小的挑战。中间的曲折故事暂且不表,经过近一年的努力,我们终于还是完成了这本书的翻译工作。

交稿前夕,正好斯托曼要来中国进行几场巡回演讲,我有幸在杭州与他进行了一次更深层次的交流。谈到本书的出版工作,斯托曼本人表现出了极大的兴趣,甚至希望我们能立刻把这本书印出来在他接下来的巡回演讲中进行分发。同时,他也迫切地希望我们能够继续翻译出版他自己修订过的本书第二版的中文版。

由于本书英文版出版时间较早,书中所引用的很多网址如今都已经失效。不过在翻译过程中我们还是保留了很多失效的网址,因为即使对应的网页现在已经无法正常访问,但从网址本身也许仍可以窥探出一些有用的信息。

感谢人民邮电出版社陈冀康编辑在本书出版过程给予的帮助,正如本书作者在跋中所写的那样,在传统出版领域出版这样一本``自由''图书还是面临着不少的挑战。陈冀康编辑的耐心和执着,最终促成了本书。感谢好友贾征主动帮忙审阅了全书的内容,也感谢家人对我翻译工作的支持。

尤其值得一提的是,这本书使用的是GFDL许可证,也就是说,这是市面上极为少有的一本``自由''图书。在GFDL条款的保护下,读者可以自由地复制、分发和修改本书。我们已经把全书的\LaTeX源文件公开发布到网上,希望通过我们的努力,抛砖引玉,让更多的人可以参与到了解、完善和传播本书内容的活动中。请访问\url{http://www.freemindworld.com/faif}获取本书的电子版本,并欢迎读者参与到本书第二版的翻译工作中来。通过读者与社区的共同努力,希望能在第一版的基础上进一步精雕细琢,完成一部至瑧完美的作品。今后如果你有机会见到斯托曼本人,不妨带上本书,他一定会非常乐意在这本记录他人生故事的``自由图书''上签名的。

\bigskip

\rightline{李凡希}

\rightline{2014年5月于杭州}





