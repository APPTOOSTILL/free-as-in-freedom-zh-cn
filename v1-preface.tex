%% Copyright (c) 2002 Sam Williams
%% Permission is granted to copy, distribute and/or modify this
%% document under the terms of the GNU Free Documentation License,
%% Version 1.3 or any later version published by the Free Software
%% Foundation; with no Invariant Sections, no Front-Cover Texts, and
%% no Back-Cover Texts. A copy of the license is included in the
%% file called ``gfdl.tex''.
\chapter{\ifdefined\eng
Preface
\fi
\ifdefined\chs
序言
\fi}

\ifdefined\eng
The work of Richard M. Stallman literally speaks for itself. From the documented source code to the published papers to the recorded speeches, few people have expressed as much willingness to lay their thoughts and their work on the line.
\fi

\ifdefined\chs
理查德·M·斯托曼的作品本身就为他自己代言。从源代码中的文档到公开发表的论文和演讲,很少人能像斯托曼那样把自己的想法和作品毫无保留的公开给大众。
\fi

\ifdefined\eng
Such openness-if one can pardon a momentary un-Stallman adjective-is refreshing. After all, we live in a society that treats information, especially personal information, as a valuable commodity. The question quickly arises. Why would anybody want to part with so much information and yet appear to demand nothing in return?
\fi

\ifdefined\chs
这样的``公开''让人耳目一新,如果可以原谅我用一个与斯托曼特质格格不如的形容词的话。再怎么说,我们仍然是生活在一种把信息尤其是个人信息当成是一种无价的物品的社会中。人们不禁要问,为什么有人会愿意公开如此多的信息,却并不求什么回报?
\fi

\ifdefined\eng
As we shall see in later chapters, Stallman does not part with his words or his work altruistically. Every program, speech, and on-the-record bon mot comes with a price, albeit not the kind of price most people are used to paying.
\fi

\ifdefined\chs
在后续的章节中我们会看到,斯托曼并不是完全无私的公开他的言论和作品。他发表的每一个程序、演讲或有记录的格言都有标价,即使不是一般人所熟悉的那种需要支付的价格。
\fi

\ifdefined\eng
I bring this up not as a warning, but as an admission. As a person who has spent the last year digging up facts on Stallman's personal history, it's more than a little intimidating going up against the Stallman oeuvre. ``Never pick a fight with a man who buys his ink by the barrel,'' goes the old Mark Twain adage. In the case of Stallman, never attempt the definitive biography of a man who trusts his every thought to the public record.
\fi

\ifdefined\chs
把丑话说在前面不是作为一种警告,而是一种坦白。作为一个花费了去年一年时间发掘有关斯托曼的个人历史的人,对他的作品做出这样的评价无异于一种小小的恐吓。马克·吐温有一句格言:不要为了不实报道与跟媒体人争吵。如果用在斯托曼身上,那就是永远不要给一个认为自己的想法永远正确的人写权威性的传记。
\fi

\ifdefined\eng
For the readers who have decided to trust a few hours of their time to exploring this book, I can confidently state that there are facts and quotes in here that one won't find in any Slashdot story or Google search. Gaining access to these facts involves paying a price, however. In the case of the book version, you can pay for these facts the traditional manner, i.e., by purchasing the book. In the case of the electronic versions, you can pay for these facts in the free software manner. Thanks to the folks at O'Reilly \& Associates, this book is being distributed under the GNU Free Documentation License, meaning you can help to improve the work or create a personalized version and release that version under the same license.
\fi

\ifdefined\chs
对于那些愿意花几个小时来阅读本书的读者,我会很有信心的说,本书中有些事实和引文是你们在Slashdot或Google搜索中无法找到的。当然,要获取到这些事实需要支付一定的费用。对于纸质书的版本,你可以用传统的方式来支付这些费用,比如买一本书。对于电子书的版本,你可以用对自由软件相似的方式来对这些事实付费。感谢O'Reilly \& Associates的同仁,得以让本书以GNU自由软件许可证的形式发行,这意味着你可以修改本书甚至发布一个你自己改编过的版本,并以相同的许可证来发布。
\fi

\ifdefined\eng
If you are reading an electronic version and prefer to accept the latter payment option, that is, if you want to improve or expand this book for future readers, I welcome your input. Starting in June, 2002, I will be publishing a bare bones HTML version of the book on the web site, http://www.faifzilla.org. My aim is to update it regularly and expand the Free as in Freedom story as events warrant. If you choose to take the latter course, please review Appendix C of this book. It provides a copy of your rights under the GNU Free Documentation License.
\fi

\ifdefined\chs
如果你正在阅读的是电子版,并愿意以自由软件的形式对本书做出贡献,也就是说,如果你愿意对本书进行改进或把它推荐给更多的读者,我会非常感激。从2002年6月开始,我会在在线商店中发布一个本书的纯HTML的版本,http://www.faifzilaa.org。我的目标是定期对它进行更新,并随着时间的流逝把Free as in Freedom的故事继续延续下去。如果你选择后一种形式,请阅读本书的附录C,这是一份完整的GNU自由文档许可证,里面描述了你所拥有的权利。
\fi

\ifdefined\eng
For those who just plan to sit back and read, online or elsewhere, I consider your attention an equally valuable form of payment. Don't be surprised, though, if you, too, find yourself looking for other ways to reward the good will that made this work possible.
\fi

\ifdefined\chs
对于只是想坐下来安安静静阅读本书的人,即使你选择的是在线电子版本而不是纸质书,我仍把你们对本书的关注看成与付费购买纸质书一样是对本书的支持。对此,你无须感到惊讶,因为你是采用了一种不同的方式来表达了你对本书的美好祝愿。
\fi

\ifdefined\eng
One final note: this is a work of journalism, but it is also a work of technical documentation. In the process of writing and editing this book, the editors and I have weighed the comments and factual input of various participants in the story, including Richard Stallman himself. We realize there are many technical details in this story that may benefit from additional or refined information. As this book is released under the GFDL, we are accepting patches just like we would with any free software program. Accepted changes will be posted electronically and will eventually be incorporated into future printed versions of this work. If you would like to contribute to the further improvement of this book, you can reach me at \url{sam@inow.com}.
\fi

\ifdefined\chs
最后,我想说明,这只是一本人物传记,但它也是一本技术文档。在撰写和编辑本书的过程中,我和编辑权衡了很多这个故事中的关联人物所提供建议和素材,包括理查德·斯托曼本人。我们意识到在这个故事中有很多的技术细节,需要由更多的素材的支撑才能更表达的更为清晰。由于本书以GFDL发布,我们也欢迎读者向给自由软件打补丁那样为本书打补丁。所有被我们接受的改动都会整合到本书的电子版中并最终可能会整合到本书的以后的印刷版本中。如果你愿意为本书提供改进意见,请发邮件到\url{sam@inow.com}。
\fi

\ifdefined\eng
Comments and Questions
\fi

\ifdefined\chs
建议与问题
\fi

\ifdefined\eng
Please address comments and questions concerning this book to the publisher:

O'Reilly \& Associates, Inc. 

1005 Gravenstein Highway North 

Sebastopol, CA 95472 

(800) 998-9938 (in the United States or Canada) 

(707) 829-0515 (international/local) 

(707) 829-0104 (fax)
\fi

\ifdefined\chs
请把你的建议和问题寄送给本书的出版商:

O'Reilly \& Associates, Inc. 

1005 Gravenstein Highway North 

Sebastopol, CA 95472 

(800) 998-9938 (in the United States or Canada) 

(707) 829-0515 (international/local) 

(707) 829-0104 (fax)
\fi

\ifdefined\eng
There is a web page for this book, which lists errata, examples, or any additional information. The site also includes a link to a forum where you can discuss the book with the author and other readers. You can access this site at:

\url{http://www.oreilly.com/catalog/freedom/}

To comment or ask technical questions about this book, send email to:

\url{bookquestions@oreilly.com}

For more information about books, conferences, Resource Centers, and the O'Reilly Network, see the O'Reilly web site at:

\url{http://www.oreilly.com}
\fi

\ifdefined\chs
我们为本书提供了一个网站,上面有本书的勘误表、示例和其它相关的资料。这个页面上还提供了与作者和其它读者进行交流的网站链接。你可以通过下面的网址来访问这个网站:

\url{http://www.oreilly.com/catalog/freedom/}

如果有任何的建议或技术问题需要咨询,请发电子邮件到:

\url{bookquestions@oreilly.com}

有关O'Reilly相关的图书、会议、资源及O'Reilly Network的相关信息,请访问O'Reilly的网站:

\url{http://www.oreilly.com}
\fi

\ifdefined\eng
Acknowledgments
\fi

\ifdefined\chs
致谢
\fi

\ifdefined\eng
Special thanks to Henning Gutmann for sticking by this book. Special thanks to Aaron Oas for suggesting the idea to Tracy in the first place. Thanks to Laurie Petrycki, Jeffrey Holcomb, and all the others at O'Reilly \& Associates. Thanks to Tim O'Reilly for backing this book. Thanks to all the first-draft reviewers: Bruce Perens, Eric Raymond, Eric Allman, Jon Orwant, Julie and Gerald Jay Sussman, Hal Abelson, and Guy Steele. I hope you enjoy this typo-free version. Thanks to Alice Lippman for the interviews, cookies, and photographs. Thanks to my family, Steve, Jane, Tish, and Dave. And finally, last but not least: thanks to Richard Stallman for having the guts and endurance to ``show us the code.''
\fi

\ifdefined\chs
由衷感谢Henning Gutmann对本书的关注,由衷感谢Aaron Oas在创作初期给Tracy的建议。感谢Laurie Petrychi、Jeffrey Holcomb和其它O'Reilly \& Associates的同仁。感谢Tim O'Reilly对本书的支持。感谢所有参与审阅本书初稿的朋友:Bruce Perens, Eric Raymond, Eric Allman, Jon Orwant, Julie和Gerald Jay Sussman, Hal Abelson和 Guy Steele。希望你们能喜欢这本没有错别字的书。感谢Alice Lippman提供访谈内容、花絮和照片。感谢我的家庭成员Steve,Jane,Tish和Dave。最后,感谢理查德·斯托曼的勇气和忍耐力让我们去了解有关他的一切。
\fi

