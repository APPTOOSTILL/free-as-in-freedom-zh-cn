%% Copyright (c) 2002, 2010 Sam Williams
%% Copyright (c) 2010 Richard M. Stallman
%% Permission is granted to copy, distribute and/or modify this
%% document under the terms of the GNU Free Documentation License,
%% Version 1.3 or any later version published by the Free Software
%% Foundation; with no Invariant Sections, no Front-Cover Texts, and
%% no Back-Cover Texts. A copy of the license is included in the
%% file called ``gfdl.tex''.
\chapter{\ifdefined\eng
Puddle of Freedom
\fi
\ifdefined\chs
自由一隅
\fi}

\ifdefined\eng
\ifdefined\vtwo
[RMS: In this chapter, I have corrected statements about facts, including facts about my thoughts and feelings, and removed some gratuitous hostility in descriptions of events.  I have preserved Williams' statements of his own impressions, except where noted.]
\fi
\fi

\ifdefined\chs
\ifdefined\vtwo
[RMS:本章中,我纠正了第一版中的一些错误陈述。包括对我想法和感受的描写。除非标明,我都保留了威廉原文中关于他个人感受的描写。]
\fi
\fi

\ifdefined\eng
Ask anyone who's spent more than a minute in Richard Stallman's presence, and you'll get the same recollection: forget the long hair. Forget the quirky demeanor. The first thing you notice is the gaze. One look into Stallman's green eyes, and you know you're in the presence of a true believer.
\fi

\ifdefined\chs
如果谁和斯托曼在一起,呆上超过一分钟,他的感受一般都是这样:不管他那一头的长发,不管他怪异的举止,给人印象最深刻的,是斯托曼凝视你的眼神。透过他绿色的眼睛,你能看到一种真实的信仰。
\fi

\ifdefined\eng
To call the Stallman gaze intense is an understatement. Stallman's eyes don't just look at you; they look through you. Even when your own eyes momentarily shift away out of simple primate politeness, Stallman's eyes remain locked-in, sizzling away at the side of your head like twin photon beams.
\fi

\ifdefined\chs
要说斯托曼眼神犀利,都要算是轻描淡写了。他不止是看着你,他简直就是要看透看穿你的心。哪怕你出于礼貌,暂时把自己的视线从他身上移开,你依旧能感受到他的眼神像灯塔一样,锁定在你的身上。
\fi

\ifdefined\eng
Maybe that's why most writers, when describing Stallman, tend to go for the religious angle. In a 1998 \textit{Salon.com} article titled ``The Saint of Free Software,'' Andrew Leonard describes Stallman's green eyes as ``radiating the power of an Old Testament prophet.''\endnote{See Andrew Leonard, ``The Saint of Free Software,'' \textit{Salon.com} (August 1998), \url{http://www.salon.com/21st/feature/1998/08/cov_31feature.html}.} A 1999 \textit{Wired} magazine article describes the Stallman beard as ``Rasputin-like,''\endnote{See Leander Kahney, ``Linux's Forgotten Man,'' \textit{Wired News} (March 5, 1999), \url{http://www.wired.com/news/print/0,1294,18291,00.html}.} while a \textit{London Guardian} profile describes the Stallman smile as the smile of ``a disciple seeing Jesus.''\endnote{See ``Programmer on moral high ground; Free software is a moral issue for Richard Stallman believes in freedom and free software,'' \textit{London Guardian} (November 6, 1999), \url{http://www.guardian.co.uk/uk/1999/nov/06/andrewbrown}.

These are just a small sampling of the religious comparisons. To date, the most extreme comparison has to go to Linus Torvalds, who, in his autobiography -- see Linus Torvalds and David Diamond, \textit{Just For Fun: The Story of an Accidental Revolutionary} (HarperCollins Publishers, Inc., 2001): 58 -- writes, ``Richard Stallman is the God of Free Software.''

Honorable mention goes to Larry Lessig, who, in a footnote description of Stallman in his book -- see Larry Lessig, \textit{The Future of Ideas} (Random House, 2001): 270 -- likens Stallman to Moses:

\begin{quote}
\ldots as with Moses, it was another leader, Linus Torvalds, who finally carried the movement into the promised land by facilitating the development of the final part of the OS puzzle. Like Moses, too, Stallman is both respected and reviled by allies within the movement. He is [an] unforgiving, and hence for many inspiring, leader of a critically important aspect of modern culture. I have deep respect for the principle and commitment of this extraordinary individual, though I also have great respect for those who are courageous enough to question his thinking and then sustain his wrath.
\end{quote}

In a final interview with Stallman, I asked him his thoughts about the religious comparisons. ``Some people do compare me with an Old Testament prophet, and the reason is Old Testament prophets said certain social practices were wrong. They wouldn't compromise on moral issues. They couldn't be bought off, and they were usually treated with contempt.''}
\fi

\ifdefined\chs
也许这就是为什么,很多描写斯托曼的文章,都会从信仰的角度开始写起。在1998年salon.com上的一篇文章,标题定为《自由软件圣徒》。作者安德鲁·伦纳德描述斯托曼那双绿色眼睛的时候,写道:``[它们] 散发出一种旧约中先知般的力量\endnote{参见安德鲁·伦纳德的文章《自由软件圣徒》,1998年8月:\url{http://www.salon.com/21st/feature/1998/08/cov_31feature.html}}。''1999年《\textit{连线}》杂志的一篇文章中,形容斯托曼的络腮长胡须好似拉斯普京\endnote{译注:拉希普京,全名葛列格里·叶菲莫维奇·拉斯普京。沙俄末期的一名颇受争议的东正教教徒。当时的沙皇尼古拉二世和皇后亚历山德拉笃信神秘主义,拉希普京便因此得宠。后被尤苏波夫亲王、狄密翠大公、普利希克维奇议员等人合谋刺杀。}的胡须一般\endnote{参见利安得·克尼的文章《被Linux遗忘的人》,《连线》,1999年3月5日: http://www.wired.com/science/discoveries/news/1999/03/18291}。《伦敦卫报》则把斯托曼的微笑说成好似``门徒看到耶稣''一般\endnote{参见《有操守的程序员;坚持自由,坚持自由软件的理查德·斯托曼》,《伦敦卫报》,1999年11月6日;http://www.guardian.co.uk/uk/1999/nov/06/andrewbrown 。把斯托曼比作宗教信徒的文章还有很多。Linux的作者莱纳斯·托瓦尔兹(Linus Torvalds)曾用更极端的语言把斯托曼比作宗教信徒。参见他2001年发表的自传《好玩而已:一场无意间发起的革命》(Just For Fun: The Story of an Accidental Revolutionary,HarperCollins Publishers, Inc., 2001),第58页。托瓦尔兹说:``斯托曼简直就是自由软件之神。''

劳伦斯·雷席格(Lawrence Lessig,昵称Larry Lessig)则把它形容得有点恐怖。雷席格所著的《The Future of Ideas》一书中,第270页脚注中曾将斯托曼比作摩西:

\begin{quote}
\ldots 同摩西一样,他也是得了贵人相助:当年斯托曼开发的操作系统只差最后一个重要组件,那就是操作系统的内核。是莱纳斯·托瓦尔兹(Linus Torvalds)开发了Linux这个自由的操作系统内核,最终将斯托曼的运动带入了当年承诺的那片“应许之地”。斯托曼也和摩西一样,在自由软件社区里,既受人尊重,也常被人指责。他为人棱角分明,也正因如此,很能激人上进。他是现代文化中不可或缺的一位领袖。斯托曼是一位卓著的人,他坚持原则和自己的信念,为此我甚是尊敬。我同样尊敬那些敢于与他共事,甚至敢于批判他想法,又能忍受他乖戾脾气的人。
\end{quote}

在本书采访的结尾,我曾问过斯托曼,想听听他对于这种将自己与宗教挂钩的说辞有什么想法。斯托曼说:``的确有些人把我和《旧约》里的先知坐对比。因为《旧约》里面一些先知把很多社会活动都预测错了。他们也不会在道德原则问题上做妥协,不会被人收买,还常常被人轻视''
}。
\fi

\ifdefined\eng
Such analogies serve a purpose, but they ultimately fall short. That's because they fail to take into account the vulnerable side of the Stallman persona. Watch the Stallman gaze for an extended period of time, and you will begin to notice a subtle change. What appears at first to be an attempt to intimidate or hypnotize reveals itself upon second and third viewing as a frustrated attempt to build and maintain contact. If his personality has a touch or ``shadow'' of autism or Asperger's Syndrome, a possibility that Stallman has entertained from time to time, his eyes certainly confirm the diagnosis. Even at their most high-beam level of intensity, they have a tendency to grow cloudy and distant, like the eyes of a wounded animal preparing to give up the ghost.
\fi

\ifdefined\chs
这些比喻有一定道理,但似乎也有不足。它们都忽视了斯托曼人格中脆弱的一面。盯着斯托曼的眼睛再久些,你会发现其中潜在的改变。一开始,是一种强势带着几分催眠般的眼神;可再看几次,就会发现那眼神其实是在寻求沟通,建立联系。如果他个性中真的有着一些自闭症的成分,那么他的这种眼神倒确实符合这种个性。哪怕当他目光如炬地盯着你的时候,你依旧能够感受到他眼神中透出的几分迷茫和疏远。恰似受伤的猛兽一般,眼中传出的有几分绝望。
\fi

\ifdefined\eng
My own first encounter with the legendary Stallman gaze dates back to the March, 1999, LinuxWorld Convention and Expo in San Jose, California. Billed as a ``coming out party'' for the ``Linux'' software community, the convention also stands out as the event that reintroduced Stallman to the technology media. Determined to push for his proper share of credit, Stallman used the event to instruct spectators and reporters alike on the history of the GNU Project and the project's overt political objectives.
\fi

\ifdefined\chs
我第一次见识传说中斯托曼的目光,是在1999年3月的第一届LinuxWorld大会上。大会在加利福尼亚州的圣何塞市举行,是一场``Linux''社区的大聚会。其中一个环节,就是在媒体面前介绍斯托曼。斯托曼决心在这次见面上简要回顾GNU工程的历史以及这个工程政治上的目标,以此说明自己在社区中的工作和价值。
\fi

\ifdefined\eng
As a reporter sent to cover the event, I received my own Stallman tutorial during a press conference announcing the release of GNOME 1.0, a free software graphic user interface. Unwittingly, I push an entire bank of hot buttons when I throw out my very first question to Stallman himself: ``Do you think GNOME's maturity will affect the commercial popularity of the Linux operating system?''
\fi

\ifdefined\chs
作为一个报道此次大会的记者,我在GNOME 1.0的媒体发布会上,被斯托曼亲自上了一课。当时,我的第一个问题在无意之中触碰到了敏感话题:``请问您觉得随着GNOME逐渐成熟,是否会推进Linux的商业应用?''
\fi

\ifdefined\eng
``I ask that you please stop calling the operating system Linux,'' Stallman responds, eyes immediately zeroing in on mine. ``The Linux kernel is just a small part of the operating system. Many of the software programs that make up the operating system you call Linux were not developed by Linus Torvalds at all. They were created by GNU Project volunteers, putting in their own personal time so that users might have a free operating system like the one we have today. To not acknowledge the contribution of those programmers is both impolite and a misrepresentation of history. That's why I ask that when you refer to the operating system, please call it by its proper name, GNU/Linux.''
\fi

\ifdefined\chs
``请你不要再把整个操作系统都叫做Linux。''斯托曼回应,眼光迅速聚焦到我的身上:``Linux是一个操作系统的内核,它仅仅是操作系统的一小部分。你所说的Linux,实际还包含了很多软件。这些软件并不都是莱纳斯·托瓦尔兹(Linus Torvalds)开发的。它们是由GNU工程的志愿者开发的,这些志愿者用自己的业余时间创造了这些软件,让用户最终可以自由使用这个操作系统。如果不提及这些开发人员的工作,那是既不礼貌,也不尊重历史的。所以我坚持把这个操作系统称作GNU/Linux,也希望请你这么称呼它。''
\fi

\ifdefined\eng
Taking the words down in my reporter's notebook, I notice an eerie silence in the crowded room. When I finally look up, I find Stallman's unblinking eyes waiting for me. Timidly, a second reporter throws out a question, making sure to use the term ``GNU/Linux'' instead of Linux. Miguel de Icaza, leader of the GNOME project, fields the question. It isn't until halfway through de Icaza's answer, however, that Stallman's eyes finally unlock from mine. As soon as they do, a mild shiver rolls down my back. When Stallman starts lecturing another reporter over a perceived error in diction, I feel a guilty tinge of relief. At least he isn't looking at me, I tell myself.
\fi

\ifdefined\chs
我把这段话记在采访本上,整个会场异常地安静。等到我抬起头来,才看到斯托曼的眼睛眨都不眨地盯着我,让我心生一丝胆怯。这时候,另外一名记者提问,正确地使用了GNU/Linux的名字,才打破僵局。GNOME项目的组长,米格尔·德·伊卡萨回答了这名记者的问题。伊卡萨回答到一半,斯托曼才把目光从我身上移开。这时我才觉得些许放松。等到斯托曼继续纠正了另外一个记者的措辞问题后,我才有种罪孽被洗清的感觉。他起码不再盯着我了---我对自己说。
\fi

\ifdefined\eng
For Stallman, such face-to-face moments would serve their purpose. By the end of the first LinuxWorld show, most reporters know better than to use the term ``Linux'' in his presence, and Wired.com is running a story comparing Stallman to a pre-Stalinist revolutionary erased from the history books by hackers and entrepreneurs eager to downplay the GNU Project's overly political objectives.\endnote{See Leander Kahney (1999).} Other articles follow, and while few reporters call the operating system GNU/Linux in print, most are quick to credit Stallman for launching the drive to build a free software operating system 15 years before.
\fi

\ifdefined\chs
从斯托曼的角度看,这种面对面的时刻是有意义的。这次大会结束的时候,各路记者在斯托曼面前已经改用GNU/Linux这名字。而《连线》的记者则开始忙着写一篇通讯,其中把斯托曼比作斯大林之前的革命者,被黑客和各大公司从历史上抹去了姓名,甚至忽略整个GNU工程,因为这些黑客和公司不喜欢GNU工程中的政治目标\endnote{参见利安得·克尼的文章《被Linux遗忘的人》(Linux's Forgotten Man),《连线》,1999年3月5日: http://www.wired.com/science/discoveries/news/1999/03/18291}。其他的报道也相继出现,尽管没有在报道中使用GNU/Linux的名字,但它们大多都提及了斯托曼的贡献,说到他十五年前发起的开发自由操作系统的项目。
\fi

\ifdefined\eng
I won't meet Stallman again for another 17 months. During the interim, Stallman will revisit Silicon Valley once more for the August, 1999 LinuxWorld show. Although not invited to speak, Stallman does manage to deliver the event's best line. Accepting the show's Linus Torvalds Award for Community Service -- an award named after Linux creator Linus Torvalds -- on behalf of the Free Software Foundation, Stallman wisecracks, ``Giving the Linus Torvalds Award to the Free Software Foundation is a bit like giving the Han Solo Award to the Rebel Alliance.''
\fi

\ifdefined\chs
再次见到斯托曼,是17个月后。在这17个月中间,斯托曼又一次前往位于硅谷的圣何塞,参加1999年8月的第二届LinuxWorld大会。尽管没有演讲,但斯托曼此次还是有目的的。他是代表自由软件基金会,领取莱纳斯·托瓦尔兹社区贡献奖。斯托曼接受奖杯的时候,打趣道:``这次代表自由软件基金会领取莱纳斯·托瓦尔兹社区贡献奖,感觉就像唐僧代表师徒四人去领取悟空奖\endnote{译注:原句为``Giving the Linus Torvalds Award to the Free Software Foundation is a bit like giving the Han Solo Award to the Rebel Alliance.''里面借用了《星球大战》中的人物关系。为了读者方便理解,翻译时做了意译。}。''
\fi

\ifdefined\eng
This time around, however, the comments fail to make much of a media dent. Midway through the week, Red Hat, Inc., a prominent GNU/Linux vendor, goes public. The news merely confirms what many reporters such as myself already suspect: ``Linux'' has become a Wall Street buzzword, much like ``e-commerce'' and ``dot-com'' before it. With the stock market approaching the Y2K rollover like a hyperbola approaching its vertical asymptote, all talk of free software or open source as a political phenomenon falls by the wayside.
\fi

\ifdefined\chs
可惜,这次的玩笑并没有吸引多少媒体的关注。倒是这周的红帽公司(Red Hat, Inc.)上市,引起了各路记者的关注。这是一家以发布GNU/Linux发行版为主的软件公司\endnote{译注:如今红帽公司的业务已经逐渐扩大,并不仅仅以发布发行版为主。但依然以GNU/Linux的各种解决方案为核心。}。这次上市,印证了我们大多数记者的猜测:``Linux''一词,会和``电子商务'',``点com''等一样,成为华尔街上的流行语。股票市场的热闹会把大家搞得像中了千年虫病毒一般,忘掉过去,丢掉历史,把自由软件或开源的政治方面抛诸脑后。
\fi

\ifdefined\eng
Maybe that's why, when LinuxWorld follows up its first two shows with a third LinuxWorld show in August, 2000, Stallman is conspicuously absent.
\fi

\ifdefined\chs
也许这就是为什么,在2000年第三届LinuxWorld大会上,大家再也没看到斯托曼的身影。
\fi

\ifdefined\eng
My second encounter with Stallman and his trademark gaze comes shortly after that third LinuxWorld show. Hearing that Stallman is going to be in Silicon Valley, I set up a lunch interview in Palo Alto, California. The meeting place seems ironic, not only because of his absence from the show but also because of the overall backdrop. Outside of Redmond, Washington, few cities offer a more direct testament to the economic value of proprietary software. Curious to see how Stallman, a man who has spent the better part of his life railing against our culture's predilection toward greed and selfishness, is coping in a city where even garage-sized bungalows run in the half-million-dollar price range, I make the drive down from Oakland.
\fi

\ifdefined\chs
我第二次再和斯托曼见面,是在第三届LinuxWorld大会之后不久。当时我听说斯托曼又要来硅谷,我就和他联系好,定在加利福尼亚州帕罗奥多市(Palo Alto)进行采访。这个地点选的很有意思。不仅仅是因为他上次缺席LinuxWorld大会,还因为这里作为硅谷重镇,是除了微软所在的华盛顿州雷蒙德市以外,最支持专有软件经济模型的几个城市之一。而斯托曼则花费了自己整个青春和大半生的时间,去与我们文化之中的自私与贪婪作斗争。我很好奇,他来到这座城市,看着车库大小的平房里,都在做着几十万美元的买卖的现实,斯托曼会作何反应。带着好奇心,我离开奥克兰市,驱车前往帕罗奥多。
\fi

\ifdefined\eng
I follow the directions Stallman has given me, until I reach the headquarters of Art.net, a nonprofit ``virtual artists collective.'' Located in a hedge-shrouded house in the northern corner of the city, the Art.net headquarters are refreshingly run-down. Suddenly, the idea of Stallman lurking in the heart of Silicon Valley doesn't seem so strange after all.
\fi

\ifdefined\chs
按照斯托曼给我的地址,我来到了Art.net的总部。这是一家汇集各路``网上艺术家''的非盈利组织。位于城市北部,坐落在一所被篱笆围起来的房子里。房子有些旧,但又透着些小清新。这氛围顿时又让我觉得,斯托曼混在硅谷,似乎也不是个很怪的想法。
\fi

\ifdefined\eng
I find Stallman sitting in a darkened room, tapping away on his gray laptop computer. He looks up as soon as I enter the room, giving me a full blast of his 200-watt gaze. When he offers a soothing ``Hello,'' I offer a return greeting. Before the words come out, however, his eyes have already shifted back to the laptop screen.
\fi

\ifdefined\chs
我看到斯托曼坐在一个背阴的房间里,正在他那台灰色笔记本电脑上敲着键盘。我一进门,他就看着我,投来了那著名的目光。互相寒暄,他又把目光聚焦回笔记本屏幕上。
\fi

\ifdefined\eng
``I'm just finishing an article on the spirit of hacking,'' Stallman says, fingers still tapping. ``Take a look.''
\fi

\ifdefined\chs
``我刚刚写了一篇关于黑客精神的文章。''斯托曼一边打字,一边说:``过来看看?''
\fi

\ifdefined\eng
I take a look. The room is dimly lit, and the text appears as greenish-white letters on a black background, a reversal of the color scheme used by most desktop word-processing programs, so it takes my eyes a moment to adjust. When they do, I find myself reading Stallman's account of a recent meal at a Korean restaurant. Before the meal, Stallman makes an interesting discovery: the person setting the table has left six chopsticks instead of the usual two in front of Stallman's place setting. Where most restaurant goers would have ignored the redundant pairs, Stallman takes it as challenge: find a way to use all six chopsticks at once. Like many software hacks, the successful solution is both clever and silly at the same time. Hence Stallman's decision to use it as an illustration.
\fi

\ifdefined\chs
我走过去开始读文章。房间阴暗,屏幕上编辑软件的界面,又被斯托曼设置成了黑底绿字。我读了几个字,眼睛才逐渐适应。文章开始回忆了斯托曼有次去一家韩国餐馆的经历。当时服务员摆桌的时候,在斯托曼面前放了三双筷子。一般人恐怕都会把多余的两双筷子拿开。可斯托曼却又借此开始玩起把戏,他企图找个法子,用上所有的筷子。和很多黑客技巧一样,最终的解决方案既聪明灵巧,可又傻里傻气。斯托曼以这个事情来作为整个文章的引子。
\fi

\ifdefined\eng
As I read the story, I feel Stallman watching me intently. I look over to notice a proud but child-like half smile on his face. When I praise the essay, my comment barely merits a raised eyebrow.
\fi

\ifdefined\chs
我读的时候,感觉斯托曼一直在盯着我。起身看了他一眼,见他一脸骄傲的表情,好似孩子般笑着站在那。我夸奖了一下他的文章,勉强让他眉毛抬了抬。
\fi

\ifdefined\eng
``I'll be ready to go in a moment,'' he says.
\fi

\ifdefined\chs
``稍等一下,我们马上就出发。''他说。
\fi

\ifdefined\eng
Stallman goes back to tapping away at his laptop. The laptop is gray and boxy, not like the sleek, modern laptops that seemed to be a programmer favorite at the recent LinuxWorld show. Above the keyboard rides a smaller, lighter keyboard, a testament to Stallman's aging hands. During the mid 1990s, the pain in Stallman's hands became so unbearable that he had to hire a typist. Today, Stallman relies on a keyboard whose keys require less pressure than a typical computer keyboard.
\fi

\ifdefined\chs
斯托曼坐回座位,开始继续敲键盘。他用的笔记本是个灰色的四方盒子,和LinuxWorld上看到的大多笔记本不同,斯托曼的这台没那么光鲜亮丽,也不是什么新款式。笔记本的键盘之上,还放着一个更小,更轻的键盘。斯托曼的一双大手,则在这个小键盘上飞舞。九十年代中期,斯托曼双手的剧痛让他无法容忍,以至于他曾一度雇用了一位打字员。今天,他用的键盘,是一种特殊的,比常规键盘按键力度更小的键盘。
\fi

\ifdefined\eng
Stallman has a tendency to block out all external stimuli while working. Watching his eyes lock onto the screen and his fingers dance, one quickly gets the sense of two old friends locked in deep conversation.
\fi

\ifdefined\chs
斯托曼在工作时,会全然不顾外界刺激。看着他眼睛盯着屏幕,手指飞舞,让人觉得斯托曼和电脑,恰似老友重逢,当下正促膝长谈。
\fi

\ifdefined\eng
The session ends with a few loud keystrokes and the slow disassembly of the laptop.
\fi

\ifdefined\chs
斯托曼用力敲了几下键盘,合上笔记本电脑,拔掉电源,结束了写作。
\fi

\ifdefined\eng
``Ready for lunch?'' Stallman asks.
\fi

\ifdefined\chs
``吃午饭去吧?''斯托曼问。
\fi

\ifdefined\eng
We walk to my car. Pleading a sore ankle, Stallman limps along slowly. Stallman blames the injury on a tendon in his left foot. The injury is three years old and has gotten so bad that Stallman, a huge fan of folk dancing, has been forced to give up all dancing activities. ``I love folk dancing intensely,'' Stallman laments. ``Not being able to dance has been a tragedy for me.''
\fi

\ifdefined\chs
斯托曼抱怨着自己脚踝的伤,和我一起徐步走到我的车前。他左脚跟腱三年前受了伤,让他这么一个民族舞爱好者从此告别任何舞蹈活动。斯托曼叹道:``我可喜欢跳舞了,不能跳舞对我来说简直是个悲剧。''
\fi

\ifdefined\eng
Stallman's body bears witness to the tragedy. Lack of exercise has left Stallman with swollen cheeks and a pot belly that was much less visible the year before. You can tell the weight gain has been dramatic, because when Stallman walks, he arches his back like a pregnant woman trying to accommodate an unfamiliar load.
\fi

\ifdefined\chs
这悲剧从斯托曼的身材上就体现出来了。由于缺乏锻炼,斯托曼开始发福。脸颊越来越圆,啤酒肚也凸出来了。你能明显看出,他发胖简直就是一瞬间完成的。他自己都还来不及适应,以至于在走路的时候,都要撑着腰,好似一个孕妇一般。
\fi

\ifdefined\eng
The walk is further slowed by Stallman's willingness to stop and smell the roses, literally. Spotting a particularly beautiful blossom, he strokes the innermost petals against his nose, takes a deep sniff, and steps back with a contented sigh.
\fi

\ifdefined\chs
走到中途,斯托曼停下脚步,在一簇玫瑰花前伏下身子。剥开花瓣,斯托曼把鼻子凑到花蕊前,深吸一口气,然后起身长叹,甚是满足。
\fi

\ifdefined\eng
``Mmm, rhinophytophilia,'' he says, rubbing his back.\endnote{At the time, I thought Stallman was referring to the flower's scientific name. Months later, I would learn that \textit{rhinophytophilia} was in fact a humorous reference to the activity -- i.e., Stallman's sticking his nose into a flower and enjoying the moment -- presenting it as the kinky practice of nasal sex with plants. For another humorous Stallman flower incident, visit: \url{http://www.stallman.org/articles/texas.html}.}
\fi

\ifdefined\chs
``嗯,rhinophytophilia。''他揉揉腰说。\endnote{开始,我还以为斯托曼说的是这花的学名。几个月后,我才发现斯托曼所谓的rhinophytophilia,原来也是他的一个玩笑。他实际是把嗅花的行为比作用鼻子和花做爱。更多斯托曼关于花的玩笑,参见:http://www.stallman.org/articles/texas.html 。译注:rhinophytophilia是斯托曼自创的一个希腊词汇。rhino意为``鼻子'';phyto意为``花'';philia意为``爱''。}
\fi

\ifdefined\eng
The drive to the restaurant takes less than three minutes. Upon recommendation from Tim Ney, former executive director of the Free Software Foundation, I have let Stallman choose the restaurant. While some reporters zero in on Stallman's monk-like lifestyle, the truth is, Stallman is a committed epicure when it comes to food. One of the fringe benefits of being a traveling missionary for the free software cause is the ability to sample delicious food from around the world. ``Visit almost any major city in the world, and chances are Richard knows the best restaurant in town,'' says Ney. ``Richard also takes great pride in knowing what's on the menu and ordering for the entire the table.''  (If they are willing, that is.)
\fi

\ifdefined\chs
开车到餐馆不到三分钟。根据前自由软件基金会执行总监蒂姆·内伊(Tim Ney)的建议,我让斯托曼选择去哪家餐馆。尽管很多报道,把斯托曼描述成苦行僧一般,可事实上,斯托曼在饮食方面,可谓是个美食家。作为一个布道全球的自由软件使者,一个额外收获就是可以尝遍各地美食。内伊介绍:``他去了几乎全世界所有的主要城市,这让他总能知道全城最好的餐馆在哪。斯托曼知道菜单上每个菜都是什么,点上一桌美食,他经常以此为荣。''
\fi

\ifdefined\eng
For today's meal, Stallman has chosen a Cantonese-style dim sum restaurant two blocks off University Avenue, Palo Alto's main drag. The choice is partially inspired by Stallman's recent visit to China, including a stop in Hong Kong, in addition to Stallman's personal aversion to spicier Hunanese and Szechuan cuisine. ``I'm not a big fan of spicy,'' Stallman admits.
\fi

\ifdefined\chs
今天,斯托曼选择了一家广式点心店。这家店和帕罗奥多市的主干道学院街(University
Ave.)相隔两个街区。选这里,一部分是因为斯托曼刚刚去过中国,期间在香港做了短暂停留,这才想当下吃点中餐。斯托曼本人并不喜欢太辣的东西,所以他也就没选川菜和湘菜。``我对辣的东西不感冒。''斯托曼坦言。
\fi

\ifdefined\eng
We arrive a few minutes after 11 a.m. and find ourselves already subject to a 20-minute wait. Given the hacker aversion to lost time, I hold my breath momentarily, fearing an outburst. Stallman, contrary to expectations, takes the news in stride.
\fi

\ifdefined\chs
快十一点的时候,我们赶到餐馆。结果发现居然在餐馆的门前已经排了长队,我们要等二十分钟才能入席。我深知黑客们不喜欢浪费时间,于是屏住呼吸,生怕斯托曼大动肝火。可出乎意料,斯托曼竟然坦然接受了这个事实。
\fi

\ifdefined\eng
``It's too bad we couldn't have found somebody else to join us,'' he tells me. ``It's always more fun to eat with a group of people.''
\fi

\ifdefined\chs
斯托曼对我说:``就只有我们俩来了,真可惜。人多了吃起来才热闹。''
\fi

\ifdefined\eng
During the wait, Stallman practices a few dance steps. His moves are tentative but skilled. We discuss current events. Stallman says his only regret about not attending LinuxWorld was missing out on a press conference announcing the launch of the GNOME Foundation. Backed by Sun Microsystems and IBM, the foundation is in many ways a vindication for Stallman, who has long championed that free software and free-market economics need not be mutually exclusive. Nevertheless, Stallman remains dissatisfied by the message that came out.
\fi

\ifdefined\chs
等待期间,斯托曼开始走起舞步。他步步小心,可还是能看出他有点儿功底。我们开始聊起时事。斯托曼说,这次缺席LinuxWorld大会,最遗憾的部分是没能出席GNOME基金会的成立发布会。这个基金会是Sun公司和IBM一手操办,它很好地诠释了斯托曼的观点:所谓自由软件,并不与自由市场和自由经济相冲突。然而,斯托曼依旧不满意大会的论调。
\fi

\ifdefined\eng
``The way it was presented, the companies were talking about Linux with no mention of the GNU Project at all,'' Stallman says.
\fi

\ifdefined\chs
``发布的时候,各个公司都在说着Linux,闭口不提GNU工程。''斯托曼说。
\fi

\ifdefined\eng
Such disappointments merely contrast the warm response coming from overseas, especially Asia, Stallman notes. A quick glance at the Stallman 2000 travel itinerary bespeaks the growing popularity of the free software message. Between recent visits to India, China, and Brazil, Stallman has spent 12 of the last 115 days on United States soil. His travels have given him an opportunity to see how the free software concept translates into different languages of cultures.
\fi

\ifdefined\chs
这种冷漠恰恰和地球另一边的热络形成对比。尤其是亚洲各国,甚是热情。看看斯托曼2000年的行程安排,就能看出自由软件的迅猛增长。在115天之中,斯托曼只在美国停留了12天,其他时间,则在印度,中国或巴西境内。他的旅行,让他见识了自由软件如何被翻译成不同语言,进入各国文化。
\fi

\ifdefined\eng
``In India many people are interested in free software, because they see it as a way to build their computing infrastructure without spending a lot of money,'' Stallman says. ``In China, the concept has been much slower to catch on. Comparing free software to free speech is harder to do when you don't have any free speech. Still, the level of interest in free software during my last visit was profound.''
\fi

\ifdefined\chs
``在印度,很多人对自由软件感兴趣。因为他们觉得这样可以用很低的成本,建设起自己的计算机基础设施。''斯托曼说:``在中国,自由软件的概念就传播得相对慢了一些。我们经常把软件自由和言论自由并提,表示这是软件用户的一种基本自由,可在中国,这种类比显然不太奏效:人们压根就没有过言论自由,自然也不拿自由这东西当回事了。不过无论如何,我这次的中国之行,还是在一定程度上提高了自由软件的关注度。''
\fi

\ifdefined\eng
The conversation shifts to Napster, the San Mateo, California software company, which has become something of a media cause célèbre in recent months. The company markets a controversial software tool that lets music fans browse and copy the music files of other music fans. Thanks to the magnifying powers of the Internet, this so-called ``peer-to-peer'' program has evolved into a de facto online jukebox, giving ordinary music fans a way to listen to MP3 music files over the computer without paying a royalty or fee, much to record companies' chagrin.
\fi

\ifdefined\chs
话题很快转到Napster,这是一家加利福尼亚州圣马特奥市的软件公司。这几个月,这家公司成了各大媒体的宠儿。这家公司开发了一款倍受争议的软件,使得音乐爱好者之间可以互相拷贝音乐文件。借着互联网的东风,这类称作P2P(即端到端)的软件逐渐流行。如今,Napster俨然成了大型音乐盒,让各路音乐爱好者可以免费欣赏音乐。这一下子就惹恼了各大唱片公司。
\fi

\ifdefined\eng
Although based on proprietary software, the Napster system draws inspiration from the long-held Stallman contention that once a work enters the digital realm -- in other words, once making a copy is less a matter of duplicating sounds or duplicating atoms and more a matter of duplicating information -- the natural human impulse to share a work becomes harder to restrict. Rather than impose additional restrictions, Napster execs have decided to take advantage of the impulse. Giving music listeners a central place to trade music files, the company has gambled on its ability to steer the resulting user traffic toward other commercial opportunities.
\fi

\ifdefined\chs
尽管Napster是个专有软件,但它的流行却印证了斯托曼的想法:一旦某个作品进入数字世界,或者说,一旦复制这个作品变得便宜简单,那么,分享行为,作为人类的一种本能,就很难被阻止。Napster给了用户一片分享音乐的空间,这家公司接着就可以借助用户量来寻求更多商业机会。
\fi

\ifdefined\eng
The sudden success of the Napster model has put the fear in traditional record companies, with good reason. Just days before my Palo Alto meeting with Stallman, U.S. District Court Judge Marilyn Patel granted a request filed by the Recording Industry Association of America for an injunction against the file-sharing service. The injunction was subsequently suspended by the U.S. Ninth District Court of Appeals, but by early 2001, the Court of Appeals, too, would find the San Mateo-based company in breach of copyright law,\endnote{See Cecily Barnes and Scott Ard, ``Court Grants Stay of Napster Injunction,'' \textit{News.com} (July 28, 2000), \url{http://news.cnet.com/news/0-1005-200-2376465.html}.} a decision RIAA spokesperson Hillary Rosen would later proclaim a ``clear victory for the creative content community and the legitimate online marketplace.''\endnote{See ``A Clear Victory for Recording Industry in Napster Case,'' RIAA press release (February 12, 2001), \url{http://www.riaa.com/PR_story.cfm?id=372}.}
\fi

\ifdefined\chs
Napster的迅猛成功让各大唱片公司开始担心。我来帕罗奥多之前的几天,玛莉莲·帕特尔法官刚刚在美国地方法院通过了一项由美国唱片协会提交的禁令。该禁令禁止了文件共享服务。而Napster之后上诉到上诉法院,使得这个禁令被美国第九巡回上诉法院暂停。但在2001年年初,上诉法院判定Napster的确违反了版权法\endnote{参见\textit{News.com}2000年7月28日Cecily Barnes与Scott Ard所著的文章《Court Grants Stay of Napster Injunction》。 \url{http://news.cnet.com/2100-1023-243817.html}}。美国唱片协会发言人希拉里·罗森对这个判决的评价是:``保护了创作者,完善了在线市场方面的法律\endnote{参见美国唱片协会于2001年2月13日发表的新闻稿《Clear Victory for Recording Industry in Napster Case》。\url{http://www.riaa.com/print.php?id=EA742466-C6DB-1D28-CF94-411172C649C7}}。''
\fi

\ifdefined\eng
For hackers such as Stallman, the Napster business model is troublesome in different ways. The company's eagerness to appropriate time-worn hacker principles such as file sharing and communal information ownership, while at the same time selling a service based on proprietary software, sends a distressing mixed message. As a person who already has a hard enough time getting his own carefully articulated message into the media stream, Stallman is understandably reticent when it comes to speaking out about the company. Still, Stallman does admit to learning a thing or two from the social side of the Napster phenomenon.
\fi

\ifdefined\chs
对于斯托曼这样的黑客来说,Napster的商业模型内包涵着复杂的信息。一方面,这家公司大力倡导黑客社区所欣赏的信息共享精神;而另一方面,又出售着基于专有软件的服务。这种复杂性,让斯托曼对这一事件少言寡语,因为他知道,自己已经花了很大功夫去传播自己的思想,而对这种部分和自己思想相同;但很大程度又有矛盾的事件发表评论,很容易被媒体断章取义,平添混乱。可斯托曼依旧承认,从Napster事件上,他也认识到了一些有价值的东西。
\fi

\ifdefined\eng
``Before Napster, I thought it might be [sufficient] for people to privately redistribute works of entertainment,'' Stallman says. ``The number of people who find Napster useful, however, tells me that the right to redistribute copies not only on a neighbor-to-neighbor basis, but to the public at large, is essential and therefore may not be taken away.''
\fi

\ifdefined\chs
``以前,我觉得人们私下分享娱乐信息就[足够]了。''斯托曼说:``可如此庞大数量的用户,在Napster上分享音乐,这让我觉得,不仅仅是私下朋友和朋友之间的共享很重要。在公共场合,公开给大家分享信息的自由,也同样重要,不可剥夺。''
\fi

\ifdefined\eng
No sooner does Stallman say this than the door to the restaurant swings open and we are invited back inside by the host. Within a few seconds, we are seated in a side corner of the restaurant next to a large mirrored wall.
\fi

\ifdefined\chs
话到此,餐厅服务员赶来,告诉我们有了空位。于是我们被领班带到桌前,坐在餐厅一角,身旁的墙壁,是一整面镜子。
\fi

\ifdefined\eng
The restaurant's menu doubles as an order form, and Stallman is quickly checking off boxes before the host has even brought water to the table. ``Deep-fried shrimp roll wrapped in bean-curd skin,'' Stallman reads. ``Bean-curd skin. It offers such an interesting texture. I think we should get it.''
\fi

\ifdefined\chs
服务员拿来菜单和笔,让我们在菜单上勾选要点的菜。还没等服务员把茶水拿上桌,斯托曼就已经开始在自己的菜单上点菜了。``香煎鮮蝦腐皮卷''斯托曼读着菜单:``腐皮,我喜欢这口感。点一个。''
\fi

\ifdefined\eng
This comment leads to an impromptu discussion of Chinese food and Stallman's recent visit to China. ``The food in China is utterly exquisite,'' Stallman says, his voice gaining an edge of emotion for the first time this morning. ``So many different things that I've never seen in the U.S., local things made from local mushrooms and local vegetables. It got to the point where I started keeping a journal just to keep track of every wonderful meal.''
\fi

\ifdefined\chs
这一下子又引来了斯托曼关于中餐和这次访问中国的即兴评论:``中国的饮食真是博大精深。''斯托曼大声说,这要算是他整个上午嗓门最大的一次了:``各种我在美国都没听说过的食物。还有很多当地特产,当地美食。当时突然我又想起来,我要办份杂志,专门记录每顿大餐的菜谱。''
\fi

\ifdefined\eng
The conversation segues into a discussion of Korean cuisine. During the same June, 2000, Asian tour, Stallman paid a visit to South Korea. His arrival ignited a mini-firestorm in the local media thanks to a Korean software conference attended by Microsoft founder and chairman Bill Gates that same week. Next to getting his photo above Gates's photo on the front page of the top Seoul newspaper, Stallman says the best thing about the trip was the food. ``I had a bowl of naeng myun, which is cold noodles,'' says Stallman. ``These were a very interesting feeling noodle. Most places don't use quite the same kind of noodles for your naeng myun, so I can say with complete certainty that this was the most exquisite naeng myun I ever had.''
\fi

\ifdefined\chs
这讨论接着又转向韩国菜。同样也是在2000年6月,斯托曼的亚洲之行中,也去了韩国。他的到来,在当地媒体引起了一个不小的轰动,这部分原因,或许还要部分归功于当时也在韩国的微软创始人比尔·盖茨。斯托曼的照片这次上了首尔的热销报纸的头版,把比尔·盖茨的照片竟然挤到了下面。除了这件事情让斯托曼非常满意之外,当地的美食也让他欢欣鼓舞。他回忆:``我吃了一碗冷面。那面条口感非常独特,我们这边都不用那种面条做冷面。我敢肯定,那冷面是我有生以来吃到的最正点的冷面。''
\fi

\ifdefined\eng
The term ``exquisite'' is high praise coming from Stallman. I know this, because a few moments after listening to Stallman rhapsodize about naeng myun, I feel his laser-beam eyes singeing the top of my right shoulder.
\fi

\ifdefined\chs
``正点''一词在斯托曼嘴里应该是很高评价了。因为正当斯托曼介绍完冷面,我突然觉察到他的眼神飘到了我的右后方。
\fi

\ifdefined\eng
``There is the most exquisite woman sitting just behind you,'' Stallman says.
\fi

\ifdefined\chs
``你后面坐了个很正点的妹子。''斯托曼说。
\fi

\ifdefined\eng
I turn to look, catching a glimpse of a woman's back. The woman is young, somewhere in her mid-20s, and is wearing a white sequined dress. She and her male lunch companion are in the final stages of paying the check. When both get up from the table to leave the restaurant, I can tell without looking, because Stallman's eyes suddenly dim in intensity.
\fi

\ifdefined\chs
我赶紧回头,瞟到了一个女人的背影。这女人很年轻,二十来岁的样子。穿了一件白色裙子,裙子上贴着闪闪发亮的饰品。她和同桌的一位男性已经用完餐,正在结帐。当两人起身离开的时候,我发现了斯托曼眼中瞬间失落的神情。
\fi

\ifdefined\eng
``Oh, no,'' he says. ``They're gone. And to think, I'll probably never even get to see her again.''
\fi

\ifdefined\chs
``啊!别啊!他们要走了。我估计以后也见不到她了。''斯托曼抱怨道。
\fi

\ifdefined\eng
After a brief sigh, Stallman recovers. The moment gives me a chance to discuss Stallman's reputation vis-à-vis the fairer sex. The reputation is a bit contradictory at times. A number of hackers report Stallman's predilection for greeting females with a kiss on the back of the hand.\endnote{See Mae Ling Mak, ``A Mae Ling Story'' (December 17, 1998), \url{http://crackmonkey.org/pipermai l/crackmonkey/1998-December/001777.html}.
So far, Mak is the only person I've found willing to speak on the record in regard to this practice, although I've heard this from a few other female sources. Mak, despite expressing initial revulsion at it, later managed to put aside her misgivings and dance with Stallman at a 1999 LinuxWorld show.} 
\fi

\ifdefined\chs
一声叹息,斯托曼心神回位。这倒给了我一个机会,问问斯托曼的感情生活和审美偏好。因为坊间传言,他可是个轻佻的公子。有些时候,很多传言还有些自相矛盾。很多黑客都声称,斯托曼和女性见面,都会以吻手背的方式来打招呼\endnote{参见Mae Ling Mak于1998年12月17日发表的《A Mae Ling Story》。\url{http://www.crackmonkey.org/pipermail/crackmonkey/1998-December/001777.html}。Mak是目前为止唯一一个愿意和我聊这事,并且公开姓名的人。除了她以外,还有几个女性跟我提到过斯托曼的这种行为。Mak一开始对于斯托曼的行为比较反感。不过在1999年的LinuxWorld上,Mak也抛弃了这种偏见,还和斯托曼一起跳了段舞。}。
\fi

\ifdefined\eng
A May 26, 2000 \textit{Salon.com} article, meanwhile, portrays Stallman as a bit of a hacker lothario. Documenting the free software-free love connection, reporter Annalee Newitz presents Stallman as rejecting traditional family values, telling her, ``I believe in love, but not monogamy.''\endnote{See Annalee Newitz, ``If Code is Free Why Not Me?'', \textit{Salon.com} (May 26, 2000), \url{http://www.salon.com/tech/feature/2000/05/26/free_love/print.html}.}
\fi

\ifdefined\chs
在Salon.com上,有一份2000年5月26日的采访,把斯托曼描写成了一位黑客界的花花公子。文中把自由软件和自由恋爱做类比,作者安娜利·奈唯姿把斯托曼说成是拒绝传统家庭观念的叛逆者,他还说:``我相信爱情,但不相信一夫一妻\endnote{参见Annalee Newitz于2000年五月26日在\textit{Salon.com}上发表的文章《If Code is Free Why Not Me?》。\url{http://www.salon.com/2000/05/26/free_love/}}。''
\fi

\ifdefined\eng
Stallman lets his menu drop a little when I bring this up. ``Well, most men seem to want sex and seem to have a rather contemptuous attitude towards women,'' he says. ``Even women they're involved with. I can't understand it at all.''
\fi

\ifdefined\chs
当我问起这方面问题时,斯托曼把菜单放下后说:``这个⋯⋯很多男人似乎都会不自觉地蔑视女性,看到女性就想到上床一类的事情。就连很多女人也自觉不自觉地融入到男性的这种倾向的思维中,可我无法理解这种态度。''
\fi

\ifdefined\eng
I mention a passage from the 1999 book \textit{Open Sources} in which Stallman confesses to wanting to name the GNU kernel after a girlfriend at the time. The girlfriend's name was Alix, a name that fit perfectly with the Unix developer convention of putting an ``x'' at the end names of operating systems and kernels -- e.g., ``Linux.'' Alix was a Unix system administrator, and had suggested to her friends, ``Someone should name a kernel after me.'' So Stallman decided to name the GNU kernel ``Alix'' as a surprise for her. The kernel's main developer renamed the kernel ``Hurd,'' but retained the name ``Alix'' for part of it.  One of Alix's friends noticed this part in a source snapshot and told her, and she was touched.  A later redesign of the Hurd eliminated that part.\endnote{See Richard Stallman, ``The GNU Operating System and the Free Software Movement,'' \textit{Open Sources} (O'Reilly \& Associates, Inc., 1999): 65. [RMS: Williams interpreted this vignette as suggesting that I am a hopeless romantic, and that my efforts were meant to impress some as-yet-unidentified woman.  No MIT hacker would believe this, since we learned quite young that most women wouldn't notice us, let alone love us, for our programming.  We programmed because it was fascinating.  Meanwhile, these events were only possible because I had a thoroughly identified girlfriend at the time.  If I was a romantic, at the time I was neither a hopeless romantic nor a hopeful romantic, but rather temporarily a successful one.

On the strength of that naive interpretation, Williams went on to compare me to Don Quijote.

For completeness' sake, here's a somewhat inarticulate quote from the first edition: ``I wasn't really trying to be romantic. It was more of a teasing thing. I mean, it was romantic, but it was also teasing, you know? It would have been a delightful surprise.'']}
\fi

\ifdefined\chs
我提到,在1999年,一本名为《开源软件文集:开源革命之声》的书中,斯托曼在一篇文章里提到,当初他想以女朋友的名字来命名GNU操作系统的内核。这位女孩子名叫Alix,因为是以x结尾,所以和Unix界对操作系统内核的命名传统非常契合,比如Linux就遵照了这个命名传统。Alix当初是个Unix系统管理员,而且曾经和她朋友开玩笑地提到过:``应该有个内核以我的名字命名。''斯托曼于是决定把GNU的内核命名为Alix,以此给她一份惊喜。可最后内核的主要开发人员把内核名字改成了Hurd,把Alix作为其中一个模块的名字。Alix的一个朋友看到源码中出现了Alix的名字,于是把这件事情告诉了她。Alix听后非常感动。之后Hurd又经历了几次重新设计,最终还是去除了Alix这个模块\endnote{参见理查德·斯托曼在《开源软件文集:开源革命之声》一书中的文章《GNU操作系统和自由软件运动》。《开源软件文集:开源革命之声》一书已由中国电力出版社于2000年出版,洪峰等人翻译。}。
\fi

\ifdefined\eng
For the first time all morning, Stallman smiles. I bring up the hand kissing. ``Yes, I do do that,'' Stallman says. ``I've found it's a way of offering some affection that a lot of women will enjoy. It's a chance to give some affection and to be appreciated for it.''
\fi

\ifdefined\chs
整个一个上午,斯托曼脸上都挂着笑容。我问起关于吻手背的事情,斯托曼说:``哦,是的。我的确这么做。我觉得这种方式可以给很多女性带来亲切感,她们也都会喜欢。这是个很好的机会,你可以借此拉近彼此距离。''
\fi

\ifdefined\eng
Affection is a thread that runs clear through Richard Stallman's life, and he is painfully candid about it when questions arise. ``There really hasn't been much affection in my life, except in my mind,'' he says. Still, the discussion quickly grows awkward. After a few one-word replies, Stallman finally lifts up his menu, cutting off the inquiry.
\fi

\ifdefined\chs
亲切感是斯托曼生命中永恒的主题。问起这个问题,他非常坦诚:``在我的生命里,确实很少体验到亲切感,只是在我头脑之中还存在着这种感受。''可聊着聊着,话题就变得尴尬。斯托曼几次回应都只是从嘴里蹦出几个单字,再问几个问题,他就举起菜单,插话道:
\fi

\ifdefined\eng
``Would you like some shu mai?'' he asks.
\fi

\ifdefined\chs
``你想吃点烧麦吗?''
\fi

\ifdefined\eng
When the food comes out, the conversation slaloms between the arriving courses. We discuss the oft-noted hacker affection for Chinese food, the weekly dinner runs into Boston's Chinatown district during Stallman's days as a staff programmer at the AI Lab, and the underlying logic of the Chinese language and its associated writing system. Each thrust on my part elicits a well-informed parry on Stallman's part.
\fi

\ifdefined\chs
上菜间隙,谈话越显尴尬。我开始和他闲聊起来,问他关于中餐的问题,问他当初他在人工智能实验室做程序员的时候,每周去唐人街吃饭的问题,还问了他关于中文口语和汉字书写之间的逻辑联系。本想借此绕回主题,可每次尝试,都被斯托曼一套太极功夫挡了回去。
\fi

\ifdefined\eng
``I heard some people speaking Shanghainese the last time I was in China,'' Stallman says. ``It was interesting to hear. It sounded quite different [from Mandarin]. I had them tell me some cognate words in Mandarin and Shanghainese. In some cases you can see the resemblance, but one question I was wondering about was whether tones would be similar. They're not. That's interesting to me, because there's a theory that the tones evolved from additional syllables that got lost and replaced. Their effect survives in the tone. If that's true, and I've seen claims that that happened within historic times, the dialects must have diverged before the loss of these final syllables.''
\fi

\ifdefined\chs
``我上次去中国的时候听到人们说上海话。''斯托曼说,``很有趣,和普通话非常不同。我问他们把同一个词分别用上海话和普通话说一遍。有些时候你能大概听出点联系。我问他们声调是不是也类似,他们说不一样。这让我觉得很有趣。因为曾经有个假说认为,不同声调曾经是不同音素,之后语言不断演化,几个音素被不同声调取代。如果这个假说成立,那么声调不同就意味着这个方言是在音素演变成声调之前就独立发展的。''
\fi

\ifdefined\eng
The first dish, a plate of pan-fried turnip cakes, has arrived. Both Stallman and I take a moment to carve up the large rectangular cakes, which smell like boiled cabbage but taste like potato latkes fried in bacon.
\fi

\ifdefined\chs
第一个上来的菜,是萝卜糕。我和斯托曼都花了些力气,才把这大块的萝卜糕切开。它闻起来像是白灼生菜,吃起来又像煎土豆饼配培根。
\fi

\ifdefined\eng
I decide to bring up the outcast issue again, wondering if Stallman's teenage years conditioned him to take unpopular stands, most notably his uphill battle since 1994 to get computer users and the media to replace the popular term ``Linux'' with ``GNU/Linux.''
\fi

\ifdefined\chs
我决定继续讨论刚才说到一半的话题,我问他,是否觉得自己的童年经历使自己养成了如此特立独行的性格。其中很出名的一个事,就是他从1994年开始,就致力于纠正广大计算机用户和各地媒体的误读,让他们使用GNU/Linux一词来替代Linux。
\fi

\ifdefined\eng
``I believe [being an outcast] did help me [to avoid bowing to popular views],'' Stallman says, chewing on a dumpling. ``I have never understood what peer pressure does to other people. I think the reason is that I was so hopelessly rejected that for me, there wasn't anything to gain by trying to follow any of the fads. It wouldn't have made any difference. I'd still be just as rejected, so I didn't try.''
\fi

\ifdefined\chs
``我觉得,[特立独行的性格]的确让我[避免了对流行观点的盲从]。''斯托曼嚼着一个饺子,继续说道,``很多人都说会感受到旁人观点的压力,而我却从不在意。我觉得我一直在拒绝盲从别人的观点。随波逐流不会有什么好处,不会带来任何改变。因此,我也没太在意别人对我的看法。''
\fi

\ifdefined\eng
Stallman points to his taste in music as a key example of his contrarian tendencies. As a teenager, when most of his high school classmates were listening to Motown and acid rock, Stallman preferred classical music. The memory leads to a rare humorous episode from Stallman's middle-school years. Following the Beatles' 1964 appearance on the Ed Sullivan Show, most of Stallman's classmates rushed out to purchase the latest Beatles albums and singles. Right then and there, Stallman says, he made a decision to boycott the Fab Four.
\fi

\ifdefined\chs
斯托曼以音乐喜好为例,向我讲述他叛逆的倾向。在他少年时期,大多数同龄的中学生都迷恋魔城音乐和摇滚乐,而斯托曼则喜欢听古典乐。他回忆1964年,自己中学生时候,披头士在《埃德·沙利文秀场》栏目出现\endnote{译注:1964年,披头士乐队第一次在美国电视上表演,当时参加的是《埃德·沙利文秀场》这一节目(Ed Sullivan Show)。从此以后披头士在美国也开始火爆。},他的同学们都冲出去买披头士的最新专辑或单曲。而就在那个时候,斯托曼决心抵制披头士音乐。
\fi

\ifdefined\eng
``I liked some of the pre-Beatles popular music,'' Stallman says. ``But I didn't like the Beatles. I especially disliked the wild way people reacted to them. It was like: who was going to have a Beatles assembly to adulate the Beatles the most?''
\fi

\ifdefined\chs
斯托曼说:``我喜欢披头士之前的一些流行乐。可我不喜欢披头士。我更不喜欢人们追求披头士的那份疯狂。当时就好像说:谁能攒齐披头士的所有专辑,谁就是最崇拜披头士的。''
\fi

\ifdefined\eng
When his Beatles boycott failed to take hold, Stallman looked for other ways to point out the herd-mentality of his peers. Stallman says he briefly considered putting together a rock band himself dedicated to satirizing the Liverpool group.
\fi

\ifdefined\chs
斯托曼的抵制行为显然没造成多大影响,他还试图找点其他法子戏弄一下盲目的同龄人。他说,他曾经考虑过要自己组织个乐队,专门调侃披头士。
\fi

\ifdefined\eng
``I wanted to call it Tokyo Rose and the Japanese Beetles.''
\fi

\ifdefined\chs
``我想把这乐队叫做`东京花',或者`日本披头士'。''
\fi

\ifdefined\eng
Given his current love for international folk music, I ask Stallman if he had a similar affinity for Bob Dylan and the other folk musicians of the early 1960s. Stallman shakes his head. ``I did like Peter, Paul and Mary,'' he says. ``That reminds me of a great filk.''
\fi

\ifdefined\chs
我知道斯托曼喜欢世界各地的民族乐,我就问他对民谣歌手鲍勃·迪伦什么看法。斯托曼摇摇头:``我倒是更喜欢彼得,保罗和玛丽。他们让我想起了伟大的诽客(filk)音乐。''
\fi

\ifdefined\eng
When I ask for a definition of ``filk,'' Stallman explains that the term is used in science fiction fandom to refer to the writing of new lyrics for songs.  (In recent decades, some filkers write melodies too.)  Classic filks include ``On Top of Spaghetti,'' a rewrite of ``On Top of Old Smokey,'' and ``Yoda,'' filk-master ``Weird'' Al Yankovic's Star Wars-oriented rendition of the Kinks tune, ``Lola.''
\fi

\ifdefined\chs
我问他什么是诽客音乐,斯托曼解释说是一个在科幻圈子里的术语,专门指代为的现成歌曲重新填词后创作的歌曲。(在最近几年,一些诽客乐作者也会自己谱曲。)传统的诽客乐有《意大利面条之上》(On Top of Spaghetti),这是借用《旧思墨客山之上》(On Top of Old Smokey)的曲调\endnote{译注:《旧思墨客山之上》(On Top of Old Smokey)是美国一首著名的民谣乐。原歌曲讲述主人公在旧思墨客山上失去真爱。《意大利面条之上》(On Top of Spaghetti)是由汤姆·格雷泽(Tom Glazer)填词,讲述主人公吃意大利面条的时候,旁人打喷嚏把顶上的肉丸子吹到了地上。}。还有由诽客乐大师,``奇士''埃尔·杨可维克(``Weird'' Al Yankovic,``奇士''是此人绰号,也被人称``奇士埃尔'')填词,改变自《罗拉》(Lola)的歌曲《尤达》(Yoda),显然,《尤达》是描写《星球大战》里的绝地武士尤达的。``
\fi

\ifdefined\eng
Stallman asks me if I would be interested in hearing the filk. As soon as I say yes, Stallman's voice begins singing in an unexpectedly clear tone, using the tune of ``Blowin' in the Wind'':
\fi

\ifdefined\chs
斯托曼问我是不是想听他唱诽客歌。我说想听。他很快开始唱起来,声音出奇地清晰。用的是《随风飘散》(Blowin' in the Wind)的曲调\endnote{译注:《随风飘散》(Blowin' in the Wind)是鲍勃·迪伦编写的著名民谣歌曲。在电影《阿甘正传》中,珍妮曾在舞台上演唱过该曲。此处的歌词翻译为意译。原文为:

\begin{verse}
How much wood could a woodchuck chuck,\\
If a woodchuck could chuck wood?\\
How many poles could a polak lock,\\
If a polak could lock poles?\\
How many knees could a negro grow,\\
If a negro could grow knees?\\
The answer, my dear,\\
is stick it in your ear.\\
The answer is, stick it in your ear\ldots
\end{verse}

整个歌词并没有什么特殊的意思,主要是一个文字游戏:将一个单词按照发音拆成两个,然后用拆出来的单词组成一些看似有些意义的话。例如,negro(黑人,黑奴)按照发音被拆成了knee(膝盖)和grow(生长)两个词,所以negro就变成了knee grow。本歌词的翻译也基本采用类似原则。
}:
\fi

\ifdefined\eng
\begin{verse}
How much wood could a woodchuck chuck,\\
If a woodchuck could chuck wood?\\
How many poles could a polak lock,\\
If a polak could lock poles?\\
How many knees could a negro grow,\\
If a negro could grow knees?\\
The answer, my dear,\\
is stick it in your ear.\\
The answer is, stick it in your ear\ldots
\end{verse}
\fi

\ifdefined\chs
\begin{verse}
一只土拨鼠要拨多少土,\\
才能称得上是土拨鼠?\\
波兰人要播多少栏目,\\
才配得上波兰这名属?\\
黑奴兄弟啊,你要黑掉多少弓弩,\\
才敢被人称作黑奴。\\
这一切的答案,亲爱的,\\
都掉进了你耳朵,粘到了听小骨,\\
答案啊,都粘到了听小骨\ldots
\end{verse}
\fi

\ifdefined\eng
The singing ends, and Stallman's lips curl into another child-like half smile. I glance around at the nearby tables. The Asian families enjoying their Sunday lunch pay little attention to the bearded alto in their midst.\endnote{For Stallman's own filks, visit \url{http://www.stallman.org/doggerel.html}. To hear Stallman singing ``The Free Software Song,'' visit \url{http://www.gnu.org/music/free-software-song.html}.} After a few moments of hesitation, I finally smile too.
\fi

\ifdefined\chs
曲罢音散,斯托曼唇角又露出了孩子般的笑容。我环视周围,旁边的几个亚洲人正享受着自己周日的饕餮,没把心思放在这位大胡子叔叔身上\endnote{斯托曼自己也创作过诽客歌词,可以在他的个人网站中找到:\url{http://www.stallman.org/doggerel.html}。想听听斯托曼唱歌的话,可以在GNU网站上找到他唱的自由软件之歌:\url{http://www.gnu.org/music/free-software-song.html}。}。犹豫了几秒,我也冲斯托曼笑了笑。
\fi

\ifdefined\eng
``Do you want that last cornball?'' Stallman asks, eyes twinkling. Before I can screw up the punch line, Stallman grabs the corn-encrusted dumpling with his two chopsticks and lifts it proudly. ``Maybe I'm the one who should get the cornball,'' he says.
\fi

\ifdefined\chs
``这个粟米球你还吃吗?''斯托曼眨着眼睛问到。我还没回答,斯托曼就拿筷子夹起粟米球往嘴里送,``那我就不客气了!''他说。
\fi

\ifdefined\eng
The food gone, our conversation assumes the dynamics of a normal interview. Stallman reclines in his chair and cradles a cup of tea in his hands. We resume talking about Napster and its relation to the free software movement. Should the principles of free software be extended to similar arenas such as music publishing? I ask.
\fi

\ifdefined\chs
饭菜吃完,我们准备进入采访的正题。斯托曼端起茶杯,靠在在椅子上。我们继续开始讨论Naspter和自由软件运动的话题。我问,自由软件的原则是否应该扩展到其他领域,比如音乐领域上?
\fi

\ifdefined\eng
``It's a mistake to transfer answers from one thing to another,'' says Stallman, contrasting songs with software programs. ``The right approach is to look at each type of work and see what conclusion you get.''
\fi

\ifdefined\chs
``把一个问题的答案无条件地嫁接到另外一个问题上,这可就不对了。''斯托曼说。说起歌曲音乐和程序,他说:``正确的思路是,根据每类不同的作品做分析,由此再得出相应的结论。''
\fi

\ifdefined\eng
When it comes to copyrighted works, Stallman says he divides the world into three categories. The first category involves ``functional'' works -- e.g., software programs, dictionaries, and textbooks. The second category involves works that might best be described as ``testimonial'' -- e.g., scientific papers and historical documents. Such works serve a purpose that would be undermined if subsequent readers or authors were free to modify the work at will.  It also includes works of personal expression -- e.g., diaries, journals, and autobiographies. To modify such documents would be to alter a person's recollections or point of view, which Stallman considers ethically unjustifiable.  The third category includes works of art and entertainment.
\fi

\ifdefined\chs
当说起版权作品的时候,斯托曼把整个领域分为三个类型。第一个类型,是所谓的``功能类''。软件,辞典,教科书都属于此类。第二个类型,可以称为``证据类''。科技论文,古籍文献都属于此类。这些作品,如果可以被任意修改,则没有了价值。这一类也同样包括对个人感受的记录,比如日记,游记,自传。修改这些作品,则会影响到对某人的回忆或观点。因此斯托曼觉得对此类作品,读者则不能有权修改。第三类,则是``艺术类'',包含了艺术和娱乐作品。
\fi

\ifdefined\eng
Of the three categories, the first should give users the unlimited right to make modified versions, while the second and third should regulate that right according to the will of the original author. Regardless of category, however, the freedom to copy and redistribute noncommercially should remain unabridged at all times, Stallman insists. If that means giving Internet users the right to generate a hundred copies of an article, image, song, or book and then email the copies to a hundred strangers, so be it. ``It's clear that private occasional redistribution must be permitted, because only a police state can stop that,'' Stallman says. ``It's antisocial to come between people and their friends. Napster has convinced me that we also need to permit, must permit, even noncommercial redistribution to the public for the fun of it. Because so many people want to do that and find it so useful.''
\fi

\ifdefined\chs
这三类作品中,对于``功能类''的作品,用户应该无条件地具备修改它们的权利;而对于``证据类''和``艺术类''作品,则要根据作者意愿,才能赋予用户一定的修改权力。斯托曼坚信,无论哪类作品,用户出于非商业目的的复制与分享行为都应该被允许。也许有人会说,一旦允许如此,就会有互联网用户把各种图片,音乐或者书籍复制成百上千份,然后把拷贝发送给几百个陌生人。但斯托曼认为,这些拷贝行为也应该被允许。``很显然,私下偶尔分享这些作品的行为,必须被允许。只有警察国家才会禁止这些行为。''斯托曼说。``把一个人和他的朋友隔离,这是反社会的行为。而Napster事件则让我觉得,我们不仅要允许私下朋友之间的分享,更应该,也必须要允许用户,出于非商业目的,把这些作品的拷贝分享给公众。因为太多的人希望如此,并且觉得这么做非常有价值。''
\fi

\ifdefined\eng
When I ask whether the courts would accept such a permissive outlook, Stallman cuts me off.
\fi

\ifdefined\chs
我问他,法庭是不是会接受这种放任的观点。斯托曼打断了我说:
\fi

\ifdefined\eng
``That's the wrong question,'' he says. ``I mean now you've changed the subject entirely from one of ethics to one of interpreting laws. And those are two totally different questions in the same field. It's useless to jump from one to the other. How the courts would interpret the existing laws is mainly in a harsh way, because that's the way these laws have been bought by publishers.''
\fi

\ifdefined\chs
``这是个错误的问题。你把这个话题从道德伦理的领域,转移成了法律领域里的释法问题。这是一个话题的两个完全不同的方面。从一个反面跳到另一个是没有任何意义的。如今法庭对于这些现有法律的解释,往往会非常严苛,因为这些法律当初都是被出版商们极度拥护才被制定出来的。''
\fi

\ifdefined\eng
The comment provides an insight into Stallman's political philosophy: just because the legal system currently backs up businesses' ability to treat copyright as the software equivalent of land title doesn't mean computer users have to play the game according to those rules. Freedom is an ethical issue, not a legal issue. ``I'm looking beyond what the existing laws are to what they should be,'' Stallman says. ``I'm not trying to draft legislation. I'm thinking about what should the law do? I consider the law prohibiting the sharing of copies with your friend the moral equivalent of Jim Crow. It does not deserve respect.''
\fi

\ifdefined\chs
这段话反映出了斯托曼的政治哲学:现有的法律系统保护各大公司,让他们能利用现有版权法保护自己的软件,但这并不意味着计算机用户必须得遵守所谓的规则。自由本身是个伦理问题,不是法律问题。他说:``我们需要超越现有法律,暂时忘记现在实际是什么样,去思考我们究竟应该要什么样。我不是要立法,而是在考虑法律应该做些什么。我觉得,禁止朋友之间分享拷贝的法律,和种族歧视法律一样,不该得到尊重。''
\fi

\ifdefined\eng
The invocation of Jim Crow prompts another question. How much influence or inspiration does Stallman draw from past political leaders? Like the civil-rights movement of the 1950s and 1960s, his attempt to drive social change is based on an appeal to timeless values: freedom, justice, and fair play.
\fi

\ifdefined\chs
说起种族歧视法,我又问了另外一个问题:他是否被曾经的那些政治领袖影响或鼓舞?比如五六十年代的民权运动,斯托曼的政治思想也是基于类似的普世价值:自由,正义,公平竞争。
\fi

\ifdefined\eng
Stallman divides his attention between my analogy and a particularly tangled strand of hair. When I stretch the analogy to the point where I'm comparing Stallman with Dr. Martin Luther King, Jr., Stallman, after breaking off a split end and popping it into his mouth, cuts me off.
\fi

\ifdefined\chs
斯托曼这会儿被一撮纠缠在一起的头发分了心,没注意到我的问题。我之后把这个问题更具体化,把斯托曼和马丁·路德·金做比较。斯托曼这时刚刚弄断头发分叉处的死结,把死结扔进了嘴里。他打断我的话,说道:
\fi

\ifdefined\eng
``I'm not in his league, but I do play the same game,'' he says, chewing.
\fi

\ifdefined\chs
``我们不是一个团体,但我们做着同样的事情。''斯托曼一边嚼着东西一边说。
\fi

\ifdefined\eng
I suggest Malcolm X as another point of comparison. Like the former Nation of Islam spokesperson, Stallman has built up a reputation for courting controversy, alienating potential allies, and preaching a message favoring self-sufficiency over cultural integration.
\fi

\ifdefined\chs
我接着提起另外一位民权活动家,马尔科姆·艾克斯(Malcolm X),把他的观点和斯托曼的做类比。马尔科姆曾是美国非洲裔人的新宗教组织``伊斯兰国度''的领袖。和他一样,斯托曼也喜欢谈论有争议的话题,疏远潜在的盟友,并且传播着一套自恰的信息,无视与现有文化体系的整合。
\fi

\ifdefined\eng
Chewing on another split end, Stallman rejects the comparison. ``My message is closer to King's message,'' he says. ``It's a universal message. It's a message of firm condemnation of certain practices that mistreat others. It's not a message of hatred for anyone. And it's not aimed at a narrow group of people. I invite anyone to value freedom and to have freedom.''
\fi

\ifdefined\chs
斯托曼嚼完嘴里的东西,表示他拒绝这种类比:``我要传达的内容更类似马丁·路德·金的思想。这是一个普世价值观。它谴责一些伤及他人的行为。我的思想并不是要人去憎恨谁,也不是针对一小撮人。我希望每个人都能珍视自由,也能拥有自由。''
\fi

\ifdefined\eng
Many criticize Stallman for rejecting handy political alliances; some psychologize this and describe it as a character trait. In the case of his well-publicized distaste for the term ``open source,'' the unwillingness to participate in recent coalition-building projects seems understandable. As a man who has spent the last two decades stumping on the behalf of free software, Stallman's political capital is deeply invested in the term. Still, comments such as the ``Han Solo'' comparison at the 1999 LinuxWorld have only reinforced Stallman's reputation, amongst those who believe virtue consists of following the crowd, as a disgruntled mossback unwilling to roll with political or marketing trends.
\fi

\ifdefined\chs
很多人都指责斯托曼,说他经常拒绝触手可得的政治盟友。有人从心理学的角度,认为这是斯托曼的一个性格特征。他拒绝使用已经被大众接受的``开源''一词,拒绝与这个圈子的人合作项目。他为自由软件奋斗了二十多年,他的名字已经深深地和自由软件绑在了一起,他的政治资本也是要不断投资在这四个字上。在LinuxWorld上领奖时,他的那翻类比,更加深了人们对他的这一印象。茫茫众生之间,很多人早已把跟风赶潮当作一种美德。而斯托曼在这些人之中,则更显得头脑守旧,不肯在政治大趋势和市场面前有半点让步。
\fi

\ifdefined\eng
``I admire and respect Richard for all the work he's done,'' says Red Hat president Robert Young, summing up Stallman's paradoxical political conduct. ``My only critique is that sometimes Richard treats his friends worse than his enemies.''
\fi

\ifdefined\chs
红帽公司总裁罗伯特·杨(Robert Young)总结理查德·斯托曼看似矛盾的政治行为时,说道:``我崇拜也尊敬理查德和他所做的一切。我对他唯一的批评就是,有些时候,他对待朋友甚至比对待敌人还要无情。''
\fi

\ifdefined\eng
\ifdefined\vtwo
[RMS: The term ``friends'' only partly fits people such as Young, and companies such as Red Hat.  It applies to some of what they did, and do: for instance, Red Hat contributes to development of free software, including some GNU programs.  But Red Hat does other things that work against the free software movement's goals -- for instance, its versions of GNU/Linux contain nonfree software.  Turning from deeds to words, referring to the whole system as ``Linux'' is unfriendly treatment of the GNU Project, and promoting ``open source'' instead of ``free software'' rejects our values.  I could work with Young and Red Hat when we were going in the same direction, but that was not often enough to make them possible allies.]
\fi
\fi

\ifdefined\chs
\ifdefined\vtwo
[RMS:
``朋友''一词只能部分地适用与罗伯特·杨和红帽公司的身上。他们做的一些事情,确实称得上是朋友。比如,红帽公司为自由软件开发贡献了很多,这其中包括很多GNU工程的程序。但红帽公司也同样做了一些违背自由软件运动主旨的事情。比如,他们的GNU/Linux发行版中,就包含一些非自由软件。从措词上看,他们把整个系统称为Linux,是对GNU工程的无视。他们宣扬``开源''的同时,不考虑自由软件,也是违背我们的价值观。如果红帽公司和罗伯特·杨能与我们同道,我乐意和他们一起共同协作。可很多时候,并非如此。]
\fi
\fi

\ifdefined\eng
Stallman's reluctance to ally the free software movement with other political causes is not due to lack of interest in them.  Visit his offices at MIT, and you instantly find a clearinghouse of left-leaning news articles covering civil-rights abuses around the globe. Visit his personal web site, \url{stallman.org}, and you'll find attacks on the Digital Millennium Copyright Act, the War on Drugs, and the World Trade Organization.  Stallman explains, ``We have to be careful of entering the free software movement into alliances with other political causes that substantial numbers of free software supporters might not agree with.  For instance, we avoid linking the free software movement with any political party because we do not want to drive away the supporters and elected officials of other parties.''
\fi

\ifdefined\chs
斯托曼这种独行侠的作风,并非是他不关注其他政治人物。在他麻省理工学院的办公室里,你能一下子看到很多左派文章,记录着全球各地公民权被侵犯的事件。登陆他的个人网站,stallman.org,你可以看到很多文章,反对千禧年版权法案,抵制反毒品运动,抵制世贸组织等等。斯托曼解释说:``我们必须要谨慎地为自由软件运动挑选政治盟友。因为有些自由软件支持者也许并不支持有些运动。比如,我们会避免把自由软件运动和其他政党建立联系,因为我们不希望这样的行为去驱散那些自由软件的支持者。''
\fi

\ifdefined\eng
Given his activist tendencies, I ask, why hasn't Stallman sought a larger voice? Why hasn't he used his visibility in the hacker world as a platform to boost his political voice?  \ifdefined\vtwo [RMS: But I do -- when I see a good opportunity.  That's why I started \url{stallman.org}.]\fi
\fi

\ifdefined\chs
鉴于斯托曼这种活动家的倾向,我问他为什么不尝试发出更大的声音?为什么不借助他在黑客圈的影响力,去推进他的政治主张?\ifdefined\vtwo [RMS: 可我其实已经在这么做,但只有在时机成熟的时候才如此。这也是我为什么建立了 stallman.org 这个个人网站] \fi
\fi

\ifdefined\eng
Stallman lets his tangled hair drop and contemplates the question for a moment. \ifdefined\vtwo [RMS:  My quoted response doesn't fit that question.  It does fit a different question, ``Why do you focus on free software rather than on the other causes you believe in?''  I suspect the question I was asked was more like that one.] \fi
\fi

\ifdefined\chs
斯托曼整理了一下头发,思考了一阵。 \ifdefined\vtwo [RMS: 我下面的回答有些所问非所答。对于我下面的话,更合适的问题也许是:``你为什么专注于自由软件,而不是其他你关注的问题?''我记得当时被问的问题似乎是这个。] \fi
\fi

\ifdefined\eng
``I hesitate to exaggerate the importance of this little puddle of freedom,'' he says. ``Because the more well-known and conventional areas of working for freedom and a better society are tremendously important. I wouldn't say that free software is as important as they are. It's the responsibility I undertook, because it dropped in my lap and I saw a way I could do something about it. But, for example, to end police brutality, to end the war on drugs, to end the kinds of racism we still have, to help everyone have a comfortable life, to protect the rights of people who do abortions, to protect us from theocracy, these are tremendously important issues, far more important than what I do. I just wish I knew how to do something about them.''
\fi

\ifdefined\chs
答道:``软件自由,仅仅是自由一隅,冰山一角。我不希望过分强调这一角。因为还有更多的领域需要自由,这一点十分重要。和这些自由相比,自由软件则显得微不足道了。自由软件是我的一份责任,因为这项自由恰好落在我熟悉的领域之中。可除此以外,还有很多极其重要的问题需要关注:警察暴力,刑讯逼供;人们对某些毒品的偏见;种族歧视;如何让人们过上安逸的生活;如何让有堕胎经历的人免遭歧视;如何避免极权政治等等。这些都和我们的生活息息相关,它们都比我的工作重要。我只是希望能了解如何在这些问题上做点什么。''
\fi

\ifdefined\eng
Once again, Stallman presents his political activity as a function of personal confidence. Given the amount of time it has taken him to develop and hone the free software movement's core tenets, Stallman is hesitant to believe he can advance the other causes he supports.
\fi

\ifdefined\chs
斯托曼之后的一席话,让我觉得他利用了各种政治活动增进了个人信心。他在自由软件上花了大量时间和精力,很多活动,他虽然支持,但若真投入其中,则要犹豫再三,要权衡自己的时间和精力。
\fi

\ifdefined\eng
``I wish I knew how to make a major difference on those bigger issues, because I would be tremendously proud if I could, but they're very hard and lots of people who are probably better than I am have been working on them and have gotten only so far,'' he says. ``But as I see it, while other people were defending against these big visible threats, I saw another threat that was unguarded. And so I went to defend against that threat. It may not be as big a threat, but I was the only one there [to oppose it].''
\fi

\ifdefined\chs
``我真心希望我能为解决这些问题,贡献出自己的一份力量。倘若我的贡献能让这些问题有所好转,我将引以为豪。可这些问题困难至极,多少比我更优秀的人,前赴后继,才只能做到今天这地步。当初我审视这一切,我发现依然有一处自由无人守护,于是我才投身至此。软件自由也许并非举足轻重,但守卫之人,仅我一个。''
\fi

\ifdefined\eng
Chewing a final split end, Stallman suggests paying the check. Before the waiter can take it away, however, Stallman pulls out a white-colored dollar bill and throws it on the pile. The bill looks so clearly counterfeit, I can't help but pick it up and read it. Sure enough, it did not come from the US Mint. Instead of bearing the image of a George Washington or Abe Lincoln, the bill's front side bears the image of a cartoon pig. Instead of the United States of America, the banner above the pig reads, ``Untied Status of Avarice.'' The bill is for zero dollars,\ifdefined\vtwo \endnote{RMS: Williams was mistaken to call this bill ``counterfeit.'' It is legal tender, worth zero dollars for payment of any debt.  Any U.S. government office will convert it into zero dollars' worth of gold.} \fi and when the waiter picks up the money, Stallman makes sure to tug on his sleeve.
\fi

\ifdefined\chs
吃完最后一口,斯托曼要结帐。服务员还没来收钱,他又掏出一张白色纸币。一眼看去,这张纸币显然是张假币,我忍不住拾起来仔细端详了一下。它自然不是美国铸币局制造的。上面画的不是华盛顿,不是林肯,而是一头卡通猪。上面写的也不是``美利坚合众国''(United States of America),而是``放纵贪婪中''(Untied Status of Avarice)。这张纸币一钱不值\ifdefined\vtwo \endnote{RMS: Williams was mistaken to call this bill ``counterfeit.'' It is legal tender, worth zero dollars for payment of any debt.  Any U.S. government office will convert it into zero dollars' worth of gold.}。\fi 服务员来的时候,斯托曼拽着他的袖子:
\fi

\ifdefined\eng
``I added an extra zero to your tip,'' Stallman says, yet another half smile creeping across his lips.
\fi

\ifdefined\chs
``我又给你加了零美元的小费。''斯托曼说着,又露出了他标志的微笑。
\fi

\ifdefined\eng
The waiter, uncomprehending or fooled by the look of the bill, smiles and scurries away.
\fi

\ifdefined\chs
这位服务员可能没有领会什么意思,或者真的被这假币迷惑住了。他也回敬笑了一下,快步走开了。
\fi

\ifdefined\eng
``I think that means we're free to go,'' Stallman says.
\fi

\ifdefined\chs
``看来我们可以离开了。''斯托曼说。
\fi

\theendnotes
\setcounter{endnote}{0}
