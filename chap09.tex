%% Copyright (c) 2002, 2010 Sam Williams
%% Copyright (c) 2010 Richard M. Stallman
%% Permission is granted to copy, distribute and/or modify this
%% document under the terms of the GNU Free Documentation License,
%% Version 1.3 or any later version published by the Free Software
%% Foundation; with no Invariant Sections, no Front-Cover Texts, and
%% no Back-Cover Texts. A copy of the license is included in the
%% file called ``gfdl.tex''.

%% ALL DONE

\chapter{\ifdefined\eng
The GNU General Public License
\fi
\ifdefined\chs
GNU通用公共许可证
\fi} \label{chapter:gpl}


\ifdefined\eng
By the spring of 1985, Richard Stallman had settled on the GNU Project's first milestone -- a Lisp-based free software version of Emacs. To meet this goal, however, he faced two challenges. First, he had to rebuild Emacs in a way that made it platform independent. Second, he had to rebuild the Emacs Commune in a similar fashion.
\fi

\ifdefined\chs
1985年新春伊始,理查德·斯托曼完成了GNU工程中第一个完整的里程碑式的软件:用Lisp语言编写的Emacs编辑器。为了能让别的用户可以真正的自由使用这个软件,他面临着两大挑战。首先,他需要改写出一个跨平台的Emacs版本;其次,他需要基于早期的``Emacs公社''探索一种全新的发布方式。
\fi

\ifdefined\eng
The dispute with UniPress had highlighted a flaw in the Emacs Commune social contract. Where users relied on Stallman's expert insight, the Commune's rules held. In areas where Stallman no longer held the position of alpha hacker -- pre-1984 Unix systems, for example -- individuals and companies were free to make their own rules.
\fi

\ifdefined\chs
与UniPress之间的纠纷暴露出了Emacs公社契约中的一些不足:这些条款都过度依赖斯托曼个人在这个社区中的地位。对于那些斯托曼所没有触及到的领域,比如1984年以前的Unix系统,其它个人和公司都可以制定他们自己的规则。
\fi

\ifdefined\eng
The tension between the freedom to modify and the freedom to exert authorial privilege had been building before GOSMACS. The Copyright Act of 1976 had overhauled U.S. copyright law, extending the legal protection of copyright to software programs. According to Section 102(b) of the Act, individuals and companies now possessed the ability to copyright the ``expression'' of a software program but not the ``actual processes or methods embodied in the program.''\endnote{See Hal Abelson, Mike Fischer, and Joanne Costello, ``Software and Copyright Law,'' updated version (1998). \\\url{http://www.swiss.ai.mit.edu/6805/articles/int-prop/software-copyright.html}} Translated, programmers and companies had the ability to treat software programs like a story or song. Other programmers could take inspiration from the work, but to make a direct copy or nonsatirical derivative, they first had to secure permission from the original creator. Although the new law guaranteed that even programs without copyright notices carried copyright protection, programmers quickly asserted their rights, attaching coypright notices to their software programs.
\fi

\ifdefined\chs
在GOSMACS之前,修改软件的自由与软件作者的特权之间就已经形成了一种矛盾的关系。1976年修订的美国版权法案把法案适用范围扩展到了软件作品上。根据该法案的第102(b)节的规定,个人或公司可以对软件程序的``表现方式''声明版权,但不能对程序中的``实际处理过程和方法''声明版权。\endnote{参见哈尔·艾贝尔森(Hal Abelson)、麦克·费雪(Mark Fischer)和乔安妮·科斯特洛(Joanne Costello)的``软件与版权法''1998年修订版。\\\url{http://www.swiss.ai.mit.edu/6805/articles/int-prop/software-copyright.html}}换句话说,程序员与软件公司可以像对待一个故事或一首歌那样来对待一个软件,可以从一个已有的软件中获取灵感,但是在没有得到原作者允许的前提下,不能直接抄袭或在它的基础上直接衍生出新的版本。尽管这部新的法律规定了软件中即使没有包含版权声明也同样受到法律保护,程序员们还是纷纷开始在他们的程序中添加上了版权声明,声张自己的权力。
\fi

\ifdefined\eng
At first, Stallman viewed these notices with alarm. Rare was the software program that didn't borrow source code from past programs, and yet, with a single stroke of the president's pen, Congress had given programmers and companies the power to assert individual authorship over communally built programs. It also injected a dose of formality into what had otherwise been an informal system. Even if hackers could demonstrate how a given program's source-code bloodlines stretched back years, if not decades, the resources and money that went into battling each copyright notice were beyond most hackers' means. Simply put, disputes that had once been settled hacker-to-hacker were now settled lawyer-to-lawyer. In such a system, companies, not hackers, held the automatic advantage.
\fi

\ifdefined\chs
刚开始,斯托曼把这些版权声明看成是一个不好的兆头。那时候,大部分程序都会从其它以前的程序中借鉴一些代码,然而,总统的大笔一挥,议会就给予了程序员和软件公司把社区开发的软件变为私有软件的权力。这就使得原来游走在灰色地带的事情变得公开合法化。即使黑客们有能力去追朔一段代码最初的来源——也许没有办法找到几十年前的源头,但至少可以往前查几年,但是对每一段代码进行这种追查所要花费的时间和金钱都远远超出了大部分黑客所能接受的范围。简单的说,这项法案使得原先黑客之间的较量变成了律师之间的较量。在这样的法律框架下,软件公司相对于黑客们来说占据了天然的有利地位。
\fi

\ifdefined\eng
Proponents of software copyright had their counter-arguments: without copyright, works might otherwise slip into the public domain. Putting a copyright notice on a work also served as a statement of quality. Programmers or companies who attached their name to the copyright attached their reputations as well. Finally, it was a contract, as well as a statement of ownership. Using copyright as a flexible form of license, an author could give away certain rights in exchange for certain forms of behavior on the part of the user. For example, an author could give away the right to suppress unauthorized copies just so long as the end user agreed not to create a commercial offshoot.
\fi

\ifdefined\chs
然而,软件版权的支持者也有他们相反的观点:如果没有版权的保护,程序员的作品可能会在不经意间进入共有领域。同时,在自己的作品中加入版权声明常常也象征着对软件质量的负责,程序员或软件公司在代码中加入了署名的版权声明也就意味着把自己的信誉与这份代码绑定在一起。另一方面,版权声明也可以看成是一种合约或是所有权的声明。软件作者可以把版权声明用作是一种具有弹性的许可证,通过版权声明放弃部分权利,换取用户方某种形式的行为。比如,软件的作者可以允许人们随意复制软件,但是不得把这些副本用于商业用途。
\fi

\ifdefined\eng
It was this last argument that eventually softened Stallman's resistance to software copyright notices. Looking back on the years leading up to the GNU Project, Stallman says he began to sense the beneficial nature of copyright sometime around the release of Emacs 15.0, the last significant pre-GNU Project upgrade of Emacs. ``I had seen email messages with copyright notices plus simple `verbatim copying permitted' licenses,'' Stallman recalls. ``Those definitely were [an] inspiration.''
\fi

\ifdefined\chs
这些观点最终使斯托曼反对软件版权声明的态度慢慢发生了变化。回顾GNU工程的历史,斯托曼说他直到差不多Emacs 15.0发布的时候才逐渐意识到使用版权来保护软件的好处,Emacs 15.0是前GNU的最后一个Emacs大版本。斯托曼回忆道:``那时我看到有些人在电子邮件中也开始使用版权声明,还加上了一行``允许原文复制''的许可声明,这触发了我的灵感。''
\fi

\ifdefined\eng
For Emacs 15, Stallman drafted a copyright that gave users the right to make and distribute copies. It also gave users the right to make modified versions, but not the right to claim sole ownership of those modified versions, as in the case of GOSMACS.
\fi

\ifdefined\chs
从Emacs 15开始,斯托曼在软件中加入了一项版权声明,允许用户随意复制并分发软件。他还允许软件的用户修改软件,但不允许用户把修改过的软件作为自己的作品并享有排他特权,就跟GOMACS一样。
\fi

\ifdefined\eng
Although helpful in codifying the social contract of the Emacs Commune, the Emacs 15 license remained too ``informal'' for the purposes of the GNU Project, Stallman says. Soon after starting work on a GNU version of Emacs, Stallman began consulting with the other members of the Free Software Foundation on how to shore up the license's language. He also consulted with the attorneys who had helped him set up the Free Software Foundation.
\fi

\ifdefined\chs
斯托曼认为,Emacs 15的版权声明虽然为Emacs公社树立了一个很好的样板,但它的内容还``不够正式'',不适合用作GNU工程的软件许可证。在开始编写GNU版本的Emacs时,斯托曼开始与自由软件基金会的成员一起探讨应该如何来完善许可证的文本。他还咨询了帮助他创建了自由软件基金会的律师。
\fi

\ifdefined\eng
Mark Fischer, a Boston attorney specializing in intellectual-property law, recalls discussing the license with Stallman during this period. ``Richard had very strong views about how it should work,'' Fischer says, ``He had two principles. The first was to make the software absolutely as open as possible. The second was to encourage others to adopt the same licensing practices.''
\fi

\ifdefined\chs
马克·费雪(Mark Fischer)是波士顿一名专注于知识产权法的律师,他回忆起当年与斯托曼讨论许可证问题的情景:``对于许可证的用处,斯托曼有非常清醒的认识。他有两条原则,第一条是要让软件变得尽可能的开放,第二条是尽可能鼓励其它人接受和使用类似的许可证协议。''
%%马克·费雪是波士顿一名专注于知识产权法的律师,他回忆起当年与斯托曼讨论许可证问题的情景:``对于许可证的用处,斯托曼有非常清醒的认识。他有两条基本原则,第一条是要让软件变得尽可能的开放(费雪说这番话时,也许是受到了开源运动中一些表述的影响,斯找曼本人从来没有打算让软件变得``开放''),``第二条是尽可能鼓励其它人接受和使用这种对软件不多加限制的软件许可证。''许可证中很多的条款都是为了符合第二条原则而设立的。
\fi

\ifdefined\eng
Encouraging others to adopt the same licensing practices meant closing off the escape hatch that had allowed privately owned versions of Emacs to emerge. To close that escape hatch, Stallman and his free software colleagues came up with a solution: users would be free to modify GNU Emacs just so long as they published their modifications. In addition, the resulting ``derivative'' works would also have carry the same GNU Emacs License.
\fi

\ifdefined\chs
鼓励其它人接受和使用类似的许可证就意味着不再允许私有的Emacs版本出现。为了实现这一目标,斯托曼与他的自由软件同僚们找了一种解决方案:用户们可以随意的修改GNU Emacs,只要求他们把他们的改动都公布出来;并且,这些衍生出来的软件作品需要与GNU Emacs采用相同的许可证。
\fi

\ifdefined\eng
The revolutionary nature of this final condition would take a while to sink in. At the time, Fischer says, he simply viewed the GNU Emacs License as a simple contract. It put a price tag on GNU Emacs' use. Instead of money, Stallman was charging users access to their own later modifications. That said, Fischer does remember the contract terms as unique.
\fi

\ifdefined\chs
这最后一个条款是一个革命性的创造,人们需要一段时间才能真正理解它。费雪回忆说,在那个时候,他只是把这个条款看成是GNU Emacs与用户之间达成的一笔交易,也是相当于是给GNU Emacs打上了价格标签。斯托曼并不向软件的最终用户收取费用,但用户需要把自己对软件所作的修改贡献出来作为他们所支付的价格。费雪认为,这条合同条款非常特别。
\fi

\ifdefined\eng
``I think asking other people to accept the price was, if not unique, highly unusual at that time,'' he says.
\fi

\ifdefined\chs
``在那个时候,让软件用户接受这样的一种软件`价格'即便不是独一无二,也是数一数二的。''费雪说。
\fi

\ifdefined\eng
The GNU Emacs License made its debut when Stallman finally released GNU Emacs in 1985. Following the release, Stallman welcomed input from the general hacker community on how to improve the license's language. One hacker to take up the offer was future software activist John Gilmore, then working as a consultant to Sun Microsystems. As part of his consulting work, Gilmore had ported Emacs over to SunOS, the company's in-house version of Unix. In the process of doing so, Gilmore had published the changes as per the demands of the GNU Emacs License. Instead of viewing the license as a liability, Gilmore saw it as clear and concise expression of the hacker ethos. ``Up until then, most licenses were very informal,'' Gilmore recalls.
\fi

\ifdefined\chs
1985年,斯托曼发布GNU Emacs的新版本时,GNU Emacs通用公共许可证第一次与公众见面。随后,斯托曼收到了很多来自黑客社区的对改进许可证用语的修改意见。约翰·吉尔摩(John Gilmore)是最早接受这种软件许可证的黑客之一,他后来成为了一名软件活动家并为Sun公司(Sun Microsystems)提供咨询服务。作为他工作的一部分,吉尔摩把Emacs移植到了SunOS上,SunOS是Sun公司内部的一种Unix发行版。在移植的过程中,吉尔摩按照GNU Emacs许可证的要求公开了他所作的修改。吉尔摩并不认为遵守GNU Emacs许可证是对他工作的一种限制,相反的,他认为这种许可证正好体现了黑客的精神气质。``不过,在那个时候,很多许可证的条款都还不是很正式。''吉尔摩回忆道。
\fi

\ifdefined\eng
As an example of this informality, Gilmore cites a copyright notice for trn, a Unix utility. Written by Larry Wall, future creator of the Perl programming language, patch made it simple for Unix programmers to insert source-code fixes -- ``patches'' in hacker jargon -- into any large program. Recognizing the utility of this feature, Wall put the following copyright notice in the program's accompanying README file:
\fi

\ifdefined\chs
吉尔摩引用了一个Unix工具软件trn的版权声明,从中也可以看出那时的许可证有多么的不正式。trn是拉里·华尔(Larry Wall)的一个作品,他后来还开发了Perl语言。有了Patch工具,Unix程序员可以很容易地把自己的代码以补丁的形式插入到一个大型软件中。华尔很清楚这一点,所以他把下面的版权声明写入了软件的README文档:
\fi

\ifdefined\eng
\begin{quote}
Copyright (c) 1985, Larry Wall\\
You may copy the trn kit in whole or in part as long as you don't try to make money off it, or pretend that you wrote it.\endnote{See Trn Kit README, \url{http://stuff.mit.edu/afs/sipb/project/trn/src/trn-3.6/README}.}
\end{quote}
\fi

\ifdefined\chs
\begin{quote}
版权所有 (C) 1985,拉里·华尔\\
你可以任意的复制trn软件包或它的一部分,只要你保证不以它来牟利,并且不得假装你是这个软件的作者。\endnote{参见Trn软件包的README文档。\url{http://stuff.mit.edu/afs/sipb/project/trn/src/trn-3.6/README}。}
\end{quote}
\fi

\ifdefined\eng
Such statements, while reflective of the hacker ethic, also reflected the difficulty of translating the loose, informal nature of that ethic into the rigid, legal language of copyright. In writing the GNU Emacs License, Stallman had done more than close up the escape hatch that permitted proprietary offshoots. He had expressed the hacker ethic in a manner understandable to both lawyer and hacker alike.
\fi

\ifdefined\chs
如此的声明,在映射着黑客的精神和气质的同时,也映射出要把这种松散、非正式的精神转换为严谨的软件版权法律语言的困难之处。在撰写GNU Emacs许可证时,斯托曼做了很多的工作,保证许可证的表述中不存在任何可能被别人利用,从而把这个软件变成私有软件的漏洞。他以一种律师和黑客都可以理解的方式,体现出了黑客的精神。
\fi

\ifdefined\eng
It wasn't long, Gilmore says, before other hackers began discussing ways to ``port'' the GNU Emacs License over to their own programs. Prompted by a conversation on Usenet, Gilmore sent an email to Stallman in November, 1986, suggesting modification:
\fi

\ifdefined\chs
吉尔摩回忆说,没过多久,就有一些黑客们开始讨论如何把GNU Emacs许可证移植到他们自己的其它软件作品上。1986年11月的一天,吉尔摩受到Usenet上一个帖子的启发,写了一封电子邮件给斯托曼,建议对GNU Emacs许可证做一些修改:
\fi

\ifdefined\eng
\begin{quote}
You should probably remove ``EMACS'' from the license and replace it with ``SOFTWARE'' or something. Soon, we hope, Emacs will not be the biggest part of the GNU system, and the license applies to all of it.\endnote{See John Gilmore, quoted from email to author.}
\end{quote}
\fi

\ifdefined\chs
\begin{quote}
或许你可以考虑把许可证中的``EMACS''字眼替换为``软件''或其它类似的名词。在不久的将来,Emacs也许就不再是GNU系统中最大的一部分,但这种许可证可以适用于整个系统。 \endnote{节选自约翰·吉尔摩寄给作者的电子邮件。}
\end{quote}
\fi

\ifdefined\eng
Gilmore wasn't the only person suggesting a more general approach. By the end of 1986, Stallman himself was at work with GNU Project's next major milestone, a source-code debugger, and was looking for ways to revamp the Emacs license so that it might apply to both programs. Stallman's solution: remove all specific references to Emacs and convert the license into a generic copyright umbrella for GNU Project software. The GNU General Public License, GPL for short, was born.
%%Gilmore wasn't the only person suggesting a more general approach. By the end of 1986, Stallman himself was at work with GNU Project's next major milestone, the source-code debugger GDB.  To release this, he had to modify the GNU Emacs license so it applied to GDB instead of GNU Emacs.  It was not a big job, but it was an opening for possible errors.  In 1989, Stallman figured out how to remove the specific references to Emacs, and express the connection between the program code and the license solely in the program's source files.  This way, any developer could apply the license to his program without changing the license. The GNU General Public License, GNU GPL for short, was born.  The GNU Project soon made it the official license of all existing GNU programs.
\fi

\ifdefined\chs
吉尔摩并不是唯一一个建议用这种更通用做法的人。1986年底,斯托曼正着手开发GNU工程中的下一个里程碑——源代码调试器,与此同时,他也同样在考虑如何修改Emacs许可证使它可以适用于Emacs和这个新的调试器。斯托曼的解决方案是:移除许可证中所有的Emacs字眼并把它转变为一个保护所有GNU软件版权的许可证。GNU 通用公共许可证就这样诞生了,缩写是GPL。
\fi

\ifdefined\eng
In fashioning the GPL, Stallman followed the software convention of using decimal numbers to indicate prototype versions and whole numbers to indicate mature versions. Stallman published Version 1.0 of the GPL in 1989 (a project Stallman was developing in 1985), almost a full year after the release of the GNU Debugger, Stallman's second major foray into the realm of Unix programming. The license contained a preamble spelling out its political intentions:
\fi

\ifdefined\chs
在打造GPL时,斯托曼使用了软件版本号的表示方式,用小数表示原型版本的许可证,用整数表示成熟的版本。1989年,斯托曼打入Unix阵营的第二个重要作品,也就是GNU调试器发布的一年后,他发布了GPL的1.0版本(斯托曼从1985年就开始了GPL这个项目)。这个许可证包含了一个表明其政治态度的序言:
\fi

\ifdefined\eng
\begin{quote}
The General Public License is designed to make sure that you have the freedom to give away or sell copies of free software, that you receive source code or can get it if you want it, that you can change the software or use pieces of it in new free programs; and that you know you can do these things.

To protect your rights, we need to make restrictions that forbid anyone to deny you these rights or to ask you to surrender the rights. These restrictions translate to certain responsibilities for you if you distribute copies of the software, or if you modify it.\endnote{See Richard Stallman, et al., ``GNU General Public License: Version 1,'' (February, 1989), \url{http://www.gnu.org/licenses/old-licenses/gpl-1.0.html}.}
\end{quote}
\fi

\ifdefined\chs
\begin{quote}
通用公共许可证被设计成确保你拥有分发或出售自由软件的权力,确保你可以获取软件的源代码,确保你可以修改软件或在别的自由软件中使用这个软件,并确保你了解你拥有以上权力。

为了保证你的权力,我们禁止任何人剥夺你的以上权力或要求你放弃这些权力。这些限制条件也作为你分发或修改这个软件时所必须承担的责任。\endnote{参见理查得·斯托曼等,``GNU通用公共许可证:第一版'' 1989年2月。\url{http://www.gnu.org/licenses/old-licenses/gpl-1.0.html}}
\end{quote}
\fi

\ifdefined\eng
In fashioning the GPL, Stallman had been forced to make an additional adjustment to the informal tenets of the old Emacs Commune. Where he had once demanded that Commune members publish any and all changes, Stallman now demanded publication only in instances when programmers circulated their derivative versions in the same public manner as Stallman. In other words, programmers who simply modified Emacs for private use no longer needed to send the source-code changes back to Stallman. In what would become a rare compromise of free software doctrine, Stallman slashed the price tag for free software. Users could innovate without Stallman looking over their shoulders just so long as they didn't bar Stallman and the rest of the hacker community from future exchanges of the same program.
\fi

\ifdefined\chs
与此同时,斯托曼也不得不对原来那个不太正式的``Emacs公社''协议进行修改。在``Emacs公社''的协议中,社区的成员需要公开任何对Emacs所进行的改动,而现在,程序员只需跟斯托曼一样以相同协议公开衍生作品的代码即可。也就是说,程序员修改了Emacs并仅供自己个人使用时,就无须把代码发给斯托曼了。斯托曼粉碎了自由软件价格标签,免得它们破坏了自由软件的精神教义。用户可以不受斯托曼的限制对软件进行创新工作,并保证了整个社区都可以获得相同的软件版本。
\fi

\ifdefined\eng
Looking back, Stallman says the GPL compromise was fueled by his own dissatisfaction with the Big Brother aspect of the original Emacs Commune social contract. As much as he liked peering into other hackers' systems, the knowledge that some future source-code maintainer might use that power to ill effect forced him to temper the GPL.
\fi

\ifdefined\chs
回首往事,斯托曼说,GPL中所做出的妥协很大程度上是因为他个人对原先``Emacs公社''中存在的``老大哥''角色感到不满。当他对黑客社区了解的越深入,他越感觉到未来的源代码维护人员可以用他们的权力迫使他让GPL做出妥协。
%%斯托曼说,作出这样的改动是因为他本人对于原来Emacs公社条款中自己的``老大哥''角色感到不满。仅管以前有很多用户会向他发来很多有用的改动,他仍然觉得强调要求用户这么做是不公平的。
\fi

\ifdefined\eng
``It was wrong to require people to publish all changes,'' says Stallman. ``It was wrong to require them to be sent to one privileged developer. That kind of centralization and privilege for one was not consistent with a society in which all had equal rights.''
\fi

\ifdefined\chs
``要求用户公开所有的改动是不对的,更何况还要求他们把这么多的改动都发给某一个拥有特权的开发者。这种中央集权的方式,与整个社区人人平等的氛围格格不入。''
\fi

\ifdefined\eng
As hacks go, the GPL stands as one of Stallman's best. It created a system of communal ownership within the normally proprietary confines of copyright law. More importantly, it demonstrated the intellectual similarity between legal code and software code. Implicit within the GPL's preamble was a profound message: instead of viewing copyright law with suspicion, hackers should view it as yet another system begging to be hacked.
\fi

\ifdefined\chs
历史证明,GPL是斯托曼最伟大的创造之一。它在现有的版权法律的框架中创造了一种平等的社区系统。更重要的是,它向世人展现了法律文书与软件代码之间的相似性。GPL的序言其实暗示了一个信息:与其用怀疑的眼光看待版权法,不如用黑客的精神渗入到这个系统中。
\fi

\ifdefined\eng
``The GPL developed much like any piece of free software with a large community discussing its structure, its respect or the opposite in their observation, needs for tweaking and even to compromise it mildly for greater acceptance,'' says Jerry Cohen, another attorney who helped Stallman with the creation of the license. ``The process worked very well and GPL in its several versions has gone from widespread skeptical and at times hostile response to widespread acceptance.''
\fi

\ifdefined\chs
``GPL的设计过程与其它自由软件开发的过程非常相似,由一个大的社区共同来讨论它的结构、它的条款。社区成员的意见可能会相左,所以还需要对它的条款进行调整甚至进行妥协,这样才能更容易为大众所接受。''另一位曾帮助斯托曼一起设计GPL的律师杰里·科恩(Jerry Cohen)说,``这样的设计方式非常有效,GPL的各个版本从被人们怀疑逐渐变为广为接受。''
\fi

\ifdefined\eng
In a 1986 interview with \textit{Byte} magazine, Stallman summed up the GPL in colorful terms. In addition to proclaiming hacker values, Stallman said, readers should also ``see it as a form of intellectual jujitsu, using the legal system that software hoarders have set up against them.''\endnote{See David Betz and Jon Edwards, ``Richard Stallman discusses his public-domain [sic] Unix-compatible software system with BYTE editors,'' \textit{BYTE} (July, 1996). (Reprinted on the GNU Project web site: \url{http://www.gnu.org/gnu/byte-interview.html}.)
 
This interview offers an interesting, not to mention candid, glimpse at Stallman's political attitudes during the earliest days of the GNU Project. It is also helpful in tracing the evolution of Stallman's rhetoric. 

Describing the purpose of the GPL, Stallman says, ``I'm trying to change the way people approach knowledge and information in general. I think that to try to own knowledge, to try to control whether people are allowed to use it, or to try to stop other people from sharing it, is sabotage.'' 

Contrast this with a statement to the author in August 2000: ``I urge you not to use the term `intellectual property' in your thinking. It will lead you to misunderstand things, because that term generalizes about copyrights, patents, and trademarks. And those things are so different in their effects that it is entirely foolish to try to talk about them at once. If you hear somebody saying something about intellectual property, without quotes, then he's not thinking very clearly and you shouldn't join.''} Years later, Stallman would describe the GPL's creation in less hostile terms. ``I was thinking about issues that were in a sense ethical and in a sense political and in a sense legal,'' he says. ``I had to try to do what could be sustained by the legal system that we're in. In spirit the job was that of legislating the basis for a new society, but since I wasn't a government, I couldn't actually change any laws. I had to try to do this by building on top of the existing legal system, which had not been designed for anything like this.''
\fi

\ifdefined\chs
1986年,斯托曼接受\textit{Byte}杂志采访时,把GPL总结为一些富有趣味的条款。除了宣传黑客价值观,斯托曼说,读者还应该``把它看成是智力的较量。通过软件开发者所建立的法律系统来对抗他们。''\endnote{参考大卫·贝茨(David Betz)和乔恩·爱德华兹(Jon Edwards)的``理查德·斯托曼与BYTE编辑关于他的公有领域Unix兼容软件系统的访谈''一文,\textit{Byte}杂志,1996年7月。(GNU工程网站上转载的地址:\url{http://www.gnu.org/gnu/byte-interview.html})

这篇访谈从一种有趣、直率的视角展示了斯托曼在GNU工程早期的一些政治态度,对于研究斯托曼推广GNU理念时采用的各种方式的演化过程具有很大的意义。

对于创造GPL的目的,斯托表示:``从宏观角度来说,我希望能改变人们获取知识和信息的方式。试图独自占有知识、控制其它人使用知识或是禁止其它人分享知识的行为,都是野蛮的行为。``

2000年,斯托曼在与本书作者的一次对话中表示:''我希望你在思考问题时不要再使用`知识产权'这个词语。这个词语具有很大的误导性,因为它混淆了版权、专利和商标这些不同的概念。这些概念之间存在着天壤之别,把它们混为一谈是一种愚蠢的行为。如果你听到有人在谈论知识产权,而且还不是引述其它人的观点,那就你不应该与这个人进行讨论,因为他根本就没有把问题想清楚。``
}几年后,斯托曼开始使用一些不那么敌对的条款来描述GPL。``我在考虑那些有关道德意识、政治意识和法律意识相关的问题,''他说,``我需要设法让它可以融入现有的法律系统。从本质上来说,这项工作是为一个新社会立法的过程,但是因为我不是政府,所以事实上我并不能改变法律。我需要设法在现有法律框架上去设计它,这在以前的设计中从来没有过。''
\fi

\ifdefined\eng
About the time Stallman was pondering the ethical, political, and legal issues associated with free software, a California hacker named Don Hopkins mailed him a manual for the 68000 microprocessor. Hopkins, a Unix hacker and fellow science-fiction buff, had borrowed the manual from Stallman a while earlier. As a display of gratitude, Hopkins decorated the return envelope with a number of stickers obtained at a local science-fiction convention. One sticker in particular caught Stallman's eye. It read, ``Copyleft (L), All Rights Reversed.'' Following the release of the first version of GPL, Stallman paid tribute to the sticker, nicknaming the free software license ``Copyleft.'' Over time, the nickname and its shorthand symbol, a backwards ``C,'' would become an official Free Software Foundation synonym for the GPL.
\fi

\ifdefined\chs
正当斯托曼在思考自由软件有关道德、政治和法律上的问题时,一位来自加州名叫堂·霍普金斯(Don Hopkins)的黑客给他寄了一份68000微处理器的手册。霍普金斯是一名Unix黑客,同时也是一名科幻小说爱好者,这本手册是他早些时候向斯托曼借阅的。为了表达对斯托曼的感激之情,他在信封上贴了一些从当地科幻小说爱好者社团获得的一些贴纸。其中,有一张贴纸上的文字吸引了斯托曼的眼球:``Copyleft (L),保留所有权利。''在GPL的第一个版本发布以后,斯托曼称赞这张贴纸给了他灵感,并把``Copyleft''一词作为自由软件许可证的呢称。以后,Copyleft这个名字以及它的标识——一个倒着写的C,成为了GPL在自由软件基金的官方标识。
\fi

\ifdefined\eng
The German sociologist Max Weber once proposed that all great religions are built upon the ``routinization'' or ``institutionalization'' of charisma. Every successful religion, Weber argued, converts the charisma or message of the original religious leader into a social, political, and ethical apparatus more easily translatable across cultures and time.
\fi

\ifdefined\chs
德国著名社会学家马克斯·韦伯(Max Web)曾经说过,所有伟大的宗教都是建立在对神的感召力的``常规化''或``制度化''的基础上的。他认为,每一种成功的宗教,都是把神的感召力或这个宗教最初创始人的想法转换成一种社会化、政治化和道德化的东西,这样这种宗教的精神才可能长时间在不同的文化之间进行传播。
\fi

\ifdefined\eng
While not religious per se, the GNU GPL certainly qualifies as an interesting example of this ``routinization'' process at work in the modern, decentralized world of software development. Since its unveiling, programmers and companies who have otherwise expressed little loyalty or allegiance to Stallman have willingly accepted the GPL bargain at face value. A few have even accepted the GPL as a preemptive protective mechanism for their own software programs. Even those who reject the GPL contract as too compulsory, still credit it as influential.
\fi

\ifdefined\chs
如果不考虑其宗教本质,GNU GPL可以看成是现代的去中心化软件开发模式中一种典型的``常规化''的例子。程序员和软件公司即使不信奉斯托曼的那套理论,也会很乐意去接受GPL表面上所呈现出来的普适价值。一部分人把GPL看成是对自己软件作品先发制人的保护机制而接受它,另一部分人虽然由于GPL的条款过于苛刻而反对它,但依然把它看成是一种很有影响力的担保方式。
\fi

\ifdefined\eng
One hacker falling into this latter group was Keith Bostic, a University of California employee at the time of the GPL 1.0 release. Bostic's department, the Computer Systems Research Group (SRG), had been involved in Unix development since the late 1970s and was responsible for many key parts of Unix, including the TCP/IP networking protocol, the cornerstone of modern Internet communications. By the late 1980s, AT\&T, the original owner of the Unix brand name, began to focus on commercializing Unix and began looking to the Berkeley Software Distribution, or BSD, the academic version of Unix developed by Bostic and his Berkeley peers, as a key source of commercial technology.
\fi

\ifdefined\chs
基思·博斯蒂克(Keith Bostic)就是属于后一个群体中的一名黑客,GPL 1.0发布的时候,他是加州大学(University of California)的一名雇员。博斯蒂克所就职的计算机系统研究小组(Computer System Research Group,CSRG),从二十世纪七十年代就开始从事Unix系统的研究,并且负责Unix中很多重要组件的开发。比如,作为现代Internet通信基石的TCP/IP网络协议就是出自这个小组。到了八十年代,Unix商标所有者AT\&T开始专注于Unix的商业化,他们选择了由博斯蒂克和他在伯克利的同事们一起开发的``伯克利软件套件''(Berkeley Software Distribution,BSD)这个Unix的学术分枝作为未来商用Unix的主要技术来源。
\fi

\ifdefined\eng
Although the Berkeley BSD source code was shared among researchers and commercial programmers with a source-code license, this commercialization presented a problem. The Berkeley code was intermixed with proprietary AT\&T code. As a result, Berkeley distributions were available only to institutions that already had a Unix source license from AT\&T. As AT\&T raised its license fees, this arrangement, which had at first seemed innocuous, became increasingly burdensome.
\fi

\ifdefined\chs
虽然伯克利BSD的源代码在研究人员和商业软件程序员群体中广为分享,但是要把它商业化仍然是一个大问题。因为伯克利的代码中也混入了很多AT\&T的私有代码,所以伯克利套件只能在那些已经向AT\&T购买了Unix源代码许可的机构使用。当AT\&T提高了它的软件许可费用后,这种看上去无辜的授权协议,越来越成为一种负担。
\fi

\ifdefined\eng
Hired in 1986, Bostic had taken on the personal project of porting BSD over to the Digital Equipment Corporation's PDP-11 computer. It was during this period, Bostic says, that he came into close interaction with Stallman during Stallman's occasional forays out to the west coast. ``I remember vividly arguing copyright with Stallman while he sat at borrowed workstations at CSRG,'' says Bostic. ``We'd go to dinner afterward and continue arguing about copyright over dinner.''
\fi

\ifdefined\chs
1986年,博斯蒂克开始进行一个把BSD移植到DEC公司(Digital Equipment Corporation)的PDP-11计算机上的个人项目。博斯蒂克回忆说,在这段时间里,斯托曼有时会来西海岸,所以他有机会与斯托曼进行一些密切的接触。``我直到现在还清晰得记得当时的情景:斯托曼坐在我们研究小组一台借来的工作站上与我争论有关软件版权的问题。争论了一段时间,我们就一起去吃晚餐,然后在晚餐过程中继续争论有关版权的问题。''博斯蒂克不无得意地说。
\fi

\ifdefined\eng
The arguments eventually took hold, although not in the way Stallman would have liked. In June, 1989, Berkeley separated its networking code from the rest of the AT\&T-owned operating system and distributed it under a University of California license. The contract terms were liberal. All a licensee had to do was give credit to the university in advertisements touting derivative programs.\endnote{The University of California's ``obnoxious advertising clause'' would later prove to be a problem. Looking for a less restrictive alternative to the GPL, some hackers used the University of California, replacing ``University of California'' with the name of their own instution. The result: free software programs that borrowed from dozens of other programs would have to cite dozens of institutions in advertisements. In 1999, after a decade of lobbying on Stallman's part, the University of California agreed to drop this clause.} In contrast to the GPL, proprietary offshoots were permissible. Only one problem hampered the license's rapid adoption: the BSD Networking release wasn't a complete operating system. People could study the code, but it could only be run in conjunction with other proprietary-licensed code.
\fi

\ifdefined\chs
这样的争论最终停止了,但并不是以一种斯托曼所喜欢的方式。1989年6月,伯克利把Unix中网络相关的代码从AT\&T拥有版权的那部分代码中剥离出来,并以加州大学许可证的形式发布。许可证的条款非常的开明,代码的使用方只需在衍生代码中注明原始代码出自于加州大学即可。\endnote{不过,加州大学这种类似于广告的条款让人生厌,并且在后来也被证明是存在问题的。有些人为了寻找一种比GPL宽松的许可证,就直接把加州大学许可证中的``加州大学''一词换成自己所在的机构。这造成的结果就是很多自由软件因为整合了很多不同机构开发的组件,就不得不把这些机构的证可证一一列出,看上去就像一堆杂乱的广告。1999年,在斯托曼坚持十年的不断的游说下,加州大学同意放弃这个条款。}与GPL不同,在加州大学许可证中,对代码的商业化使用是被允许的。唯一制约这种许可证广为应用的原因是:BSD的网络部分并不是一个完整的操作系统。人家可以学习这份代码,但它只能与其它私有代码放到一起才能真正运行起来。
\fi

\ifdefined\eng
Over the next few years, Bostic and other University of California employees worked to replace the missing components and turn BSD into a complete, freely redistributable operating system. Although delayed by a legal challenge from Unix Systems Laboratories -- the AT\&T spin-off that retained ownership of the Unix brand name -- the effort would finally bear fruit in the early 1990s. Even before then, however, many of the Berkeley utilities would make their way into Stallman's GNU Project.
\fi

\ifdefined\chs
在接下来的几年中,博斯蒂克和其它加州大学的同事们一起,重新编写BSD中所缺少的组件,把BSD变为一个完整的可以自由分发的操作系统。虽然这项工作因为AT\&T下属的Unix系统实验室(Unix System Laboratories)对于Unix这个品牌的所有权而陷入了一些法律上的困境,但是到了九十年代初,这个项目几乎已经完成了。甚至在这之前,很多伯克利开发的工具软件已经开始进入到斯托曼的GNU工程中。
\fi

\ifdefined\eng
``I think it's highly unlikely that we ever would have gone as strongly as we did without the GNU influence,'' says Bostic, looking back. ``It was clearly something where they were pushing hard and we liked the idea.''
\fi

\ifdefined\chs
博斯蒂克回忆说:``如果没有GNU的影响,我想我们不可能如此坚定的走得那么远。GNU中有一种力量在推动我们的前进,那就是它让我们很欣赏的理念。''
\fi

\ifdefined\eng
By the end of the 1980s, the GPL was beginning to exert a gravitational effect on the free software community. A program didn't have to carry the GPL to qualify as free software -- witness the case of the BSD utilities -- but putting a program under the GPL sent a definite message. ``I think the very existence of the GPL inspired people to think through whether they were making free software, and how they would license it,'' says Bruce Perens, creator of Electric Fence, a popular Unix utility, and future leader of the Debian GNU/Linux development team. A few years after the release of the GPL, Perens says he decided to discard Electric Fence's homegrown license in favor of Stallman's lawyer-vetted copyright. ``It was actually pretty easy to do,'' Perens recalls.
\fi

\ifdefined\chs
到了八十年代未,GPL开始在自由软件社区中产生了如地心引力般的效果。一个程序要成为自由软件并不一定要以GPL发布,比如那些BSD的工具软件,但是如果以GPL来发布,就会更直接的表明自己作为自由软件身份。布鲁斯·佩伦斯(Bruce Perens)说:``我觉得GPL的存在,激发了人们去思考自己的软件作品是否可以成为自由软件,以及应该以什么样的许可证来发布它。'' 佩伦斯是一个知名的Unix工具软件Electric Fence的开发者,也是Debian GNU/Linux开发团队的领头人。在GPL发布后的几年,佩伦斯说他决定废弃Electric Fence所使用的自己土法炮制的软件许可证而改用斯托曼所创造的、由律师审查过的许可证。他回忆说:``事实上这是非常容易做到的。''
\fi

\ifdefined\eng
Rich Morin, the programmer who had viewed Stallman's initial GNU announcement with a degree of skepticism, recalls being impressed by the software that began to gather under the GPL umbrella. As the leader of a SunOS user group, one of Morin's primary duties during the 1980s had been to send out distribution tapes containing the best freeware or free software utilities. The job often mandated calling up original program authors to verify whether their programs were copyright protected or whether they had been consigned to the public domain. Around 1989, Morin says, he began to notice that the best software programs typically fell under the GPL license. ``As a software distributor, as soon as I saw the word GPL, I knew I was home free,'' recalls Morin.
\fi

\ifdefined\chs
作为一名程序员,李奇·莫林对斯托曼最初的GNU宣言在一定程度上是抱有怀疑的,但是他很惊讶的发现各种软件开始慢慢聚集到了GPL的大伞之下。作为一名SunOS用户组的领头人,莫林在80年代期间的一项重要工作就是分发一些载有优质免费软件和自由软件工具的磁带。为了完成这项工作,他需要去联系软件的原始作者,询问他们的作品是否有版权保护,以及他们是否愿意把这个软件放入公有领域。莫林说,到了1989年左右,很多出色的软件程序都开始使用GPL许可证了。``作为一名软件分发人员,我一看到GPL这个词,就知道我不用担心授权的问题了。''莫林回忆说。
\fi

\ifdefined\eng
To compensate for the prior hassles that went into compiling distribution tapes to the Sun User Group, Morin had charged recipients a convenience fee. Now, with programs moving over to the GPL, Morin was suddenly getting his tapes put together in half the time, turning a tidy profit in the process. Sensing a commercial opportunity, Morin rechristened his hobby as a business: Prime Time Freeware.
\fi

\ifdefined\chs
为了补偿在Sun用户组中制作分发磁带的成本,莫林会向订阅者收取少量的费用。现在,这些程序都改用GPL了,莫林收集软件并制作磁带只需要花以前一半的时间,从此就有了一点小小的赢利。莫林有着灵敏的商业嗅觉,自此,他把自己的爱好变成了一项事业,并创办了Prime Time Freeware公司。
\fi

\ifdefined\eng
Such commercial exploitation was completely within the confines of the free software agenda. ``When we speak of free software, we are referring to freedom, not price,'' advised Stallman in the GPL's preamble. By the late 1980s, Stallman had refined it to a more simple mnemonic: ``Don't think free as in free beer; think free as in free speech.''
\fi

\ifdefined\chs
这样的商业探索完全是在自由软件推广计划框架之中的。斯托曼在GPL的序言中说``当我们说自由软件这个词的时候,我们在说的是自由,而不是免费\endnote{译注:自由软件的英文是Free Software,Free这个词在英文中既有``自由''的意思,也有``免费''的意思。}''。到了80年代后期,斯托曼用一句更容易记忆的话来表述这个观点``不要把`Free'理解成是`免费啤酒'中的`免费',它是`自由演讲'中的`自由'''。
\fi

\ifdefined\eng
For the most part, businesses ignored Stallman's entreaties. Still, for a few entrepreneurs, the freedom associated with free software was the same freedom associated with free markets. Take software ownership out of the commercial equation, and you had a situation where even the smallest software company was free to compete against the IBMs and DECs of the world.
\fi

\ifdefined\chs
但是从总体上来看,商业界还是无视了斯托曼的诉求。对于少数企业家来说,自由软件所关注的自由依然与自由市场中的自由是一样的。在商业市场中如果不考虑软件的所有权,你就算是一个很小的软件公司,也可以与IBM和DEC这样的大公司自由竞争。
\fi

\ifdefined\eng
One of the first entrepreneurs to grasp this concept was Michael Tiemann, a software programmer and graduate student at Stanford University. During the 1980s, Tiemann had followed the GNU Project like an aspiring jazz musician following a favorite artist. It wasn't until the release of the GNU C Compiler in 1987, however, that he began to grasp the full potential of free software. Dubbing GCC a ``bombshell,'' Tiemann says the program's own existence underlined Stallman's determination as a programmer.
\fi

\ifdefined\chs
迈克尔·蒂曼(Michael Tiemann)是最初少数几个抓住了这个理念的企业家之一,他是一名程序员,同时也是斯坦福大学的一名研究生。在80年代,蒂曼曾经一度像一名爵士乐发烧友追随一名他所喜爱的艺术家一样追随着GNU工程。1987年GNU C编译器的发布,让他开始真正认识到自由软件内在的巨大潜力。GCC的诞生可以看成是一个爆炸性的新闻,蒂曼觉得,GCC的出现进一步巩固了斯托曼作为一个传奇程序员的地位。
\fi

\ifdefined\eng
``Just as every writer dreams of writing the great American novel, every programmer back in the 1980s talked about writing the great American compiler,'' Tiemman recalls. ``Suddenly Stallman had done it. It was very humbling.''
\fi

\ifdefined\chs
``就像所有的作家都梦想要写出一部伟大的小说一样,80年代时,每个程序员都在梦想要写出一个伟大的编译器。'' 蒂曼回忆说,``斯托曼以迅雷不及掩耳之势完成了这项工作,这实在是很让其他的程序员蒙羞。''
\fi

\ifdefined\eng
``You talk about single points of failure, GCC was it,'' echoes Bostic. ``Nobody had a compiler back then, until GCC came along.''
\fi

\ifdefined\chs
``你在说的事其实是一种所谓的单点故障,GCC就是其中一个,''博斯蒂克回应说,``那时候没有别的编译器,GCC就是唯一的一个。''
\fi

\ifdefined\eng
Rather than compete with Stallman, Tiemann decided to build on top of his work. The original version of GCC weighed in at 110,000 lines of code, but Tiemann recalls the program as surprisingly easy to understand. So easy in fact that Tiemann says it took less than five days to master and another week to port the software to a new hardware platform, National Semiconductor's 32032 microchip. Over the next year, Tiemann began playing around with the source code, creating a native compiler for the C+ programming language. One day, while delivering a lecture on the program at Bell Labs, Tiemann ran into some AT\&T developers struggling to pull off the same thing.
\fi

\ifdefined\chs
在蒂曼看来,与其与斯托曼正面竞争,不如在他的作品基础上进行二次开发。GCC最早的版本大概有110000行代码,但是在蒂曼眼里,这些代码都很容易理解。他只用了不到五天的时间就掌握了这些代码,并且只花了一周时间就把它移植到了另一个新的硬件平台上——国家半导体公司(National Semiconductor)的32032微处理器上。次年,蒂曼就开始进一步修改GCC的源代码,并开发了一个原生的C++语言的编译器。有一天,他在贝尔实验室教授一门编程课程时,蒂曼发现有些AT\&T的开发者还在努力地做着相同的事情。
\fi

\ifdefined\eng
``There were about 40 or 50 people in the room, and I asked how many people were working on the native code compiler,'' Tiemann recalls. ``My host said the information was confidential but added that if I took a look around the room I might get a good general idea.''
\fi

\ifdefined\chs
``教室里大概有四五十人,我问他们有多少人在参与开发原生的编译器,''蒂曼回忆说,``主持人说具体的开发人员的人数是保密的,但是如果看看这个教室里的人,就可以得出一个大概的结论。''
\fi

\ifdefined\eng
It wasn't long after, Tiemann says, that the light bulb went off in his head. ``I had been working on that project for six months,'' Tiemann says. I just thought to myself, whether it's me or the code this is a level of efficiency that the free market should be ready to reward.''
\fi

\ifdefined\chs
蒂曼继续回忆说,不久以后,他就灵光一闪有了个想法:``我已经在这个项目上投入了半年的时间。我开始思考,能如此高效的开发出新的编译器,这到底是我自己的功劳,还是这份代码的功劳?谁更应该在自由市场中获得嘉奖?''
\fi

\ifdefined\eng
Tiemann found added inspiration in the \textit{GNU Manifesto}, which, while excoriating the greed of some software vendors, encourages other vendors to consider the advantages of free software from a consumer point of view. By removing the power of monopoly from the commerical software question, the GPL makes it possible for the smartest vendors to compete on the basis of service and consulting, the two most profit-rich corners of the software marketplace.
\fi

\ifdefined\chs
蒂曼从《\textit{GNU宣言}》中获得了更多灵感,这虽然会让一部分软件开发商的利益受损,但是它鼓励其它软件开发商更多从消费者的角度出发,更好的利用自由软件。虽然GPL让商业软件无法垄断市场,但是它让一些更有眼光的开发商可以从提供服务和咨询的角度去开拓业务、开展竞争,而这些其实是软件市场中利润最高的几个领域之一。
\fi

\ifdefined\eng
In a 1999 essay, Tiemann recalls the impact of Stallman's Manifesto. ``It read like a socialist polemic, but I saw something different. I saw a business plan in disguise.''\endnote{See Michael Tiemann, ``Future of Cygnus Solutions: An Entrepreneur's Account,'' \textit{Open Sources} (O'Reilly \& Associates, Inc., 1999): 139, \url{http://www.oreilly.com/catalog/opensources/book/tiemans.html}.}
\fi

\ifdefined\chs
在1999年的一篇文章中,蒂曼回忆起斯托曼的宣言带来的影响。``它听上去像是一个社会主义者的呐喊,但是它们还是大不相同的。我在里面发现了一个隐蔽的商业计划。'' \endnote{参考O'Reilly \& Associates, Inc.于1999年出版的《\textit{开源}》一书中第139页,迈克尔·蒂曼所写的``Cygnus Solutions公司的未来''一文。\url{http://www.oreilly.com/catalog/opensources/book/tiemans.html}}
\fi

\ifdefined\eng
Teaming up with John Gilmore, another GNU Project fan, Tiemann launched a software consulting service dedicated to customizing GNU programs. Dubbed Cygnus Support, the company signed its first development contract in February, 1990. By the end of the year, the company had \$725,000 worth of support and development contracts.
\fi

\ifdefined\chs
约翰·吉尔摩也是一位GNU工程的粉丝,蒂曼和他一起成立了一个软件咨询服务公司,专注于提供对GNU程序的定制化服务,这就是后来的Cygnus Support公司。1990年2月,Cygnus Support与客户签订了第一份软件开发合同。同年年底,该公司已经签订了价值超过72.5万美元的软件支持和开发合同。
\fi

\ifdefined\eng
GNU Emacs, GDB, and GCC were the ``big three'' of developer-oriented tools, but they weren't the only ones developed by Stallman during the GNU Project's first half decade. By 1990, Stallman had also generated GNU versions of the Bourne Shell (rechristened the Bourne Again Shell, or BASH), YACC (rechristened Bison), and awk (rechristened gawk). Like GCC , every GNU program had to be designed to run on multiple systems, not just a single vendor's platform. In the process of making programs more flexible, Stallman and his collaborators often made them more useful as well.
\fi

\ifdefined\chs
GNU Emacs、GDB和GCC是三个最为重要的开发工具,不过它们并不是斯托曼在GNU工程的最初五年时间中开发的唯一软件。截止1990年,斯托曼还开发了GNU版本的Bourne Shell(改名为Bourne Again Shell, BASH),YACC(改名为Bison)和awk(改名为gawk)。与GCC一样,这些GNU程序都被设计为可以在各种系统上运行,而不是局限于某一个开发商的平台。斯托曼和他的同事们不但把程序做得更具弹性,也把它们做得越来越有用。
\fi

\ifdefined\eng
Recalling the GNU universalist approach, Prime Time Freeware's Morin points to a critical, albeit mundane, software package called hello. ``It's the hello world program which is five lines of C, packaged up as if it were a GNU distribution,'' Morin says. ``And so it's got the Texinfo stuff and the configure stuff. It's got all the other software engineering goo that the GNU Project has come up with to allow packages to port to all these different environments smoothly. That's tremendously important work, and it affects not only all of [Stallman's] software, but also all of the other GNU Project software.''
\fi

\ifdefined\chs
Prime Time Freeware公司的莫林提到了一个很简单但很重要的自由软件包:hello,这个软件包体现了GNU软件大一统的设计理念。莫林说:``hello软件包其实就是一个只有五行代码的`hello world' C语言程序,但这个程序按照GNU发行包的标准进行了打包。里面包含了标准的Texinfo的文档和configuration配置脚本。这个软件包为其它的软件开发者提供了一个很好的示例,展示了GNU工程是如何让软件包可以平滑的移植到不同的环境中的。这件事非常重要,这不但影响到所有斯托曼的软件作品,也影响到其它所有的GNU工程的软件。''
\fi

\ifdefined\eng
According to Stallman, improving software programs was secondary to building them in the first place. ``With each piece I may or may not find a way to improve it,'' said Stallman to Byte. ``To some extent I am getting the benefit of reimplementation, which makes many systems much better. To some extent it's because I have been in the field a long time and worked on many other systems. I therefore have many ideas to bring to bear.''\endnote{See Richard Stallman, \textit{BYTE} (1986).}
\fi

\ifdefined\chs
在斯托曼看来,开发新的软件是第一要务,而改进软件则是位于第二位的。斯托曼在接受\textit{Byte}杂志采访时说:``对于某个软件作品来说,我可以去改进它。但有时重新实现整个软件也许是更好的做法,因为这样会让整个系统变得更好。从某种程度上来说,我在这个领域已经工作了很多年,并且接触过很多不同的系统,因此,我有很多想法可以对其产生积极的影响。''\endnote{参考\textit{BYTE}杂志1986年对理查德·斯托曼的采访。}
\fi

\ifdefined\eng
Nevertheless, as GNU tools made their mark in the late 1980s, Stallman's AI Lab-honed reputation for design fastidiousness soon became legendary throughout the entire software-development community.
\fi

\ifdefined\chs
不管怎么说,GNU的工具在80年代后期发挥出了它的优势,斯托曼所在的人工智能实验室也以能设计出严密的程序而在整个软件开发社区中建立了良好的声誉。
\fi

\ifdefined\eng
Jeremy Allison, a Sun user during the late 1980s and programmer destined to run his own free software project, Samba, in the 1990s, recalls that reputation with a laugh. During the late 1980s, Allison began using Emacs. Inspired by the program's community-development model, Allison says he sent in a snippet of source code only to have it rejected by Stallman.
\fi

\ifdefined\chs
杰瑞米·埃里森(Jeremy Allison),20世纪80年代一名Sun的用户,同时也是一名程序员,是Samba这个自由软件项目的维护者。他在90年代时曾经嘲笑过GNU的工具。在80年代后期,埃里森开始使用Emacs,并对它的社区开发模式产生了兴趣。当时,埃里森故意给斯托曼发去了一段很可能会被他拒绝的代码。
\fi

\ifdefined\eng
``It was like the \textit{Onion} headline,'' Allison says. ``\hspace{0.01in}`Child's prayers to God answered: No.'\hspace{0.01in}''
\fi

\ifdefined\chs
埃里森说:''这就像是\textit{The Onion}的头条新闻,孩子们在祈祷,但是上帝说`不'''。
\fi

\ifdefined\eng
Stallman's growing stature as a software programmer, however, was balanced by his struggles as a project manager. Although the GNU Project moved from success to success in creation of developer-oriented tools, its inability to generate a working kernel-the central ``traffic cop'' program in all Unix systems that determines which devices and applications get access to the microprocessor and when-was starting to elicit grumbles as the 1980s came to a close. As with most GNU Project efforts, Stallman had started kernel development by looking for an existing program to modify. According to a January 1987 ``Gnusletter,'' Stallman was already working to overhaul TRIX, a Unix kernel developed at MIT.
\fi

\ifdefined\chs
斯托曼作为一名程序员的名气越来越大,但他也在努力尝试转型为一名项目经理。尽管GNU工程在开发各种系统工具上捷报频传,它还是没有能够开发出一个真正可用的内核。所谓``内核'',是指在所有的Unix系统中起到``交通警察''作用的核心程序,它决定了各种设备和程序应该如何去使用处理器和各种资源。到了80年代末,GNU工程因为没有自主开发的内核引发了越来多的抱怨。跟其它GNU工程的软件一样,斯托曼先是试图寻找一个现有的程序,通过修改它来开展内核的开发工作。根据1987年1月的``GNusletter''所描述的,斯托曼已经开始尝试修改TRIX,TRIX是一个麻省理工学院开发的Unix内核。
\fi

\ifdefined\eng
A review of GNU Project ``GNUsletters'' of the late 1980s reflects the management tension. In January, 1987, Stallman announced to the world that the GNU Project was working to overhaul TRIX, a Unix kernel developed at MIT. A year later, in February of 1988, the GNU Project announced that it had shifted its attentions to Mach, a lightweight ``micro-kernel'' developed at Carnegie Mellon. All told, however, official GNU Project kernel development wouldn't commence until 1990. \endnote{See ``HURD History.''\url{http://www.gnu.org/software/hurd/history.html}}
\fi

\ifdefined\chs
回顾80年代末GNU工程的``GNUsletter''中的新闻,软件开发管理上的压力在那个时候已经渐渐显现出来。1987年1月,斯托曼宣布GNU工程正在尝试修改TRIX这个由MIT开发的Unix内核。一年以后,1988年2月,GNU工程又宣布他们已经把注意力转Mach,一个由卡内基美隆大学开发的``微内核''。这也就意味着,在1990年以前,GNU工程官方的内核开发事实上并没有真正开始。\endnote{参考``HURD历史''一文。\url{http://www.gnu.org/software/hurd/history.html}}
\fi

\ifdefined\eng
The delays in kernel development were just one of many concerns weighing on Stallman during this period. In 1989, Lotus Development Corporation filed suit against rival software company, Paperback Software International, for copying menu commands in Lotus' popular 1-2-3 Spreadsheet program. Lotus' suit, coupled with the Apple -Microsoft ``look and feel'' battle, provided a troublesome backdrop for the GNU Project. Although both suits fell outside the scope of the GNU Project, both revolved around operating systems and software applications developed for the personal computer, not Unix-compatible hardware systems-they threatened to impose a chilling effect on the entire culture of software development. Determined to do something, Stallman recruited a few programmer friends and composed a magazine ad blasting the lawsuits. He then followed up the ad by helping to organize a group to protest the corporations filing the suit. Calling itself the League of Programming Freedom, the group held protests outside the offices of Lotus, Inc. and the Boston courtroom hosting the Lotus trial.
\fi

\ifdefined\chs
这个时期,内核开发的延迟只是困扰斯托曼的众多问题中的一个。1989年莲花公司(Lotus Development Corporation)起诉了他的一家竞争对手——Paperback Software International公司,控告他抄袭了莲花公司的1-2-3电子表格程序中的菜单设计。莲花公司的官司,加上苹果和微软之间的``视觉体验''的斗争,给GNU工程带来了颇多麻烦。虽然这两场官司跟GNU工程都没有直接的关系,因为他们都主要是与个人电脑上的操作系统和应用软件相关,而不是关注于类Unix的硬件系统,但是他们对整个软件开发的文化产生了深远的影响。斯托曼觉得自己不能袖手旁观,他召集了几个程序员朋友在一本杂志上刊登了一则广告。然后他基于这则广告,成立了一个抗议小组,反对那些参与诉讼的公司。他把自己的这个组织命名为``自由编程联盟'',他们在莲花公司门口以及波士顿审理莲花公司这起案子的法庭外发起了抗议。
\fi

\ifdefined\eng
The protests were notable.\endnote{According to a League for Programming Freedom press release at \url{http://progfree.org/Links/prep.ai.mit.edu/demo.final.release}, the protests were notable for featuring the first hexadecimal protest chant:

\begin{verse}
1-2-3-4, toss the lawyers out the door\\
5-6-7-8, innovate don't litigate\\
9-A-B-C, 1-2-3 is not for me\\
D-E-F-O, look and feel have got to go\\
\end{verse}} They document the evolving nature of software industry. Applications had quietly replaced operating systems as the primary corporate battleground. In its unfulfilled quest to build a free software operating system, the GNU Project seemed hopelessly behind the times. Indeed, the very fact that Stallman had felt it necessary to put together an entirely new group dedicated to battling the ``look and feel'' lawsuits reinforced that obsolescence in the eyes of some observers.
\fi

\ifdefined\chs
这次抗议非常引人注目。\endnote{根据自由编程联盟的发布的文章(\url{http://progfree.org/Links/prep.ai.mit.edu/demo.final.release}),这次抗议很引人注目的原因之一是他们使用了一个十六进制的口号:

\begin{verse}
1-2-3-4, 把律师们抛出门外\\
5-6-7-8, 创新不应被起诉\\
9-A-B-C, 我们不用1-2-3\\
D-E-F-O, 再见``视觉体验''之争\\
\end{verse}} 他们的行动在软件工业发展史上留下了一笔。应用软件正悄悄地取代操作系统成为软件公司竞争的主要战场。然而,GNU工程想要建立一个自由操作系统的愿景都尚未实现,这使它看起来有点跟不上时代了。不过也正是这个原因,让斯托曼觉得有必要组织一股新的力量来对付这次对``视觉体验''的诉讼,但这在一些观察家眼中,则更显得GNU工程正在变得更加过时。
\fi

\ifdefined\eng
In 1990, the John D. and Catherine T. MacArthur Foundation cerified Stallman's genius status when it granted Stallman a MacArthur fellowship, therefore making him a recipient for the organization's so-called ``genius grant.'' The grant, a \$240,000 reward for launching the GNU Project and giving voice to the free software philosophy, relieved a number of short-term concerns. First and foremost, it gave Stallman, a nonsalaried employee of the FSF who had been supporting himself through consulting contracts, the ability to devote more time to writing GNU code.\endnote{I use the term ``writing'' here loosely. About the time of the MacArthur award, Stallman began suffering chronic pain in his hands and was dictating his work to FSF-employed typists. Although some have speculated that the hand pain was the result of repetitive stress injury, or RSI, an injury common among software programmers, Stallman is not 100\% sure. ``It was NOT carpal tunnel syndrome,'' he writes. ``My hand problem was in the hands themselves, not in the wrists.'' Stallman has since learned to work without typists after switching to a keyboard with a lighter touch.}
\fi

\ifdefined\chs
1990年,麦克阿瑟基金会授予了斯托曼麦克阿瑟院士的称号,嘉奖他的出众的能力,并授予他``天才奖金''。这份24万美元的奖金用于资助GNU工程和宣传自由软件哲学,这帮助GNU工程缓解了一些短期的困难。最重要的一点就是,它让斯托曼这个在自由软件基金会不拿薪水而仅靠做咨询来养活自己的员工,可以有更多的时间专注于编写GNU的代码。\endnote{在这里用``编写''这个词并不是很严密。斯托曼获得麦克阿瑟奖的时候,他的手患上了一种慢性病,经常疼痛,所以那时大部分打字的工作都是由FSF雇佣的一名打字员来完成的。有些人认为斯托曼所得的病就是在程序员群体中很常见的重复使力伤害症(RSI),但他自己对此并不是很确定。他说:``这不是腕管综合症,因为我是手疼,不是手腕关节疼。''后来斯托曼开始换用一种按键更为轻松的键盘,免去了请打字员录入的麻烦。}
\fi

\ifdefined\eng
Ironically, the award also made it possible for Stallman to vote. Months before the award, a fire in Stallman's apartment house had consumed his few earthly possessions. By the time of the award, Stallman was listing himself as a ``squatter''\endnote{See Reuven Lerner, ``Stallman wins \$240,000 MacArthur award,'' MIT, \textit{The Tech} (July 18, 1990). \url{http://the-tech.mit.edu/V110/N30/rms.30n.html}} at 545 Technology Square. ``[The registrar of voters] didn't want to accept that as my address,'' Stallman would later recall. ``A newspaper article about the MacArthur grant said that and then they let me register.''\endnote{See Michael Gross, ``Richard Stallman: High School Misfit, Symbol of Free Software, MacArthur-certified Genius'' (1999).}
\fi

\ifdefined\chs
麦克阿瑟奖还戏剧性的赋予了斯托曼选举权。在获奖前不久,发生在斯托曼公寓里的一场大火几乎让他损失了一切财产。在获奖的时候,斯托曼还只是在545科技广场存身的一个流浪汉\endnote{参考鲁文·勒纳(Reuven Lerner)``斯托曼荣获24万美元麦克阿瑟奖''一文,麻省理工学院,\textit{The Tech}杂志,1990年7月18日。\url{http://the-tech.mit.edu/V110/N30/rms.30n.html}}。``选举中心不同意我把那里作为我的住址,''斯托曼后回忆说,``不过报纸上那篇有关麦克阿瑟奖的新闻帮我证实了这个现状,于是他们就让我注册了。''\endnote{参考迈克尔·格罗斯(Michael Gross)``理查德·斯托曼:高中的弃儿、自由软件的象征、麦克阿瑟奖的天才获奖者''一文,1999年。}
\fi

\ifdefined\eng
Most importantly, the MacArthur money gave Stallman more freedom. Already dedicated to the issue of software freedom, Stallman chose to use the additional freedom to increase his travels in support of the GNU Project mission.
\fi

\ifdefined\chs
最关键的是,麦克阿瑟捐助的资金让斯托曼有了更多的自由,他可以全身心的投入软件自由的事业了。有了这些资助,他决定更多的去四处宣讲GNU工程的使命。
\fi

\ifdefined\eng
Interestingly, the ultimate success of the GNU Project and the free software movement in general would stem from one of these trips. In 1990, Stallman paid a visit to the Polytechnic University in Helsinki, Finland. Among the audience members was 21-year-old Linus Torvalds, future developer of the Linux kernel-the free software kernel destined to fill the GNU Project's most sizable gap.
\fi

\ifdefined\chs
有意思的是,GNU工程和自由软件运动最终的胜利差不多就是源于这些旅行。1990年,斯托曼去芬兰的赫尔辛基大学访问。在听众中,有一名叫林纳斯·托瓦兹的学生,那时他只有21岁。后来托瓦兹发明了Linux内核,Linux内核完美的填补了GNU工程中所空缺的那一部分。
\fi

\ifdefined\eng
A student at the nearby University of Helsinki at the time, Torvalds regarded Stallman with bemusement. ``I saw, for the first time in my life, the stereotypical long-haired, bearded hacker type,'' recalls Torvalds in his 2001 autobiography Just for Fun. ``We don't have much of them in Helsinki.''\endnote{See Linus Torvalds and David Diamond, \textit{Just For Fun: The Story of an Accidentaly Revolutionary} (HarperCollins Publishers, Inc., 2001): 58-59. %%Although presumably accurate in regard to Torvalds' life, what the book says about Stallman is sometimes wrong.  For instance, it says that Stallman ``wants to make everything open source,'' and that he ``complains about other people not using the GPL.''  In fact, Stallman advocates free software, not open source.  He urges authors to choose the GNU GPL, in most circumstances, but says that all free software licenses are ethical.
}
\fi

\ifdefined\chs
托瓦兹那时在附近的赫尔辛基大学上学,斯托曼给他留下的第一印象是颇为奇怪的。``我这辈子头一次看到这样一个长头发、长胡子的古怪黑客。在赫尔辛基很少能见到这样的人。''托瓦兹在2001年的自传《乐者为王》中回忆到。\endnote{参考林纳斯·托瓦兹与大卫·戴蒙(David Diamond)《\textit{乐者为王:一次不经意间的发生的革命}》一书,哈珀·柯林斯出版集团,2001年,58-59页。%%仅管这本中有关托瓦兹的生活的描述都很准确,但是这本书中对斯托曼的描写很多是不正确的。比如,在这本书中说斯托曼``打算把所有软件都变成开源的'',并且``抱怨其它人不积极使用GPL''。事实上,斯托曼关注的是自由软件运动,而不是``开源'';他希望软件作者尽可能去使用GPL,但是他同时也认为各种自由软件许可证都是符合他的道德斯望的。
}
\fi

\ifdefined\eng
While not exactly attuned to the ``sociopolitical'' side of the Stallman agenda, Torvalds nevertheless appreciated the agenda's underlying logic: no programmer writes error-free code. By sharing software, hackers put a program's improvement ahead of individual motivations such as greed or ego protection.
\fi

\ifdefined\chs
虽然托瓦兹并不了解斯托曼的宏大计划中的政治性的一面,但他仍然很欣赏这个计划在内在逻辑:没有人可以写出没有错误的代码。通过共享软件,黑客们把不断改进程序作为终极的目标,甚至超越了个人野心。
\fi

\ifdefined\eng
Like many programmers of his generation, Torvalds had cut his teeth not on mainframe computers like the IBM 7094, but on a motley assortment of home-built computer systems. As university student, Torvalds had made the step up from C programming to Unix, using the university's MicroVAX. This ladder-like progression had given Torvalds a different perspective on the barriers to machine access. For Stallman, the chief barriers were bureaucracy and privilege. For Torvalds, the chief barriers were geography and the harsh Helsinki winter. Forced to trek across the University of Helsinki just to log in to his Unix account, Torvalds quickly began looking for a way to log in from the warm confines of his off-campus apartment.
\fi

\ifdefined\chs
与他同辈的很多程序员一样,托瓦兹并不关注类似于IBM 7094这样的大型机,而是对那些五花八门配置的家用电脑更感兴趣 。作为一个大学生,托瓦兹是通过在学校的MicroVAX上学习C语言,一步步进入Unix的世界的。这种阶梯式的学习曲线使托瓦兹对于访问大型机的不方便之处有着一种不同的视角。对于斯托曼来说,主要的不方便是官僚作风和特权控制。对于托瓦兹来说,则是地理位置和赫尔辛基寒冷的冬天。为了避免穿过整个赫尔辛基大学的校园去登陆他的Unix系统,托瓦兹开始设法寻找一种可以在他温暖的校外公寓中远程访问学校的Unix系统的方法。
\fi

\ifdefined\eng
The search led Torvalds to the operating system Minix, a lightweight version of Unix developed for instructional purposes by Dutch university professor Andrew Tanenbaum. The program fit within the memory confines of a 386 PC, the most powerful machine Torvalds could afford, but still lacked a few necessary features. It most notably lacked terminal emulation, the feature that allowed Torvalds' machine to mimic a university terminal, making it possible to log in to the MicroVAX from home.
\fi

\ifdefined\chs
不久,托瓦兹就发现了Minix操作系统,Minix操作系统是一名荷兰的大学教授安德鲁·塔嫩鲍姆(Andrew Tanenbaum)所开发的一个用于教学用途的轻量级Unix系统。这个系统可以在386电脑上正常运行,这正好是托瓦兹能买得起的最强大的机器,但是,这个系统仍然缺乏一些必要的功能。最明显的一点是他没有终端模拟的功能,有了这项功能才能使托瓦兹足不出户模拟出一个学校电脑的终端并登陆到学校的MicroVAX计算机上。
\fi

\ifdefined\eng
During the summer of 1991, Torvalds rewrote Minix from the ground up, adding other features as he did so. By the end of the summer, Torvalds was referring to his evolving work as the ``GNU/Emacs of terminal emulation programs.''\endnote{See Linus Torvalds and David Diamond, \textit{Just For Fun: The Story of an Accidentaly Revolutionary} (HarperCollins Publishers, Inc., 2001): 58-59.} Feeling confident, he solicited a Minix newsgroup for copies of the POSIX standards, the software blue prints that determined whether a program was Unix compatible. A few weeks later, Torvalds was posting a message eerily reminiscent of Stallman's original 1983 GNU posting:
\fi

\ifdefined\chs
1991年的夏天,托瓦兹把Minix从头重写了一遍,并加了上了一些他所开发的新特性。到了夏天快要过去的时候,他完成了这个名为``GNU/Emacs of terminal emulation programs''的程序。\endnote{参考林纳斯·托瓦兹与大卫·戴蒙《\textit{乐者为王:一次不经意间的发生的革命}》一书,哈珀·柯林斯出版集团,2001年,78页。}他对这个作品很有信心,他开始在一个Minix新闻组上寻找POSIX标准,POSIX标准定义了与Unix兼容的程序应有的行为。几周后,托瓦兹发布了一则让人可以联想到斯托曼在1983年启动GNU工程时的所发的帖子的消息:
\fi

\ifdefined\eng
\begin{quote}
Hello everybody out there using minix-

I'm doing a (free) operating system (just a hobby, won't be big and professional like gnu for 386 (486) AT clones). This has been brewing since April, and is starting to get ready. I'd like any feedback on things people like/dislike in minix, as my OS resembles it somewhat (same physical layout of the file-system (due to practical reasons) among other things). \endnote{See ``Linux 10th Anniversary.'' \url{http://www.linux10.org/history/}}
\end{quote}
\fi

\ifdefined\chs
\begin{quote}
使用Minix操作系统的各位大家好。

我正在开发一个(自由的)操作系统(只是出于兴趣,这个系统不会像386 (486) AT上的gnu系统那么庞大和专业)。这个系统从4月份开始开发,现在已经逐渐可用了。我期待听到各位喜欢或不喜欢Minix的人们的反馈,因为这个系统参考了它的不少设计(比如出于实用的需要,它们的文件系统采用了相同的物理布局,除此之外还有一些别的相同之处)\endnote{参考``Linux 10周年''。\url{http://www.linux10.org/history/}}
\end{quote}
\fi

\ifdefined\eng
The posting drew a smattering of responses and within a month, Torvalds had posted a 0.01 version of the operating system-i.e., the earliest possible version fit for outside review-on an Internet FTP site. In the course of doing so, Torvalds had to come up with a name for the new system. On his own PC hard drive, Torvalds had saved the program as Linux, a name that paid its respects to the software convention of giving each Unix variant a name that ended with the letter X. Deeming the name too ``egotistical,'' Torvalds changed it to Freax, only to have the FTP site manager change it back.
\fi

\ifdefined\chs
在不到一个月的时间内,这个帖子收到了大量的回复。托瓦兹在Internet上的一个FTP服务器上发布了这个操作系统的0.01版,这是最早向公众开放征求意见的版本。为了实现这个目标,托瓦兹需要给这个新的系统起一个名字。在他自己的电脑硬盘上,这个程序被命名为Linux,这个名字体现了它是一个Unix的变形版本,所以名字以字母X结尾。托瓦兹认为这个名字太过于自恋,所以在FTP上把它的名字改成了Freax,但FTP管理员又把名字改了回来。
\fi

\ifdefined\eng
Although Torvalds had set out build a full operating system, both he and other developers knew at the time that most of the functional tools needed to do so were already available, thanks to the work of GNU, BSD, and other free software developers. One of the first tools the Linux development team took advantage of was the GNU C Compiler, a tool that made it possible to process programs written in the C programming language.
\fi

\ifdefined\chs
尽管托瓦兹准备开发一个完整的操作系统,但他和其它开发人员都意识到,大部分重要的工具软件其实都已经有了现成的版本,这些软件是由GNU、BSD和无数自由软件开发者共同完成的。对于Linux的开发团队来说,GNU C编译器是这些工具软件中最重要的一个,有了它才使用C语言进行开发成为可能。
\fi

\ifdefined\eng
Integrating GCC improved the performance of Linux. It also raised issues. Although the GPL's ``viral'' powers didn't apply to the Linux kernel, Torvald's willingness to borrow GCC for the purposes of his own free software operating system indicated a certain obligation to let other users borrow back. As Torvalds would later put it: ``I had hoisted myself up on the shoulders of giants.''\endnote{See Linus Torvalds and David Diamond, \textit{Just For Fun: The Story of an Accidentaly Revolutionary} (HarperCollins Publishers, Inc., 2001): 96-97.} Not surprisingly, he began to think about what would happen when other people looked to him for similar support. A decade after the decision, Torvalds echoes the Free Software Foundation's Robert Chassel when he sums up his thoughts at the time:
\fi

\ifdefined\chs
整合GCC提高了Linux的性能,但也带来了一些问题。仅管GPL的``传染性''并不影响到Linux内核,但托瓦兹在他的自由软件操作系统中使用了GCC,也造成其它用户产生了从Linux借用更多东西的愿望。事后托瓦兹说,``我需要站在巨人们的肩膀上''。\endnote{参考林纳斯·托瓦兹与大卫·戴蒙《\textit{乐者为王:一次不经意间的发生的革命}》一书,哈珀·柯林斯出版集团,2001年,96-97页。} 因此,他开始思考当别人开始利用Linux时并需要更多依赖Linux时事情会变成什么样子。在做出这个决定的十年后,托瓦兹回应了自由软件基金会的罗伯特·查瑟尔(Robert Chassel),把他的想法总结如下:
\fi

\ifdefined\eng
\begin{quote}
You put six months of your life into this thing and you want to make it available and you want to get something out of it, but you don't want people to take advantage of it. I wanted people to be able to see [Linux], and to make changes and improvements to their hearts' content. But I also wanted to make sure that what I got out of it was to see what they were doing. I wanted to always have access to the sources so that if they made improvements, I could make those improvements myself.\endnote{See Linus Torvalds and David Diamond, \textit{Just For Fun: The Story of an Accidentaly Revolutionary} (HarperCollins Publishers, Inc., 2001): 94-95.}
\end{quote}
\fi

\ifdefined\chs
\begin{quote}
你花了你生命中宝贵的半年时间去做一个东西,并且希望它能广为流传并从中获利,但你不希望别人用它来牟利。我希望人们都能了解Linux,并且能对它进行修改,让它更符合每个人的期望。但我同样希望能知道别人都拿它来做了些什么,我希望我一直可以看到那些别人所改动过的源代码,这样的话,我也能有机会对它继续进行改进。\endnote{参考林纳斯·托瓦兹与大卫·戴蒙《\textit{乐者为王:一次不经意间的发生的革命}》一书,哈珀·柯林斯出版集团,2001年,94-95页。}
\end{quote}
\fi

\ifdefined\eng
When it was time to release the 0.12 version of Linux, the first to include a fully integrated version of GCC, Torvalds decided to voice his allegiance with the free software movement. He discarded the old kernel license and replaced it with the GPL. The decision triggered a porting spree, as Torvalds and his collaborators looked to other GNU programs to fold into the growing Linux stew. Within three years, Linux developers were offering their first production release, Linux 1.0, including fully modified versions of GCC, GDB, and a host of BSD tools.
\fi

\ifdefined\chs
Linux的0.12版本中第一次完成的集成了GCC,托瓦兹决定在这次发布时宣布他与自由软件运动的联盟。他废除了Linux内核原有的许可证,并替换为GPL。这个决定带来了一场移植软件的狂欢,托瓦兹和其它Linux爱好者寻找了各种GNU程序并把它们移植到了Linux中。短短不到3年的时间,Linux开发者发布了第一个可以用于生产环境的Linux版本,1.0。里面包含了经过大面积修改的GCC、GDB和一系列BSD工具。
\fi


\ifdefined\eng
By 1994, the amalgamated operating system had earned enough respect in the hacker world to make some observers wonder if Torvalds hadn't given away the farm by switching to the GPL in the project's initial months. In the first issue of Linux Journal, publisher Robert Young sat down with Torvalds for an interview. When Young asked the Finnish programmer if he felt regret at giving up private ownership of the Linux source code, Torvalds said no. ``Even with 20/20 hindsight,'' Torvalds said, he considered the GPL ``one of the very best design decisions'' made during the early stages of the Linux project.\endnote{See Robert Young, ``Interview with Linus, the Author of Linux,'' \textit{Linux Journal} (March 1, 1994). \url{http://www.linuxjournal.com/article.php?sid=2736}}
\fi

\ifdefined\chs
到了1994年,这个混杂了各种程序的操作系统在黑客圈子里赢得了极大的关注,很多观察家因此认为托瓦兹在把Linux内核许可证切换到GPL后的最初几个月中,并没有把程序真正的交给社区。在第一期《Linux通讯》杂志中,杂志编辑罗伯特·杨(Robert Young)对托瓦兹进行了一次面对面的采访。当杨向这名芬兰程序员提问是否后悔放弃了对Linux源代码的私有权时,托瓦兹的回答是否定的。托瓦兹说:``即使是做一次事后诸葛亮,我仍觉得选择GPL是Linux项目初期一个最佳的设计决策。''\endnote{参考罗伯特·杨``Linux之父林纳斯访谈'',\textit{Linux通讯}杂志,1994年3月1日。\url{http://www.linuxjournal.com/article.php?sid=2736}}
\fi

\ifdefined\eng
That the decision had been made with zero appeal or deference to Stallman and the Free Software Foundation speaks to the GPL's growing portability. Although it would take a few years to be recognized by Stallman, the explosiveness of Linux development conjured flashbacks of Emacs. This time around, however, the innovation triggering the explosion wasn't a software hack like Control-R but the novelty of running a Unix-like system on the PC architecture. The motives may have been different, but the end result certainly fit the ethical specifications: a fully functional operating system composed entirely of free software.
\fi

\ifdefined\chs
这个决策与斯托曼和自由软件基金会所宣称的GPL的普适性完全没有冲突。虽然斯托曼一开始并没有注意到这点,但是Linux爆发式的成长反过来让Emacs受到了更多的关注。这一次的创新,决非是一个像Control-R这样的小小的软件技巧,而是一个让类Unix系统可以在PC上运行的新奇事物。虽然动机不同,但是结果是一致的:创造一个只由自由软件构成的完整的操作系统。
\fi

\ifdefined\eng
As his initial email message to the comp.os.minix newsgroup indicates, it would take a few months before Torvalds saw Linux as anything less than a holdover until the GNU developers delivered on the HURD kernel. This initial unwillingness to see Linux in political terms would represent a major blow to the Free Software Foundation.
\fi

\ifdefined\chs
就像托瓦兹在最初发表到comp.os.minix新闻组的电子邮件所写的那样,他最初确实只是把Linux作为是GNU开发者发布HURD内核前的一个替代品。这个毫无政治目的的原始想法对于自由软件基金会而言,是个打击。
\fi

\ifdefined\eng
As far as Torvalds was concerned, he was simply the latest in a long line of kids taking apart and reassembling things just for fun. Nevertheless, when summing up the runaway success of a project that could have just as easily spent the rest of its days on an abandoned computer hard drive, Torvalds credits his younger self for having the wisdom to give up control and accept the GPL bargain.
\fi

\ifdefined\chs
托瓦兹自己觉得,他只是跟很多小朋友一样,喜欢把玩具拆开再组装起来。但事实上,他还利用闲暇时光在一块废弃的硬盘上创造出一个成功的项目,他放弃了对Linux的控制权,接受GPL的恩泽,这一切都足以证明年轻的托瓦兹所拥有的过人智慧。
\fi

\ifdefined\eng
``I may not have seen the light,'' writes Torvalds, reflecting on Stallman's 1991 Polytechnic University speech and his subsequent decision to switch to the GPL. ``But I guess something from his speech sunk in .''\endnote{See Linus Torvalds and David Diamond, \textit{Just For Fun: The Story of an Accidentaly Revolutionary} (HarperCollins Publishers, Inc., 2001): 59.}
\fi

\ifdefined\chs
托瓦兹回忆起1991年斯托曼的演讲以及其后他决定对Linux使用GPL的往事,他写道:``我也许没有看到指路的明灯,但我从他的演讲中领悟到了一些更为深刻的东西。''\endnote{参考林纳斯·托瓦兹与大卫·戴蒙《\textit{乐者为王:一次不经意间的发生的革命}》一书,哈珀·柯林斯出版集团,2001年,59页。}
\fi

\theendnotes
\setcounter{endnote}{0}


%% \ifdefined\eng
%% The tension between the freedom to modify and authorial privilege had been building before Gosmacs. The Copyright Act of 1976 had overhauled U.S. copyright law, extending the legal coverage of copyright to software programs. According to Section 102(b) of the Act, individuals and companies could copyright the ``expression'' of a software program but not the ``actual processes or methods embodied in the program.''\endnote{See Hal Abelson, Mike Fischer, and Joanne Costello, ``Software and Copyright Law,'' updated version (1997), \url{http://groups.csail.mit.edu/mac/classes/6.805/articles/int-prop/software-copyright.html}.}
%% \fi

%% \ifdefined\chs
%% 在Gosmacs之前,自由修改软件与软件作者的特权之间就已经形成了一种对立的关系。在1976的版权运动中,修订后的美国版权法案把适用范围扩展到了软件作品上。根据该法案的第102(b)节的规定,个人或公司可以对软件程序的``实现方式''声明版权,但不能对``程序中的实际处理过程或方法''声明版权。\endnote{参见哈尔·艾贝尔森、麦克·费雪和乔安妮·科斯特洛的《软件与版权法》,1997年修订版。\url{http://groups.csail.mit.edu/mac/classes/6.805/articles/int-prop/software-copyright.html}}
%% \fi

%% \ifdefined\eng
%% Translated, this treated a program much like an algebra textbook: its author can claim copyright on the text but not on the mathematical ideas of algebra or the pedagogical technique employed to explain it.  Thus, regardless of what Stallman said about using the code of the original Emacs, other programmers were legally entitled to write their own implementations of the ideas and commands of Emacs, and they did.  Gosmacs was one of 30-odd imitations of the original Emacs developed for various computer systems.
%% \fi

%% \ifdefined\chs
%% 换句话说,如果以几何教科书为例:作者可以对书籍的文本声明版权,但不能对几何中的数学原理或用于解释这些原理的教学描述声明版权。所以,不管理查德对Emacs的使用做如何的规定,其它的程序员都可以合法的为Emacs开发一些新的功能或命令,而且,他们确实也这样做了。Gosmacs就是为不同系统开发的30多种基于原始Emacs的衍生版本之一。
%% \fi

%% \ifdefined\eng
%% The Emacs Commune applied only to the code of the original Emacs program written by Stallman himself.  Even if it had been legally enforced, it would not have applied to separately developed imitations such as Gosmacs.  Making Gosmacs nonfree was unethical according to the ethical ideas of the free software movement, because (as proprietary software) it did not respect its users' freedom, but this issue had nothing to do with where the ideas in Gosmacs came from.
%% \fi

%% \ifdefined\chs
%% ``Emacs公社''只适用于理查德本人所写的原始Emacs版本。即使通过法律的手段来实施这种协议,它也不适用于类似于Gosmacs这种独立开发的衍生版本。依据自由软件运动的哲学,把Gosmacs变成非自由的软件是一件不道德的事情,因为作为一个私有软件,它将无法让它的用户享有自由的权力。TODO但这个问题与Gosmacs内在的思想的最初来源没有任何的关系。
%% \fi

%% \ifdefined\eng
%% Under copyright, programmers who wanted to copy code from an existing program (even with changes) had to obtain permission from the original developer. The new law applied copyright even in the absence of copyright notices -- though hackers generally did not know this -- and the copyright notices too began appearing.
%% \fi

%% \ifdefined\chs
%% 在版权的保护下,如果有人想从一个程序中直接复用代码(包括修改后复用),都需要得到程序原始作者的许可。新的版权法案适用于所有软件作品,即使这个软件没有明确的声明版权。早期的黑客们并不清楚这一点,但为软件作品添加版权声明的做法却从那个时候开始慢慢得以普及。
%% \fi

%% \ifdefined\eng

%% Stallman saw these notices as the flags of an invading, occupying army. Rare was the program that didn't borrow source code from past programs, and yet, with a single stroke of the president's pen, the U.S. government had given programmers and companies the legal power to forbid such reuse.  Copyright also injected a dose of formality into what had otherwise been an informal system. Simply put, disputes that had once been settled hacker-to-hacker were now to be settled lawyer-to-lawyer. In such a system, companies, not hackers, held the automatic advantage.  Some saw placing one's name in a copyright notice as taking responsibility for the quality of the code, but the copyright notice usually has a company's name, and there are other ways for individuals to say what code they wrote.
%% \fi

%% \ifdefined\chs
%% 理查得把这些版权声明的出现看成是一个不好的兆头。那时候,大部分程序都会从以前的程序中借鉴一些代码,然而,总统的大笔一挥,美国政府就剥夺了程序员和软件公司重用其它软件代码的权力。版权法案向原本不太正式的软件版权领域注入了一剂强心针。简单的说,这项法案使得原先黑客之间的较量变成了律师之间的较量。在这样的法律框架下,软件公司相对于黑客来说占据了天然的有利地位。有些人把在版权声明加入自己的名字看成是对代码质量的一种担保,不过大多数情况下,版权声明中都只会写公司的名字,对于个人来说,只能使用其它的方式来标记他们写过的代码。
%% \fi

%% \ifdefined\eng

%% However, Stallman also noticed, in the years leading up to the GNU Project, that copyright allowed an author to grant permission for certain activities covered by copyright, and place conditions on them too.  ``I had seen email messages with copyright notices plus simple `verbatim copying permitted' licenses,'' he recalls. ``Those definitely were [an] inspiration.''  These licenses carried the condition not to remove the license.  Stallman's idea was to take this a few steps further.  For example, a permission notice could allow users to redistribute even modified versions, with the condition that these versions carry the same permission.
%% \fi

%% \ifdefined\chs
%% 然而,在领导GNU工程的这些年中,斯托曼注意到,版权法案使得软件的作者可以在其保护的范围中允许他人进行一些相关的活动,也可以对这些活动规定一些条件。他回忆说:``我见过一些附有版权声明的电子邮件,并且标明`允许原样复制'。这给我了一些灵感。''这些许可证中通常带有一个条款,那就是在任何情况下需要保留这个许可证文本。斯托曼决定对此发扬光大。比如,可以允许用户任何分发软件,甚至分发修改后的版本,但是这些版本都必须具备相同的许可证,允许再次分发。
%% \fi

%% \ifdefined\eng

%% Thus Stallman concluded that use of copyright was not necessarily unethical.  What was bad about software copyright was the way it was typically used, and designed to be used: to deny the user essential freedoms.  Most authors imagined no other way to use it.  But copyright could be used in a different way: to make a program free and assure its continued freedom.
%% \fi

%% \ifdefined\chs
%% 所以,斯托曼认为,用版权来保护软件作品并不是不道德的。只不过是人们使用版权的方式不太好:以版权保护的名义剥夺用户的自由。版权的出现似乎就是为了这个目的,大部分的软件作者也想不到别的使用版权的方式。但是,其实版权完全可以用其它的方式来使用:用版权来保证软件的自由,并保证它一直是自由的。
%% \fi

%% \ifdefined\eng
%% By GNU Emacs 16, in early 1985, Stallman drafted a copyright-based license that gave users the right to make and distribute copies. It also gave users the right to make and distribute modified versions, but only under the same license.  They could not exercise the unlimited power of copyright over those modified versions, so they could not make their versions proprietary as Gosmacs was.  And they had to make the source code available.  Those conditions closed the legal gap that would otherwise allow restricted, nonfree versions of GNU Emacs to emerge.
%% \fi

%% \ifdefined\chs
%% 1985年,GNU Emacs 16发布,斯托曼起草了一份基于版权的许可证,允许用户随意复制并分发该软件。他还允许软件的用户分发修改后软件,但是要求分发后的软件与原来的软件使用相同的许可证。其它人不能夺取修改后的软件的版本,也就 不能像Gosmacs一样把软件变为私有软件。并且,许可证还要求软件的源代码必须是可以获取的。这些限制条件就杜绝了非自由的GNU Emacs版本的出现。
%% \fi


