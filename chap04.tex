%% Copyright (c) 2002, 2010 Sam Williams
%% Copyright (c) 2010 Richard M. Stallman
%% Permission is granted to copy, distribute and/or modify this
%% document under the terms of the GNU Free Documentation License,
%% Version 1.3 or any later version published by the Free Software
%% Foundation; with no Invariant Sections, no Front-Cover Texts, and
%% no Back-Cover Texts. A copy of the license is included in the
%% file called ``gfdl.tex''.

\chapter{\ifdefined\eng
Impeach God
\fi
\ifdefined\chs
逆天行道,选举上帝
\fi}

\ifdefined\eng
Although their relationship was fraught with tension, Richard Stallman would inherit one noteworthy trait from his mother: a passion for progressive politics.
\fi

\ifdefined\chs
尽管理查德·斯托曼和母亲的关系比较紧张,但他还是继承了母亲的一个特质:对激进政治极度热情。
\fi

\ifdefined\eng
It was an inherited trait that would take several decades to emerge, however. For the first few years of his life, Stallman lived in what he now admits was a ``political vacuum.''\endnote{See Michael Gross, ``Richard Stallman: High School Misfit, Symbol of Free Software, MacArthur-certified Genius'' (1999).} Like most Americans during the Eisenhower age, the Stallman family spent the Fifties trying to recapture the normalcy lost during the wartime years of the 1940s.
\fi

\ifdefined\chs
可这个特质要在十几年之后才在斯托曼身上显现出来。他早年的生活,用他现在的话说,是与``政治绝缘''的\endnote{参见迈克尔·格劳斯的采访:《斯托曼·斯托曼:中学校园里的怪才,自由软件界的旗手,麦克阿瑟奖获得者》(1999年)。}。正如艾森豪威尔时期的大多数美国家庭一样,五十年代,斯托曼一家试图重拾二战以前的那份宁静、安逸的正常生活。
\fi

\ifdefined\eng
``Richard's father and I were Democrats but happy enough to leave it at that,'' says Lippman, recalling the family's years in Queens. ``We didn't get involved much in local or national politics.''
\fi

\ifdefined\chs
``理查德的父亲和我虽然都是民主党人,可我们觉得那段时间的生活还算过得去。''李普曼回忆着当年和理查德父亲住在纽约市皇后区的日子,``我们当时并没有参与太多的当地或者全国的政治活动。''
\fi

\ifdefined\eng
That all began to change, however, in the late 1950s when Alice divorced Daniel Stallman. The move back to Manhattan represented more than a change of address; it represented a new, independent identity and a jarring loss of tranquility.
\fi

\ifdefined\chs
然而,五十年代末期,李普曼的离异打破了这一切。她要重新搬回纽约市曼哈顿区,这不仅仅是改个住址,它更意味着一份全新的独立生活的开始,更意味着重回喧嚣的大都市生活。
\fi

\ifdefined\eng
``I think my first taste of political activism came when I went to the Queens public library and discovered there was only a single book on divorce in the whole library,'' recalls Lippman. ``It was very controlled by the Catholic church, at least in Elmhurst, where we lived. I think that was the first inkling I had of the forces that quietly control our lives.''
\fi

\ifdefined\chs
``关于政治活动,我最初的体验是在离婚期间。当时我走进皇后区公共图书馆,却发现里面只有一本关于离婚的书。那里当时被天主教严格控制着,至少在我们住的阿母赫斯特区是这样的。这是我第一次意识到,我们的日常生活,被周围的各种权威力量悄悄地控制着。''李普曼说。
\fi

\ifdefined\eng
Returning to her childhood neighborhood, Manhattan's Upper West Side, Lippman was shocked by the changes that had taken place since her departure to Hunter College a decade and a half before. The skyrocketing demand for post-war housing had turned the neighborhood into a political battleground. On one side stood the pro-development city-hall politicians and businessmen hoping to rebuild many of the neighborhood's blocks to accommodate the growing number of white-collar workers moving into the city. On the other side stood the poor Irish and Puerto Rican tenants who had found an affordable haven in the neighborhood.
\fi

\ifdefined\chs
重回童年居住过的曼哈顿上西城,李普曼被周围的变化震惊了。十五年前,她离家去亨特学院读书。如今,由于战后住房资源紧张,当地居民对高楼有大量需求,把几个街区变成了一个政治决斗场,把人们分成了对立的两组。一组人,主要是政客和商人,他们希望把这一片拆迁扩建,以应付越来越多的白领移居到这里;另一组人,主要是来自爱尔兰或波多黎各的房客,他们都不富裕,所幸已经在周围找到了便宜舒适的住房。
\fi

\ifdefined\eng
At first, Lippman didn't know which side to choose. As a new resident, she felt the need for new housing. As a single mother with minimal income, however, she shared the poorer tenants' concern over the growing number of development projects catering mainly to wealthy residents. Indignant, Lippman began looking for ways to combat the political machine that was attempting to turn her neighborhood into a clone of the Upper East Side.
\fi

\ifdefined\chs
一开始,李普曼不知道站到哪一边。作为一个新来的住户,她觉得的确有必要扩建。可作为一个拿着微薄收入的单亲母亲,她和那些房客有着一样的忧虑,觉得周围越来越多的项目,都是为有钱人开发的。心中忿忿不平,李普曼开始寻找机会,来和庞大的政治机器做斗争,避免让政客把上西城变成上东城那样,只供有钱人吃喝玩乐。
\fi

\ifdefined\eng
Lippman says her first visit to the local Democratic party headquarters came in 1958. Looking for a day-care center to take care of her son while she worked, she had been appalled by the conditions encountered at one of the city-owned centers that catered to low-income residents. ``All I remember is the stench of rotten milk, the dark hallways, the paucity of supplies. I had been a teacher in private nursery schools. The contrast was so great. We took one look at that room and left. That stirred me up.''
\fi

\ifdefined\chs
李普曼第一次去民主党总部是在1958年。当时,她希望为理查德找个日托所,以便在自己工作的时候,有人照顾理查德。在本市的低收入人员帮助中心里,她一下子被那里的环境吓到了。``我就记得当时那股酸臭的坏牛奶味儿,黑洞洞的走廊,还有那么一丁点的救助物资。我以前是护士学校的老师,那里的环境和这里反差太大了。我们看了一眼那个日托所的房间,就立马走人了。那环境太让我恶心。''
\fi

\ifdefined\eng
The visit to the party headquarters proved disappointing, however. Describing it as ``the proverbial smoke-filled room,'' Lippman says she became aware for the first time that corruption within the party might actually be the reason behind the city's thinly disguised hostility toward poor residents. Instead of going back to the headquarters, Lippman decided to join up with one of the many clubs aimed at reforming the Democratic party and ousting the last vestiges of the Tammany Hall machine. Dubbed the Woodrow Wilson/FDR Reform Democratic Club, Lippman and her club began showing up at planning and city-council meetings, demanding a greater say.
\fi

\ifdefined\chs
这次造访民主党总部,让李普曼很失望。``大家说的一点也没错,乌烟瘴气,不是个干净地方。''李普曼说,这是她第一次意识到,党内腐败也许才是一切症结所在,大家也许才因此鄙视甚至仇视穷人。李普曼没有再踏入民主党总部一步。她看到当地众多俱乐部,都志在促进民主党的党内改革。李普曼加入了其中的一个俱乐部:伍德罗·威尔逊/FDR民主党改革俱乐部,从此与坦慕尼协会余党\endnote{译注:坦慕尼协会曾是纽约市民主党的政治机器。曾一度控制了曼哈顿一带民主党候选人的提名。二十世纪三十年代因操纵选票丑闻而一蹶不振。但在五十年代又有起色,六十年代左右逐渐退出政治舞台。}作斗争。李普曼开始出现在俱乐部内部的例行会议中,和市政委员会会议会上,以此她取得了更多话语权。
\fi

\ifdefined\eng
``Our primary goal was to fight Tammany Hall, Carmine DeSapio and his henchman,''\endnote{Carmine DeSapio holds the dubious distinction of being the first Italian-American boss of Tammany Hall, the New York City political machine. For more information on DeSapio and the politics of post-war New York, see John Davenport, ``Skinning the Tiger: Carmine DeSapio and the End of the Tammany Era,'' \textit{New York Affairs} (1975): 3:1.} says Lippman. ``I was the representative to the city council and was very much involved in creating a viable urban-renewal plan that went beyond simply adding more luxury housing to the neighborhood.''
\fi

\ifdefined\chs
李普曼说:``我们的主要目标,是对抗以卡米思·德·萨皮奥(Carmine De Sapio)为首的坦慕尼协会和它的追随者\endnote{卡米思·德·萨皮奥(Carmine De Sapio)曾是坦慕尼协会的第一个意大利裔主席。关于他和纽约二战后的政治格局,参见约翰·达文波特的文章《擒虎:卡米思·德·萨皮奥,和坦慕尼协会的最后时光》,《纽约事务》(1975):3:1。John Davenport, "Skinning the Tiger: Carmine DeSapio and the End of the Tammany Era," New York Affairs (1975): 3:1.}。我是市政委员会的一名代表委员。主要提议修建更多廉价住房,反对单纯修建舒适豪宅。''
\fi

\ifdefined\eng
Such involvement would blossom into greater political activity during the 1960s. By 1965, Lippman had become an ``outspoken'' supporter for political candidates like William Fitts Ryan, a Democrat elected to Congress with the help of reform clubs and one of the first U.S. representatives to speak out against the Vietnam War.
\fi

\ifdefined\chs
这种参与在六十年代变成了更大的政治活动。1965年,李普曼已经开始公开支持一些民主党的议会候选人,比如威廉·菲茨·赖安,他在各个民主党改革俱乐部的帮助下,进入议会,并成为当时第一批反对越战的议员。
\fi

\ifdefined\eng
It wasn't long before Lippman, too, was an outspoken opponent of U.S. involvement in Indochina. ``I was against the Vietnam War from the time Kennedy sent troops,'' she says. ``I had read the stories by reporters and journalists sent to cover the early stages of the conflict. I really believed their forecast that it would become a quagmire.''
\fi

\ifdefined\chs
不久之后,李普曼也开始公开反对美国涉足印度支那问题。``肯尼迪政府把军队送去越南开始,我就一直反对越战。从一开始,我就读各种来自前线的报道文章。很多文章都预测美国政府会因此深陷泥潭,我也非常赞同这个观点。''
\fi

\ifdefined\eng
Such opposition permeated the Stallman-Lippman household. In 1967, Lippman remarried. Her new husband, Maurice Lippman, a major in the Air National Guard, resigned his commission to demonstrate his opposition to the war. Lippman's stepson, Andrew Lippman, was at MIT and temporarily eligible for a student deferment. Still, the threat of induction should that deferment disappear, as it eventually did, made the risk of U.S. escalation all the more immediate. Finally, there was Richard who, though younger, faced the prospect of being drafted as the war lasted into the 1970s.
\fi

\ifdefined\chs
这种反战之声充斥在理查德的家中。1967年,李普曼再婚。她的新任丈夫莫里斯·李普曼是一名空军少校,越战开始他辞职以示反战。莫里斯的儿子,李普曼的继子,安德鲁·李普曼当时正在麻省理工学院读书。他暂时还可以申请延期入伍。可随着战事升级,战争时间一久,入伍时间到期,他还是要去服役。当战事延续到七十年代,理查德虽然年纪还小,可也不得不考虑入伍的问题。他要么选择去越南参战,要么选择到加拿大避开兵役。
\fi

\ifdefined\eng
``Vietnam was a major issue in our household,'' says Lippman. ``We talked about it constantly: what would we do if the war continued, what steps Richard or his stepbrother would take if they got drafted. We were all opposed to the war and the draft. We really thought it was immoral.''
\fi

\ifdefined\chs
李普曼说:``越南问题成了家里的重点,我们总是不停地讨论它:如果战争持续下去我们怎么办?要是理查德或者安德鲁收到征兵令怎么办。我们全家都反对这征兵令,更反对这场战争。我们从心里觉得这场战争是不道德的。''
\fi

\ifdefined\eng
For Stallman, the Vietnam War elicited a complex mixture of emotions: confusion, horror, and, ultimately, a profound sense of political impotence. As a kid who could barely cope in the mild authoritarian universe of private school, Stallman experienced a shiver whenever the thought of Army boot camp presented itself. He did not think he could get through it and emerge sane.
\fi

\ifdefined\chs
对于理查德来说,越战给他带来的情感是复杂的:恐惧,不解,以及最终给他带来的一种政治上无能为力的感受。理查德如此地厌恶权威,他甚至无法忍受私立学校的威权制度。别说参军,哪怕就只让他想想军队里的训练,都会令他不寒而栗。
\fi

\ifdefined\eng
``I was devastated by the fear, but I couldn't imagine what to do and didn't have the guts to go demonstrate,'' recalls Stallman, whose March 16th birthday earned him a low number in the dreaded draft lottery.  This did not affect him immediately, since he had a college deferment, one of the last before the U.S. stopped granting them; but it would affect him in a few years. ``I couldn't envision moving to Canada or Sweden. The idea of getting up by myself and moving somewhere. How could I do that? I didn't know how to live by myself. I wasn't the kind of person who felt confident in approaching things like that.''
\fi

\ifdefined\chs
``我当时被吓坏了。可我也不知道能做什么,该怎么做。更没胆量去上街参加游行。''理查德如是说。1971年,政府最终取消了学生延期入伍的政策。抽签征兵\endnote{译注:抽签征兵,1969年至1975年期间,美国实行的一种征兵制度。根据生日,在适龄青年男性中,随机分配征兵顺序。顺序在前的人将最先收到征兵令。}的结果让理查德很是担心:他3月18日的生日,征兵顺序中比较靠前。``我很难想象要移民到加大拿或者瑞典。要一个人离开过日子,我当时可不行。我根本不知道怎么照顾自己,我对自己在生活自理方面的能力非常不自信。''
\fi

\ifdefined\eng
Stallman says he was impressed by the family members who did speak out. Recalling a sticker, printed and distributed by his father, likening the My Lai massacre to similar Nazi atrocities in World War II, he says he was ``excited'' by his father's gesture of outrage. ``I admired him for doing it,'' Stallman says. ``But I didn't imagine that I could do anything. I was afraid that the juggernaut of the draft was going to destroy me.''
\fi

\ifdefined\chs
理查德说,他至今依然感慨家人在公开场合的反战行为。他记得父亲车上有一个车贴,把美莱村屠杀\endnote{译注:美莱村屠杀,1968年3月16日,美军因怀疑村民掩护越共逃亡,在越南广义省美莱村展开了屠杀。}比作纳粹的大屠杀。这个车贴是父亲亲自制作的,他还做了很多这样的车贴,分发给大家。理查德说,他很受父亲情绪的鼓舞。``我很崇拜他这么做。''理查德说,``可我没想过自己能做什么。当时我很害怕自己的一辈子就这样被征兵令毁了。''
\fi

\ifdefined\eng
However, Stallman says he was turned off by the tone and direction of much of that movement. Like other members of the Science Honors Program, he saw the weekend demonstrations at Columbia as little more than a distracting spectacle.\endnote{Chess, another Columbia Science Honors Program alum, describes the protests as ``background noise.'' ``We were all political,'' he says, ``but the SHP was important. We would never have skipped it for a demonstration.''} Ultimately, Stallman says, the irrational forces driving the anti-war movement became indistinguishable from the irrational forces driving the rest of youth culture. Instead of worshiping the Beatles, girls in Stallman's age group were suddenly worshiping firebrands like Abbie Hoffman and Jerry Rubin. To a kid already struggling to comprehend his teenage peers, slogans like ``make love not war'' had a taunting quality. Stallman did not want to make war, at least not in Southeast Asia, but nobody was inviting him to make love either.
\fi

\ifdefined\chs
可理查德也逐渐开始不喜欢反战运动的方向和调调。正如其他科学之星计划的学生一样,理查德每周末都会看到聚集在哥伦比亚大学的抗议群众\endnote{哥伦比亚科学之星计划的另一个学生,柴斯回忆当时的抗议游行为``背景噪声''。``我们都关心政治。可科学之星计划对我们来说更重要。我们从没有翘课去游行。''}。最终,理查德形容说,各种非理智力量把反战运动变得与各种其他青年人的非理智活动一样可怕。一瞬间,理查德那个年级的姑娘们都不再崇拜披头士,改去崇拜各路反战领袖,如阿比·霍夫曼(Abbie Hoffman)和杰里·鲁宾(Jerry Rubin)。理查德,这个中学生里的异类,面对各种流行趋势已然有些应接不暇了,如今又来了``要做爱不作战''这种花哨口号,实在让他觉得甚是沮丧。他显然不喜欢战争,至少不喜欢这次在东南亚的战争。当然这也并不意味着就有会有个姑娘愿意与他一起共度春宵。
\fi

\ifdefined\eng
``I didn't like the counter culture much,'' Stallman recalls. ``I didn't like the music. I didn't like the drugs. I was scared of the drugs. I especially didn't like the anti-intellectualism, and I didn't like the prejudice against technology. After all, I loved a computer. And I didn't like the mindless anti-Americanism that I often encountered. There were people whose thinking was so simplistic that if they disapproved of the conduct of the U.S. in the Vietnam War, they had to support the North Vietnamese. They couldn't imagine a more complicated position, I guess.''
\fi

\ifdefined\chs
``我并不喜欢这种对抗主流的文化。''理查德坦言,``我不喜欢他们的音乐,不喜欢毒品。我可害怕毒品。我更厌恶他们那套反理性反知识的论调,厌恶他们对各种技术的偏见。因为无论怎么说,我还是喜欢计算机的。我还经常遇到各种没头脑的反美意见,这也让我很反感。有些人的头脑真的太过简单了,他们觉得如果要反对美国参加越战,就意味着支持北越政权。他们就天真到想不出另外一种可能。''
\fi

\ifdefined\eng
Such comments underline a trait that would become the key to Stallman's own political maturation. For Stallman, political confidence was directly proportionate to personal confidence. By 1970, Stallman had become confident in few things outside the realm of math and science. Nevertheless, confidence in math gave him enough of a foundation to examine the extremes of the anti-war movement in purely logical terms.  Doing so, Stallman found the logic wanting. Although opposed to the war in Vietnam, Stallman saw no reason to disavow war as a means for defending liberty or correcting injustice.
\fi

\ifdefined\chs
这样的评论标志着理查德自己政治思想的逐步成熟。对于理查德来说,政治上的信心和自信心成正比。到了1970,理查德已经在数学和科学以外的一些领域里树立起了自信。当然,在数学方面的信心和能力,作为基础,让理查德可以从纯逻辑的角度分析越战。这种分析最终带来的结论,让理查德觉得,尽管反对越战,但不可否认,战争在维护自由和捍卫正义方面起到的积极作用。
\fi

\ifdefined\eng
In the 1980s, a more confident Stallman decided to make up for his past inactivity by participating in mass rallies for abortion rights in Washington DC.  ``I became dissatisfied with my earlier self for failing in my duty to protest the Vietnam War,'' he explains.
\fi

\ifdefined\chs
八十年代,理查德已经不再沉默。他带着这份自信,在华盛顿市参加了维护堕胎权利的聚会。``我开始不满从前的自己,觉得不该像当初那样逃避游行抗议,这是我的权利,更是我的责任。''
\fi

\ifdefined\eng
In 1970, Stallman left behind the nightly dinnertime conversations about politics and the Vietnam War as he departed for Harvard. Looking back, Stallman describes the transition from his mother's Manhattan apartment to life in a Cambridge dorm as an ``escape.'' At Harvard, he could go to his room and have peace whenever he wanted it. Peers who watched Stallman make the transition, however, saw little to suggest a liberating experience.
\fi

\ifdefined\chs
1970年,理查德离开了家,离开了每晚家中餐桌上的越战和政治讨论,去了哈佛大学。理查德回忆,从曼哈顿母亲的公寓,转到马萨诸塞州剑桥市的哈佛大学宿舍,对自己来说是一种``逃离''。可同龄人来看,他的这次大逃离并没有给他带来多少解放。
\fi

\ifdefined\eng
``He seemed pretty miserable for the first while at Harvard,'' recalls Dan Chess, a classmate in the Science Honors Program who also matriculated at Harvard. ``You could tell that human interaction was really difficult for him, and there was no way of avoiding it at Harvard. Harvard was an intensely social kind of place.''
\fi

\ifdefined\chs
``他在哈佛的第一年很痛苦。''丹·柴斯,另外一位去了哈佛大学的科学之星的学生,回忆道,``简单的人际交往对于理查德来说都是非常困难的。可这种交往在哈佛又是无法避免的。哈佛校园就是个大社交场。''
\fi

\ifdefined\eng
To ease the transition, Stallman fell back on his strengths: math and science. Like most members of the Science Honors Program, Stallman breezed through the qualifying exam for Math 55, the legendary ``boot camp'' class for freshman mathematics ``concentrators'' at Harvard. Within the class, members of the Science Honors Program formed a durable unit. ``We were the math mafia,'' says Chess with a laugh. ``Harvard was nothing, at least compared with the SHP.''
\fi

\ifdefined\chs
为了让这种过渡更顺畅,理查德重新埋头在自己的老伙伴中间:数学和自然科学。与很多科学之星计划的成员一道,理查德轻松通过数学55的课前考试,获得了选修数学55的资格。所谓数学55,是哈佛大学一门数学类课程的官方编号。它专门为未来数学家设计,以难度大而闻名全校。在这门课上,科学之星的成员凑成了一个小团体。``我们数学党的。''柴斯笑道:``跟科学之星比,哈佛不在话下。''
\fi

\ifdefined\eng
To earn the right to boast, however, Stallman, Chess, and the other SHP alumni had to get through Math 55. Promising four years worth of math in two semesters, the course favored only the truly devout. ``It was an amazing class,'' says David Harbater, a former ``math mafia'' member and now a professor of mathematics at the University of Pennsylvania. ``It's probably safe to say there has never been a class for beginning college students that was that intense and that advanced. The phrase I say to people just to get it across is that, among other things, by the second semester we were discussing the differential geometry of Banach manifolds. That's usually when their eyes bug out, because most people don't start talking about Banach manifolds until their second year of graduate school.''
\fi

\ifdefined\chs
可要修成正果,理查德,柴斯和其他科学之星成员,必须要通过数学55。这个课程把本该四年学完的数学知识,全部放在两个学期里。只有真正肯下功夫的人,才能啃下这块硬骨头。``那真是门让人神往的课。''戴维·哈伯特(David Harbater)回忆道。他曾是这门课上``数学党''的成员,如今已是宾夕法尼亚大学数学系教授。``很保守地说,几乎没有几个给大学新生设计的课程,能这么难,这么深,课业量也很少有这么大的。我经常给人这么形容这门课的难度:第二个学期,我们已经开始讲巴拿赫空间的微分几何了。一般人听了肯定瞪大眼睛,因为绝大多数人到了研究生第二年才开始说点巴拿赫空间的东西。''
\fi

\ifdefined\eng
Starting with 75 students, the class quickly melted down to 20 by the end of the second semester. Of that 20, says Harbater, ``only 10 really knew what they were doing.'' Of that 10, 8 would go on to become future mathematics professors, 1 would go on to teach physics.
\fi

\ifdefined\chs
那年这门课刚一开始有75个学生,到第二学期结束,就只有20个学生还在坚持\endnote{译注:美国大学允许学生在学期开始之前选课,上过几周,如果觉得不合适,可以放弃这门课。所以一个课程在开始的时候会有很多人上,但到了学期结束只有一部分人坚持选下了这门课,拿到成绩。}。戴维·哈伯特说,这20个学生当中,``只有10个人真正听懂了课上讲的是什么。这10个人里,8个人后来做了数学教授,另外有一个做了物理教授。''
\fi

\ifdefined\eng
``The other one,'' emphasizes Harbater, ``was Richard Stallman.''
\fi

\ifdefined\chs
``最后那一个,''戴维·哈伯特强调,``就是理查德·斯托曼。''
\fi

\ifdefined\eng
Seth Breidbart, a fellow Math 55 classmate, remembers Stallman distinguishing himself from his peers even then.
\fi

\ifdefined\chs
赛思·布莱德巴特也是数学55课上的学生,他记得当时理查德就显出了与众不同的特质。
\fi

\ifdefined\eng
``He was a stickler in some very strange ways,'' says Breidbart. There is a standard technique in math which everybody does wrong. It's an abuse of notation where you have to define a function for something and what you do is you define a function and then you prove that it's well defined. Except the first time he did and presented it, he defined a relation and proved that it's a function. It's the exact same proof, but he used the correct terminology, which no one else did. That's just the way he was.''
\fi

\ifdefined\chs
``遇到一些事,他总是特别能坚持。当时,我们经常用一种错误的方法来解题。每次我们要定义一个函数的时候,我们都会先定义一个函数,然后证明这个函数是良构的。这么做本身是用错了术语。理查德第一次也犯了这个错。可后来,他都是先定义一个关系,然后证明这个关系是一个函数。其实证明的过程都一样,但他用了正确的符号和术语。而其他没有一个人这么做。他就是这么个人。''
\fi

\ifdefined\eng
It was in Math 55 that Richard Stallman began to cultivate a reputation for brilliance. Breidbart agrees, but Chess, whose competitive streak refused to yield, says the realization that Stallman might be the best mathematician in the class didn't set in until the next year. ``It was during a class on Real Analysis,'' says Chess, now a math professor at Hunter College. ``I actually remember in a proof about complex valued measures that Richard came up with an idea that was basically a metaphor from the calculus of variations. It was the first time I ever saw somebody solve a problem in a brilliantly original way.''
\fi

\ifdefined\chs
数学55的课堂上,理查德又一次展现了他的聪明才智。布莱德巴特早早就承认了这一点。而柴斯好胜的个性则很难让他服输,但到了第二年,他还是不得不承认,理查德也许是整个课堂中最优秀的数学家。如今,柴斯已是亨特学院数学系的教授,他回忆:``在一堂实验分析的课上,理查德想出了一个关于复测度的证明,那个证明实际上是借鉴了变分法里的一些技巧。我是第一次看到有人能自己想出来,用这么漂亮的方法解决这个问题的。''
\fi

\ifdefined\eng
For Chess, it was a troubling moment. Like a bird flying into a clear glass window, it would take a while to realize that some levels of insight were simply off limits.
\fi

\ifdefined\chs
对于柴斯来说,这的确是个恼人的时刻。他就好像一只小鸟飞到了透明玻璃前,他要花些时间,才会不甘心地意识到,有些地方,虽然看得到,却未必是自己能力所及的。
\fi

\ifdefined\eng
``That's the thing about mathematics,'' says Chess. ``You don't have to be a first-rank mathematician to recognize first-rate mathematical talent. I could tell I was up there, but I could also tell I wasn't at the first rank. If Richard had chosen to be a mathematician, he would have been a first-rank mathematician.''\endnote{Stallman doubts this, however. ``One of the reasons I moved from math and physics to programming is that I never learned how to discover anything new in the former two.  I only learned to study what others had done.  In programming, I could do something useful every day.''}
\fi

\ifdefined\chs
``这恐怕就是数学之美吧。''柴斯说:``你不必需要成为一流的数学家,就可以欣赏数学天才的作品。我就处在这么一个尴尬的地方。我算不上一流数学家。如果理查德当初选择做数学家的话,他肯定会是一流的。''\endnote{理查德对此持怀疑态度。``我离开数学和物理界,转去搞计算机,其中一个重要原因,是我觉得在数学和物理领域很难有什么新的突破。我只能每天学习别人以前做了什么。在计算机领域,我每天都可以有些新东西。''}
\fi

\ifdefined\eng
For Stallman, success in the classroom was balanced by the same lack of success in the social arena. Even as other members of the math mafia gathered to take on the Math 55 problem sets, Stallman preferred to work alone. The same went for living arrangements. On the housing application for Harvard, Stallman clearly spelled out his preferences. ``I said I preferred an invisible, inaudible, intangible roommate,'' he says. In a rare stroke of bureaucratic foresight, Harvard's housing office accepted the request, giving Stallman a one-room single for his freshman year.
\fi

\ifdefined\chs
有得就必然有失,理查德在课堂上的得意,换不来社交上的成功。其他数学党的成员一般都会凑在一起讨论数学55的作业,而理查德则自己一个人应付作业题。住宿问题上理查德也坚持如此。在哈佛大学住宿申请的表格上,理查德清晰地说出了自己对室友的要求:``我说我希望能有个几乎看不见,听不见,摸不着的室友。''宿舍管理部门这次倒是难得如此识相,竟然接受了理查德的要求,在他入学的第一年,给他安排了一个单人间。
\fi

\ifdefined\eng
Breidbart, the only math-mafia member to share a dorm with Stallman that freshman year, says Stallman slowly but surely learned how to interact with other students. He recalls how other dorm mates, impressed by Stallman's logical acumen, began welcoming his input whenever an intellectual debate broke out in the dining club or dorm commons.
\fi

\ifdefined\chs
布莱德巴特是数学党成员中,唯一一个在大一期间和理查德同住一个宿舍楼的。他回忆,理查德当时的确慢慢地学着如何和别人沟通交往。他记得,当时其他几个宿舍的学生,都被理查德严密的逻辑分析能力感染,都喜欢跟他凑在宿舍楼的大厅或者餐厅里,谈天说地。
\fi

\ifdefined\eng
``We had the usual bull sessions about solving the world's problems or what would be the result of something,'' recalls Breidbart. ``Say somebody discovers an immortality serum. What do you do? What are the political results? If you give it to everybody, the world gets overcrowded and everybody dies. If you limit it, if you say everyone who's alive now can have it but their children can't, then you end up with an underclass of people without it. Richard was just better able than most to see the unforeseen circumstances of any decision.''
\fi

\ifdefined\chs
布莱德巴特说:``我们凑一起经常会扯上一些异想天开的事情,比如如何解决各种世界问题,或者某种东西如果成真了会怎么样。好比说,有人发明了一种长生不老药会如何。你该怎么办?这东西的政治影响会是什么?如果说每人都吃了这个药,最后谁都不会老,最后人越来越多,到头来还是会因为资源困乏,大家都得死掉。老是不会老了,但终究也会死。可如果限制发布这个药,好比你可以说,只有现在活着的人能拿到这个药,新生儿不能吃长生不老药,那最后世界上就会被分出一批下等人。理查德总能比大家更早看出各种决策的优劣。''
\fi

\ifdefined\eng
Stallman remembers the discussions vividly. ``I was always in favor of immortality,'' he says. ``How else would we be able to see what the world is like 200 years from now?'' Curious, he began asking various acquaintances whether they would want immortality if offered it. ``I was shocked that most people regarded immortality as a bad thing.'' Many said that death was good because there was no use living a decrepit life, and that aging was good because it got people prepared for death, without recognizing the circularity of the combination.
\fi

\ifdefined\chs
理查德至今还清晰地记得当时的讨论:``我总是对长生不老的话题感兴趣。我们要是能看到200年之后,世界是什么样子,那会如何?''出于好奇,理查德找了几个朋友,问他们愿意不愿意长生不老:``我很惊讶,不少人觉得长生不老并非什么好事情。''很多人说,死亡从某种意义上讲是个好事,因为一直活下去的话生命会显得越来越没有意义。而苍老的好处则是让人们能逐渐意识到死亡的到来。
\fi

\ifdefined\eng
Although perceived as a first-rank mathematician and first-rate informal debater, Stallman shied away from clear-cut competitive events that might have sealed his brilliant reputation. Near the end of freshman year at Harvard, Breidbart recalls how Stallman conspicuously ducked the Putnam exam, a prestigious test open to math students throughout the U.S. and Canada. In addition to giving students a chance to measure their knowledge in relation to their peers, the Putnam served as a chief recruiting tool for academic math departments. According to campus legend, the top scorer automatically qualified for a graduate fellowship at any school of his choice, including Harvard.
\fi

\ifdefined\chs
尽管众人已经公认,理查德会是个一流数学家,也是个非正式的辩论能手,可他自己却一直避免参加各种有明确排名或分界线的竞赛。布莱德巴特清晰地记得,理查德大一的时候,大家眼看着他回避参加帕特南竞赛(Putnam Competition)。帕特南竞赛是专门针对美国和加拿大数学系的本科生的竞赛。参赛者借此可以知道自己的数学水平,更重要的,比赛的结果经常被各大院校当作选拔研究生和博士生的依据。按照学生中间的流言来看,如果拿到好成绩,就可以去任何一所大学读研究生或博士,而且是免学费,全额奖学金。当然,这也包括哈佛大学。
\fi

\ifdefined\eng
Like Math 55, the Putnam was a brutal test of merit. A six-hour exam in two parts, it seemed explicitly designed to separate the wheat from the chaff. Breidbart, a veteran of both the Science Honors Program and Math 55, describes it as easily the most difficult test he ever took. ``Just to give you an idea of how difficult it was,'' says Breidbart, ``the top score was a 120, and my score the first year was in the 30s. That score was still good enough to place me 101st in the country.''
\fi

\ifdefined\chs
和数学55一样,帕特南竞赛也是各残酷的竞技场。两场考试,一共六个小时。这比赛是铁了心要把学生分出三六九等。布莱德巴特,这位参加了科学之星计划,也上了数学55的学生,依然觉得这个竞赛是他参加过的最难的考试。``就跟你这么说吧,满分120,我第一年的时候分数是30几分。就这分数,我还能排上第101名。''
\fi

\ifdefined\eng
Surprised that Stallman, the best student in the class, had skipped the test, Breidbart says he and a fellow classmate cornered him in the dining common and demanded an explanation. ``He said he was afraid of not doing well,'' Breidbart recalls.
\fi

\ifdefined\chs
可让人吃惊的是,理查德,这位全班数学最好的学生,竟然不参加这个比赛。布莱德巴特记得当时他们听到这个消息之后,几个人把理查德围在墙角,让他解释为什么不参加。``他说他怕做得不好。''布莱德巴特回忆。
\fi

\ifdefined\eng
Breidbart and the friend quickly wrote down a few problems from memory and gave them to Stallman. ``He solved all of them,'' Breidbart says, ``leading me to conclude that by not doing well, he either meant coming in second or getting something wrong.''
\fi

\ifdefined\chs
一听这话,布莱德巴特和其他几个学生迅速地写下了几个帕特南竞赛得题目,扔给理查德,让他做。``他全都做对了。我当时就觉得,所谓`做得不好',对他来说也许就是拿个亚军,或者做错一道题。''
\fi

\ifdefined\eng
Stallman remembers the episode a bit differently. ``I remember that they did bring me the questions and it's possible that I solved one of them, but I'm pretty sure I didn't solve them all,'' he says. Nevertheless, Stallman agrees with Breidbart's recollection that fear was the primary reason for not taking the test. Despite a demonstrated willingness to point out the intellectual weaknesses of his peers and professors in the classroom, Stallman hated and feared the notion of head-to-head competition -- so why not just avoid it?
\fi

\ifdefined\chs
理查德对这个事情的记忆则略有出入。``我确实记得他们给我题目了,可我好像记得我就做出来一道,反正肯定没全做对。''无论如何,理查德承认,正如布莱德巴特说的,他当时的确是出于害怕才不参加这竞赛的。理查德只是在班上指出同学或者老师的错误。但他却不喜欢,甚至害怕参加任何正面竞争。结果,他就总是避开类似的比赛。
\fi

\ifdefined\eng
``It's the same reason I never liked chess,'' says Stallman. ``Whenever I'd play, I would become so consumed by the fear of making a single mistake and losing that I would start making stupid mistakes very early in the game. The fear became a self-fulfilling prophecy.''  He avoided the problem by not playing chess.
\fi

\ifdefined\chs
理查德说:``同样的原因,我也不喜欢下棋。每次我下棋的时候,我都害怕那种一步走错,满盘皆输的情况。而这种恐惧,最后总是成真。''于是,理查德也就只能靠避免下棋,来解决这个问题。
\fi

\ifdefined\eng
Whether such fears ultimately prompted Stallman to shy away from a mathematical career is a moot issue. By the end of his freshman year at Harvard, Stallman had other interests pulling him away from the field. Computer programming, a latent fascination throughout Stallman's high-school years, was becoming a full-fledged passion. Where other math students sought occasional refuge in art and history classes, Stallman sought it in the computer-science laboratory.
\fi

\ifdefined\chs
究竟是不是因为这种恐惧,才没让理查德成为数学家,我们不得而知。但无论如何,大一结束的时候,理查德已经有了新的兴趣:编程。这兴趣从理查德中学的时候就逐渐形成,但潜伏了很久,最终在大一后期成为了他的一个主要兴趣,让他投入了自己的全部热情。其他数学系的学生都靠上艺术和历史课来放松,理查德则跑去机房缓解压力。
\fi

\ifdefined\eng
For Stallman, the first taste of real computer programming at the IBM New York Scientific Center had triggered a desire to learn more. ``Toward the end of my first year at Harvard school, I started to have enough courage to go visit computer labs and see what they had. I'd ask them if they had extra copies of any manuals that I could read.'' Taking the manuals home, Stallman would examine the machine specifications to learn about the range of different computer designs.
\fi

\ifdefined\chs
当年第一次在IBM纽约科学中心编程的经历,诱使着理查德去了解更多。他说:``在哈佛学习快一年的时候,我开始跑去哈佛的几个计算机实验室,看看他们那里有什么新东西。刚到那里,我就问他们能不能给我一份使用手册。''把这些使用手册拿回家,理查德开始自习阅读,比较各个机器这间的差别和联系。
\fi

\ifdefined\eng
One day, near the end of his freshman year, Stallman heard about a special laboratory near MIT. The laboratory was located on the ninth floor of a building in Tech Square, the mostly-commercial office park MIT had built across the street from the campus. According to the rumors, the lab itself was dedicated to the cutting-edge science of artificial intelligence and boasted the cutting-edge machines and software to match.
\fi

\ifdefined\chs
第一年快结束的时候,理查德有天听说在麻省理工学院,有个特殊的实验室。那个实验室就在校区旁边的技术广场大厦九层。传说,这个实验室是专门为顶尖的人工智能研究而设立,里面各种高级计算机和软件。
\fi

\ifdefined\eng
Intrigued, Stallman decided to pay a visit.
\fi

\ifdefined\chs
理查德被这一切吸引住了,决定亲自前往一探究竟。
\fi

\ifdefined\eng
The trip was short, about 2 miles on foot, 10 minutes by train, but as Stallman would soon find out, MIT and Harvard can feel like opposite poles of the same planet. With its maze-like tangle of interconnected office buildings, the Institute's campus offered an aesthetic yin to Harvard's spacious colonial-village yang.  Of the two, the maze of MIT was much more Stallman's style. The same could be said for the student body, a geeky collection of ex-high school misfits known more for its predilection for pranks than its politically powerful alumni.
\fi

\ifdefined\chs
从哈佛大学到那里并不远,走路大概三公里,坐地铁只要十分钟。可等到了那里,理查德才发现,麻省理工学院和哈佛大学,简直就像两个世界。麻省理工学院里真可说是``五步一楼,十步一阁''。各个建筑纠缠在一起;相比之下,哈佛大学则显得明快宽敞。两所大学,一张一弛,一阴一阳。这不仅体现在建筑风格上,在校风学风上也是如此。麻省理工学院的学生,多是中学里的怪才,他们好开玩笑,喜欢恶作剧;而哈佛大学的学生,则更多是家有深厚背景的孩子,或是从小就有政治抱负的青年。
\fi

\ifdefined\eng
The yin-yang relationship extended to the AI Lab as well. Unlike Harvard computer labs, there was no grad-student gatekeeper, no clipboard waiting list for terminal access, no atmosphere of ``look but don't touch.'' Instead, Stallman found only a collection of open terminals and robotic arms, presumably the artifacts of some AI experiment. When he encountered a lab employee, he asked if the lab had any spare manuals it could loan to an inquisitive student. ``They had some, but a lot of things weren't documented,'' Stallman recalls. ``They were hackers, after all,'' he adds wryly, referring to hackers' tendency to move on to a new project without documenting the last one.
\fi

\ifdefined\chs
这种学校风格同样延伸到计算机实验室中。与哈佛的各个计算机实验室不同,麻省理工大学的这个人工智能实验室,没有门卫;没有拿着小本子,记下等候使用终端的人,给他们排队的人;也没有贴着``严禁触摸''标识的那种氛围。到了那里,理查德就看到了几个空着没人用的终端,和几个机械臂,提醒着人们这里正在做着人工智能方面的试验。理查德找到一位在这里上班的雇员,问他能不能送自己一份使用手册。理查德回忆:``那人说他手头确实有几份闲置的手册,可好多东西都没写在手册里。怎么说,他们也是黑客。''理查德笑道,暗指黑客们往往会直奔下一个项目,而不去把之前的项目记录在文档里。
\fi

\ifdefined\eng
Stallman left with something even better than a manual: A job.  His first project was to write a PDP-11 simulator that would run on a PDP-10. He came back to the AI Lab the next week, grabbing an available terminal, and began writing the code.
\fi

\ifdefined\chs
在对方的帮助下,理查德最终得到了一份比手册更好的东西:一个在实验室的工作职位\endnote{译注:美国学生一般会在校内或校外找些工作,一方面可以积累经验,但很重要的一方面是可以挣钱交学费。}。他的第一个工作任务,是在PDP-10上写一个PDP-11的模拟器。于是,他第二周又来到人工智能实验室,找了个终端,开始写代码。
\fi

\ifdefined\eng
Looking back, Stallman sees nothing unusual in the AI Lab's willingness to accept an unproven outsider at first glance. ``That's the way it was back then,'' he says. ``That's the way it still is now. I'll hire somebody when I meet him if I see he's good. Why wait? Stuffy people who insist on putting bureaucracy into everything really miss the point. If a person is good, he shouldn't have to go through a long, detailed hiring process; he should be sitting at a computer writing code.''
\fi

\ifdefined\chs
回想起来,理查德觉得,人工智能实验室把一份工作``随便''扔给一个素不相识的外校学生这件事,并没什么不妥。``当年都是这样,如今也是如此。我见一个人,他要是很优秀,我就雇他。等什么呢?很多刻板的人喜欢在每件事上都安插各种官僚流程。这实在太愚蠢。要是一个人真的很优秀,他就不会花时间准备各种冗长拖沓的面试。他应该踏踏实实坐电脑前头写代码。''
\fi

\ifdefined\eng
To get a taste of ``bureaucratic and stuffy,'' Stallman need only visit the computer labs at Harvard. There, access to the terminals was doled out according to academic rank. As an undergrad, Stallman sometimes had to wait for hours. The waiting wasn't difficult, but it was frustrating. Waiting for a public terminal, knowing all the while that a half dozen equally usable machines were sitting idle inside professors' locked offices, seemed the height of irrational waste. Although Stallman continued to pay the occasional visit to the Harvard computer labs, he preferred the more egalitarian policies of the AI Lab. ``It was a breath of fresh air,'' he says. ``At the AI Lab, people seemed more concerned about work than status.''
\fi

\ifdefined\chs
要想了解什么是``刻板''和``官僚流程'',理查德说只要去哈佛的计算机实验室看看就行了。在那里,使用终端的先后顺序,是按照年级排序的。作为一个本科生,理查德经常要等好几个小时才能用上计算机。等待本身并不是件难事,但却是个很无聊的事情。尤其是你明明知道,在各个教授的办公室里,都有一个能用的计算机终端,可每个办公室的房门都是紧锁的,你就只能在那里按部就班地等着用那几台公共终端。在理查德看来,这实在是暴殄天物。尽管他后来也会偶尔去哈佛大学的计算机实验室,但他显然更喜欢人工智能实验室的平等环境。他说:``在那里总算能呼吸到新鲜空气了。人工智能实验室里,人们更关心手头的工作,而不是各自的地位状态。''
\fi

\ifdefined\eng
Stallman quickly learned that the AI Lab's first-come, first-served policy owed much to the efforts of a vigilant few. Many were holdovers from the days of Project MAC, the Department of Defense-funded research program that had given birth to the first time-share operating systems. A few were already legends in the computing world. There was Richard Greenblatt, the lab's in-house Lisp expert and author of MacHack, the computer chess program that had once humbled AI critic Hubert Dreyfus. There was Gerald Sussman, original author of the robotic block-stacking program HACKER. And there was Bill Gosper, the in-house math whiz already in the midst of an 18-month hacking bender triggered by the philosophical implications of the computer game LIFE.\endnote{See Steven Levy, \textit{Hackers} (Penguin USA [paperback], 1984): 144.\\Levy devotes about five pages to describing Gosper's fascination with LIFE, a math-based software game first created by British mathematician John Conway. I heartily recommend this book as a supplement, perhaps even a prerequisite, to this one.}
\fi

\ifdefined\chs
理查德很快了解到,人工智能实验室这种先到先用的原则,很大一部分要归功于当年一批警觉的小团体。其中很多人,参与了当年国防部资助的MAC项目(Porject MAC)。在这个项目里,他们开发了第一个分时操作系统。而这个小团体中的有些人,如今已经在世界计算机史册上留了名。包括实验室中的Lisp专家理查德·格林布拉特(Richard Greenblatt),他设计了国际象棋程序MacHack,让休伯特·德莱弗斯(Hubert Dreyfus)关于人工智能的观点受到重创\endnote{译注:休伯特·德莱弗斯(Hubert Dreyfus),美国哲学家。曾在1965年断言,计算机下国际象棋不可能战胜一个十岁孩子。参见他的论文《Alchemy and Artificial Intelligence》};还有著名的杰拉尔德·萨斯曼(Gerald Sussman),他曾设计HACKER这个程序,利用机器人解决堆垛问题;也有著名的数学怪才比尔·高斯伯(Bill Gosper),他当年发现了生命游戏(LIFE game)中的模式并因此获奖\endnote{参见史蒂芬·李维(Steven Levy),《黑客》(1984年,美国企鹅出版社),第144页。史蒂芬用了五页来描述高斯伯如何迷恋生命游戏。生命游戏是英国数学家约翰·康威建立的数学模型。强烈建议读一下《黑客》一书,可以作为本书的补充读物。}。
\fi

\ifdefined\eng
Members of the tight-knit group called themselves ``hackers.'' Over time, they extended the ``hacker'' description to Stallman as well. In the process of doing so, they inculcated Stallman in the ethical traditions of the ``hacker ethic.'' In their fascination with exploring the limits of what they could make a computer do, hackers might sit at a terminal for 36 hours straight if fascinated with a challenge. Most importantly, they demanded access to the computer (when no one else was using it) and the most useful information about it. Hackers spoke openly about changing the world through software, and Stallman learned the instinctual hacker disdain for any obstacle that prevented a hacker from fulfilling this noble cause. Chief among these obstacles were poor software, academic bureaucracy, and selfish behavior.
\fi

\ifdefined\chs
这批小团体里的人把自己称作``黑客''。当然将来,他们也会把理查德归类为``黑客''。在成为``黑客''的过程中,理查德被逐渐地引进黑客文化的小圈子里。这群人,会痴迷于不断挖掘计算机的极限。他们遇到挑战,可以一连36个小时坐在终端显示器前面工作。最重要的,他们对计算机和各种相关信息都有无尽的需求。他们会公开大谈如何用计算机和软件改变世界。理查德也逐渐学到,黑客们会本能地蔑视任何阻止他们实现这一伟大梦想的障碍和困难。这些障碍之中,最为重要的有以下几个方面:一是低劣的软件,二是各种学术官僚主义,三是自私自利的垄断的行为。
\fi

\ifdefined\eng
Stallman also learned the lore, stories of how hackers, when presented with an obstacle, had circumvented it in creative ways. This included various ways that hackers had opened professors' offices to ``liberate'' sequestered terminals. Unlike their pampered Harvard counterparts, MIT faculty members knew better than to treat the AI Lab's limited stock of terminals as private property. If a faculty member made the mistake of locking away a terminal for the night, hackers were quick to make the terminal accessible again -- and to remonstrate with the professor for having mistreated the community.  Some hackers did this by picking locks (``lock hacking''), some by removing ceiling tiles and climbing over the wall.  On the 9th floor, with its false floor for the computers' cables, some spelunked under it.  ``I was actually shown a cart with a heavy cylinder of metal on it that had been used to break down the door of one professor's office,''\endnote{Gerald Sussman, an MIT faculty member and hacker whose work at the AI Lab predates Stallman's, disputes this story. According to Sussman, the hackers never broke any doors to retrieve terminals.} Stallman says.
\fi

\ifdefined\chs
理查德在这里学到的东西里,还包括黑客们解决问题的各种新颖技巧和相关的轶事。这其中,就包括黑客们如何打开教授们紧锁的办公室门,``解放''被囚禁的计算机终端。哈佛大学的计算机终端每天都被娇生惯养;相比之下,麻省理工学院的教授们更了解如何应付有限的终端资源。要是哪位教授,把某个终端锁在自己的办公室里过夜,黑客们就会想法把终端重新搞到手。之后,还会大摇大摆走到教授面前,劝诫他不要如此对待实验室的同僚。黑客们的做法多种多样,有些人会直接撬锁,他们自称这是``黑''锁术;有些人则会把天花板打开,爬到天花板上面,从天花板和房顶中间的通风空隙中爬入教授的办公室;在大厦的第九层,地板和楼层地面中间有空隙,本来是用于铺设机房的电线,而黑客们则会撬开地板,从下面溜进办公室。理查德说:``我当初还给他们展示过一架手推车,上面放着一个实心的圆柱形金属,这东西后来就曾用来撞开办公室的门\endnote{杰拉尔德·萨斯曼(Gerald Sussman),他既是麻省理工学院的教授,也同时是人工智能实验室里的一名黑客。他在人工智能实验室工作的时候,理查德还没来。根据他的回忆,当时黑客们根本没有破门进入教授的办公室。}。''
\fi

\ifdefined\eng
The hackers' insistence served a useful purpose by preventing the professors from egotistically obstructing the lab's work.  The hackers did not disregard people's particular needs, but insisted that these be met in ways that didn't obstruct everyone else.  For instance, professors occasionally said they had something in their offices which had to be protected from theft.  The hackers responded, ``No one will object if you lock your office, although that's not very friendly, as long as you don't lock away the lab's terminal in it.''
\fi

\ifdefined\chs
黑客们如此坚持,使得教授们很难起私心而阻碍整个实验室的工作。当然黑客们也没有无视教授们的个人需求,只是教授的需求不能影响到别人。比如,有些教授有时候会辩解称,自己办公室里有重要的私人物品,所以必须要锁门。黑客们则会回应:``你锁上办公室的门没问题,但是别把实验室的终端也锁在里面。''
\fi

\ifdefined\eng
Although the academic people greatly outnumbered the hackers in the AI Lab, the hacker ethic prevailed. The hackers were the lab staff and students who had designed and built parts of the computers, and written nearly all the software that users used.  They kept everything working, too.  Their work was essential, and they refused to be downtrodden.  They worked on personal pet projects as well as features users had asked for, but sometimes the pet projects revolved around improving the machines and software even further. Like teenage hot-rodders, most hackers viewed tinkering with machines as its own form of entertainment.
\fi

\ifdefined\chs
虽然在人数上,黑客们处于劣势,但黑客的文化在不断追求、不断进步的计算机领域还是占了上风。这些黑客,很多是实验室的成员,或者是学生。实验室的计算机上,有些硬件就是他们设计的;而几乎所有软件,都是这些黑客写的。他们不仅创造了整个系统,还维护着这个系统的运行。他们的工作至关重要,因此他们拒绝来自任何方面的压抑。他们会根据其他用户的需求写程序,也会同时维护着自己的个人项目。有时候,这些个人项目会越来越庞大,甚至影响深远。黑客们每天为计算机系统增砖添瓦,并以此为乐。
\fi

\ifdefined\eng
Nowhere was this tinkering impulse better reflected than in the operating system that powered the lab's central PDP-10 computer. Dubbed ITS, short for the Incompatible Time Sharing system, the operating system incorporated the hacking ethic into its very design. Hackers had built it as a protest to Project MAC's original operating system, the Compatible Time Sharing System, CTSS, and named it accordingly. At the time, hackers felt the CTSS design too restrictive, limiting programmers' power to modify and improve the program's own internal architecture if needed. According to one legend passed down by hackers, the decision to build ITS had political overtones as well. Unlike CTSS, which had been designed for the IBM 7094, ITS was built specifically for the PDP-6. In letting hackers write the system themselves, AI Lab administrators guaranteed that only hackers would feel comfortable using the PDP-6. In the feudal world of academic research, the gambit worked. Although the PDP-6 was co-owned in conjunction with other departments, AI researchers soon had it to themselves.  Using ITS and the PDP-6 as a foundation, the Lab had been able to declare independence from Project MAC shortly before Stallman's arrival.\endnote{\textit{Ibid.}}
\fi

\ifdefined\chs
这种工作的结果之一,就是运行在实验室核心的PDP-10计算机上的操作系统。它被称做ITS,全称``非兼容型分时系统''(Incompatible Time Sharing system)。这个系统从头到脚都体现着黑客的文化。它的诞生,本来是用来抗议MAC项目的操作系统:CTSS,即``兼容型分时系统''(Compatible Time Sharing System)。这种抗议从名字中的``非兼容型''几个字上就能看出来。当时,黑客们觉得CTSS的很多设计太过严苛,因为它限制了程序员改进现有程序的行为。沿袭着黑客们一贯以来的作风,设计ITS的决定本身就是政治上正确的。CTSS本身是为IBM 7094设计的,ITS则是为PDP-6设计。为了让黑客们更好地写出自己的操作系统,人工智能实验室的管理员保证可以让黑客们使用PDP-6计算机。在分封制度盛行的学术圈,这策略确实奏效。虽然PDP-6以前是和其他几个系共用的,可后来最终还是归了人工智能的研究人员。理查德来之前的那段时间,黑客们终于利用ITS和这台PDP-6做基础,正式宣告自己独立于MAC项目\endnote{\textit{Ibid.}}。
\fi

\ifdefined\eng
By 1971, ITS had moved to the newer but compatible PDP-10, leaving the PDP-6 for special stand-alone uses. The AI PDP-10 had a very large memory for 1971, equivalent to a little over a megabyte; in the late 70s it was doubled.   Project MAC had bought two other PDP-10s; all were located on the 9th floor, and they all ran ITS.  The hardware-inclined hackers designed and built a major hardware addition for these PDP-10s, implementing paged virtual memory, a feature lacking in the standard PDP-10.\endnote{I apologize for the whirlwind summary of ITS' genesis, an operating system many hackers still regard as the epitome of the hacker ethos. For more information on the program's political significance, see Simson Garfinkel, \textit{Architects of the Information Society: Thirty-Five Years of the Laboratory for Computer Science at MIT} (MIT Press, 1999).}
\fi

\ifdefined\chs
1971年,ITS被安装在更新一代的PDP-10上。那台老的PDP-6则被单独拿出来以做他用。人工智能实验室这台PDP-10有大约1MB的内存,在当时算是非常大的了。到七十年代末,又给它的内存加了一倍。MAC项目也为实验室带来了两台PDP-10,它们都被放在大厦九楼,运行着ITS系统。PDP-10本身没有内存分页机制。为此,对硬件感兴趣的黑客们为这三台PDP-10设计了内存分页模块\endnote{原谅我草草地带过了ITS的历史。这个操作系统至今依然被很多黑客视为黑客文化的缩影。关于这个操作系统的更多历史,参见西姆森·加芬克尔(Simson Garfinkel)的书《信息社会的建筑师:麻省理工学院计算机实验室的三十五年》(麻省理工学院出版社,1999年)。Architects of the Information Society: Thirty-Five Years of the Laboratory for Computer Science at MIT}。
\fi

\ifdefined\eng
As an apprentice hacker, Stallman quickly became enamored with ITS. Although forbidding to some non-hackers, ITS boasted features most commercial operating systems wouldn't offer for years (or even to this day), features such as multitasking, applying the debugger immediately to any running program, and full-screen editing capability. 
\fi

\ifdefined\chs
作为一名黑客学徒,理查德一下子被ITS迷住了。尽管ITS会吓跑一些门外汉,可它却比其他商业操作系统提早几年实现了很多特性,甚至有些特性是今天很多商业操作系统都没有提供的。这包括多任务系统,在线调试任何运行中的程序,全屏编辑等等。
\fi

\ifdefined\eng
``ITS had a very elegant internal mechanism for one program to examine another,'' says Stallman, recalling the program. ``You could examine all sorts of status about another program in a very clean, well-specified way.''  This was convenient
not only for debugging, but also for programs to start, stop or control other programs.
\fi

\ifdefined\chs
理查德回忆:``ITS内部有一套非常优雅的机制,可以让一个程序实时监测另一个运行中的程序。你可以获得另外某个程序的各种信息,内容清晰,接口明确。''这个功能不仅对调试程序有很大帮助,而且可以轻松地开关和控制别的程序。
\fi

\ifdefined\eng
Another favorite feature would allow the one program to freeze another program's job cleanly, between instructions. In other operating systems, comparable operations might stop the program in the middle of a system call, with internal status that the user could not see and that had no well-defined meaning. In ITS, this feature made sure that monitoring the step-by-step operation of a program was reliable and consistent.
\fi

\ifdefined\chs
另外一个功能,就是允许一个程序暂停另一个程序,让那个程序停在指定的两个指令之间。在别的操作系统里,类似的功能只能让程序停留在某个系统调用中,把程序停在这个状态,会有很多内部隐藏状态用户都看不到。而在ITS里,这一功能则可以保证一条指令一条指令地监视另一个程序运行,并且能做到停顿之后,一样能继续运行。
\fi

\ifdefined\eng
``If you said, `Stop the job,' it would always be stopped in user mode. It would be stopped between two user-mode instructions, and everything about the job would be consistent for that point,'' Stallman says. ``If you said, `Resume the job,' it would continue properly. Not only that, but if you were to change the (explicitly visible) status of the job and continue it, and later change it back, everything would be consistent. There was no hidden status anywhere.''
\fi

\ifdefined\chs
理查德说:``如果你说:`把这个任务暂停',那它肯定会停在用户态。而且会停在两条用户态的指令中间。任何运行过程中的变量状态,都会保留。你再说:`继续跑它',它就真的会继续跑下去,不会因为暂停而搞乱程序,导致前后不一致。还不止如此,你还能把某个变量或者哪个状态给修改了。让它继续跑,然后再把它改回去,这程序就能和当初一样。所有状态都是用户可见的,没有什么隐藏状态。''
\fi

\ifdefined\eng
Starting in September 1971, hacking at the AI Lab had become a regular part of Stallman's weekly school schedule. From Sunday through Friday, Stallman was at Harvard. As soon as Friday afternoon arrived, however, he was on the subway, heading down to MIT for the weekend. Stallman usually made sure to arrive well before the ritual food run. Joining five or six other hackers in their nightly quest for Chinese food, he would jump inside a beat-up car and head across the Harvard Bridge into nearby Boston. For the next hour or so, he and his hacker colleagues would discuss everything from ITS to the internal logic of the Chinese language and pictograph system. Following dinner, the group would return to MIT and hack code until dawn, or perhaps go to Chinatown again at 3 a.m.
\fi

\ifdefined\chs
从1971年开始,理查德每周都会定期跑去人工智能实验室。周日到周五,理查德会呆在哈佛大学。一到了周五下午,他就坐上地铁,直奔麻省理工学院,在那里度过整个周末。他一般会在晚饭时间赶到,然后约上五六个黑客,一起去吃中餐。他们跳进一辆旧车上,穿过哈佛大桥,直奔波士顿。接下来的四个小时里,他们凑在一起就开始谈天说地。话题从ITS系统,一直到汉语和象形文字的内在逻辑。晚饭结束,这群人再回到麻省理工,写代码一直到天亮。期间也许会在半夜三点多的时候,再跑去唐人街一趟。
\fi

\ifdefined\eng
Stallman might stay up all morning hacking, or might sleep Saturday morning on a couch. On waking he would hack some more, have another Chinese dinner, then go back to Harvard.  Sometimes he would stay through Sunday as well.  These Chinese dinners were not only delicious; they also provided sustenance lacking in the Harvard dining halls, where on the average only one meal a day included anything he could stomach. (Breakfast did not enter the count, since he didn't like most breakfast foods and was normally asleep at that hour.)
\fi

\ifdefined\chs
第二天上午,理查德也许会继续折腾代码,也有可能找个沙发睡上一觉。醒来之后,他会继续写代码,找点中餐吃吃,然后回到哈佛。有时候,他会再多呆一天,等到周日再回去。这些中餐不仅美味,而且比哈佛大学食堂的菜内容丰富,营养全面。吃上一顿能管一天。所以理查德他们一般一天也就吃一顿饭。(对于他来说,早餐一般不在考虑范围之内。一方面,早餐中的食物他不喜欢;另一方面,早餐时间他一般都会睡大觉。)
\fi

\ifdefined\eng
For the geeky outcast who rarely associated with his high-school peers, it was a heady experience to be hanging out with people who shared the same predilection for computers, science fiction, and Chinese food. ``I remember many sunrises seen from a car coming back from Chinatown,'' Stallman would recall nostalgically, 15 years after the fact in a speech at the Swedish Royal Technical Institute. ``It was actually a very beautiful thing to see a sunrise, 'cause that's such a calm time of day. It's a wonderful time of day to get ready to go to bed. It's so nice to walk home with the light just brightening and the birds starting to chirp; you can get a real feeling of gentle satisfaction, of tranquility about the work that you have done that night.''\endnote{See Richard Stallman, ``RMS lecture at KTH (Sweden),'' (October 30, 1986), \url{http://www.gnu.org/philosophy/stallman-kth.html}.}
\fi

\ifdefined\chs
作为一个中学时期很少有社交活动的宅男,如今和这么多跟自己类似的人混在一起,确实让理查德甚是陶醉。他们这些人,都对计算机感兴趣,都喜欢科幻小说,喜欢吃中餐。``我记得很多次,我们从唐人街回来,在车上看日出。''十五年后,在瑞典皇家技术研究所的一次演讲上,理查德回想当年,一切依旧历历在目:``欣赏日出真的是个很享受的事情。那是一天之中最平静的时刻。也是一天之中上床睡觉的好时候。倘若这时候徒步走回家,伴着几缕阳光,耳畔回响着鸟鸣,再想想你整整一夜完成的工作,你会有种由内而外地舒适和满足感。''\endnote{参见理查德在瑞典皇家技术研究所的演讲(1986年10月30日):\url{http://www.gnu.org/philosophy/stallman-kth.html}}
\fi

\ifdefined\eng
The more Stallman hung out with the hackers, the more he adopted the hacker world view. Already committed to the notion of personal liberty, Stallman began to infuse his actions with a sense of communal duty. When others violated the communal code, Stallman was quick to speak out. Within a year of his first visit, Stallman was the one opening locked offices to recover the sequestered terminals that belonged to the lab community as a whole. In true hacker fashion, Stallman also sought to make his own personal contribution to the art. One of the most artful door-opening tricks, commonly attributed to Greenblatt, involved bending a stiff wire into several right angles and attaching a strip of tape to one end. Sliding the wire under the door, a hacker could twist and rotate the wire so that the tape touched the inside doorknob. Provided the tape stuck, a hacker could turn the doorknob by pulling the handle formed from the outside end of the wire.
\fi

\ifdefined\chs
理查德和这群黑客呆得越来越久,他也逐渐开始了解和接纳黑客们的世界观。曾经他坚持个人自由,如今,在这基础之上,又注入了公共责任的概念。当其他人违反了这种公共准则,理查德会毫不留情地当面指出。在麻省理工学院的第一年,理查德也曾想法打开教授办公室的门,拿回属于实验室的计算机终端。作为一个黑客,理查德也贡献了自己的开门技巧。以前大家一直用着一个开门的方法,据说是格林布拉特(Greenblatt)最先使用的。这个方法要求用到一条硬的电线,把这条电线折几个弯,从一盒磁带里抽出磁条,拴在弯过的电线一头,再把磁条做成一个绳套。开门的时候,先从门下的缝隙里把带磁条一头的电线躺着穿过门缝,等电线和磁条都进了门,再把电线立起来。来回晃几下电线,直到磁条做的绳套能套在门把手上。一旦套紧,黑客就可以把电线往后拉,这样,门就从里面被打开了。
\fi

\ifdefined\eng
When Stallman tried the trick, he found it hard to execute. Getting the tape to stick wasn't always easy, and twisting the wire in a way that turned the doorknob was similarly difficult. Stallman thought about another method: sliding away ceiling tiles to climb over the wall. This always worked, if there was a desk
to jump down onto, but it generally covered the hacker in itchy fiberglass.  Was there a way to correct that flaw?  Stallman considered an alternative approach. What if, instead of slipping a wire under the door, a hacker slid away two ceiling panels and reached over the wall with a wire?
\fi

\ifdefined\chs
当理查德自己亲自尝试这个方法的时候,他发现这法子操作起来难度不小。要让磁条套上门把手,可是个体力活。而之前要怎么把电线折出几个弯,也是有学问的。为此,理查德开始另寻他法:他开始想法把天花板上的板子移开,然后爬进天花板和房顶中间的通风空隙中。爬到办公室上方,跳到桌子上,一切就大功告成了。这法子一直都行得通,可这一路,总会沾上一身让人痒痒的玻璃纤维。理查德开始考虑怎么解决这个问题。他想到,如果不把电线从门下缝隙滑过去,而是从天花板上悬下去开门,会不会可以呢?
\fi

\ifdefined\eng
Stallman took it upon himself to try it out. Instead of using a wire, Stallman draped out a long U-shaped loop of magnetic tape with a short U of adhesive tape attached sticky-side-up at the base. Reaching across over the door jamb, he dangled the tape until it looped under the inside doorknob. Lifting the tape until the adhesive stuck, he then pulled on one end of the tape, thus turning the doorknob. Sure enough, the door opened. Stallman had added a new twist to the art of getting into a locked room.
\fi

\ifdefined\chs
理查德自己开始尝试。他并没有用电线,而是把一盒磁带里的磁条拿出来,拿着两端,向下悬成一个U形,把磁条有粘性的部分冲上。移开天花板上的两块板子,站到高处,把U形的磁条从通风道甩到房间里。磁条会从天花板上垂下去,晃悠几下,把U形磁条套在门把手上。接着把磁条往上一拉,门就顺势打开了。就这样,理查德又为黑客开门技巧做了贡献。
\fi

\ifdefined\eng
``Sometimes you had to kick the door after you turned the door knob,'' says Stallman, recalling a slight imperfection of the new method. ``It took a little bit of balance to pull it off while standing on a chair on a desk.''
\fi

\ifdefined\chs
``有时候,门锁打开后你还得踹两脚才能把门弄开。''理查德回忆起这个方法的一点瑕疵:``拉磁条的时候你得小心点,一般是站在一个椅子上,下面还要再垫个桌子。这时候保持平衡就很重要。''
\fi

\ifdefined\eng
Such activities reflected a growing willingness on Stallman's part to speak and act out in defense of political beliefs. The AI Lab's spirit of direct action had proved inspirational enough for Stallman to break out of the timid impotence of his teenage years. Opening up an office to free a terminal wasn't the same as taking part in a protest march, but it was effective in a way that most protests weren't: it solved the problem at hand.
\fi

\ifdefined\chs
这些活动从一个方面反映出理查德期望表达自己的政治诉求。人工智能实验室的实干精神,让理查德打破了中学以来的沉默。打开办公室的门和上街游行虽然不可同日而语,但从效率的角度分析,与游行比起来,它快速有效,能直接解决问题。
\fi

\ifdefined\eng
By the time of his last years at Harvard, Stallman was beginning to apply the whimsical and irreverent lessons of the AI Lab back at school.
\fi

\ifdefined\chs
在哈佛大学的最后一年,理查德把人工智能实验室的那套古灵精怪的行事理念用到了哈佛的校园里。
\fi

\ifdefined\eng
``Did he tell you about the snake?'' his mother asks at one point during an interview. ``He and his dorm mates put a snake up for student election. Apparently it got a considerable number of votes.''
\fi

\ifdefined\chs
``他跟你说那条蛇了吗?''他母亲在一次采访中突然问起,``他还有他宿舍那边的几个人一起,把一条蛇拿去参加了学生会选举。而且最后那条蛇还拿到不少选票。''
\fi

\ifdefined\eng
The snake was a candidate for election within Currier House, Stallman's dorm, not the campus-wide student council. Stallman does remember the snake attracting a fair number of votes, thanks in large part to the fact that both the snake and its owner both shared the same last name. ``People may have voted for it because they thought they were voting for the owner,'' Stallman says. ``Campaign posters said that the snake was `slithering for' the office. We also said it was an `at large' candidate, since it had climbed into the wall through the ventilating unit a few weeks before and nobody knew where it was.''
\fi

\ifdefined\chs
根据理查德的回忆,那条蛇实际是拿去参加了卡瑞尔楼的学生干部竞选。那是理查德住的宿舍楼,并非全校范围内的学生竞选。理查德的确记得当时那条蛇拿到了不少选票,但很大一个原因,是因为这条蛇的姓和它主人一样。``人们也许以为是给这条蛇的主人在投票。''理查德说,``我们的选举海报上写那条蛇是个很有`潜力'的候选人。因为竞选开始前几周,它刚刚`潜'到通风口里,我们好几个人极尽全``力''也没能找到它。''
\fi

\ifdefined\eng
Stallman and friends also ``nominated'' the house master's 3-year-old son. ``His platform was mandatory retirement at age seven,'' Stallman recalls. Such pranks paled in comparison to the fake-candidate pranks on the MIT campus, however. One of the most successful fake-candidate pranks was a cat named Woodstock, which actually managed to outdraw most of the human candidates in a campus-wide election. ``They never announced how many votes Woodstock got, and they treated those votes as spoiled ballots,'' Stallman recalls. ``But the large number of spoiled ballots in that election suggested that Woodstock had actually won. A couple of years later, Woodstock was suspiciously run over by a car. Nobody knows if the driver was working for the MIT administration.'' Stallman says he had nothing to do with Woodstock's candidacy, ``but I admired it.''\endnote{In an email shortly after this book went into its final edit cycle, Stallman says he drew political inspiration from the Harvard campus as well. ``In my first year of Harvard, in a Chinese History class, I read the story of the first revolt against the Qin dynasty,'' he says.  (That's the one whose cruel founder burnt all the books and was buried with the terra cotta warriors.)  ``The story is not reliable history, but it was very moving.''}
\fi

\ifdefined\chs
理查德和几个朋友还提名了楼管的三岁儿子。``给他选举时候提出的政治纲领,就是七岁强制退休。''不过,理查德的这些恶作剧与麻省理工学院里的选举恶作剧比起来,实在是小巫见大巫了。在麻省理工学院最成功的一次选举恶作剧,是提名一只名叫伍德斯托克的猫参加全校学生委员会的选举。最后这只猫得到的选票比许多候选人的选票都多。``他们最后也没有宣布伍德斯托克到底得了多少票。他们把这些选票当作了废票。可选举结果一出,大家看到了一大堆的废票,从废票的数量上看,伍德斯托克应该是可以赢得这次选举了。几年之后,伍德斯托克据说是在马路上被一辆车压死了。''理查德如此回忆道。他说他当年和伍德斯托克的选举没有一点关系,``可我很崇拜这事\endnote{本书最后一审期间,理查德发来邮件说,他从哈佛大学也得到了很多政治上的鼓舞。理查德说:``我大一的时候,修了中国历史这门课。读了很多关于反对秦朝暴政的故事。虽然很多故事也许没有史料依据,但依旧振奋人心。''}。''
\fi

\ifdefined\eng
At the AI Lab, Stallman's political activities had a sharper-edged tone. During the 1970s, hackers faced the constant challenge of faculty members and administrators pulling an end-run around ITS and its hacker-friendly design. ITS allowed anyone to sit down at a console and do anything at all, even order the system to shut down in five minutes. If someone ordered a shutdown with no good reason, some other user canceled it. In the mid-1970s some faculty members (usually those who had formed their attitudes elsewhere) began calling for a file security system to limit access to their data.   Other operating systems had such features, so those faculty members had become accustomed to living under security, and to the feeling that it was protecting them from something dangerous. But the AI Lab, through the insistence of Stallman and other hackers, remained a security-free zone.
\fi

\ifdefined\chs
在人工智能实验室,理查德的政治活动更加积极,产生的矛盾也更尖锐。在七十年代,学校的一些教授和管理员一度发出通知,说即将停止在实验室的计算机上运行ITS,也将停止支持很多黑客们喜欢的设计。ITS允许任何人坐在终端前,做任何事情。包括把整个计算机关机。这在当年,可是个大事。因为很多人都会通过终端使用同一台计算机,关机意味着其他人也必须暂停手上的工作。不过,如果别人关机,你会收到即将关机的通知。要是你手头的工作还没完成,你也可以取消这次关机操作。到了七十年代中期,一些教授开始对文件的安全保护提出要求,要求他们能够控制哪些用户才能访问自己的文件。可人工智能实验室的这些黑客们,还是坚持要维护一个没有控制的系统。
\fi

\ifdefined\eng
Stallman presented both ethical and practical arguments against adding security. On the ethical side, Stallman appealed to the AI Lab community's traditions of intellectual openness and trust. On the practical side, he pointed to the internal structure of ITS, which was built to foster hacking and cooperation rather than to keep every user under control.  Any attempt to reverse that design would require a major overhaul. To make it even more difficult, he used up the last empty field in each file's descriptor for a feature to record which user had most recently changed the file. This feature left no place to store file security information, but it was so useful that nobody could seriously propose to remove it.
\fi

\ifdefined\chs
于是,理查德对这些安全控制特性从道德和实践上给予了否定。从道德上说,人工智能实验室的这个圈子里,大家都有着开放和信任的传统。从实践上看,ITS系统本身也是为了开发和协作而设计的,并非要控制哪些用户。任何与这初衷相违背的特性都需要额外的精力来开发维护,完全没有必要。为了让安全特性更难实现,理查德把文件描述符的最后一点富裕空间,用来存储用户的最后修改时间。最后修改时间是个很重要的信息,谁也不想把它去掉。可这么一来,整个文件描述符也就没有任何地方来存储安全信息了。
\fi

\ifdefined\eng
``The hackers who wrote the Incompatible Timesharing System decided that file protection was usually used by a self-styled system manager to get power over everyone else,'' Stallman would later explain. ``They didn't want anyone to be able to get power over them that way, so they didn't implement that kind of a feature. The result was, that whenever something in the system was broken, you could always fix it'' (since access control did not stand in your way).\endnote{See Richard Stallman (1986).}
\fi

\ifdefined\chs
``设计ITS的黑客们认为,文件访问控制这个特性,是为自私的系统管理员设计的,这样他们这些管理员就可以凌驾在任何用户之上。''理查德事后解释:``黑客们可不希望有谁被任何人控制约束,所以他们不会实现这样的功能。这么做的结果是,一旦系统出了毛病,任何人都能赶在第一时间修复它。''(因为没有访问控制功能去阻止你去访问任何文件。)\endnote{参见上述的理查德在瑞典皇家技术研究所的演讲。}
\fi

\ifdefined\eng
Through such vigilance, hackers managed to keep the AI Lab's machines security-free. In one group at the nearby MIT Laboratory for Computer Sciences, however, security-minded faculty members won the day. The DM group installed its first password system in 1977. Once again, Stallman took it upon himself to correct what he saw as ethical laxity. Gaining access to the software code that controlled the password system, Stallman wrote a program to decrypt the encrypted passwords that the system recorded.  Then he started an email campaign, asking users to choose the null string as their passwords. If the user had chosen ``starfish,'' for example, the email message looked something like this:
\fi

\ifdefined\chs
这种警觉,让黑客们得以在人工智能实验室营造出一个没有管制的世界。可在旁边的麻省理工学院计算机科学实验室里,对安全敏感的教授们则取得了胜利。动态建模组最先在1977年在系统中使用了密码登陆。又一次,斯托曼挺身而出,纠正了这种自己看来``不道德''的行为。他先获得了密码登录系统代码的访问权限,然后写了一个程序,解密已经加密存储的密码。然后他开始给系统上的每个用户发邮件,劝说他们放弃使用密码。这封邮件大概是这样的:
\fi

\ifdefined\eng
\begin{quote}
I see you chose the password ``starfish''. I suggest that you switch to the password ``carriage return'', which is what I use. It's easier to type, and also opposes the idea of passwords and security.
\end{quote}
\fi

\ifdefined\chs
\begin{quote}
我看到你用``starfish''做密码了,我建议你把密码直接改成``回车'',我用的就是这个密码。这密码简单好记,而且可以让你忘了密码和安全这些繁杂琐事。
\end{quote}
\fi

\ifdefined\eng
The users who chose ``carriage return'' -- that is, users who simply pressed the Enter or Return button, entering a blank string instead of a unique password -- left their accounts accessible to the world at large, just as all accounts had been, not long before. That was the point: by refusing to lock the shiny new locks on their accounts, they ridiculed the idea of having locks. They knew that the weak security implemented on that machine would not exclude any real intruders, and that this did not matter, because there was no reason to be concerned about intruders, and that no one wanted to intrude anyway, only to visit.
\fi

\ifdefined\chs
所谓用``回车''做密码,就等同于没有密码。这个密码让其他用户也可以访问自己的帐号。这和以前没有密码的日子没什么分别。这就是理查德和黑客们要表达的:他们拒绝任何光鲜亮美的枷锁,他们会嘲弄任何企图使用枷锁的想法。他们知道,这种安全保护能力很弱的系统,不可能防范任何入侵者。可这无所谓,因为完全没必要考虑什么入侵者,从来就没有入侵者,只有访客。
\fi

\ifdefined\eng
Stallman, speaking in an interview for the 1984 book \textit{Hackers}, proudly noted that one-fifth of the LCS staff accepted this argument and employed the null-string password.\endnote{See Steven Levy, \textit{Hackers} (Penguin USA [paperback], 1984): 417.}
\fi

\ifdefined\chs
理查德在一次为1984年出版的《黑客》一书的采访中,很骄傲地提到,当年计算机科学实验室里,有五分之一的人换了``回车''做密码\endnote{参见史蒂芬·李维(Steven Levy),《黑客》(1984年,美国企鹅出版社),第417页。}。
\fi

\ifdefined\eng
Stallman's null-string campaign, and his resistance to security in general, would ultimately be defeated. By the early 1980s, even the AI Lab's machines were sporting password security systems. Even so, it represented a major milestone in terms of Stallman's personal and political maturation. Seen in the context of Stallman's later career, it represents a significant step in the development of the timid teenager, afraid to speak out even on issues of life-threatening importance, into the adult activist who would soon turn needling and cajoling into a full-time occupation.
\fi

\ifdefined\chs
理查德的这种反密码抗议,以及各种拒绝安全保护的行为,最终还是要失败的。八十年代起,哪怕是人工智能实验室,也开始支持密码登陆系统。可即便如此,理查德的行为标志着他个人政治思维成熟的里程碑。理查德迈出了一大步。当年他还是怯懦的中学生时,哪怕遇到决定生死的大事,他也不敢上街公开反对。如今,他则成了一个活跃人士,把各种嘻笑怒骂当作家常便饭。
\fi

\ifdefined\eng
In voicing his opposition to computer security, Stallman drew on many of the key ideas that had shaped his early life: hunger for knowledge, distaste for authority, and frustration over prejudice and secret rules that rendered some people outcasts. He would also draw on the ethical concepts that would shape his adult life: responsibility to the community, trust, and the hacker spirit of direct action. Expressed in software-computing terms, the null string represents the 1.0 version of the Richard Stallman political worldview -- incomplete in a few places but, for the most part, fully mature.
\fi

\ifdefined\chs
反对计算机安全这事,体现了理查德早年性格中的几个关键特质:对知识如饥似渴,对权威厌恶嘲弄;可又烦恼于别人对自己的各种偏见不解,被一些人看作异类。这些也正反映出了未来主导他行动的道德基础:对社区的责任感和信任;还有遇到问题,直接行动的黑客精神。用软件界的行话说,反密码抗议行动体现的正是理查德1.0版。虽然有些地方还很不完善,但大部分已经成熟。
\fi

\ifdefined\eng
Looking back, Stallman hesitates to impart too much significance to an event so early in his hacking career. ``In that early stage there were a lot of people who shared my feelings,'' he says. ``The large number of people who adopted the null string as their password was a sign that many people agreed that it was the proper thing to do. I was simply inclined to be an activist about it.''
\fi

\ifdefined\chs
理查德自己,则不太赞成把他这样的早年的黑客经历强调得太多。他说:``早年间,很多人都和我有同样的想法。当时那么多人最后选择了不用密码,这事实本身就说明他们很多人认为这么做是对的。我只是在这方面比较活跃。''
\fi

\ifdefined\eng
Stallman does credit the AI Lab for awakening that activist spirit, however. As a teenager, Stallman had observed political events with little idea as to how he could do or say anything of importance. As a young adult, Stallman was speaking out on matters in which he felt supremely confident, matters such as software design, responsibility to the community, and individual freedom. ``I joined this community which had a way of life which involved respecting each other's freedom,'' he says. ``It didn't take me long to figure out that that was a good thing. It took me longer to come to the conclusion that this was a moral issue.''
\fi

\ifdefined\chs
不过理查德依旧很感激人工智能实验室氛围唤醒了他作为政治活动家潜质。曾经还是少年的时候,理查德目睹了很多政治事件,他很难想象自己能做出什么有影响的事情。如今虽只过弱冠之年,他却已经可以在自己的领域里指手划脚,自信地为整个社区负责,也为个人自由出头。他说:``我加入了这个圈子,这个圈子的文化非常强调尊重每个人的自由。我很快就意识到这是个好事,可花了很长的时间后,我才觉得这是一种道德责任。''
\fi

\ifdefined\eng
Hacking at the AI Lab wasn't the only activity helping to boost Stallman's esteem. At the start of his junior year at Harvard, Stallman began participating in a recreational international folk dance group which had just been started in Currier House. He was not going to try it, considering himself incapable of dancing, but a friend pointed out, ``You don't know you can't if you haven't tried.'' To his amazement, he was good at it and enjoyed it. What started as an experiment became another passion alongside hacking and studying; also, occasionally, a way to meet women, though it didn't lead to a date during his college career.  While dancing, Stallman no longer felt like the awkward, uncoordinated 10-year-old whose attempts to play football had ended in frustration. He felt confident, agile, and alive. In the early 80s, Stallman went further and joined the MIT Folk Dance Performing Group.   Dancing for audiences, dressed in an imitation of the traditional garb of a Balkan peasant, he found being in front of an audience fun, and discovered an aptitude for being on stage which later helped him in public speaking.
\fi

\ifdefined\chs
理查德不仅在人工智能实验室获得了黑客们的尊重。在哈佛的第一年,理查德参加了一个世界民间舞蹈小组。这个小组刚刚在卡瑞尔宿舍楼成立。理查德一开始也没打算加入,因为他觉得自己根本不适合跳舞。可他朋友却说:``你不试试怎么知道呢?''后来他真的去试了,出乎意料的是,他不仅很擅长跳舞,而且很享受其中的乐趣。这个无心之举最后竟然发展成了他的另一大爱好。这也给了他不少机会和女孩子搭讪,可惜在大学期间,每次搭讪都没能换来单独的约会。跳舞让理查德明白自己不再是那个手脚不协调,不会玩橄榄球的10岁少年了。他由此变得更加自信,灵活,有活力。在八十年代早期,理查德走得更远了:他加入了麻省理工学院民间舞蹈表演队。他穿着巴尔干半岛的传统服饰,在观众面前表演舞蹈。这让他甚是陶醉。舞台上面对观众的经历,也为他日后在各种公共场合中演讲打下了基础。
\fi

\ifdefined\eng
Although the dancing and hacking did little to improve Stallman's social standing, they helped him overcome the sense of exclusion that had clouded his pre-Harvard life. In 1977, attending a science-fiction convention for the first time, he came across Nancy the Buttonmaker, who makes calligraphic buttons saying whatever you wish. Excited, Stallman ordered a button with the words ``Impeach God'' emblazoned on it.
\fi

\ifdefined\chs
尽管跳舞和写代码没能提高多少他的社交能力,可刚到哈佛时那种疏远不自然的感觉却从此烟消云散。1977年,在一次科幻小说聚会上,他看到了南希徽章制作机,你可以随便写点格言座右铭什么的,把它制作成徽章。理查德很是兴奋,他当场制作了一个徽章,上面写着``选举上帝''。
\fi

\ifdefined\eng
For Stallman, the ``Impeach God'' message worked on many levels. An atheist since early childhood, Stallman first saw it as an attempt to start a ``second front'' in the ongoing debate on religion. ``Back then everybody was arguing about whether a god existed,'' Stallman recalls. ```Impeach God' approached the subject from a completely different viewpoint. If a god was so powerful as to create the world and yet did nothing to correct the problems in it, why would we ever want to worship such a god? Wouldn't it be more just to put it on trial?''
\fi

\ifdefined\chs
在理查德看来,所谓``选举上帝''可以从很多层面上来解释。作为一个从小就坚持无神论的人,理查德把``选举上帝''看作是信仰争论的另一种角度。``那时候,很多人都在讨论上帝究竟是不是存在。`选举上帝'四个字给了人们一个全新的角度。如果说,咱们这位上帝能耐这么大,大到可以创造整个世界,可却把这个世界中的问题扔在那里,不闻不问,那我们何必要膜拜这么一个上帝呢?这样的上帝难道不该放到法庭上审讯一番吗?''
\fi

\ifdefined\eng
At the same time, ``Impeach God'' was a reference to the the Watergate scandal of the 1970s, in effect comparing a tyrannical deity to Nixon.  Watergate affected Stallman deeply. As a child, Stallman had grown up resenting authority. Now, as an adult, his mistrust had been solidified by the culture of the AI Lab hacker community. To the hackers, Watergate was merely a Shakespearean rendition of the daily power struggles that made life such a hassle for those without privilege. It was an outsized parable for what happened when people traded liberty and openness for security and convenience.
\fi

\ifdefined\chs
另一方面,``选举上帝''也影射了七十年代的水门事件。它把尼克松比作暴虐的天神。水门事件对理查德的影响很深。理查德从小就厌恶权威。如今长大成人,这份厌恶在人工智能实验室的黑客精神影响下,更加牢固。黑客们看来,水门事件好似莎士比亚剧一般,诠释着上层权力斗争如何戏虐无权的人。它恰似一幅缩影,展现着人们如何出于安全和方便的考虑,放弃自由和开放。
\fi

\ifdefined\eng
Buoyed by growing confidence, Stallman wore the button proudly. People curious enough to ask him about it received a well-prepared spiel. ``My name is Jehovah,'' Stallman would say. ``I have a secret plan to end injustice and suffering, but due to heavenly security reasons I can't tell you the workings of my plan. I see the big picture and you don't, and you know I'm good because I told you so. So put your faith in me and obey me without question. If you don't obey, that means you're evil, so I'll put you on my enemies list and throw you in a pit where the Infernal Revenue Service will audit your taxes every year for all eternity.''
\fi

\ifdefined\chs
戴上徽章,理查德自信满满。好奇的路人倘若问上一句,必然换来一篇准备已久的长篇大论:``我乃耶和华,从天而来,替天行道。除忧治病,铲逆为公。我眼中所见之事,并非肉眼凡胎所能看;我心中所思之物,也非莽夫俗子所能解。神谕无价,安全第一,行道之法,乃是天机,万万不可泄露。你等世人,尽管加信于我,毋疑毋问,顺天听命。旦有不从,即为妖孽,从此名姓入簿,打入非常之所,不得入三界,列五行。妖孽之地,行严法,加重税,世代如此,不得翻案。''
\fi

\ifdefined\eng
Those who interpreted the spiel as a parody of the Watergate hearings only got half the message. For Stallman, the other half of the message was something only his fellow hackers seemed to be hearing. One hundred years after Lord Acton warned about absolute power corrupting absolutely, Americans seemed to have forgotten the first part of Acton's truism: power, itself, corrupts. Rather than point out the numerous examples of petty corruption, Stallman felt content voicing his outrage toward an entire system that trusted power in the first place.
\fi

\ifdefined\chs
如果只是觉得这段话是对水门事件的恶搞,那只听懂了一半。另外一半,恐怕只有那群黑客的同类能听懂。百年之前,阿克頓男爵(Lord Acton)曾经发出警告:绝对权力导致绝对的腐败。而很多美国人,恐怕忘了这个警告的前半部分:权利本身导致腐败。理查德并没有直接列举各种腐败案例,他直指过分信任权利,从而酿成腐败的整个社会。
\fi

\ifdefined\eng
``I figured, why stop with the small fry,'' says Stallman, recalling the button and its message. ``If we went after Nixon, why not go after Mr. Big? The way I see it, any being that has power and abuses it deserves to have that power taken away.''
\fi

\ifdefined\chs
理查德回忆起徽章的事情,说:``我觉得,为什么把讨论就停留在这里了?如果仅仅看到水门这一桩事件,你也许只是不再信任尼克松和他的幕僚。可明天你也许就又开始信任尼克柏,尼克树,你依旧信任权力。我对待这些掌权者都是一个态度:倘若他滥用权力,就活该在哪天被剥夺权力。''
\fi

\theendnotes
\setcounter{endnote}{0}
